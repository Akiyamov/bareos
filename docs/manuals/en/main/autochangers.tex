%%
%%

\chapter{Autochanger Support}
\label{AutochangersChapter}
\index[general]{Support!Autochanger}
\index[general]{Autochanger!Support}

Bareos provides autochanger support for reading and writing tapes.  In
order to work with an autochanger, Bareos requires a number of things, each of
which is explained in more detail after this list:

\begin{itemize}
\item The package \package{bareos-storage-tape} must be installed.

\item A script that actually controls the autochanger according  to commands
   sent by Bareos. Bareos contains the script \command{mtx-changer}, that utilize the command \command{mtx}.
   It's config file is normally located at \file{/etc/bareos/mtx-changer.conf}

\item That each Volume (tape) to be used must be defined in the Catalog and
   have a Slot number assigned to it so that Bareos knows where the Volume is
   in the autochanger. This is generally done with the \bcommand{label}{} command, but
   can also done after the tape is labeled using the \bcommand{update}{slots}
   command.  See below for more details. You must pre-label the tapes manually
   before using them.

\item Modifications to your Storage daemon's Device configuration  resource to
   identify that the device is a changer, as well  as a few other parameters.

\item You need to ensure that your Storage daemon
   has access permissions to both the tape drive and the control device.
   On Linux, the system user \user{bareos} is added to the groups \group{disk} and \group{tape},
   so that it should have the permission to access the library.

\item Set \linkResourceDirectiveValue{Dir}{Storage}{Auto Changer}{yes}.
\end{itemize}

Bareos uses its own \command{mtx-changer} script to interface with a program
that actually does the tape changing.  Thus in principle, \command{mtx-changer}
can be adapted to function with any autochanger program, or you can
call any other script or program. The current
version of \command{mtx-changer} works with the \command{mtx} program.
FreeBSD users might need to adapt this script to use \command{chio}.
For more details, refer to the \ilink{Testing Autochanger}{AutochangerTesting} chapter.

Bareos also supports autochangers with barcode
readers. This support includes two Console commands: \bcommand{label}{barcodes}
and \bcommand{update}{slots}. For more details on these commands, see the chapter about \nameref{Barcodes}.

Current Bareos autochanger support does not include cleaning, stackers, or
silos. Stackers and silos are not supported because Bareos expects to
be able to access the Slots randomly.
However, if you are very careful to setup Bareos to access the Volumes
in the autochanger sequentially, you may be able to make Bareos
work with stackers (gravity feed and such).

In principle, if \command{mtx} will operate your changer correctly, then it is
just a question of adapting the \command{mtx-changer} script (or selecting one
already adapted) for proper interfacing.

% link does not exists any more:
%You can find a list of autochangers
%supported by {\bf mtx} at the following link:
%\elink{http://mtx.opensource-sw.net/compatibility.php}
%{http://mtx.opensource-sw.net/compatibility.php}.

%In addition, apparently certain versions of mtx, for example, version
%1.3.11 limit the number of slots to a maximum of 64. The solution was to
%use version 1.3.10.

If you are having troubles,
please use the {\bf auto} command in the \command{btape} program
to test the functioning of your autochanger with Bareos.
Please remember, that on most distributions, the \bareosSd
runs as user \user{bareos} and not as \user{root}.
You will need to ensure that the Storage daemon has sufficient
permissions to access the autochanger.

Some users have reported that the the Storage daemon blocks under certain
circumstances in trying to mount a volume on a drive that has a different
volume loaded.  As best we can determine, this is simply a matter of
waiting a bit.  The drive was previously in use writing a Volume, and
sometimes the drive will remain BLOCKED for a good deal of time (up to 7
minutes on a slow drive) waiting for the cassette to rewind and to unload
before the drive can be used with a different Volume.

\section{Knowing What SCSI Devices You Have}
\label{SCSI devices}
\index[general]{SCSI devices}
\index[general]{Devices!SCSI}
\index[general]{Devices!Detecting}

\subsection{Linux}
Under Linux, you can

\footnotesize
\begin{verbatim}
cat /proc/scsi/scsi
\end{verbatim}
\normalsize

to see what SCSI devices you have available. You can also:

\footnotesize
\begin{verbatim}
cat /proc/scsi/sg/device_hdr /proc/scsi/sg/devices
\end{verbatim}
\normalsize

to find out how to specify their control address ({\bf /dev/sg0} for the
first, {\bf /dev/sg1} for the second, ...) on the \linkResourceDirective{Sd}{Autochanger}{Changer Device}
Bareos directive.

You can also use the excellent  {\bf lsscsi} tool.
\footnotesize
\begin{verbatim}
$ lsscsi -g
 [1:0:2:0]    tape    SEAGATE  ULTRIUM06242-XXX 1619  /dev/st0  /dev/sg9
 [1:0:14:0]   mediumx STK      L180             0315  /dev/sch0 /dev/sg10
 [2:0:3:0]    tape    HP       Ultrium 3-SCSI   G24S  /dev/st1  /dev/sg11
 [3:0:0:0]    enclosu HP       A6255A           HP04  -         /dev/sg3
 [3:0:1:0]    disk    HP 36.4G ST336753FC       HP00  /dev/sdd  /dev/sg4
\end{verbatim}
\normalsize

% missing link:
%For more detailed information on what SCSI devices you have please see
%the \ilink{Linux SCSI Tricks}{SCSITricks}  section of the Tape Testing
%chapter of this manual.

\subsection{FreeBSD}
Under FreeBSD, use the following command to list the SCSI devices as well as the \path|/dev/passN| that you will use on
the Bareos \linkResourceDirective{Sd}{Autochanger}{Changer Device} directive:


\footnotesize
\begin{verbatim}
camcontrol devlist
\end{verbatim}
\normalsize

Please check that your Storage daemon has permission to access this
device.

The following tip for FreeBSD users comes from Danny Butroyd:
on reboot Bareos will NOT have permission to
control the device \file{/dev/pass0} (assuming this is your changer device).
To get around this just edit the \file{/etc/devfs.conf} file and add the
following to the bottom:
\footnotesize
\begin{verbatim}
own     pass0   root:bareos
perm    pass0   0666
own     nsa0.0  root:bareos
perm    nsa0.0    0666
\end{verbatim}
\normalsize

This gives the bareos group permission to write to the nsa0.0 device
too just to be on the safe side.   To bring these changes into effect
just run:-

\begin{commands}{}
/etc/rc.d/devfs restart
\end{commands}

Basically this will stop you having to manually change permissions on these
devices to make Bareos work when operating the AutoChanger after a reboot.

\subsection{Solaris}

On Solaris, the changer device will typically be some file under \file{/dev/rdsk}.



\section{Slots}
\index[general]{Slots}
\label{Slots}

To properly address autochangers, Bareos must know which Volume is in each
{\bf slot} of the autochanger. Slots are where the changer cartridges reside
when not loaded into the drive. Bareos numbers these slots from one to the
number of cartridges contained in the autochanger.

Bareos will not automatically use a Volume in your autochanger unless it is
labeled and the slot number is stored in the catalog and the Volume is marked
as InChanger. This is because it must know where each volume is to
be able to load the volume.
For each Volume in your
changer, you will, using the Console program, assign a slot. This information
is kept in Bareos's catalog database along with the other data for the
volume. If no slot is given, or the slot is set to zero, Bareos will not
attempt to use the autochanger even if all the necessary configuration records
are present. When doing a \bcommand{mount}{} command on an autochanger, you must
specify which slot you want mounted.  If the drive is loaded with a tape
from another slot, it will unload it and load the correct tape, but
normally, no tape will be loaded because an \bcommand{unmount}{} command causes
Bareos to unload the tape in the drive.


You can check if the Slot number and InChanger flag by:
\begin{bconsole}{list volumes}
*list volumes
\end{bconsole}



\section{Multiple Devices}
\label{sec:MultipleDevices}
\index[general]{Devices!Multiple}
\index[general]{Multiple Devices}

Some autochangers have more than one read/write device (drive). The
\ilink{Autochanger resource}{AutochangerRes} permits you to group Device resources, where each device
represents a drive. The Director may still reference the Devices (drives)
directly, but doing so, bypasses the proper functioning of the
drives together.  Instead, the Director (in the Storage resource)
should reference the Autochanger resource name. Doing so permits
the Storage daemon to ensure that only one drive uses the mtx-changer
script at a time, and also that two drives don't reference the
same Volume.

Multi-drive requires the use of the \linkResourceDirective{Sd}{Device}{Drive Index} directive.
Drive numbers or the Device Index are numbered beginning
at zero, which is the default. To use the second Drive in an autochanger, you
need to define a second Device resource, set the \linkResourceDirective{Sd}{Device}{Drive Index}
and set the \linkResourceDirective{Sd}{Device}{Archive Device}.

As a default, Bareos jobs will prefer to write to a Volume that is
already mounted. If you have a multiple drive autochanger and you want
Bareos to write to more than one Volume in the same Pool at the same
time, you will need to set \linkResourceDirectiveValue{Dir}{Job}{Prefer Mounted Volumes}{no}.
This will cause the Storage daemon to maximize the use of drives.



\section{Device Configuration Records}
\index[general]{Device Configuration Records}

Configuration of autochangers within Bareos is done in the Device resource of
the Storage daemon.

Following records control how Bareos uses the autochanger:

\begin{description}
    \item[\linkResourceDirective{Sd}{Device}{Autochanger}]
        Specifies if the current device belongs to an
        autochanger resource.

    \item[\linkResourceDirective{Sd}{Autochanger}{Changer Command} (\linkResourceDirective{Sd}{Device}{Changer Command})]

    \item[\linkResourceDirective{Sd}{Autochanger}{Changer Device} (\linkResourceDirective{Sd}{Device}{Changer Device})]

    \item[\linkResourceDirective{Sd}{Device}{Drive Index}]
        Individual driver number, starting at 0.

    \item[\linkResourceDirective{Sd}{Device}{Maximum Changer Wait}]
\end{description}



\section{Specifying Slots When Labeling}
\index[general]{Specifying Slots When Labeling}
\index[general]{Label!Specifying Slots When Labeling}
\label{SpecifyingSlots}

If you add an {\bf Autochanger = yes} record to the Storage resource in your
Director's configuration file, the Bareos Console will automatically prompt
you for the slot number when the Volume is in the changer when
you {\bf add} or {\bf label} tapes for that Storage device. If your
{\bf mtx-changer} script is properly installed, Bareos will automatically
load the correct tape during the label command.

You must also set
{\bf Autochanger = yes} in the Storage daemon's Device resource
as we have described above in order for the autochanger to be used.
%\TODO{check: why not in the example?}
Please see
\linkResourceDirective{Dir}{Storage}{Auto Changer}
and
\linkResourceDirective{Sd}{Device}{Autochanger}
for more details on these records.

Thus all stages of dealing with tapes can be totally automated. It is also
possible to set or change the Slot using the {\bf update} command in the
Console and selecting {\bf Volume Parameters} to update.

Even though all the above configuration statements are specified and correct,
Bareos will attempt to access the autochanger only if a {\bf slot} is non-zero
in the catalog Volume record (with the Volume name).

If your autochanger has barcode labels, you can label all the Volumes in
your autochanger one after another by using the \bcommand{label}{barcodes} command.
For each tape in the changer containing a barcode, Bareos will mount the tape
and then label it with the same name as the barcode. An appropriate Media
record will also be created in the catalog. Any barcode that begins with the
same characters as specified on the "CleaningPrefix=xxx" command, will be
treated as a cleaning tape, and will not be labeled.
For example with:

\footnotesize
\begin{verbatim}
Pool {
  Name ...
  Cleaning Prefix = "CLN"
}
\end{verbatim}
\normalsize

Any slot containing a barcode of CLNxxxx will be treated as a cleaning tape
and will not be mounted.


\section{Changing Cartridges}
\index[general]{Cartridges!Changing}
If you wish to insert or remove cartridges in your autochanger or
you manually run the {\bf mtx} program, you must first tell Bareos
to release the autochanger by doing:

\footnotesize
\begin{verbatim}
unmount
(change cartridges and/or run mtx)
mount
\end{verbatim}
\normalsize

If you do not do the unmount before making such a change, Bareos
will become completely confused about what is in the autochanger
and may stop function because it expects to have exclusive use
of the autochanger while it has the drive mounted.


\section{Dealing with Multiple Magazines}
\index[general]{Magazines!Dealing with Multiple}

If you have several magazines or if you insert or remove cartridges from a
magazine, you should notify Bareos of this. By doing so, Bareos will as
a preference, use Volumes that it knows to be in the autochanger before
accessing Volumes that are not in the autochanger. This prevents unneeded
operator intervention.

If your autochanger has barcodes (machine readable tape labels), the task of
informing Bareos is simple. Every time, you change a magazine, or add or
remove a cartridge from the magazine, simply use following commands in the Console program:

\footnotesize
\begin{verbatim}
unmount
(remove magazine)
(insert new magazine)
update slots
mount
\end{verbatim}
\normalsize

This will cause Bareos to request the autochanger to
return the current Volume names in the magazine. This will be done without
actually accessing or reading the Volumes because the barcode reader does this
during inventory when the autochanger is first turned on. Bareos will ensure
that any Volumes that are currently marked as being in the magazine are marked
as no longer in the magazine, and the new list of Volumes will be marked as
being in the magazine. In addition, the Slot numbers of the Volumes will be
corrected in Bareos's catalog if they are incorrect (added or moved).

If you do not have a barcode reader on your autochanger, you have several
alternatives.

\begin{enumerate}
\item You can manually set the Slot and InChanger flag using  the {\bf update
   volume} command in the Console (quite  painful).

\item You can issue a

\footnotesize
\begin{verbatim}
update slots scan
\end{verbatim}
\normalsize

   command that will cause Bareos to read the label on each  of the cartridges in
   the magazine in turn and update the  information (Slot, InChanger flag) in the
   catalog. This  is quite effective but does take time to load each cartridge
   into the drive in turn and read the Volume label.

%\item You can modify the mtx-changer script so that it simulates  an
%   autochanger with barcodes. See below for more details.
\end{enumerate}

\hide{
% unwanted, commented out
\section{Simulating Barcodes in your Autochanger}
\index[general]{Autochanger!Simulating Barcodes in your}
\index[general]{Simulating Barcodes in your Autochanger}
\label{simulating}

You can simulate barcodes in your autochanger by making the {\bf mtx-changer}
script return the same information that an autochanger with barcodes would do.
This is done by commenting out the one and only line in the {\bf list)} case,
which is:

\footnotesize
\begin{verbatim}
  ${MTX} -f $ctl status | grep " *Storage Element [0-9]*:.*Full" | awk "{print \$3 \$4}" | sed "s/Full *\(:VolumeTag=\)*//"
\end{verbatim}
\normalsize

at approximately line 99 by putting a \# in column one of that line, or by
simply deleting it. Then in its place add a new line that prints the contents
of a file. For example:

\footnotesize
\begin{verbatim}
cat /etc/bareos/changer.volumes
\end{verbatim}
\normalsize

Be sure to include a full path to the file, which can have any name. The
contents of the file must be of the following format:

\footnotesize
\begin{verbatim}
1:Volume1
2:Volume2
3:Volume3
...
\end{verbatim}
\normalsize

Where the 1, 2, 3 are the slot numbers and Volume1, Volume2, ... are the
Volume names in those slots. You can have multiple files that represent the
Volumes in different magazines, and when you change magazines, simply copy the
contents of the correct file into your {\bf /etc/bareos/changer.volumes} file.
There is no need to stop and start Bareos when you change magazines, simply
put the correct data in the file, then run the {\bf update slots} command, and
your autochanger will appear to Bareos to be an autochanger with barcodes.
}


\section{Update Slots Command}
\index[general]{Console!Command!update slots}
\label{updateslots}

If you change only one cartridge in the magazine, you may not want to scan all
Volumes, so the {\bf update slots} command (as well as the {\bf update slots
scan} command) has the additional form:

\footnotesize
\begin{verbatim}
update slots=n1,n2,n3-n4, ...
\end{verbatim}
\normalsize

where the keyword {\bf scan} can be appended or not. The n1,n2, ... represent
Slot numbers to be updated and the form n3-n4 represents a range of Slot
numbers to be updated (e.g. 4-7 will update Slots 4,5,6, and 7).

This form is particularly useful if you want to do a scan (time expensive) and
restrict the update to one or two slots.

For example, the command:

\footnotesize
\begin{verbatim}
update slots=1,6 scan
\end{verbatim}
\normalsize

will cause Bareos to load the Volume in Slot 1, read its Volume label and
update the Catalog. It will do the same for the Volume in Slot 6. The command:


\footnotesize
\begin{verbatim}
update slots=1-3,6
\end{verbatim}
\normalsize

will read the barcoded Volume names for slots 1,2,3 and 6 and make the
appropriate updates in the Catalog. If you don't have a barcode reader the above command will
not find any Volume names so will do nothing.


\section{Using the Autochanger}
\index[general]{Autochanger!Using the}
\label{using}

Let's assume that you have properly defined the necessary Storage daemon
Device records, and you have added the {\bf Autochanger = yes} record to the
Storage resource in your Director's configuration file.

Now you fill your autochanger with say six blank tapes.

What do you do to make Bareos access those tapes?

One strategy is to prelabel each of the tapes. Do so by starting Bareos, then
with the Console program, enter the {\bf label} command:

\footnotesize
\begin{verbatim}
./bconsole
Connecting to Director rufus:8101
1000 OK: rufus-dir Version: 1.26 (4 October 2002)
*label
\end{verbatim}
\normalsize

it will then print something like:

\footnotesize
\begin{verbatim}
Using default Catalog name=BackupDB DB=bareos
The defined Storage resources are:
     1: Autochanger
     2: File
Select Storage resource (1-2): 1
\end{verbatim}
\normalsize

I select the autochanger (1), and it prints:

\footnotesize
\begin{verbatim}
Enter new Volume name: TestVolume1
Enter slot (0 for none): 1
\end{verbatim}
\normalsize

where I entered {\bf TestVolume1} for the tape name, and slot {\bf 1} for the
slot. It then asks:

\footnotesize
\begin{verbatim}
Defined Pools:
     1: Default
     2: File
Select the Pool (1-2): 1
\end{verbatim}
\normalsize

I select the Default pool. This will be automatically done if you only have a
single pool, then Bareos will proceed to unload any loaded volume, load the
volume in slot 1 and label it. In this example, nothing was in the drive, so
it printed:

\footnotesize
\begin{verbatim}
Connecting to Storage daemon Autochanger at localhost:9103 ...
Sending label command ...
3903 Issuing autochanger "load slot 1" command.
3000 OK label. Volume=TestVolume1 Device=/dev/nst0
Media record for Volume=TestVolume1 successfully created.
Requesting mount Autochanger ...
3001 Device /dev/nst0 is mounted with Volume TestVolume1
You have messages.
*
\end{verbatim}
\normalsize

You may then proceed to label the other volumes. The messages will change
slightly because Bareos will unload the volume (just labeled TestVolume1)
before loading the next volume to be labeled.

Once all your Volumes are labeled, Bareos will automatically load them as they
are needed.

To "see" how you have labeled your Volumes, simply enter the {\bf list
volumes} command from the Console program, which should print something like
the following:

\footnotesize
\begin{verbatim}
*{\bf list volumes}
Using default Catalog name=BackupDB DB=bareos
Defined Pools:
     1: Default
     2: File
Select the Pool (1-2): 1
+-------+----------+--------+---------+-------+--------+----------+-------+------+
| MedId | VolName  | MedTyp | VolStat | Bites | LstWrt | VolReten | Recyc | Slot |
+-------+----------+--------+---------+-------+--------+----------+-------+------+
| 1     | TestVol1 | DDS-4  | Append  | 0     | 0      | 30672000 | 0     | 1    |
| 2     | TestVol2 | DDS-4  | Append  | 0     | 0      | 30672000 | 0     | 2    |
| 3     | TestVol3 | DDS-4  | Append  | 0     | 0      | 30672000 | 0     | 3    |
| ...                                                                            |
+-------+----------+--------+---------+-------+--------+----------+-------+------+
\end{verbatim}
\normalsize



\section{Barcode Support}
\index[general]{Support!Barcode}
\index[general]{Barcode Support}
\label{Barcodes}

Bareos provides barcode support with two Console commands, {\bf label
barcodes} and {\bf update slots}.

The {\bf label barcodes} will cause Bareos to read the barcodes of all the
cassettes that are currently installed in the magazine (cassette holder) using
the {\bf mtx-changer} {\bf list} command. Each cassette is mounted in turn and
labeled with the same Volume name as the barcode.

The {\bf update slots} command will first obtain the list of cassettes and
their barcodes from {\bf mtx-changer}. Then it will find each volume in turn
in the catalog database corresponding to the barcodes and set its Slot to
correspond to the value just read. If the Volume is not in the catalog, then
nothing will be done. This command is useful for synchronizing Bareos with the
current magazine in case you have changed magazines or in case you have moved
cassettes from one slot to another. If the autochanger is empty, nothing will
be done.

The {\bf Cleaning Prefix} statement can be used in the Pool resource to define
a Volume name prefix, which if it matches that of the Volume (barcode) will
cause that Volume to be marked with a VolStatus of {\bf Cleaning}. This will
prevent Bareos from attempting to write on the Volume.

\section{Use bconsole to display Autochanger content}

The {\bf status slots storage=xxx} command displays autochanger content.

\footnotesize
\begin{verbatim}
 Slot |  Volume Name    |  Status  |      Type         |    Pool        |  Loaded |
------+-----------------+----------+-------------------+----------------+---------|
    1 |           00001 |   Append |  DiskChangerMedia |        Default |    0    |
    2 |           00002 |   Append |  DiskChangerMedia |        Default |    0    |
    3*|           00003 |   Append |  DiskChangerMedia |        Scratch |    0    |
    4 |                 |          |                   |                |    0    |
\end{verbatim}
\normalsize

If you see a {\bf *} near the slot number, you have to run {\bf update slots}
command to synchronize autochanger content with your catalog.



\section{Bareos Autochanger Interface}
\index[general]{Autochanger!Interface}
\label{autochanger-interface}

Bareos calls the autochanger script that you specify on the {\bf Changer
Command} statement. Normally this script will be the {\bf mtx-changer} script
that we provide, but it can in fact be any program. The only requirement
for the script is that it must understand the commands that
Bareos uses, which are {\bf loaded}, {\bf load}, {\bf
unload}, {\bf list}, and {\bf slots}. In addition,
each of those commands must return the information in the precise format as
specified below:

\footnotesize
\begin{verbatim}
- Currently the changer commands used are:
    loaded -- returns number of the slot that is loaded, base 1,
              in the drive or 0 if the drive is empty.
    load   -- loads a specified slot (note, some autochangers
              require a 30 second pause after this command) into
              the drive.
    unload -- unloads the device (returns cassette to its slot).
    list   -- returns one line for each cassette in the autochanger
              in the format <slot>:<barcode>. Where
              the {\bf slot} is the non-zero integer representing
              the slot number, and {\bf barcode} is the barcode
              associated with the cassette if it exists and if you
              autoloader supports barcodes. Otherwise the barcode
              field is blank.
    slots  -- returns total number of slots in the autochanger.
\end{verbatim}
\normalsize

Bareos checks the exit status of the program called, and if it is zero, the
data is accepted. If the exit status is non-zero, Bareos will print an
error message and request the tape be manually mounted on the drive.


\section{Tapespeed and blocksizes}
\index[general]{Tuning!Tape}
\index[general]{Tuning!blocksize}
\index[general]{Tape!speed}
\index[general]{Blocksize!optimize}
\label{Tapespeed and blocksizes}
% referenced from https://www.bareos.org/en/Whitepapers/articles/Speed_Tuning_of_Tape_Drives.html
\label{setblocksizes}

The \bareosWhitepaperTapeSpeedTuning shows that the two parameters
\configdirective{Maximum File Size} and \configdirective{Maximum Block Size}
of the device have significant influence on the tape speed.

While it is no problem to change the \linkResourceDirective{Sd}{Device}{Maximum File Size} parameter,
unfortunately it is not possible to change the \linkResourceDirective{Sd}{Device}{Maximum Block Size}
parameter, because the previously written tapes would become unreadable
in the new setup. It would require that the \linkResourceDirective{Sd}{Device}{Maximum Block Size} parameter
is switched back to the old value to be able to read the old volumes, but
of course then the new volumes would be unreadable.

Why is that the case?

The problem is that Bareos writes the label block (header) in the same block
size that is configured in the \linkResourceDirective{Sd}{Device}{Maximum Block Size} parameter in the device.
Per default, this value is 63k, so usually a tape written by Bareos looks like this:

\begin{verbatim}
|-------------------
|label block  (63k)|
|-------------------
|data block  1(63k)|
|data block  2(63k)|
|...               |
|data block  n(63k)|
--------------------
\end{verbatim}
Setting the maximum block size to e.g. 512k, would lead to the following:
\begin{verbatim}
|-------------------
|label block (512k)|
|-------------------
|data block 1(512k)|
|data block 2(512k)|
|...               |
|data block n(512k)|
--------------------
\end{verbatim}

As you can see, every block is written with the maximum block size, also the label block.

The problem that arises here is that reading a block header with a wrong block size causes
a read error which is interpreted as an non-existent label by Bareos.

This is a potential source of data loss, because in normal operation, Bareos refuses to
relabel an already labeled volume to be sure to not overwrite data that is still needed.
If Bareos cannot read the volume label, this security mechanism does not work and you might
label tapes already labeled accidentally.

To solve this problem, the block size handling was changed in Bareos \sinceVersion{sd}{Maximum Block Size}{14.2.0} in the following way:
\begin{itemize}
\item  The tape label block is always written in the standard 63k (64512) block size.
\item  The following blocks are then written in the block size configured in the \configdirective{Maximum Block Size} directive.
\item  To be able to change the block size in an existing environment, it is now
    possible to set the \linkResourceDirective{Dir}{Pool}{Maximum Block Size}
    and
    \linkResourceDirective{Dir}{Pool}{Minimum Block Size}
    in the pool resource. This setting is automatically promoted to each medium in
that pool as usual (i.e. when a medium is labeled for that pool or if a volume is transferred to that pool from the scratch pool).
When a volume is mounted, the volume's block size is
used to write and read the data blocks that follow the header block.
\end{itemize}

The following picture shows the result:
\begin{verbatim}
|--------------------------------|
|label block (label block size)  |
|--------------------------------|
|data block 1(maximum block size)|
|data block 2(maximum block size)|
|...                             |
|data block n(maximum block size)|
---------------------------------|
\end{verbatim}
We have a label block with a certain size (63k per default to be compatible to old installations),
and the following data blocks are written with another blocksize.


This approach has the following advantages:
\begin{itemize}
\item If nothing is configured, existing installations keep on working without problems.
\item If you want to switch an existing installation that uses the default
block size and move to a new (usually bigger) block size, you can do that
easily by creating a new pool, where \linkResourceDirective{Dir}{Pool}{Maximum Block Size} is set to the new
value that you wish to use in the future:
\end{itemize}


\begin{bconfig}{Pool Ressource: setting Maximum Block Size}
Pool {
   Name = LTO-4-1M
      Pool Type = Backup
      Recycle = yes                       # Bareos can automatically recycle Volumes
      AutoPrune = yes                     # Prune expired volumes
      Volume Retention = 1 Month          # How long should the Full Backups be kept? (#06)
      Maximum Block Size = 1048576
      Recycle Pool = Scratch
}

\end{bconfig}

Now configure your backups that they will write into the newly defined pool in the
future, and your backups will be written with the new block size.

Your existing tapes can be automatically transferred to the new pool when they expire
via the \ilink{Scratch Pool}{TheScratchPool} mechanism. When a tape in your old pool expires, it is
transferred to the scratch pool if you set
{\bf Recycle Pool = Scratch}. When your new pool needs a new volume, it will get it
from the scratch pool and apply the new pool's properties to that tape which also
include  \linkResourceDirective{Dir}{Pool}{Maximum Block Size} and  \linkResourceDirective{Dir}{Pool}{Minimum Block Size}.

This way you can smoothly switch your tapes to a new block size while you can still
restore the data on your old tapes at any time.

\subsection*{Possible Problems}
There is only one case where the new block handling will cause problems, and this
is if you have used bigger block sizes already in your setup. As we now defined the
label block to always be 63k, all labels will not be readable.

To also solve this problem, the directive \linkResourceDirective{Sd}{Device}{Label Block Size} can be used to define
a different label block size.  That way, everything should work smoothly as all label
blocks will be readable again.


\subsection*{How can I find out which block size was used when the tape was written?}

At least on Linux, you can see if Bareos tries to read the blocks with the
wrong block size. In that case, you get a kernel message like the following in
your system's messages:
\begin{verbatim}
[542132.410170] st1: Failed to read 1048576 byte block with 64512 byte transfer.
\end{verbatim}
Here, the block was written with 1M block size but we only read 64k.

\subsection*{Direct access to Volumes with with non-default block sizes}
\label{direct-access-to-volumes-with-non-default-blocksizes}
\index[general]{bls!block size}
\index[general]{bextract!block size}
\index[general]{Command!bls!block size}
\index[general]{Command!bextract!block size}


\command{bls} and \command{bextract} can directly access Bareos volumes without catalog database.
This means that these programs don't have information about the used block size.

To be able to read a volume written with an arbitrary block size, you need to
set the  \linkResourceDirective{Sd}{Device}{Label Block Size} (to be able to to read the label block) and the
 \linkResourceDirective{Sd}{Device}{Maximum Block Size} (to be able to read the data blocks) setting in the
device definition used by those tools to be able to open the medium.

Example using \command{bls} with a tape that was written with another blocksize than the
\variable{DEFAULT_BLOCK_SIZE} (63k), but with the default label block size of 63k:
\begin{commands}{bls with non-default block size}
<command>bls</command> <parameter>FC-Drive-1 -V A00007L4</parameter>
bls: butil.c:289-0 Using device: "FC-Drive-1" for reading.
25-Feb 12:47 bls JobId 0: No slot defined in catalog (slot=0) for Volume "A00007L4" on "FC-Drive-1" (/dev/tape/by-id/scsi-350011d00018a5f03-nst).
25-Feb 12:47 bls JobId 0: Cartridge change or "update slots" may be required.
25-Feb 12:47 bls JobId 0: Ready to read from volume "A00007L4" on device "FC-Drive-1" (/dev/tape/by-id/scsi-350011d00018a5f03-nst).
25-Feb 12:47 bls JobId 0: Error: block.c:1004 Read error on fd=3 at file:blk 0:1 on device "FC-Drive-1" (/dev/tape/by-id/scsi-350011d00018a5f03-nst). ERR=Cannot allocate memory.
 Bareos status: file=0 block=1
 Device status: ONLINE IM_REP_EN file=0 block=2
0 files found.
\end{commands}

As can be seen, \command{bls} manages to read the label block as it knows what volume is mounted
(Ready to read from volume \parameter{A00007L4}), but fails to read the data blocks.

\begin{commands}{dmesg}
<command>dmesg</command>
[...]
st2: Failed to read 131072 byte block with 64512 byte transfer.
[...]
\end{commands}

This shows that the block size for the data blocks that we need is 131072.

Now we have to set this block size in the \file{bareos-sd.conf}, device
resource as \linkResourceDirective{Sd}{Device}{Maximum Block Size}:

\begin{bconfig}{Storage Device Resource: setting Maximum Block Size}
Device {
  Name = FC-Drive-1
  Drive Index = 0
  Media Type = LTO-4
  Archive Device = /dev/tape/by-id/scsi-350011d00018a5f03-nst
  AutomaticMount = yes
  AlwaysOpen = yes
  RemovableMedia = yes
  RandomAccess = no
  AutoChanger = yes
  Maximum Block Size = 131072
}
\end{bconfig}


Now we can call bls again, and everything works as expected:
\begin{commands}{bls with non-default block size}
<command>bls</command> <parameter>FC-Drive-1 -V A00007L4</parameter>
bls: butil.c:289-0 Using device: "FC-Drive-1" for reading.
25-Feb 12:49 bls JobId 0: No slot defined in catalog (slot=0) for Volume "A00007L4" on "FC-Drive-1" (/dev/tape/by-id/scsi-350011d00018a5f03-nst).
25-Feb 12:49 bls JobId 0: Cartridge change or "update slots" may be required.
25-Feb 12:49 bls JobId 0: Ready to read from volume "A00007L4" on device "FC-Drive-1" (/dev/tape/by-id/scsi-350011d00018a5f03-nst).
bls JobId 203: [...]
\end{commands}


\subsection*{How to configure the block sizes in your environment}
The following chart shows how to set the directives for {\bf maximum block size} and {\bf label block size}
depending on how your current setup is:

\begin{center}
\includegraphics[width=0.8\linewidth]{\idir blocksize-decisionchart}
\end{center}

\section{Tape Drive Cleaning}

Bareos has no build-in functionality for tape drive cleaning.
Fortunately this is not required as most modern tape libraries
have build in auto-cleaning functionality.
This functionality might require an empty tape drive,
so the tape library gets aware, that it is currently not used.
However, by default Bareos keeps tapes in the drives,
in case the same tape is required again.

The directive \linkResourceDirective{Dir}{Pool}{Cleaning Prefix}
is only used for making sure that Bareos does not try to write backups on a cleaning tape.

If your tape libraries auto-cleaning won't work when there are tapes in the drives,
it's probably best to set up an admin job that removes the tapes from the drives.
This job has to run, when no other backups do run.
A job definition for an admin job to do that may look like this:

\begin{bareosConfigResource}{bareos-dir}{job}{ReleaseAllTapeDrives}
Job {
    Name = ReleaseAllTapeDrives
    JobDefs = DefaultJob
    Schedule = "WeeklyCycleAfterBackup"
    Type = Admin
    Priority = 200

    RunScript {
        Runs When = Before
        Runs On Client = no
        Console = "release storage=Tape alldrives"
    }
}
\end{bareosConfigResource}

Replace \resourcename{Dir}{Storage}{Tape} by the storage name of your tape library.
Use the highest \linkResourceDirective{Dir}{Job}{Priority}
value to make sure no other jobs are running.
In the default configuration for example, the \job{CatalogBackup} job
has Priority = 100. The higher the number, the lower the job priority.
