\chapter{File Deduplication using Base Jobs}
\label{basejobs}
\index[general]{Base Jobs}
\index[general]{File Deduplication}

A base job is sort of like a Full save except that you will want the FileSet to
contain only files that are unlikely to change in the future (i.e.  a snapshot
of most of your system after installing it).  After the base job has been run,
when you are doing a Full save, you specify one or more Base jobs to be used.
All files that have been backed up in the Base job/jobs but not modified will
then be excluded from the backup.  During a restore, the Base jobs will be
automatically pulled in where necessary.

Imagine having 100
nearly identical Windows or Linux machine containing the OS and user files.
Now for the OS part, a Base job will be backed up once, and rather than making
100 copies of the OS, there will be only one.  If one or more of the systems
have some files updated, no problem, they will be automatically backuped.

A new Job directive \configdirective{Base=JobX,JobY,...} permits to specify the list of
files that will be used during Full backup as base.

\begin{verbatim}
Job {
   Name = BackupLinux
   Level= Base
   ...
}

Job {
   Name = BackupZog4
   Base = BackupZog4, BackupLinux
   Accurate = yes
   ...
}
\end{verbatim}

In this example, the job \texttt{BackupZog4} will use the most recent version
of all files contained in \texttt{BackupZog4} and \texttt{BackupLinux}
jobs. Base jobs should have run with \configdirective{Level=Base} to be used.

By default, Bareos will compare permissions bits, user and group fields,
modification time, size and the checksum of the file to choose between the
current backup and the BaseJob file list. You can change this behavior with the
\texttt{BaseJob} FileSet option. This option works like the \configdirective{Verify}, that is described in the \ilink{FileSet}{FileSetResource} chapter.

\begin{verbatim}
FileSet {
  Name = Full
  Include = {
    Options {
       BaseJob  = pmugcs5
       Accurate = mcs
       Verify   = pin5
    }
    File = /
  }
}
\end{verbatim}

\warning{The current implementation doesn't permit to scan
volume with \command{bscan}. The result wouldn't permit to restore files easily.}
