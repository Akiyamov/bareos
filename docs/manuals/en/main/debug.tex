\chapter{Debugging}

%\chapter{What To Do When Bareos Crashes (Kaboom)}
\index[general]{Crash}
\index[general]{Debug!crash}

This chapter describes, how to debug Bareos, when the program crashes.
If you are just interested about how to get more information about a running Bareos daemon,
please read \nameref{sec:debug-messages}.

If you are running on a Linux system, and you have a set of working
configuration files, it is very unlikely that {\bf Bareos} will crash. As with
all software, however, it is inevitable that someday, it may crash.

This chapter explains what you should do if one of the three {\bf Bareos}
daemons (Director, File, Storage) crashes.  When we speak of crashing, we
mean that the daemon terminates abnormally because of an error.  There are
many cases where Bareos detects errors (such as PIPE errors) and will fail
a job. These are not considered crashes.  In addition, under certain
conditions, Bareos will detect a fatal in the configuration, such as
lack of permission to read/write the working directory. In that case,
Bareos will force itself to crash with a SEGFAULT. However, before
crashing, Bareos will normally display a message indicating why.
For more details, please read on.

\section{Traceback}
\index[general]{Traceback}

Each of the three Bareos daemons has a built-in exception handler which, in
case of an error, will attempt to produce a traceback. If successful the
traceback will be emailed to you.

For this to work, you need to ensure that a few things are setup correctly on
your system:

\begin{enumerate}
\item You must have a version of Bareos with debug information and not stripped of debugging symbols.
  When using a packaged version of Bareos, this requires to install the Bareos debug packages 
  (\package{bareos-debug} on RPM based systems, \package{bareos-dbg} on Debian based systems).

\item On Linux, \command{gdb} (the GNU debugger) must be installed. 
   On some systems such as Solaris, \command{gdb} may be replaced by \command{dbx}.

\item By default, btraceback uses \command{bsmtp} to send the traceback via email.
   Therefore it expects a local mail transfer daemon running. 
   It send the traceback to \email{root@localhost} via \host{localhost}.

\item Some Linux distributions, e.g. Ubuntu\index[general]{Platform!Ubuntu!Debug}, 
    disable the possibility to examine the memory of other processes.
    While this is a good idea for hardening a system, our debug mechanismen will fail.
    To disable this feature, run (as root):
\begin{commands}{disable ptrace protection to enable debugging (required on Ubuntu Linux)}
<command> </command><parameter>test -e /proc/sys/kernel/yama/ptrace_scope && echo 0 > /proc/sys/kernel/yama/ptrace_scope</parameter>
\end{commands}
 
\end{enumerate}

If all the above conditions are met, the daemon that crashes will produce a
traceback report and email it. If the above conditions are not true,
you can run the debugger by hand as described below.

\section{Testing The Traceback}
\index[general]{Traceback!Test}

To "manually" test the traceback feature, you simply start {\bf Bareos} then
obtain the {\bf PID} of the main daemon thread (there are multiple threads).
The output produced here will look different depending on what OS and what
version of the kernel you are running.

\begin{commands}{get the process ID of a running Bareos daemon}
<command> </command><parameter>ps fax | grep bareos-dir</parameter>
 2103 ?        S      0:00 /usr/sbin/bareos-dir
\end{commands}

which in this case is 2103. Then while Bareos is running, you call the program
giving it the path to the Bareos executable and the {\bf PID}. In this case,
it is:

\begin{commands}{get traceback of running Bareos director daemon}
<command> </command><parameter>btraceback /usr/sbin/bareos-dir 2103</parameter>
\end{commands}

It should produce an email showing you the current state of the daemon (in
this case the Director), and then exit leaving {\bf Bareos} running as if
nothing happened. If this is not the case, you will need to correct the
problem by modifying the \command{btraceback} script.


\subsection{Getting A Traceback On Other Systems}

It should be possible to produce a similar traceback on systems other than
Linux, either using \command{gdb} or some other debugger. 
Solaris\index[general]{Platform!Solaris!Debug} with \command{dbx}
loaded works quite fine. On other systems, you will need to modify the 
\command{btraceback} program to invoke the correct debugger, and possibly correct the
\file{btraceback.gdb} script to have appropriate commands for your debugger.
Please keep in mind that for any debugger to
work, it will most likely need to run as root.

\section{Manually Running Bareos Under The Debugger}

If for some reason you cannot get the automatic traceback, or if you want to
interactively examine the variable contents after a crash, you can run Bareos
under the debugger. Assuming you want to run the Storage daemon under the
debugger (the technique is the same for the other daemons, only the name
changes), you would do the following:

\begin{enumerate}
\item The Director and the File daemon should  be running but
   the Storage daemon should not.

\item Start the Storage daemon under the debugger:

\begin{commands}{run the Bareos Storage daemon in the debugger}
<command>gdb</command><parameter> --args /usr/sbin/bareos-sd -f -s -d 200</parameter>
(gdb) <input>run</input>
\end{commands}

Parameter:
\begin{description}
 \item[-f] foreground
 \item[-s] no signals
 \item[-d nnn] debug level 
\end{description}
See section \ilink{daemon command line options}{daemon-command-line-options} for a detailed list of options.

\item At this point, Bareos will be fully operational.

\item In another shell command window, start the Console program  and do what
   is necessary to cause Bareos to die.

\item When Bareos crashes, the {\bf gdb} shell window will  become active and
   {\bf gdb} will show you the error that  occurred.

\item To get a general traceback of all threads, issue the following  command:

\begin{commands}{run the Bareos Storage daemon in the debugger}
(gdb) <input>thread apply all bt</input>
\end{commands}

After that you can issue any debugging command.
\end{enumerate}
