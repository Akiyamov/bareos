\chapter{Customizing the Configuration}
\label{ConfigureChapter}
\index[general]{Customizing the Configuration}

Each Bareos component (Director, Client, Storage, Console) has its own configuration
containing a set of resource definitions. These resources are very
similar from one service to another, but may contain different directives
(records) depending on the component. For example, in the Director configuration,
the \nameref{DirectorResourceDirector} defines the name of the Director, a number
of global Director parameters and his password. In the File daemon
configuration, the \nameref{ClientResourceDirector} specifies which Directors are
permitted to use the File daemon.

If you install all Bareos daemons (Director, Storage and File Daemon) onto one system,
the Bareos package tries its best to generate a working configuration as a basis for your individual configuration.

The details of each resource and the directives permitted therein are
described in the following chapters.

The following configuration files must be present:

\begin{itemize}
\item
   \nameref{DirectorChapter} -- to define the resources
   necessary for the Director. You define all the Clients  and Storage daemons
   that you use in this configuration file.
\item
   \nameref{StoredConfChapter} -- to define the resources to
   be used by each Storage daemon. Normally, you will have a single Storage
   daemon that controls your disk storage or tape drives. However, if you have
   tape drives on several machines, you will have at least one Storage daemon
   per machine.
\item
   \nameref{FiledConfChapter} -- to define the resources for
   each client to be backed up. That is, you will have a separate  Client
   resource file on each machine that runs a File daemon.
\item
   \nameref{ConsoleConfChapter} -- to define the resources for
   the Console program (user interface to the Director).  It defines which
Directors are  available so that you may interact with them.
\end{itemize}


\section{Configuration Path Layout}
\label{sec:ConfigurationPathLayout}
\index[general]{Configuration!Directories}
\index[general]{Configuration!Subdirectories}

When a Bareos component starts, it reads its configuration.
In Bareos $<$ 16.2.2 only configuration files (which optionally can include other files) are supported.
Since Bareos \sinceVersion{}{Subdirectory Configuration Scheme}{16.2.2} also configuration subdirectories are supported.

\subsection{Naming}

In this section, the following naming is used:

\begin{itemize}
    \item \path|$CONFIGDIR|\hide{$} refers to the base configuration directory.
        Bareos Linux packages use \configPathUnix.
    \item A component is one of the following Bareos programs:
    \begin{itemize}
        \item bareos-dir
        \item bareos-sd
        \item bareos-fd
        \item bareos-traymonitor
        \item bconsole
        \item bat (only legacy config file: bat.conf)
        \item Bareos tools, like \nameref{sec:VolumeUtilityCommands} and others.
    \end{itemize}
    \item \path|$COMPONENT|\hide{$} refers to one of the listed components.
%
%     \item Legacy config file (still fully supported, with some
%           limitation when using the configuration API):
%         \begin{itemize}
%             \item \path|$CONFIGDIR/$COMPONENT.conf|
%         \end{itemize}
\end{itemize}

\subsection{What configuration will be used?}
\label{sec:ConfigurationFileOrConfigurationSubDirectories}

When starting a Bareos component, it will look for its configuration.
Bareos components allow the configuration file/directory to be specified as a command line parameter \path|-c $PATH|\hide{$}.

\begin{itemize}
    \item configuration path parameter is not given (default)
        \begin{itemize}
        \item \path|$CONFIGDIR/$COMPONENT.conf| is a file
            \begin{itemize}
            \item the configuration is read from the file \path|$CONFIGDIR/$COMPONENT.conf|
            \end{itemize}
        \item \path|$CONFIGDIR/$COMPONENT.d/| is a directory
            \begin{itemize}
            \item the configuration is read from \path|$CONFIGDIR/$COMPONENT.d/*/*.conf| (subdirectory configuration)
            \end{itemize}
        \end{itemize}
    \item configuration path parameter is given (\path|-c $PATH|)
        \begin{itemize}
        \item \path|$PATH| is a file
            \begin{itemize}
            \item the configuration is read from the file specified in \path|$PATH|
            \end{itemize}
        \item \path|$PATH| is a directory
            \begin{itemize}
            \item the configuration is read from \path|$PATH/$COMPONENT.d/*/*.conf| (subdirectory configuration)
            \end{itemize}
        \end{itemize}
\end{itemize}

As the \path|$CONFIGDIR|\hide{$} differs between platforms or is overwritten by the path parameter,
the documentation will often refer to the configuration without the leading path
(e.g. \path|$COMPONENT.d/*/*.conf|\hide{$} instead of \path|$CONFIGDIR/$COMPONENT.d/*/*.conf|).

\begin{center}
\includegraphics[width=0.8\linewidth]{\idir bareos-read-configuration}
\end{center}


When subdirectory configuration is used,
all files matching \path|$PATH/$COMPONENT.d/*/*.conf| will be read, see \nameref{sec:ConfigurationSubdirectories}.

\subsubsection{Relation between Bareos components and configuration}

\begin{center}
\begin{tabular}{ l || l | l }
Bareos component &
\shortstack[l]{Configuration File \\ (default path on Unix)} &
\shortstack[l]{Subdirectory Configuration Scheme\\ (default path on Unix) \\ since Bareos $>=$ 16.2.2} \\
\hline
\hline

bareos-dir                   & \file{bareos-dir.conf}       & \file{bareos-dir.d} \\
\nameref{DirectorChapter}    & (\configFileDirUnix)         & (\configDirectoryDirUnix) \\
\hline

bareos-sd                    & \file{bareos-sd.conf}        & \file{bareos-sd.d} \\
\nameref{StoredConfChapter}  & (\configFileSdUnix)          & (\configDirectorySdUnix) \\
\hline

bareos-fd                    & \file{bareos-fd.conf}        & \file{bareos-fd.d} \\
\nameref{FiledConfChapter}   & (\configFileFdUnix)          & (\configDirectoryFdUnix) \\
\hline

bconsole                     & \file{bconsole.conf}         & \file{bconsole.d} \\
\nameref{ConsoleConfChapter} & (\configFileBconsoleUnix)    & (\configDirectoryBconsoleUnix) \\
\hline

bareos-traymonitor           & \file{tray-monitor.conf}     & \file{tray-monitor.d} \\
\nameref{sec:MonitorConfig}  & (\configFileTrayMonitorUnix) & (\configDirectoryTrayMonitorUnix) \\
\hline

bat                          & \file{bat.conf}              & (not supported) \\
                             & ({\configFileBatUnix})       &  \\
\hline

\nameref{sec:VolumeUtilityCommands} & \file{bareos-sd.conf}        & \file{bareos-sd.d} \\
(use the bareos-sd configuration)   & (\configFileSdUnix)          & (\configDirectorySdUnix) \\

\end{tabular}
\end{center}



\subsection{Subdirectory Configuration Scheme}
\label{sec:SubdirectoryConfigurationScheme}
\label{sec:ConfigurationSubdirectories}
% ConfigurationIncludeDirectory is referenced from the Bareos code.
\label{ConfigurationIncludeDirectory}

If the subdirectory configuration is used, instead of a single configuration file,
all files matching \path|$COMPONENT.d/*/*.conf|\hide{$} are read as a configuration,
see \nameref{sec:ConfigurationFileOrConfigurationSubDirectories}.

\subsubsection{Reason for the Subdirectory Configuration Scheme}

In Bareos $<$ 16.2.2, Bareos uses one configuration file per component.

Most larger Bareos environments split their configuration into separate
files, making it easier to manage the configuration.

Also some extra packages (bareos-webui, plugins, ...) require a configuration,
which must be included into the \bareosDir or \bareosSd configuration.
The subdirectory approach makes it easier to add or modify the configuration resources of different Bareos packages.

The Bareos \ilink{configure}{sec:bcommandConfigure} command
requires a configuration directory structure, as provided by the subdirectory approach.

From Bareos \sinceVersion{}{Subdirectory Configuration Scheme used as Default}{16.2.4} on,
new installations will use configuration subdirectories by default.


\subsubsection{Resource file conventions}
    \label{sec:ConfigurationResourceFileConventions}

\begin{itemize}
\item Each configuration resource has to use its own configuration file.
\item The path of a resource file is \path|$COMPONENT.d/$RESOURCETYPE/$RESOURCENAME.conf|.
\item The name of the configuration file is identical with the resource name:
    \begin{itemize}
    \item e.g.
        \begin{itemize}
        \item \path|bareos-dir.d/director/bareos-dir.conf|
        \item \path|bareos-dir.d/pool/Full.conf|
        \end{itemize}
    \item Exceptions:
        \begin{itemize}
        \item The resource file \path|bareos-fd.d/client/myself.conf| always has the file name \path|myself.conf|,
                while the name is normally set to the hostname of the system. 
        \end{itemize}
    \end{itemize}
\item Example resource files:
    \begin{itemize}
    \item Additional packages can contain configuration files that are automatically included. However, most additional configuration resources require configuration. When a resource file requires configuration, it has to be included as an example file:
        \begin{itemize}
        \item \path|$CONFIGDIR/$COMPONENT.d/$RESOURCE/$NAME.conf.example|
        \item For example, the \bareosWebui entails one config resource and one config resource example for the \bareosDir:
            \begin{itemize}
            \item \path|$CONFIGDIR/bareos-director.d/profile/webui-admin.conf|
            \item \path|$CONFIGDIR/bareos-director.d/console/admin.conf.example|\hide{$}
            \end{itemize}
        \end{itemize}
    \end{itemize}
\item \hypertarget{sec:deleteConfigurationResourceFiles}Disable/remove configuration resource files:
    \begin{itemize}
    \item Normally you should not remove resources that are already in use (jobs, clients, ...). Instead you should disable them by adding the directive \configline{Enable = no}. Otherwise you take the risk that orphaned entries are kept in the Bareos catalog. However, if a resource has not been used or all references have been cleared from the database, they can also be removed from the configuration.
    \warning{If you want to remove a configuration resource that is part of a Bareos package,
                    replace the resource configuration file by an empty file.
                    This prevents the resource from reappearing in the course of a package update.}
    \end{itemize}
\end{itemize}



\subsubsection{Using Subdirectories Configuration Scheme}

\paragraph{New installation}

\begin{itemize}
    \item The Subdirectories Configuration Scheme is used
            by default from Bareos \sinceVersion{}{Subdirectory Configuration Scheme used as Default}{16.2.4} onwards.
    \item They will be usable immediately after installing a Bareos component.
    \item If additional packages entail example configuration files (\path|$NAME.conf.example|),
        copy them to \path|$NAME.conf|, modify it as required and reload or restart the component.
\end{itemize}

\paragraph{Updates from Bareos $<$ 16.2.4}
    \label{sec:UpdateToConfigurationSubdirectories}

\begin{itemize}
\item When updating to a Bareos version containing the Subdirectories Configuration,
            the existing configuration will not be touched and is still the default configuration.
    \begin{itemize}
    \item \warning{Problems can occur if you have implemented an own wildcard mechanism to load your configuration
            from the same subdirectories as used by the new packages (\path|$CONFIGDIR/$COMPONENT.d/*/*.conf|).
            In this case, newly installed configuration resource files can alter
            your current configuration by adding resources.}
            Best create a copy of your configuration directory before updating Bareos
            and modify your existing configuration file to use that other directory.
    \end{itemize}
\item As long as the old configuration file (\path|$CONFIGDIR/$COMPONENT.conf|) exists, it will be used.
\item The correct way of migrating to the new configuration scheme would be
            to split the configuration file into resources,
            store them in the resource directories and then remove the original configuration file.
    \begin{itemize}
    \item For migrating the \bareosDir configuration, the script \bareosMigrateConfigSh exists.
        Being called, it connects via \command{bconsole} to a running \bareosDir and creates subdirectories with the resource configuration files.
        \begin{commands}{bareos-migrate-config.sh}
# prepare temporary directory
mkdir /tmp/baroes-dir.d
cd /tmp/baroes-dir.d

# download migration script
wget https://raw.githubusercontent.com/bareos/bareos-contrib/master/misc/bareos-migrate-config/bareos-migrate-config.sh

# execute the script
bash bareos-migrate-config.sh

# backup old configuration
mv /etc/bareos/bareos-dir.conf /etc/bareos/bareos-dir.conf.bak
mv /etc/bareos/bareos-dir.d /etc/bareos/bareos-dir.d.bak

# make sure, that all packaged configuration resources exists,
# otherwise they will be added when updating Bareos.
for i in `find  /etc/bareos/bareos-dir.d.bak/ -name *.conf -type f -printf "%P\n"`; do touch "$i"; done

# install newly generated configuration
cp -a /tmp/bareos-dir.d /etc/bareos/
        \end{commands}
        Restart the \bareosDir and verify your configuration.
        Also make sure, that all resource configuration files coming from Bareos packages exists, in doubt as empty files, see \hyperlink{sec:deleteConfigurationResourceFiles}{remove configuration resource files}.

    \item Another way, without splitting the configuration into resource files is:
        \begin{itemize}
        \item \begin{commands}{move configuration to subdirectory}
mkdir $CONFIGDIR/$COMPONENT.d/migrate && mv $CONFIGDIR/$COMPONENT.conf $CONFIGDIR/$COMPONENT.d/migrate
        \end{commands}
        \item Resources defined in both, the new configuration directory scheme
                    and the old configuration file, must be removed from one of the places,
                    best from the old configuration file,
                    after verifying that the settings are identical with the new settings.
        \end{itemize}
    \end{itemize}
\end{itemize}

\section{Configuration File Format}

A configuration file consists of one or more resources (see \nameref{sec:ConfigurationResourceFormat}).

Bareos programs can work with
\begin{itemize}
  \item all resources defined in one configuration file
  \item configuration files that include other configuration files (see \nameref{Includes})
  \item \nameref{sec:ConfigurationSubdirectories}, where each configuration file contains exactly one resource definition
\end{itemize}



\subsection{Character Sets}
\index[general]{Character Sets}
Bareos is designed to handle most character sets of the world,
US ASCII, German, French, Chinese, ...  However, it does this by
encoding everything in UTF-8, and it expects all configuration files
(including those read on Win32 machines) to be in UTF-8 format.
UTF-8 is typically the default on Linux machines, but not on all
Unix machines, nor on Windows, so you must take some care to ensure
that your locale is set properly before starting Bareos.

\index[general]{Windows!Configuration Files!UTF-8}
To ensure that Bareos configuration files can be correctly read including
foreign characters, the {\bf LANG} environment variable
must end in {\bf .UTF-8}. A full example is {\bf en\_US.UTF-8}. The
exact syntax may vary a bit from OS to OS, so that the way you have to define
it will differ from the example.  On most newer Win32 machines you can use \command{notepad}
to edit the conf files, then choose output encoding UTF-8.

Bareos assumes that all filenames are in UTF-8 format on Linux and
Unix machines. On Win32 they are in Unicode (UTF-16) and will hence
be automatically converted to UTF-8 format.

\subsection{Comments}
\label{Comments}
\index[general]{Configuration!Comments}

When reading a configuration, blank lines are ignored and everything
after a hash sign (\#) until the end of the line is taken to be a comment.

\subsection{Semicolons}
A semicolon (;) is a logical end of line and anything after the semicolon is
considered as the next statement. If a statement appears on a line by itself,
a semicolon is not necessary to terminate it, so generally in the examples in
this manual, you will not see many semicolons.

\subsection{Including other Configuration Files}
\label{Includes}
\index[general]{Including other Configuration Files}
\index[general]{Files!Including other Configuration}
\index[general]{Configuration!Including Files}

If you wish to break your configuration file into smaller pieces, you can do
so by including other files using the syntax \configdirective{@filename}
where \file{filename} is the full path and filename of another file.
The \configdirective{@filename}
specification can be given anywhere a primitive token would appear.

\begin{bconfig}{include a configuration file}
@/etc/bareos/extra/clients.conf
\end{bconfig}

Since Bareos \sinceVersion{}{Including configuration files by wildcard}{16.2.1} wildcards in pathes are supported:
\begin{bconfig}{include multiple configuration files}
@/etc/bareos/extra/*.conf
\end{bconfig}

% Before 
% this could be archived by
% If you wish include all files in a specific directory, you can use the following:
% \begin{bconfig}{include configuration files}
% # Include subfiles associated with configuration of clients.
% # They define the bulk of the Clients, Jobs, and FileSets.
% # Remember to "reload" the Director after adding a client file.
% #
% @|"sh -c 'for f in /etc/bareos/clientdefs/*.conf ; do echo @${f} ; done'"
% \end{bconfig}
% \hide{$}

By using \configdirective{@|command} it is also possible to include the output of a script as a configuration:
\begin{bconfig}{use the output of a script as configuration}
@|"/etc/bareos/generate_configuration_to_stdout.sh"
\end{bconfig}
or if a parameter should be used:
\begin{bconfig}{use the output of a script with parameter as a configuration}
@|"sh -c '/etc/bareos/generate_client_configuration_to_stdout.sh clientname=client1.example.com'"
\end{bconfig}
The scripts are called at the start of the daemon. You should use this with care.


\section{Resource}
\label{sec:ConfigurationResourceFormat}
\index[general]{Configuration!Resource}

A resource is defined as the resource type (see \nameref{ResTypes}),
followed by an open brace (\path|{|), a number of \nameref{sec:ConfigurationResourceDirective}s, and ended by a closing brace (\path|}|)

% \hide{
% \begin{bconfig}{Resource}
% $RESOURCETYPE {
%     Name  = $RESOURCENAME
%     $KEY1 = $VALUE1
%     $KEY2 = $VALUE2
% }
% \end{bconfig}
% }

Each resource definition MUST contain a \configdirective{Name} directive.
It can contain a \configdirective{Description} directive.
The \configdirective{Name} directive is used to
uniquely identify the resource.
The \configdirective{Description} directive can be used
during the display of the Resource to provide easier human recognition. For
example:

\begin{bconfig}{Resource example}
Director {
  Name = "bareos-dir"
  Description = "Main Bareos Director"
  Query File = "/usr/lib/bareos/scripts/query.sql"
}
\end{bconfig}

defines the Director resource with the name \parameter{bareos-dir} and a query file \file{/usr/lib/bareos/scripts/query.sql}.

\index[general]{Configuration!Naming Convention}

When naming resources, for some resource types naming conventions should be applied:
\begin{description}
    \item[Client] names should be postfixed with \name{-fd}
    \item[Storage] names should be postfixed with \name{-sd}
    \item[Director] names should be postfixed with \name{-dir}
\end{description}
These conventions helps a lot when reading log messages.


\subsection{Resource Directive}
\label{sec:ConfigurationResourceDirective}

Each directive contained
within the resource (within the curly braces \path|{}|) is composed of a \nameref{sec:ConfigurationResourceDirectiveKeyword} followed by
an equal sign (=) followed by a \nameref{sec:ConfigurationResourceDirectiveValue}. The keywords must be one of
the known Bareos resource record keywords.


\subsection{Resource Directive Keyword}
\label{sec:ConfigurationResourceDirectiveKeyword}

A resource directive keyword is the part before the equal sign (\path|=|) in a \nameref{sec:ConfigurationResourceDirective}.
The following sections will list all available directives by Bareos component resources.

\subsubsection{Upper and Lower Case and Spaces}

Case (upper/lower) and spaces are ignored in the resource directive
keywords.

Within the keyword (i.e. before the equal sign), spaces are not significant.
Thus the keywords: {\bf name}, {\bf Name}, and {\bf N a m e} are all
identical.


\subsection{Resource Directive Value}
\label{sec:ConfigurationResourceDirectiveValue}

A resource directive value is the part after the equal sign (\path|=|) in a \nameref{sec:ConfigurationResourceDirective}.

\subsubsection{Spaces}

Spaces after the equal sign and before the first character of the value are
ignored. Other spaces within a value may be significant (not ignored)
and may require quoting.


\subsubsection{Quotes}
\label{sec:Quotes}
In general, if you want spaces in a name to the
right of the first equal sign (=), you must enclose that name within double
quotes. Otherwise quotes are not generally necessary because once defined,
quoted strings and unquoted strings are all equal.

 Within a quoted string, any character following a
backslash (\textbackslash{}) is taken as itself (handy for inserting
backslashes and double quotes (")).

\warning{If the configure directive is used to define a number, the number is never to be put between surrounding quotes. This is even true if the number is specified together with its unit, like \parameter{365 days}.
}

\subsubsection{Numbers}

Numbers are not to be quoted, see \nameref{sec:Quotes}.
Also do not prepend numbers by zeros (0), as these are not parsed in the expected manner (write 1 instead of 01).

\subsubsection{Data Types}
\index[general]{Configuration!Data Types}
\index[general]{Data Type}
\label{DataTypes}

When parsing the resource directives, Bareos classifies the data according to
the types listed below.

\begin{description}

\item [acl]
    \index[general]{Data Type!acl}
    \label{DataTypeAcl}
This directive defines what is permitted to be accessed.
It does this by using a list of regular expressions, separated by commas (\argument{,})
or using multiple directives.
If \argument{!} is prepended, the expression is negated.
The special keyword \parameter{*all*} allows unrestricted access.

Depending on the type of the ACL, the regular expressions can be either resource names, paths or console commands.

Since Bareos \sinceVersion{dir}{ACL: strict regular expression handling}{16.2.4} regular expression are handled more strictly. Before also substring matches has been accepted.

\label{sec:CommandAclExample}
For clarification, we demonstrate the usage of ACLs by some examples for \linkResourceDirective{Dir}{Console}{Command ACL}:
\begin{bconfig}{Allow only the help command}
Command ACL = help
\end{bconfig}

\begin{bconfig}{Allow the help and the list command}
Command ACL = help, list
\end{bconfig}

\begin{bconfig}{Allow the help and the (not existing) iDoNotExist command}
Command ACL = help, iDoNotExist
\end{bconfig}

\begin{bconfig}{Allow all commands (special keyword)}
Command ACL = *all*
\end{bconfig}

\begin{bconfig}{Allow all commands except sqlquery and commands starting with u}
Command ACL = !sqlquery, !u.*, *all*
\end{bconfig}

Same:
\begin{bconfig}{Some as above. Specifying it in multiple lines doesn't change the meaning}
Command ACL = !sqlquery, !u.*
Command ACL = *all*
\end{bconfig}

\begin{bconfig}{Additional deny the setip and setdebug commands}
Command ACL = !sqlquery
Command ACL = !u.*
Comamnd ACL = !set(ip|debug)
Comamnd ACL = *all*
\end{bconfig}

\warning{
ACL checking stops at the first match. So the following definition allows all commands, which might not be what you expected:
}
\begin{bconfig}{Wrong: Allows all commands}
# WARNING: this configuration ignores !sqlquery, as *all* is matched before.
Command ACL = *all*, !sqlquery
\end{bconfig}

\item [auth-type]
    \index[general]{Data Type!auth-type}
    \label{DataTypeAuthType}
Specifies the authentication type that must be supplied when connecting to
a backup protocol that uses a specific authentication type.
Currently only used for \nameref{NDMPResource}.

The following values are allowed:
\begin{description}
\item[None] - Use no password
\item[Clear] - Use clear text password
\item[MD5] - Use MD5 hashing
\end{description}


\item [integer]
    \index[general]{Data Type!integer}
    \label{DataTypeInteger}
   A 32 bit integer value. It may be positive or negative.

   Don't use quotes around the number, see \nameref{sec:Quotes}.


\item [long integer]
    \index[general]{Data Type!long integer}
    \label{DataTypeLongInteger}
   A 64 bit integer value. Typically these  are values such as bytes that can
exceed 4 billion and thus  require a 64 bit value.

   Don't use quotes around the number, see \nameref{sec:Quotes}.

\item [job protocol]
    \index[general]{Data Type!job protocol}
    \label{DataTypeJobProtocol}

The protocol to run a the job.
Following protocols are available:
\begin{description}
    \item[Native] Native Bareos job protocol.
    \item[NDMP] Deprecated. Alias for \NdmpBareos.
    \item[NDMP\_BAREOS] Since Bareos \sinceVersion{dir}{NDMP BAREOS}{17.2.3}. See \nameref{sec:NdmpBareos}.
    \item[NDMP\_NATIVE] Since Bareos \sinceVersion{dir}{NDMP NATIVE}{17.2.3}. See \nameref{sec:NdmpNative}.
\end{description}



\item [name]
    \index[general]{Data Type!name}
    \label{DataTypeName}
   A keyword or name consisting of alphanumeric characters, including the
hyphen, underscore, and dollar  characters. The first character of a {\bf
name} must be  a letter.  A name has a maximum length currently set to 127
bytes.

Please note that Bareos resource names as well as certain other
names (e.g. Volume names) must contain only letters (including ISO accented
letters), numbers, and a few special characters (space, underscore, ...).
All other characters and punctuation are invalid.


\item [password]
    \index[general]{Data Type!password}
    \label{DataTypePassword}
   This is a Bareos password and it is stored internally in MD5 hashed format.


% \item [path]
%     \index[general]{Data Type!path}
%     \label{DataTypePath}
%     File name, including path.

\item [path]
    \index[general]{Data Type!path}
    \label{DataTypeDirectory}
   A path is either a quoted or  non-quoted string. A path will be
passed to your  standard shell for expansion when it is scanned. Thus
constructs such as {\bf \$HOME} are interpreted to be  their correct values.
The path can either reference to a file or a directory.


\item [positive integer]
    \index[general]{Data Type!positive integer}
    \label{DataTypePositiveInteger}
   A 32 bit positive integer value.

   Don't use quotes around the number, see \nameref{sec:Quotes}.


\item [speed]
    \index[general]{Data Type!speed}
    \label{DataTypeSpeed}
    The speed parameter can be specified as k/s, kb/s, m/s or mb/s.

    Don't use quotes around the parameter, see \nameref{sec:Quotes}.


\item [string]
    \index[general]{Data Type!string}
    \label{DataTypeString}
   A quoted string containing virtually any  character including spaces, or a
non-quoted string. A  string may be of any length. Strings are typically
values  that correspond to filenames, directories, or system  command names. A
backslash (\textbackslash{}) turns the next character into  itself, so to
include a double quote in a string, you precede the  double quote with a
backslash. Likewise to include a backslash.


\item [string-list]
    \index[general]{Data Type!string list}
    \label{DataTypeStringList}
    Multiple strings, specified in multiple directives, or in a single directive, separated by commas (\textbf{,}).

\item [strname]
    \index[general]{Data Type!strname}
    \label{DataTypeStrname}
is similar to a \dtName, except that the name may be quoted and
can thus contain  additional characters including spaces.


\item [net-address]
    \index[general]{Data Type!net-address}
    \label{DataTypeNetAddress}
is either a domain name or an IP address specified as a
dotted quadruple in string or quoted string format.
This directive only permits a single address to be specified.
The \dtNetPort\ must be specificly separated.
If multiple net-addresses are needed, please assess if it is also possible to specify \dtNetAddresses\ (plural).


\item [net-addresses]
    \index[general]{Data Type!net-addresses}
    \label{DataTypeNetAddresses}
Specify a set of net-addresses and net-ports.
Probably the simplest way to explain
this is to show an example:

\begin{bconfig}{net-addresses}
Addresses  = {
    ip = { addr = 1.2.3.4; port = 1205;}
    ipv4 = {
        addr = 1.2.3.4; port = http;}
    ipv6 = {
        addr = 1.2.3.4;
        port = 1205;
    }
    ip = {
        addr = 1.2.3.4
        port = 1205
    }
    ip = { addr = 1.2.3.4 }
    ip = { addr = 201:220:222::2 }
    ip = {
        addr = server.example.com
    }
}
\end{bconfig}

where ip, ip4, ip6, addr, and port are all keywords. Note, that  the address
can be specified as either a dotted quadruple, or  in IPv6 colon notation, or as
a symbolic name (only in the ip specification).  Also, the port can be specified
as a number or as the mnemonic value from  the \file{/etc/services} file.  If a port
is not specified, the default one will be used. If an ip  section is specified,
the resolution can be made either by IPv4 or  IPv6. If ip4 is specified, then
only IPv4 resolutions will be permitted,  and likewise with ip6.


\item [net-port]
    \index[general]{Data Type!net-port}
    \label{DataTypeNetPort}
    Specify a network port (a positive  integer).

    Don't use quotes around the parameter, see \nameref{sec:Quotes}.


\item [resource]
    \index[general]{Data Type!resource}
    \label{DataTypeRes}
A resource defines a relation to the name of another resource.


\item [size]
    \index[general]{Data Type!size}
    \label{DataTypeSize}
    \label{Size1}
A size specified as bytes. Typically, this is  a floating point scientific
input format followed by an optional modifier. The  floating point input is
stored as a 64 bit integer value.  If a modifier is present, it must
immediately follow the  value with no intervening spaces. The following
modifiers are permitted:

\begin{description}
\item [k]
   1,024 (kilobytes)

\item [kb]
   1,000 (kilobytes)

\item [m]
   1,048,576 (megabytes)

\item [mb]
   1,000,000 (megabytes)

\item [g]
   1,073,741,824 (gigabytes)

\item [gb]
   1,000,000,000 (gigabytes)
\end{description}

    Don't use quotes around the parameter, see \nameref{sec:Quotes}.


\item [time]
    \index[general]{Data Type!time}
    \label{DataTypeTime}
    \label{Time}
A time or duration specified in seconds.  The time is stored internally as
a 64 bit integer value, but it is specified in two parts: a number part and
a modifier part.  The number can be an integer or a floating point number.
If it is entered in floating point notation, it will be rounded to the
nearest integer.  The modifier is mandatory and follows the number part,
either with or without intervening spaces.  The following modifiers are
permitted:

\begin{description}

\item [seconds]
   \index[dir]{seconds}

\item [minutes]
   \index[dir]{minutes} (60 seconds)

\item [hours]
   \index[dir]{hours} (3600 seconds)

\item [days]
   \index[dir]{days} (3600*24 seconds)

\item [weeks]
   \index[dir]{weeks} (3600*24*7 seconds)

\item [months]
   \index[dir]{months} (3600*24*30 seconds)

\item [quarters]
   \index[dir]{quarters} (3600*24*91 seconds)

\item [years]
   \index[dir]{years} (3600*24*365 seconds)

\end{description}

Any abbreviation of these modifiers is also permitted (i.e.  {\bf seconds}
may be specified as {\bf sec} or {\bf s}).  A specification of {\bf m} will
be taken as months.

The specification of a time may have as many number/modifier parts as you
wish.  For example:

\footnotesize
\begin{verbatim}
1 week 2 days 3 hours 10 mins
1 month 2 days 30 sec
\end{verbatim}
\normalsize

are valid date specifications.

    Don't use quotes around the parameter, see \nameref{sec:Quotes}.


\item [audit-command-list]
    \index[general]{Data Type!audit command list}
    \label{DataTypeAuditCommandList}
    Specifies the commands that should be logged on execution (audited).
    E.g.
\begin{bconfig}{}
Audit Events = label
Audit Events = restore
\end{bconfig}
    Based on the type \dtStringList.



\item [\yesno]
    \index[general]{Data Type!\yesno}
    \index[general]{Data Type!boolean}
    \label{DataTypeYesNo}
   Either a \parameter{yes} or a \parameter{no} (or \parameter{true} or \parameter{false}).


\end{description}




\subsubsection{Variable Expansion}
    \label{VarsChapter}

Depending on the directive, Bareos will expand to the following variables:

\paragraph{Variable Expansion on Volume Labels}
\label{sec:VariableExpansionVolumeLabels}

When labeling a new volume (see \linkResourceDirective{Dir}{Pool}{Label Format}), following Bareos internal variables can be used:

\begin{tabular}{p{2cm}p{7cm}}
\textbf{Internal Variable} & \textbf{Description} \\
\variable{$Year} & Year \\
\variable{$Month} & Month: 1-12 \\
\variable{$Day} & Day: 1-31 \\
\variable{$Hour} & Hour: 0-24 \\
\variable{$Minute} & Minute: 0-59 \\
\variable{$Second} & Second: 0-59 \\
\variable{$WeekDay} & Day of the week: 0-6, using 0 for Sunday\\
\variable{$Job} & Name of the Job \\
\variable{$Dir} & Name of the Director \\
\variable{$Level} & Job Level \\
\variable{$Type} & Job Type \\
\variable{$JobId} & JobId \\
\variable{$JobName} & unique name of a job\\
\variable{$Storage} & Name of the Storage Daemon\\
\variable{$Client} &  Name of the Clients \\
\variable{$NumVols} & Numbers of volumes in the pool\\
\variable{$Pool} &  Name of the Pool  \\
\variable{$Catalog} &  Name of the Catalog\\
\variable{$MediaType} &  Type of the media
\end{tabular}
\hide{$}

Additional, normal environment variables can be used, e.g.
\variable{$HOME} oder \variable{$HOSTNAME}.

With the exception of Job specific variables, you can trigger the variable expansion
by using the \ilink{var command}{var}.



\paragraph{Variable Expansion in Autochanger Commands}

At the configuration of autochanger commands the following variables can be used:


\begin{tabular}{p{2cm}p{7cm}}
\textbf{Variable} & \textbf{Description} \\
\parameter{\%a} & Archive Device Name\\
\parameter{\%c} & Changer Device Name\\
\parameter{\%d} & Changer Drive Index\\
\parameter{\%f} & Client's Name\\
\parameter{\%j} & Job Name\\
\parameter{\%o} & Command\\
\parameter{\%s} & Slot Base 0\\
\parameter{\%S} & Slot Base 1\\
\parameter{\%v} & Volume Name
\end{tabular}



\paragraph{Variable Expansion in Mount Commands}

At the configuration of mount commands the following variables can be used:



\begin{tabular}{p{2cm}p{7cm}}
\textbf{Variable} & \textbf{Description} \\
\parameter{\%a} & Archive Device Name\\
\parameter{\%e} & Erase\\
\parameter{\%n} & Part Number\\
\parameter{\%m} & Mount Point\\
\parameter{\%v} & Last Part Name
\end{tabular}



\paragraph{Variable Expansion on RunScripts}

Variable Expansion on RunScripts is described at \linkResourceDirective{Dir}{Job}{Run Script}.



\paragraph{Variable Expansion in Mail and Operator Commands}

At the configuration of mail and operator commands the following variables can be used:

\begin{tabular}{p{2cm}p{7cm}}
\textbf{Variable} & \textbf{Description} \\
\parameter{\%c} & Client's Name\\
\parameter{\%d} & Director's Name\\
\parameter{\%e} & Job Exit Code\\
\parameter{\%i} & JobId\\
\parameter{\%j} & Unique Job Id\\
\parameter{\%l} & Job Level\\
\parameter{\%n} & Unadorned Job Name\\
\parameter{\%s} & Since Time\\
\parameter{\%t} & Job Type (Backup, ...)\\
\parameter{\%r} & Recipients\\
\parameter{\%v} & Read Volume Name\\
\parameter{\%V} & Write Volume Name\\
\parameter{\%b} & Job Bytes\\
\parameter{\%B} & Job Bytes in human readable format \\
\parameter{\%F} & Job Files
\end{tabular}


\subsection{Resource Types}
\index[general]{Types!Resource}
\index[general]{Resource Types}
\label{ResTypes}

% TODO: is ths section really useful or should it be removed?

The following table lists all current Bareos resource types. It shows what
resources must be defined for each service (daemon). The default configuration
files will already contain at least one example of each permitted resource.

\addcontentsline{lot}{table}{Resource Types}
\begin{longtable}{|l||c|c|c|c|}
 \hline
\multicolumn{1}{|c|| }{\bf Resource } &
\multicolumn{1}{c| }{ \ilink{Director}{DirectorConfChapter} } &
\multicolumn{1}{c| }{ \ilink{Client}{FiledConfChapter} } &
\multicolumn{1}{c| }{ \ilink{Storage}{StoredConfChapter} } &
\multicolumn{1}{c| }{ \ilink{Console}{ConsoleConfChapter}  } \\
 \hline
 \hline
{Autochanger} &                                 &                                 & \cmlink{StorageResourceAutochanger} &  \\
\hline
{Catalog }  & \cmlink{DirectorResourceCatalog}  &                                 &    &    \\
 \hline
{Client  }  & \cmlink{DirectorResourceClient}   & \cmlink{ClientResourceClient}   &    &    \\
 \hline
{Console }  & \cmlink{DirectorResourceConsole}  &                                 &                                  & \cmlink{ConsoleResourceConsole} \\
 \hline
{Device  }  &                                   &                                 & \cmlink{StorageResourceDevice}   &    \\
 \hline
{Director } & \cmlink{DirectorResourceDirector} & \cmlink{ClientResourceDirector} & \cmlink{StorageResourceDirector} & \cmlink{ConsoleResourceDirector} \\
 \hline
{FileSet }  & \cmlink{DirectorResourceFileSet}  &                                 &    &    \\
 \hline
{Job}       & \cmlink{DirectorResourceJob}      &                                 &    &    \\
 \hline
{JobDefs }  & \cmlink{DirectorResourceJobDefs}  &                                 &    &    \\
 \hline
{Message }  & \cmlink{ResourceMessages}         & \cmlink{ResourceMessages}       & \cmlink{ResourceMessages} &    \\
 \hline
{NDMP }     &                                   &                                 & \cmlink{StorageResourceNDMP} &    \\
 \hline
{Pool  }    & \cmlink{DirectorResourcePool}     &                                 &    &    \\
 \hline
{Profile}   & \cmlink{DirectorResourceProfile}  &                                 &    &    \\
 \hline
{Schedule } & \cmlink{DirectorResourceSchedule} &                                 &    &    \\
 \hline
{Storage }  & \cmlink{DirectorResourceStorage}  &                                 & \cmlink{StorageResourceStorage} & \\
\hline
\end{longtable}

\section{Names, Passwords and Authorization}
\label{Names}
\index[general]{Authorization!Names and Passwords}
\index[general]{Passwords}

In order for one daemon to contact another daemon, it must authorize itself
with a password. In most cases, the password corresponds to a particular name,
so both the name and the password must match to be authorized. Passwords are
plain text, any text.  They are not generated by any special process; just
use random text.

The default configuration files are automatically defined for correct
authorization with random passwords. If you add to or modify these files, you
will need to take care to keep them consistent.

\label{sec:resource-relation}
\begin{figure}[htbp]
\begin{center}
\includegraphics[width=0.8\textwidth]{\idir Conf-Diagram}
\caption{Relation between resource names and passwords}
%references to this label do not work reliably
%\label{fig:password}
\end{center}
\end{figure}

%The diagram \ref{fig:password} illustrates what names/passwords in which resources
%must match up.

In the left column, you can see the Director, Storage, and Client resources and their corresponding names and passwords -- these are all in \file{bareos-dir.conf}. In
the right column the corresponding values in the
Console, Storage daemon (SD), and File daemon (FD) configuration files are shown.

Please note that the address \host{fw-sd}, that appears in the Storage
resource of the Director,
is passed to the File daemon in symbolic form. The File daemon then resolves it
to an IP address. For this reason you must use either an IP address or a
resolvable fully qualified name. A name such as \host{localhost}, not being a fully
qualified name, will resolve in the File daemon to the \host{localhost} of the File
daemon, which is most likely not what is desired. The password used for the
File daemon to authorize with the Storage daemon is a temporary password
unique to each Job created by the daemons and is not specified in any .conf
file.
