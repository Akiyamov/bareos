%%
%%

\chapter{Catalog Maintenance}
\label{CatMaintenanceChapter}
\index[general]{Maintenance!Catalog}
\index[general]{Catalog Maintenance}

\section{Catalog Database}

Bareos stores its catalog in a database.
Different database backends are offered:
\begin{itemize}
    \item PostgreSQL (recommended)
    \item MariaDB/MySQL (see \nameref{sec:MysqlSupport})
    \item Sqlite (only for testing)
\end{itemize}

What database will be used, can be configured in the \bareosDir configuration, see the \nameref{DirectorResourceCatalog}.

The database often runs on the same server as the \bareosDir.
However, it is also possible to run it on a different system.
This might require some more manual configuration.



\subsection{dbconfig-common (Debian)}
\index[general]{Platform!Debian!dbconfig-common}
\index[general]{Platform!Ubuntu!dbconfig-common}
\label{sec:dbconfig}

Since Bareos \sinceVersion{dir}{dbconfig-common (Debian)}{14.2.0} the Debian (and Ubuntu) based packages support the \package{dbconfig-common} mechanism to create and update the Bareos database, according to the user choices.

The first choice is, if \package{dbconfig-common} should be used at all.
If you decide against it, the database must be configured manually, see \nameref{CatMaintenanceManualConfiguration}.

If you decided to use \package{dbconfig-common}, the next question will only be asked, if more than one
Bareos database backend (\package{bareos-database-*}) is installed.
If this is the case, select the database backend you want to use.


\begin{center}
\includegraphics[width=0.45\textwidth]{\idir dbconfig-1-enable}
\includegraphics[width=0.45\textwidth]{\idir dbconfig-2-select-database-type}
\end{center}

Depending on the selected database backend,
more questions about how to access the database will be asked.
Often, the default values are suitable.

The \package{dbconfig-common} configuration (and credentials) is done by the \package{bareos-database-common} package.
Settings are stored in the file \file{/etc/dbconfig-common/bareos-database-common.conf}.

The Bareos database backend will get automatically configured in \file{/etc/bareos/bareos-dir.conf}.
A later reconfiguration might require manual adapt changes.

\warning{When using the PostgreSQL backend and updating to Bareos $<$ 14.2.3, it is necessary to manually grant database permissions (\command{grant_bareos_privileges}), normally by}
\begin{commands}{}
<command> </command><parameter>su - postgres -c /usr/lib/bareos/scripts/grant_bareos_privileges</parameter>
\end{commands}
For details see chapter \nameref{CatMaintenanceManualConfiguration}.



\subsection{Manual Configuration}
    \label{CatMaintenanceManualConfiguration}

Bareos comes with a number of scripts to prepare and update the databases. All these scripts are located in the Bareos script directory, normally at \scriptPathUnix.

\begin{center}
\begin{tabular}{| l | l | l |}
\hline
\textbf{Script}                & \textbf{Stage} & \textbf{Description} \\
\hline
\hline
\file{create_bareos_database}  & installation & create Bareos database \\
\file{make_bareos_tables}      & installation & create Bareos tables \\
\file{grant_bareos_privileges} & installation & grant database access privileges \\
\hline
\file{update_bareos_tables} [-f]   & update       & update the database schema \\
\hline
\file{drop_bareos_tables}      & deinstallation & remove Bareos database tables \\
\file{drop_bareos_database}    & deinstallation & remove Bareos database \\
\hline
\file{make_catalog_backup.pl}  & backup         & backup the Bareos database, default on Linux \\
\file{make_catalog_backup}     & backup         & backup the Bareos database for systems without Perl \\
\file{delete_catalog_backup}   & backup helper  & remove the temporary Bareos database backup  file \\
\hline
\end{tabular}
\end{center}

The database preparation scripts have following configuration options:
% \begin{center}
% \begin{tabular}{| l | l | l |}
% \hline
% \textbf{Parameter}      & \textbf{Values} & \textbf{Default} \\
% \hline
% \hline
% \parameter{db_type}     & parameter, dbdriver from \configFileDirUnix, installed database backends & \\
% \parameter{db_name}     & environment variable \variable{db_name}, dbdriver from \configFileDirUnix & bareos \\
% \parameter{db_user}     & environment variable \variable{db_user} & bareos \\
% \parameter{db_password} & environment variable \variable{db_password} & \textit{none} \\
% \hline
% \end{tabular}
% \end{center}

\begin{description}
    \item[db\_type] \hfill
        \begin{itemize}
            \item command line parameter \$1
            \item \linkResourceDirective{Dir}{Catalog}{DB Driver} from the configuration
            \item installed database backends
            \item fallback: postgresql
        \end{itemize}
    \item[db\_name] \hfill
        \begin{itemize}
            \item environment variable \variable{db_name}
            \item \linkResourceDirective{Dir}{Catalog}{DB Name} from the configuration
            \item default: bareos
        \end{itemize}
    \item[db\_user] \hfill
        \begin{itemize}
            \item environment variable \variable{db_user}
            \item \linkResourceDirective{Dir}{Catalog}{DB User} from the configuration
            \item default: bareos
        \end{itemize}
    \item[db\_password] \hfill
        \begin{itemize}
            \item environment variable \variable{db_password}
            \item \linkResourceDirective{Dir}{Catalog}{DB Password} from the configuration
            \item default: \textit{none}
        \end{itemize}
\end{description}

Reading the settings from the configuration require read permission for the current user.
The normal PostgreSQL administrator user (\user{postgres}) don't have these permissions.
So if you plan to use non-default database settings, you might add the user \user{postgres} to the group \group{bareos}.

The database preparation scripts need to have password-less administrator access to the database.
Depending on the distribution you're using, this require additional configuration.
See the following section about howto achieve this for the different database systems.


To view and test the currently configured settings, use following commands:
\begin{commands}{Show current database configuration}
<command> </command><parameter>/usr/sbin/bareos-dbcheck -B</parameter>
catalog=MyCatalog
db_name=bareos
db_driver=mysql
db_user=bareos
db_password=YourPassword
db_address=
db_port=0
db_socket=
db_type=MySQL
working_dir=/var/lib/bareos
\end{commands}

\begin{commands}{Test the database connection. Example: wrong password}
<command> </command><parameter>/usr/sbin/bareos-dir -t -f -d 500</parameter>
[...]
bareos-dir: mysql.c:204-0 Error 1045 (28000): Access denied for user 'bareos'@'localhost' (using password: YES)
bareos-dir: dird.c:1114-0 Could not open Catalog "MyCatalog", database "bareos".
bareos-dir: dird.c:1119-0 mysql.c:200 Unable to connect to MySQL server.
Database=bareos User=bareos
MySQL connect failed either server not running or your authorization is incorrect.
bareos-dir: mysql.c:239-0 closedb ref=0 connected=0 db=0
25-Apr 16:25 bareos-dir ERROR TERMINATION
Please correct the configuration in /etc/bareos/bareos-dir.d/*/*.conf
\end{commands}




\subsubsection{PostgreSQL}

On most distributions, PostgreSQL uses ident to allow access to the database system.
The database administrator account is the Unix user \user{postgres}.
Normally, this user can access the database without password, as the ident mechanism is used to identify the user.

If this works on your system can be verified by
\begin{commands}{Access the local PostgreSQL database}
su - postgres
psql
\end{commands}

If your database is configured to to require a password, this must be definied in the file
\elink{\file{~/.pgpass}}{http://www.postgresql.org/docs/8.2/static/libpq-pgpass.html}
in the following syntax: \parameter{HOST:PORT:DATABASE:USER:PASSWORD}, e.g.
\begin{config}{PostgreSQL access credentials}
localhost:*:bareos:bareos:secret
\end{config}
The permission of this file must be 0600 (\command{chmod 0600 ~/.pgpass}).

Again, verify that you have specified the correct settings by calling the \command{psql} command.
If this connects you to the database, your credentials are good.
Exit the PostgreSQL client and run the Bareos database preparation scripts:
\begin{commands}{Setup Bareos catalog database}
su - postgres
/usr/lib/bareos/scripts/create_bareos_database
/usr/lib/bareos/scripts/make_bareos_tables
/usr/lib/bareos/scripts/grant_bareos_privileges
\end{commands}

% \warning{The user \user{postgres} normally can't read the Bareos director config file \configFileDirUnix. If you're using non-standard settings (like different database name, ...), make sure, it got read access to it.}

The encoding of the bareos database must be \parameter{SQL_ASCII}.
The command \command{create_bareos_database} automatically creates the database with this encoding.
This can be verified by the command \command{psql -l}, which  shows information about existing databases:
\begin{commands}{List existing databases}
<command>psql</command><parameter> -l</parameter>
        List of databases
   Name    |  Owner   | Encoding
-----------+----------+-----------
 bareos    | postgres | SQL_ASCII
 postgres  | postgres | UTF8
 template0 | postgres | UTF8
 template1 | postgres | UTF8
(4 rows)
\end{commands}

The owner of the database may vary. The Bareos database maintance scripts don't change the default owner of the Bareos database, so it stays at the PostgreSQL administration user. The \command{grant_bareos_privileges} script grant the required permissions to the Bareos database user. In contrast, when installing (not updating) using \ilink{dbconfig}{sec:dbconfig}, the database owner will be identical with the Bareos database user.

% \warning{When updating from Bareos $<$ 14.2 to Bareos $<$ 14.2.3, it is necessary to manually grant database permissions, using the \command{grant_bareos_privileges} command.}

% When updating from Bareos $<$ 14.2 to a version using dbconfig (Debian based distribution),
% you should change the database owner to the Bareos database user:
% \begin{commands}{Change the database owner to the Bareos database user (Debian)}
% root@linux:~# <input>su - postgres</input>
% postgres@linux:~# <input>psql</input>
% psql (9.1.14)
% Type "help" for help.
% 
% postgres=> <input>\l</input>
%                              List of databases
%    Name    |  Owner   | Encoding  | Collate | Ctype |   Access privileges   
% -----------+----------+-----------+---------+-------+-----------------------
%  bareos    | postgres | SQL_ASCII | C       | C     | 
%  postgres  | postgres | SQL_ASCII | C       | C     | 
%  template0 | postgres | SQL_ASCII | C       | C     | =c/postgres          +
%            |          |           |         |       | postgres=CTc/postgres
%  template1 | postgres | SQL_ASCII | C       | C     | =c/postgres          +
%            |          |           |         |       | postgres=CTc/postgres
% (4 rows)
% 
% postgres=> <input>ALTER DATABASE bareos OWNER TO bareos;</input>
% ALTER DATABASE
% postgres=> <input>\l</input>
%                              List of databases
%    Name    |  Owner   | Encoding  | Collate | Ctype |   Access privileges   
% -----------+----------+-----------+---------+-------+-----------------------
%  bareos    | bareos   | SQL_ASCII | C       | C     | 
%  postgres  | postgres | SQL_ASCII | C       | C     | 
%  template0 | postgres | SQL_ASCII | C       | C     | =c/postgres          +
%            |          |           |         |       | postgres=CTc/postgres
%  template1 | postgres | SQL_ASCII | C       | C     | =c/postgres          +
%            |          |           |         |       | postgres=CTc/postgres
% (4 rows)
% \end{commands}

By default, using PostgreSQL ident, a Unix user can access a database of the same name.
Therefore the user \user{bareos} can access the database \path|bareos|.



\begin{commands}{Verify Bareos database on PostgreSQL as Unix user bareos (bareos-13.2.3)}
root@linux:~# <input>su - bareos -s /bin/sh</input>
bareos@linux:~# <input>psql</input>
Welcome to psql 8.3.23, the PostgreSQL interactive terminal.

Type:  \copyright for distribution terms
       \h for help with SQL commands
       \? for help with psql commands
       \g or terminate with semicolon to execute query
       \q to quit

bareos=> <input>\dt</input>
                 List of relations
 Schema |          Name          | Type  |  Owner
--------+------------------------+-------+----------
 public | basefiles              | table | postgres
 public | cdimages               | table | postgres
 public | client                 | table | postgres
 public | counters               | table | postgres
 public | device                 | table | postgres
 public | devicestats            | table | postgres
 public | file                   | table | postgres
 public | filename               | table | postgres
 public | fileset                | table | postgres
 public | job                    | table | postgres
 public | jobhisto               | table | postgres
 public | jobmedia               | table | postgres
 public | jobstats               | table | postgres
 public | location               | table | postgres
 public | locationlog            | table | postgres
 public | log                    | table | postgres
 public | media                  | table | postgres
 public | mediatype              | table | postgres
 public | ndmpjobenvironment     | table | postgres
 public | ndmplevelmap           | table | postgres
 public | path                   | table | postgres
 public | pathhierarchy          | table | postgres
 public | pathvisibility         | table | postgres
 public | pool                   | table | postgres
 public | quota                  | table | postgres
 public | restoreobject          | table | postgres
 public | status                 | table | postgres
 public | storage                | table | postgres
 public | unsavedfiles           | table | postgres
 public | version                | table | postgres
(30 rows)

bareos=> <input>select * from Version;</input>
 versionid
-----------
      2002
(1 row)

bareos=> <input>\du</input>
                                 List of roles
   Role name   | Superuser | Create role | Create DB | Connections | Member of
---------------+-----------+-------------+-----------+-------------+-----------
 bareos        | no        | no          | no        | no limit    | {}
 postgres      | yes       | yes         | yes       | no limit    | {}
(2 rows)

bareos=> <input>\dp</input>
                 Access privileges for database "bareos"
 Schema |               Name                |   Type   |  Access privileges
--------+-----------------------------------+----------+--------------------------------------
 public | basefiles                         | table    | {root=arwdxt/root,bareos=arwdxt/root}
 public | basefiles_baseid_seq              | sequence | {root=rwU/root,bareos=rw/root}
...

bareos=>
\end{commands}


\subsubsection{MySQL}
    \label{catalog-maintenance-mysql}

MySQL user authentication is username, password and host-based.
The database administrator is the user \user{root}.

On some distributions access to the MySQL database is allowed password-less as database user \user{root},
on other distributions, a password is required.
On productive systems you normally want to have password secured access.

The bareos database preparation scripts require password-less access to the database.
To guarantee this, create a MySQL credentials file \elink{\file{~/.my.cnf}}{http://dev.mysql.com/doc/refman/4.1/en/password-security.html} with the credentials of the database administrator:
\begin{config}{MySQL credentials file .my.cnf}
[client]
host=localhost
user=root
password=<input>YourPasswordForAccessingMysqlAsRoot</input>
\end{config}
Alternatively you can specifiy your database password by adding it to the file \file{/etc/my.cnf}.

Verify that you have specified the correct settings by calling the \command{mysql} command.
If this connects you to the database, your credentials are good.
Exit the MySQL client.

For the Bareos database connection, you should specify a database password. 
Otherwise the Bareos database user gets the permission to connect without password.
This is not recommended.
Choose a database password and add it into the Bareos Director configuration file \configFileDirUnix:
\begin{bconfig}{Bareos catalog configuration}
...
#
# Generic catalog service
#
Catalog {
  Name = MyCatalog
  dbdriver = "mysql"
  dbname = "bareos"
  dbuser = "bareos"
  dbpassword = "YourSecretPassword"
}
...
\end{bconfig}

After this, run the Bareos database preparation scripts.
For Bareos $<=$ 13.2.2, the database password must be specified as environment variable \variable{db_password}.
From \sinceVersion{dir}{MySQL password from configuration file}{13.2.3}
the database password is read from the configuration, if no environment variable is given.

\begin{commands}{Setup Bareos catalog database}
export db_password=<input>YourSecretPassword</input>
/usr/lib/bareos/scripts/create_bareos_database
/usr/lib/bareos/scripts/make_bareos_tables
/usr/lib/bareos/scripts/grant_bareos_privileges
\end{commands}

After this, you can use the \command{mysql} command to verify that your database setup is okay
and works with your the Bareos database user.
The result should look similar as this (here Bareos 13.2 is used on SLES11):

\begin{commands}{Verify Bareos database on MySQL}
root@linux:~# <input>mysql --user=bareos --password=YourSecretPassword bareos</input>
Welcome to the MySQL monitor.  Commands end with ; or \g.
Your MySQL connection id is 162
Server version: 5.5.32 SUSE MySQL package

Copyright (c) 2000, 2013, Oracle and/or its affiliates. All rights reserved.

Oracle is a registered trademark of Oracle Corporation and/or its
affiliates. Other names may be trademarks of their respective
owners.

Type 'help;' or '\h' for help. Type '\c' to clear the current input statement.

mysql> <input>show tables;</input>
+--------------------+
| Tables_in_bareos   |
+--------------------+
| BaseFiles          |
| CDImages           |
| Client             |
| Counters           |
| Device             |
| DeviceStats        |
| File               |
| FileSet            |
| Filename           |
| Job                |
| JobHisto           |
| JobMedia           |
| JobStats           |
| Location           |
| LocationLog        |
| Log                |
| Media              |
| MediaType          |
| NDMPJobEnvironment |
| NDMPLevelMap       |
| Path               |
| PathHierarchy      |
| PathVisibility     |
| Pool               |
| Quota              |
| RestoreObject      |
| Status             |
| Storage            |
| UnsavedFiles       |
| Version            |
+--------------------+
30 rows in set (0.00 sec)

mysql> <input>describe Job;</input>
+-----------------+---------------------+------+-----+---------+----------------+
| Field           | Type                | Null | Key | Default | Extra          |
+-----------------+---------------------+------+-----+---------+----------------+
| JobId           | int(10) unsigned    | NO   | PRI | NULL    | auto_increment |
| Job             | tinyblob            | NO   |     | NULL    |                |
| Name            | tinyblob            | NO   | MUL | NULL    |                |
| Type            | binary(1)           | NO   |     | NULL    |                |
| Level           | binary(1)           | NO   |     | NULL    |                |
| ClientId        | int(11)             | YES  |     | 0       |                |
| JobStatus       | binary(1)           | NO   |     | NULL    |                |
| SchedTime       | datetime            | YES  |     | NULL    |                |
| StartTime       | datetime            | YES  |     | NULL    |                |
| EndTime         | datetime            | YES  |     | NULL    |                |
| RealEndTime     | datetime            | YES  |     | NULL    |                |
| JobTDate        | bigint(20) unsigned | YES  |     | 0       |                |
| VolSessionId    | int(10) unsigned    | YES  |     | 0       |                |
| VolSessionTime  | int(10) unsigned    | YES  |     | 0       |                |
| JobFiles        | int(10) unsigned    | YES  |     | 0       |                |
| JobBytes        | bigint(20) unsigned | YES  |     | 0       |                |
| ReadBytes       | bigint(20) unsigned | YES  |     | 0       |                |
| JobErrors       | int(10) unsigned    | YES  |     | 0       |                |
| JobMissingFiles | int(10) unsigned    | YES  |     | 0       |                |
| PoolId          | int(10) unsigned    | YES  |     | 0       |                |
| FileSetId       | int(10) unsigned    | YES  |     | 0       |                |
| PriorJobId      | int(10) unsigned    | YES  |     | 0       |                |
| PurgedFiles     | tinyint(4)          | YES  |     | 0       |                |
| HasBase         | tinyint(4)          | YES  |     | 0       |                |
| HasCache        | tinyint(4)          | YES  |     | 0       |                |
| Reviewed        | tinyint(4)          | YES  |     | 0       |                |
| Comment         | blob                | YES  |     | NULL    |                |
+-----------------+---------------------+------+-----+---------+----------------+
27 rows in set (0,00 sec)

mysql> <input>select * from Version;</input>
+-----------+
| VersionId |
+-----------+
|      2002 |
+-----------+
1 row in set (0.00 sec)

mysql> <input>exit</input>
Bye
\end{commands}


\paragraph{Modify database credentials}

If you want to change the Bareos database credentials, do the following:

\begin{itemize}
    \item stop the Bareos director
    \item modify the configuration
    \item rerun the grant script \command{grant_bareos_privileges} (or modify database user directly)
    \item start the Bareos director
\end{itemize}

Modify the configuration, set a new password:

\begin{bareosConfigResource}{bareos-dir}{Catalog}{MyCatalog}
Catalog {
  Name = MyCatalog
  dbdriver = "mysql"
  dbname = "bareos"
  dbuser = "bareos"
  dbpassword = "MyNewSecretPassword"
}
\end{bareosConfigResource}

Rerun the Bareos grant script \command{grant_bareos_privileges} ...

\begin{commands}{Modify database privileges}
export db_password=<input>MyNewSecretPassword</input>
/usr/lib/bareos/scripts/grant_bareos_privileges
\end{commands}


\hide{
% table and commands have changed in MySQL 5.7.6.
% As things now get more complicated (different command for different version)
% we skip this topic here.
... or modify the database users directly:

\begin{commands}{Show Bareos database users}
<command>mysql</command>
mysql> <input>SELECT user,host,password FROM mysql.user WHERE user='bareos';</input>
+--------+-----------+-------------------------------------------+
| user   | host      | password                                  |
+--------+-----------+-------------------------------------------+
| bareos | 127.0.0.1 | *CD8C42695AC221807E2BA599FC392C650155C16C |
| bareos | localhost | *CD8C42695AC221807E2BA599FC392C650155C16C |
| bareos | ::1       | *CD8C42695AC221807E2BA599FC392C650155C16C |
+--------+-----------+-------------------------------------------+
3 rows in set (0.00 sec)

mysql> <input>UPDATE mysql.user SET Password=PASSWORD('MyNewSecretPassword') where User='bareos';</input>
Query OK, 3 rows affected (0.00 sec)
Rows matched: 3  Changed: 3  Warnings: 0

mysql> <input>FLUSH PRIVILEGES;</input>
Query OK, 0 rows affected (0.00 sec)

mysql> <input>SELECT user,host,password FROM mysql.user WHERE user='bareos';</input>
+--------+-----------+-------------------------------------------+
| user   | host      | password                                  |
+--------+-----------+-------------------------------------------+
| bareos | 127.0.0.1 | *2119D34B0C0F7452E952EE3A73A7CAA30C1B1852 |
| bareos | localhost | *2119D34B0C0F7452E952EE3A73A7CAA30C1B1852 |
| bareos | ::1       | *2119D34B0C0F7452E952EE3A73A7CAA30C1B1852 |
+--------+-----------+-------------------------------------------+
3 rows in set (0.00 sec)

mysql>
\end{commands}
}


\subsubsection{Sqlite}

There are different versions of Sqlite available. When we use the term Sqlite, we will always refer to Sqlite3.

Sqlite is a file based database. Access via network connection is not supported.
Because its setup is easy, it is a good database for testing.
However please don't use it in a production environment.

Sqlite stores a database in a single file.
Bareos creates this file at \file{/var/lib/bareos/bareos.db}.

Sqlite does not offer access permissions. The only permissions that do apply are the Unix file permissions.

The database is accessable by following command:
\begin{commands}{Verify Bareos database on Sqlite3 (bareos-13.2.3)}
<command>sqlite3</command><input> /var/lib/bareos/bareos.db</input>
SQLite version 3.7.6.3
Enter ".help" for instructions
Enter SQL statements terminated with a ";"
sqlite> <input>.tables</input>
BaseFiles           Filename            Media               Pool
CDImages            Job                 MediaType           Quota
Client              JobHisto            NDMPJobEnvironment  RestoreObject
Counters            JobMedia            NDMPLevelMap        Status
Device              JobStats            NextId              Storage
DeviceStats         Location            Path                UnsavedFiles
File                LocationLog         PathHierarchy       Version
FileSet             Log                 PathVisibility
sqlite> <input>select * from Version;</input>
2002
sqlite>
\end{commands}




\section{Retention Periods}

\subsection{Database Size}
\index[general]{Size!Database}
\index[general]{Database Size}

As mentioned above, if you do not do automatic pruning, your Catalog will grow
each time you run a Job. Normally, you should decide how long you want File
records to be maintained in the Catalog and set the {\bf File Retention}
period to that time. Then you can either wait and see how big your Catalog
gets or make a calculation assuming approximately 154 bytes for each File
saved and knowing the number of Files that are saved during each backup and
the number of Clients you backup.

For example, suppose you do a backup of two systems, each with 100,000 files.
Suppose further that you do a Full backup weekly and an Incremental every day,
and that the Incremental backup typically saves 4,000 files. The size of your
database after a month can roughly be calculated as:

\footnotesize
\begin{verbatim}
Size = 154 * No. Systems * (100,000 * 4 + 10,000 * 26)
\end{verbatim}
\normalsize

where we have assumed four weeks in a month and 26 incremental backups per month.
This would give the following:

\footnotesize
\begin{verbatim}
Size = 154 * 2 * (100,000 * 4 + 10,000 * 26) = 203,280,000 bytes
\end{verbatim}
\normalsize

So for the above two systems, we should expect to have a database size of
approximately 200 Megabytes. Of course, this will vary according to how many
files are actually backed up.

You will note that the File table (containing the file attributes) make up
the large bulk of the number of records as well as the space used.
As a consequence, the most important Retention
period will be the {\bf File Retention} period.


Without proper setup and maintenance, your Catalog may continue to grow
indefinitely as you run Jobs and backup Files, and/or it may become
very inefficient and slow. How fast the size of your
Catalog grows depends on the number of Jobs you run and how many files they
backup. By deleting records within the database, you can make space available
for the new records that will be added during the next Job. By constantly
deleting old expired records (dates older than the Retention period), your
database size will remain constant.


\subsection{Setting Retention Periods}
\index[general]{Setting Retention Periods}
\index[general]{Periods!Setting Retention}
\label{Retention}

Bareos uses three Retention periods: the {\bf File Retention} period,
the {\bf Job Retention} period, and the {\bf Volume Retention} period. Of
these three, the File Retention period is by far the most important in
determining how large your database will become.

The {\bf File Retention} and the {\bf Job Retention} are specified in each
Client resource as is shown below. The {\bf Volume Retention} period is
specified in the Pool resource, and the details are given in the next chapter
of this manual.

\begin{description}

\item [File Retention = {\textless}time-period-specification{\textgreater}]
   \index[general]{File Retention}
   \index[general]{Retention!File}
   The  File Retention record defines the length of time that  Bareos will keep
File records in the Catalog database.  When this time period expires, and if
{\bf AutoPrune} is set to {\bf yes}, Bareos will prune (remove) File records
that  are older than the specified File Retention period. The pruning  will
occur at the end of a backup Job for the given Client.  Note that the Client
database record contains a copy of the  File and Job retention periods, but
Bareos uses the  current values found in the Director's Client resource to  do
the pruning.

Since File records in the database account for probably 80 percent of the
size of the database, you should carefully determine exactly what File
Retention period you need. Once the File records have been removed from
the database, you will no longer be able to restore individual files
in a Job. However, as long as the
Job record still exists, you will be able to restore all files in the
job.

Retention periods are specified in seconds, but as a convenience, there are
a number of modifiers that permit easy specification in terms of minutes,
hours, days, weeks, months, quarters, or years on the record.  See the
\ilink{Configuration chapter}{Time} of this manual for additional details
of modifier specification.

The default File retention period is 60 days.

\item [Job Retention = {\textless}time-period-specification{\textgreater}]
   \index[general]{Job!Retention}
   \index[general]{Retention!Job}
   The Job Retention record defines the length of time that {\bf Bareos}
will keep Job records in the Catalog database.  When this time period
expires, and if {\bf AutoPrune} is set to {\bf yes} Bareos will prune
(remove) Job records that are older than the specified Job Retention
period.  Note, if a Job record is selected for pruning, all associated File
and JobMedia records will also be pruned regardless of the File Retention
period set.  As a consequence, you normally will set the File retention
period to be less than the Job retention period.

As mentioned above, once the File records are removed from the database,
you will no longer be able to restore individual files from the Job.
However, as long as the Job record remains in the database, you will be
able to restore all the files backuped for the Job.
As a consequence, it is generally a good idea to retain the Job
records much longer than the File records.

The retention period is specified in seconds, but as a convenience, there
are a number of modifiers that permit easy specification in terms of
minutes, hours, days, weeks, months, quarters, or years.
See the \ilink{Configuration chapter}{Time} of this manual for additional details of
modifier specification.

The default Job Retention period is 180 days.

\item \linkResourceDirective{Dir}{Client}{Auto Prune}
   \index[general]{AutoPrune}
   \index[general]{Job!Retention!AutoPrune}
   If set to  {\bf yes},
Bareos will automatically apply
the File retention period and the Job  retention period for the Client at the
end of the Job.
If you turn this off by setting it to {\bf no}, your  Catalog will grow each
time you run a Job.
\end{description}


\subsubsection{Job Statistics}

\label{sec:JobStatistics}
\index[general]{Statistics}
\index[general]{Job!Statistics}


Bareos catalog contains lot of information about your IT infrastructure, how
many files, their size, the number of video or music files etc. Using Bareos
catalog during the day to get them permit to save resources on your servers.

In this chapter, you will find tips and information to measure Bareos
efficiency and report statistics.

If you want to have statistics on your backups to
provide some Service Level Agreement indicators, you could use a few
SQL queries on the Job table to report how many:

\begin{itemize}
\item jobs have run
\item jobs have been successful
\item files have been backed up
\item ...
\end{itemize}

However, these statistics are accurate only if your job retention is greater
than your statistics period. Ie, if jobs are purged from the catalog, you won't
be able to use them.

Now, you can use the \bcommand{update}{stats [days=num]} console command to fill
the JobHistory table with new Job records. If you want to be sure to take in
account only good jobs, ie if one of your important job has failed but
you have fixed the problem and restarted it on time, you probably want to
delete the first bad job record and keep only the successful one. For
that simply let your staff do the job, and update JobHistory table after two or
three days depending on your organization using the \parameter{[days=num]} option.

These statistics records aren't used for restoring, but mainly for
capacity planning, billings, etc.

The \linkResourceDirective{Dir}{Director}{Statistics Retention} defines
the length of time that Bareos will keep statistics job records in the Catalog
database after the Job End time. This information is stored in the \texttt{JobHistory} table.
When this time
period expires, and if user runs \bcommand{prune}{stats} command, Bareos will
prune (remove) Job records that are older than the specified period.

You can use the following Job resource in your nightly \job{BackupCatalog}
job to maintain statistics.

\begin{bareosConfigResource}{bareos-dir}{Job}{BackupCatalog}
Job {
  Name = BackupCatalog
  ...
  RunScript {
    Console = "update stats days=3"
    Console = "prune stats yes"
    RunsWhen = After
    RunsOnClient = no
  }
}
\end{bareosConfigResource}


\section{PostgreSQL}

\index[general]{Database!PostgreSQL}
\index[general]{PostgreSQL}

\subsection{Compacting Your PostgreSQL Database}
\index[general]{Database!PostgreSQL!Compacting}
\label{CompactingPostgres}

Over time, as noted above, your database will tend to grow until Bareos starts
deleting old expired records based on retention periods. After that starts,
it is expected that the database size remains constant, provided that the amount
of clients and files being backed up is constant.

Note that PostgreSQL uses multiversion concurrency control (MVCC), so that
an UPDATE or DELETE of a row does not immediately remove the old version of the
row.  Space occupied by outdated or deleted row versions is only reclaimed for
reuse by new rows when running \textbf{VACUUM}. Such outdated or deleted row versions
are also referred to as \emph{dead tuples}.

Since PostgreSQL Version 8.3, autovacuum is enabled by default, so that setting
up a cron job to run VACUUM is not necesary in most of the cases. Note that
there are two variants of VACUUM: standard VACUUM and VACUUM FULL. Standard
VACUUM only marks old row versions for reuse, it does not free any allocated
disk space to the operating system. Only VACUUM FULL can free up disk space,
but it requires exclusive table locks so that it can not be used in parallel
with production database operations and temporarily requires up to as much
additional disk space that the table being processed occupies.

All database programs have some means of writing the database out in ASCII
format and then reloading it. Doing so will re-create the database from
scratch producing a compacted result, so below, we show you how you can do
this for PostgreSQL.

For a PostgreSQL database, you could write the Bareos database as an
ASCII file (\file{bareos.sql}) then reload it by doing the following:

\begin{commands}{}
pg_dump -c bareos > bareos.sql
cat bareos.sql | psql bareos
rm -f bareos.sql
\end{commands}

Depending on the size of your database, this will take more or less time and a
fair amount of disk space. For example, you can \command{cd} to the location of
the Bareos database (typically \directory{/var/lib/pgsql/data} or possible
\directory{/usr/local/pgsql/data}) and check the size.

Except from special cases PostgreSQL does not
need to be dumped/restored to keep the database efficient.  A normal
process of vacuuming will prevent the database from getting too
large.  If you want to fine-tweak the database storage, commands such
as VACUUM, VACUUM FULL, REINDEX, and CLUSTER exist specifically to keep you
from having to do a dump/restore.

More details on this subject can be found in the PostgreSQL documentation.
The page \url{http://www.postgresql.org/docs/} contains links to the documentation
for all PostgreSQL versions. The section \emph{Routine Vacuuming} explains
how VACUUM works and why it is required, see
\url{http://www.postgresql.org/docs/current/static/routine-vacuuming.html}
for the current PostgreSQL version.

\subsubsection{What To Do When The Database Keeps Growing}
\label{PostgresSize}
Especially when a high number of files are beeing backed up or when working with
high retention periods, it is probable that autovacuuming will not work.
When starting to use Bareos with an empty Database, it is normal that the file
table and other tables grow, but the growth rate should drop as soon as jobs are deleted by
retention or pruning. The file table is usually the largest table in Bareos.

The reason for autovacuuming not beeing triggered is then probably the default
setting of \texttt{autovacuum\_vacuum\_scale\_factor = 0.2}, the current value can
be shown with the following query or looked up in \texttt{postgresql.conf}:

\begin{commands}{SQL statement to show the autovacuum\_vacuum\_scale\_factor parameter}
bareos=# show autovacuum_vacuum_scale_factor;
 autovacuum_vacuum_scale_factor
 --------------------------------
  0.2
  (1 row)
\end{commands}

In essence, this means that a VACUUM is only triggered when 20\% of table size
are obsolete. Consequently, the larger the table is, the less frequently VACUUM
will be triggered by autovacuum. This make sense because vacuuming has a
performance impact. While it is possible to override the autovacuum parameters
on a table-by-table basis, it can then still be triggered at any time.

To learn more details about autovacuum see
\url{http://www.postgresql.org/docs/current/static/routine-vacuuming.html#AUTOVACUUM}

The following example shows how to configure running VACUUM on the file table by
using an admin-job in Bareos. The job will be scheduled to run at a time that should
not run in parallel with normal backup jobs, here by scheduling it to run after
the BackupCatalog job.

First step is to check the amount of dead tuples and if autovacuum triggers VACUUM:

\begin{commands}{Check dead tuples and vacuuming on PostgreSQL}
bareos=# SELECT relname, n_dead_tup, last_vacuum, last_autovacuum, last_analyze, last_autoanalyze
FROM pg_stat_user_tables WHERE n_dead_tup > 0 ORDER BY n_dead_tup DESC;
-[ RECORD 1 ]----+------------------------------
relname          | file
n_dead_tup       | 2955116
last_vacuum      |
last_autovacuum  |
last_analyze     |
last_autoanalyze |
-[ RECORD 2 ]----+------------------------------
relname          | log
n_dead_tup       | 111298
last_vacuum      |
last_autovacuum  |
last_analyze     |
last_autoanalyze |
-[ RECORD 3 ]----+------------------------------
relname          | job
n_dead_tup       | 1785
last_vacuum      |
last_autovacuum  | 2015-01-08 01:13:20.70894+01
last_analyze     |
last_autoanalyze | 2014-12-27 18:00:58.639319+01
...
\end{commands}

In the above example, the file table has a high number of dead tuples and it
has not been vacuumed. Same for the log table, but the dead tuple count is not
very high. On the job table autovacuum has been triggered.

Note that the statistics views in PostgreSQL are not persistent, their values
are reset on restart of the PostgreSQL service.

To setup a scheduled admin job for vacuuming the file table, the following must be done:

\begin{enumerate}
\item Create a file with the SQL statements for example\\
\texttt{/usr/local/lib/bareos/scripts/postgresql\_file\_table\_maintenance.sql}\\
with the following content:
\begin{commands}{SQL Script for vacuuming the file table on PostgreSQL}
\t \x
SELECT relname, n_dead_tup, last_vacuum, last_autovacuum, last_analyze, last_autoanalyze
FROM pg_stat_user_tables WHERE relname='file';
VACUUM VERBOSE ANALYZE file;
SELECT relname, n_dead_tup, last_vacuum, last_autovacuum, last_analyze, last_autoanalyze
FROM pg_stat_user_tables WHERE relname='file';
\t \x
SELECT table_name,
  pg_size_pretty(pg_total_relation_size(table_name)) AS total_sz,
  pg_size_pretty(pg_total_relation_size(table_name) - pg_relation_size(table_name)) AS idx_sz
  FROM ( SELECT ('"' || relname || '"' ) AS table_name
    FROM pg_stat_user_tables WHERE relname != 'batch' ) AS all_tables
  ORDER BY pg_total_relation_size(table_name) DESC LIMIT 5;
\end{commands}
The SELECT statements are for informational purposes only, the final statement
shows the total and index disk usage of the 5 largest tables.

\item Create a shell script that runs the SQL statements, for example\\
\texttt{/usr/local/lib/bareos/scripts/postgresql\_file\_table\_maintenance.sh}\\
with the following content:
\begin{commands}{SQL Script for vacuuming the file table on PostgreSQL}
#!/bin/sh
psql bareos < /usr/local/lib/bareos/scripts/postgresql_file_table_maintenance.sql
\end{commands}

\item As in PostgreSQL only the database owner or a database superuser is allowed
to run VACUUM, the script will be run as the \texttt{postgres} user. To permit
the \texttt{bareos} user to run the script via \texttt{sudo}, write the following
sudo rule to a file by executing \texttt{visudo -f /etc/sudoers.d/bareos\_postgres\_vacuum}:
\begin{commands}{sudo rule for allowing bareos to run a script as postgres}
bareos ALL = (postgres) NOPASSWD: /usr/local/lib/bareos/scripts/postgresql_file_table_maintenance.sh
\end{commands}
and make sure that \texttt{/etc/sudoers} includes it, usually by the line
\footnotesize
\begin{verbatim}
#includedir /etc/sudoers.d
\end{verbatim}
\normalsize

\item Create the following admin job in the director configuration
\begin{commands}{SQL Script for vacuuming the file table on PostgreSQL}
# PostgreSQL file table maintenance job
Job {
  Name = FileTableMaintJob
  JobDefs = DefaultJob
  Schedule = "WeeklyCycleAfterBackup"
  Type = Admin
  Priority = 20

  RunScript {
    RunsWhen = Before
    RunsOnClient = no
    Fail Job On Error = yes
    Command = "sudo -u postgres /usr/local/lib/bareos/scripts/postgresql_file_table_maintenance.sh"
  }
}
\end{commands}
In this example the job will be run by the schedule WeeklyCycleAfterBackup,
the \texttt{Priority} should be set to a higher value than \texttt{Priority}
in the BackupCatalog job.
\end{enumerate}


\subsection{Repairing Your PostgreSQL Database}
\label{RepairingPSQL}
\index[general]{Database!Repairing Your PostgreSQL}
\index[general]{Repairing Your PostgreSQL Database}

The same considerations apply as for \nameref{RepairingMySQL}.
Consult the PostgreSQL documents for how to repair the database.

For Bareos specific problems,
consider using \nameref{bareos-dbcheck} program.




\section{MySQL/MariaDB}

\index[general]{Database!MySQL}
\index[general]{MySQL}

\subsection{MySQL/MariaDB Support}
\index[general]{MariaDB|see{MySQL}}
\label{sec:MysqlSupport}

As MariaDB is a fork of MySQL, we use MySQL as synonym for MariaDB and fully support it.
We test our packages against the preferred MySQL fork that a distribution provides.

\hide{
supporting MySQL >= 5.7 since bareos-17.2.4
\limitation{MySQL}{MySQL $\geq$ 5.7 not supported}{%
MySQL 5.7 did change it behavior in some respects. The result is, that the Bareos database creation scripts do not work any more.
For the time being, we advise to use MariaDB instead, which is also the default on most Linux distributions.
See \ticket{705}.
}
}

\subsection{Compacting Your MySQL Database}
\index[general]{Database!MySQL!Compacting}
\label{CompactingMySQL}

Over time, as noted above, your database will tend to grow.
Even though Bareos regularly prunes files, {\bf MySQL} does not automatically
reuse the space, and instead continues growing.

It is assumed that you are using the {\bf InnoDB} database engine (which is the default since MySQL Version 5.5).

It is recommended that you use the {\bf OPTIMIZE TABLE} and {\bf ANALYZE TABLE}
statements regularly. This is to make sure that all indices are up to date
and to recycle space inside the database files.


You can do this via the {\bf mysqlcheck} command:
\footnotesize
\begin{verbatim}
mysqlcheck -a -o -A
\end{verbatim}
\normalsize

Please note that the database files are never shrunk by {\bf MySQL}. If you really need to shrink the database files, you need to recreate the database. This only works if you use per-table tablespaces by setting the {\bf innodb\_file\_per\_table} configuration option.
See
\elink{http://dev.mysql.com/doc/refman/5.5/en/innodb-multiple-tablespaces.html}
{http://dev.mysql.com/doc/refman/5.5/en/innodb-multiple-tablespaces.html} for details.

\hide{
\footnotesize
\begin{verbatim}
mysqldump -f --opt bareos > bareos.sql
mysql bareos < bareos.sql
rm -f bareos.sql
\end{verbatim}
\normalsize

Depending on the size of your database, this will take more or less time and a
fair amount of disk space.
}

\subsection{Repairing Your MySQL Database}
\index[general]{Database!Repairing Your MySQL}
\index[general]{Repairing Your MySQL Database}
\label{RepairingMySQL}

If you find that you are getting errors writing to your MySQL database, or
Bareos hangs each time it tries to access the database, you should consider
running MySQL's database check and repair routines.

This can be done by running the \command{mysqlcheck} command:
\footnotesize
\begin{verbatim}
mysqlcheck --all-databases
\end{verbatim}
\normalsize

If the errors you are getting are simply SQL warnings, then you might try
running \command{bareos-dbcheck} before (or possibly after) using the MySQL database repair
program. It can clean up many of the orphaned record problems, and certain
other inconsistencies in the Bareos database.

A typical cause of MySQL database problems is if your partition fills. In
such a case, you will need to create additional space on the partition.



\subsection{MySQL Table is Full}
\index[general]{Database!MySQL Table is Full}
\index[general]{MySQL Table is Full}

If you are running into the error {\bf The table 'File' is full ...},
it is probably because on version 4.x MySQL, the table is limited by
default to a maximum size of 4 GB and you have probably run into
the limit. The solution can be found at:
\elink{http://dev.mysql.com/doc/refman/5.0/en/full-table.html}
{http://dev.mysql.com/doc/refman/5.0/en/full-table.html}

You can display the maximum length of your table with:

\footnotesize
\begin{verbatim}
mysql bareos
SHOW TABLE STATUS FROM bareos like "File";
\end{verbatim}
\normalsize

If the column labeled "Max\_data\_length" is around 4Gb, this is likely
to be the source of your problem, and you can modify it with:

\footnotesize
\begin{verbatim}
mysql bareos
ALTER TABLE File MAX_ROWS=281474976710656;
\end{verbatim}
\normalsize


\subsection{MySQL Server Has Gone Away}
\index[general]{Database!MySQL Server Has Gone Away}
\index[general]{MySQL Server Has Gone Away}
If you are having problems with the MySQL server disconnecting or with
messages saying that your MySQL server has gone away, then please read
the MySQL documentation, which can be found at:

\elink{http://dev.mysql.com/doc/refman/5.0/en/gone-away.html}
{http://dev.mysql.com/doc/refman/5.0/en/gone-away.html}

\subsection{MySQL Temporary Tables}
When doing backups with large numbers of files, MySQL creates some
temporary tables.  When these tables are small they can be held in
system memory, but as they approach some size, they
spool off to disk.  The default location for these temp tables is
/tmp.  Once that space fills up, Bareos daemons such as the Storage
daemon doing spooling can get strange errors. E.g.

\footnotesize
\begin{verbatim}
Fatal error: spool.c:402 Spool data read error.
Fatal error: backup.c:892 Network send error to SD. ERR=Connection reset by
peer
\end{verbatim}
\normalsize

What you need to do is setup MySQL to use a different (larger) temp
directory, which can be set in the /etc/my.cnf with these variables
set:

\footnotesize
\begin{verbatim}
  tmpdir=/path/to/larger/tmpdir
  bdb_tmpdir=/path/to/larger/tmpdir
\end{verbatim}
\normalsize



\subsection{MySQL: Lock Wait Timeout}

In large environments, the Bareos \mysql backend may run in a lock wait timeout.
This can be seen as Bareos message, e.g.:

\begin{bmessage}{Bareos error message because of \mysql lock time timeout}
Fatal error: sql_create.c:899 Fill File table Query failed: INSERT INTO File (FileIndex, JobId, PathId, FilenameId, LStat, MD5, DeltaSeq) SELECT batch.FileIndex, batch.JobId, Path.PathId, Filename.FilenameId,batch.LStat, batch.MD5, batch.DeltaSeq FROM batch JOIN Path ON (batch.Path = Path.Path) JOIN Filename ON (batch.Name = Filename.Name): ERR=Lock wait timeout exceeded; try restarting transaction
\end{bmessage}

In this case the \mysql \configline{innodb_lock_wait_timeout} must be increased.
A value of 300 should be sufficient.

\begin{config}{/etc/my.cnf.d/server.cnf}
...
[mysqld]
innodb_lock_wait_timeout = 300
...
\end{config}



% TODO: hidden, because outdated information. should we improve or remove it?
\hide{
\label{DatabasePerformance}
\section{Database Performance Issues}
\index[general]{Database Performance Issues}
\index[general]{Performance!Database}

There are a considerable number of ways each of the databases can be
tuned to improve the performance. Going from an untuned database to one
that is properly tuned can make a difference of a factor of 100 or more
in the time to insert or search for records.

For each of the databases, you may get significant improvements by adding
additional indexes. The comments in the Bareos make\_xxx\_tables give some
indications as to what indexes may be appropriate.  Please see below
for specific instructions on checking indexes.

For MySQL, what is very important is to use the examine the
my.cnf file (usually in /etc/my.cnf).
You may obtain significant performances by switching to
the my-large.cnf or my-huge.cnf files that come with the MySQL source
code.

For SQLite3, one significant factor in improving the performance is
to ensure that there is a "PRAGMA synchronous = NORMAL;" statement.
This reduces the number of times that the database flushes the in memory
cache to disk. There are other settings for this PRAGMA that can
give even further performance improvements at the risk of a database
corruption if your system crashes.

For PostgreSQL, you might want to consider turning fsync off.  Of course
doing so can cause corrupted databases in the event of a machine crash.
There are many different ways that you can tune PostgreSQL, the
following document discusses a few of them:
\elink{
http://www.varlena.com/varlena/GeneralBits/Tidbits/perf.html}
{http://www.varlena.com/varlena/GeneralBits/Tidbits/perf.html}.

There is also a PostgreSQL FAQ question number 3.3 that may
answer some of your questions about how to improve performance
of the PostgreSQL engine:
\elink{
http://www.postgresql.org/docs/faqs.FAQ.html\#3.3}
{http://www.postgresql.org/docs/faqs.FAQ.html\#3.3}.
% TODO: verify above is correct. is this okay for book?

Also for PostgreSQL, look at what "effective\_cache\_size". For a 2GB memory
machine, you probably want to set it at 131072, but don't set it too high.
In addition, for a 2GB system, work\_mem = 256000 and
maintenance\_work\_mem = 256000 seem to be reasonable values.  Make
sure your checkpoint\_segments is set to at least 8.
}

\hide{
\section{Performance Issues Indexes}
\index[general]{Database Performance Issues Indexes}
\index[general]{Performance!Database}

\TODO{This chapter needs verification/updating.}

One of the most important considerations for improving performance on
the Bareos database is to ensure that it has all the appropriate indexes.
Several users have reported finding that their database did not have
all the indexes in the default configuration.  In addition, you may
find that because of your own usage patterns, you need additional indexes.

The most important indexes for performance are the two indexes on the
{\bf File} table.  The first index is on {\bf FileId} and is automatically
made because it is the unique key used to access the table.  The other
one is the (JobId, PathId, Filename) index.  If these Indexes
are not present, your performance may suffer a lot.

\subsection{PostgreSQL Indexes}
On PostgreSQL, you can check to see if you have the proper indexes using
the following commands:

\footnotesize
\begin{verbatim}
psql bareos
select * from pg_indexes where tablename='file';
\end{verbatim}
\normalsize

If the indexes are not present, especially the JobId index, you can
create them with the following commands:

\footnotesize
\begin{verbatim}
psql bareos
CREATE INDEX file_jobid_idx on file (jobid);
\end{verbatim}
\normalsize

Make sure that you doesn't have an index on File (filenameid, pathid).

\subsection{MySQL Indexes}
On MySQL, you can check if you have the proper indexes by:

\footnotesize
\begin{verbatim}
mysql bareos
show index from File;
\end{verbatim}
\normalsize

If the indexes are not present, especially the JobId index, you can
create them with the following commands:

\footnotesize
\begin{verbatim}
mysql bareos
CREATE INDEX file_jobid_idx on File (JobId);
\end{verbatim}
\normalsize

Though normally not a problem, you should ensure that the indexes
defined for Filename and Path are both set to 255 characters. Some users
reported performance problems when their indexes were set to 50 characters.
To check, do:

\footnotesize
\begin{verbatim}
mysql bareos
show index from Filename;
show index from Path;
\end{verbatim}
\normalsize

and what is important is that for Filename, you have an index with
Key\_name "Name" and Sub\_part "255". For Path, you should have a Key\_name
"Path" and Sub\_part "255".  If one or the other does not exist or the
Sub\_part is less that 255, you can drop and recreate the appropriate
index with:

\footnotesize
\begin{verbatim}
mysql bareos
DROP INDEX Path on Path;
CREATE INDEX Path on Path (Path(255));

DROP INDEX Name on Filename;
CREATE INDEX Name on Filename (Name(255));
\end{verbatim}
\normalsize


\subsection{SQLite Indexes}
On SQLite, you can check if you have the proper indexes by:

\footnotesize
\begin{verbatim}
sqlite <path>/bareos.db
select * from sqlite_master where type='index' and tbl_name='File';
\end{verbatim}
\normalsize

If the indexes are not present, especially the JobId index, you can
create them with the following commands:

\footnotesize
\begin{verbatim}
sqlite <path>/bareos.db
CREATE INDEX file_jobid_idx on File (JobId);
\end{verbatim}
\normalsize
}


\section{Backing Up Your Bareos Database}
\index[general]{Backup!Bareos database}
\index[general]{Backup!Catalog}
\index[general]{Database!Backup Bareos database}
\label{BackingUpBareos}

If ever the machine on which your Bareos database crashes, and you need to
restore from backup tapes, one of your first priorities will probably be to
recover the database. Although Bareos will happily backup your catalog
database if it is specified in the FileSet, this is not a very good way to do
it, because the database will be saved while Bareos is modifying it. Thus the
database may be in an instable state. Worse yet, you will backup the database
before all the Bareos updates have been applied.

To resolve these problems, you need to backup the database after all the backup
jobs have been run. In addition, you will want to make a copy while Bareos is
not modifying it. To do so, you can use two scripts provided in the release
{\bf make\_catalog\_backup} and {\bf delete\_catalog\_backup}. These files
will be automatically generated along with all the other Bareos scripts. The
first script will make an ASCII copy of your Bareos database into {\bf
bareos.sql} in the working directory you specified in your configuration, and
the second will delete the {\bf bareos.sql} file.

The basic sequence of events to make this work correctly is as follows:

\begin{itemize}
\item Run all your nightly backups
\item After running your nightly backups, run a Catalog backup Job
\item The Catalog backup job must be scheduled after your last nightly backup

\item You use \linkResourceDirective{Dir}{Job}{Run Before Job} to create the ASCII  backup file
      and \linkResourceDirective{Dir}{Job}{Run After Job} to clean up
\end{itemize}

Assuming that you start all your nightly backup jobs at 1:05 am (and that they
run one after another), you can do the catalog backup with the following
additional Director configuration statements:

\begin{bareosConfigResource}{bareos-dir}{Job}{BackupCatalog}
# Backup the catalog database (after the nightly save)
Job {
  Name = "BackupCatalog"
  Type = Backup
  Client=rufus-fd
  FileSet="Catalog"
  Schedule = "WeeklyCycleAfterBackup"
  Storage = DLTDrive
  Messages = Standard
  Pool = Default
  # This creates an ASCII copy of the catalog
  # Arguments to make_catalog_backup.pl are:
  #  make_catalog_backup.pl <catalog-name>
  RunBeforeJob = "/usr/lib/bareos/scripts/make_catalog_backup.pl MyCatalog"
  # This deletes the copy of the catalog
  RunAfterJob  = "/usr/lib/bareos/scripts/delete_catalog_backup"
  # This sends the bootstrap via mail for disaster recovery.
  # Should be sent to another system, please change recipient accordingly
  Write Bootstrap = "|/usr/sbin/bsmtp -h localhost -f \"\(Bareos\) \" -s \"Bootstrap for Job %j\" root@localhost"
}
\end{bareosConfigResource}

\begin{bareosConfigResource}{bareos-dir}{Schedule}{WeeklyCycleAfterBackup}
# This schedule does the catalog. It starts after the WeeklyCycle
Schedule {
  Name = "WeeklyCycleAfterBackup"
  Run = Level=Full sun-sat at 1:10
}
\end{bareosConfigResource}

\begin{bareosConfigResource}{bareos-dir}{FileSet}{Catalog}
# This is the backup of the catalog
FileSet {
  Name = "Catalog"
  Include {
    Options {
      signature=MD5
    }
    File = "/var/lib/bareos/bareos.sql" # database dump
    File = "/etc/bareos"                # configuration
  }
}
\end{bareosConfigResource}

It is preferable
to write/send the \ilink{bootstrap}{BootstrapChapter} file to another computer. It will allow
you to quickly recover the database backup should that be necessary.  If
you do not have a bootstrap file, it is still possible to recover your
database backup, but it will be more work and take longer.
