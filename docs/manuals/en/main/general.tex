
\chapter{What is Bareos?}
\label{GeneralChapter}

Bareos is a set of computer programs that permits the system
administrator to manage backup, recovery, and verification of computer data
across a network of computers of different kinds. Bareos can also run entirely
upon a single computer and can backup to various types of media, including tape
and disk.

In technical terms, it is a
network Client/Server based backup program. Bareos is relatively easy to use
and efficient, while offering many advanced storage management features that
make it easy to find and recover lost or damaged files. Due to its modular
design, Bareos is scalable from small single computer systems to systems
consisting of hundreds of computers located over a large network.

\section{History}
\label{History}
Bareos is a \elink{fork}{http://www.bareos.org/en/faq/items/why_fork.html} of the open source project 
\elink{Bacula}{http://www.bacula.org} version 5.2. 
In 2010 the Bacula community developer Marco van Wieringen started
to collect rejected or neglected community contributions in his own branch.
This branch was later on the base of Bareos and since then was enriched by
a lot of new features.

This documentation also bases on the \elink{original Bacula documentation}{http://www.bacula.org/5.2.x-manuals/en/main/main/}, it is technically
also a fork of the documenation created following the rules of the GNU Free Documentation License.

Original author of Bacula and it's documentation is Kern Sibbald. We thank Kern and all contributors to Bacula and it's documentation.
We maintain a list of contributors to Bacula (until the time we've started the fork) and Bareos in our \elink{AUTHORS}{https://github.com/bareos/bareos/blob/master/AUTHORS} file.


\section{Who Needs Bareos?}

If you are currently using a program such as tar, dump, or
bru to backup your computer data, and you would like a network solution, more
flexibility, or catalog services, Bareos will most likely provide the
additional features you want. However, if you are new to Unix systems or do
not have offsetting experience with a sophisticated backup package, the Bareos project does not
recommend using Bareos as it is much more difficult to setup and use than
tar or dump.

If you want Bareos to behave like the above mentioned simple
programs and write over any tape that you put in the drive, then you will find
working with Bareos difficult. Bareos is designed to protect your data
following the rules you specify, and this means reusing a tape only
as the last resort. It is possible to "force" Bareos to write
over any tape in the drive, but it is easier and more efficient to use a
simpler program for that kind of operation.

If you would like a backup program that can write
to multiple volumes (i.e. is not limited by your tape drive capacity), Bareos
can most likely fill your needs.

If you are currently using a sophisticated commercial package such as Legato
Networker, ARCserveIT, Arkeia, IBM Tivoli Storage Manager or PerfectBackup+, you may be interested in
Bareos, which provides many of the same features and is free software
available under the GNU AGPLv3 software license.

\section{Bareos Components or Services}

Bareos is made up of the following major components or services:
Director, Console, File, Storage, and Monitor services.


\subsection*{Bareos Director}
   \label{DirDef}
   The Director is the central control program for all the other daemons.
   It schedules and supervises
   all the backup, restore, verify and archive operations.  The system
   administrator uses the Bareos Director to schedule backups and to
   recover files. The Director runs as a daemon
   (or service) in the background.
   \label{UADef}

\subsection*{Bareos Console}
   The Bareos Console (\command{bconsole}) is the program that allows the
   administrator or user to communicate with the \bareosDir.
   It runs in a shell window (i.e.  TTY interface).
   Most system administrators will find this completely adequate.
   For more details see the \nameref{sec:bconsole}.

\subsection*{Bareos File Daemon}
   \label{FDDef}
   The \bareosFd is a program that must be installed on each (Client) machine that should be backed up.
   At the request of the \bareosDir, it finds the files to be backed up and sends them (their
   data) to the \bareosSd.

   It is specific to the operating system on which it runs and is responsible for providing the
   file attributes and data when requested by the \bareosDir.

   The \bareosFd is also responsible for the file system dependent part of
   restoring the file attributes and data during a recovery operation.
   This program runs as a daemon on the machine to be backed up.

\subsection*{Bareos Storage Daemon}
    \label{SDDef}
    The \bareosSd is responsible, at the \bareosDir request, for accepting
    data from a \bareosFd and storing the
    file attributes and data to the physical backup media or volumes.
    In the case of a restore request,
    it is responsible to find the data and send it to the \bareosFd.

    There can be multiple \bareosSd in your environment, all controlled by the same \bareosDir.

    The Storage services runs as
    a daemon on the machine that has the backup device (such as a tape
    drive).

\subsection*{Catalog}
   \label{DBDefinition}
   The Catalog services are comprised of the software programs
   responsible for maintaining the file indexes and volume databases for
   all files backed up.  The Catalog services permit the system
   administrator or user to quickly locate and restore any desired file.
   The Catalog services sets Bareos apart from simple backup programs like
   tar and bru, because the catalog maintains a record of all Volumes used,
   all Jobs run, and all Files saved, permitting efficient restoration and
   Volume management.  Bareos currently supports three different databases,
   MySQL, PostgreSQL, and SQLite, one of which must be chosen when building
   Bareos.

   The three SQL databases currently supported (MySQL, PostgreSQL or
   SQLite) provide quite a number of features, including rapid indexing,
   arbitrary queries, and security.  Although the Bareos project plans to support other
   major SQL databases, the current Bareos implementation interfaces only
   to MySQL, PostgreSQL and SQLite.

To perform a successful save or restore, the following four daemons must be
configured and running: the Director daemon, the File daemon, the Storage
daemon, and the Catalog service (MySQL, PostgreSQL or SQLite).

\section{Bareos Version Numbers and Releases}
    \index[general]{Version numbers}
    \index[general]{Releases}

Bareos version numbers consists of three parts: YY.Q.C

\begin{center}
\begin{tabular}{p{0.1\textwidth} p{0.8\textwidth}}
YY & year (last two digits) \\
Q  & quarter of the year \\
YY.Q & year and quarter of the code freeze.
        After this, as a general rule, no new feature should get introduced to this Bareos branch.
        Subsequent release are for bugfixing. \\
C & Release counter. For every subsequent release, this counter is incremented.
        Beginning with 16.2, numbers from 1 to 3 represents the month of the quarter during development.
        After the code freeze, the number is set to 4. So, stable releases get number from 4 onwards.
        Maintenance releases get numbers starting from 5 onwards. \\
\end{tabular}
\end{center}

Following information can be determined from the Bareos release bareos-16.2.4:
\begin{itemize}
\item 16.2: Code freeze have been in the second quarter of 2016
\item 4: this is the first stable release of the bareos-16.2 branch
\end{itemize}

For details about the different releases see \nameref{releasenotes}.

\section{Bareos Packages}
    \label{sec:BareosPackages}

Following Bareos Linux packages are available (release 17.2.4):

% generated by
% cd bareos:/bareos-13.2/SLE_11_SP3
% rpm -qp --qf '%{NAME} | %{SUMMARY} \\\\\n' */*bareos*.rpm | sed -e 's/&/and/g' -e 's/|/\&/' | sort | uniq
% cd /srv/obs/repos/bareos:/bareos-15.2/RHEL_7
% rpm -qp --qf '%{NAME} | %{SUMMARY} \\\\\n' */*bareos*.rpm | sed -e 's/&/and/g' -e 's/|/\&/' | grep -v debuginfo | sort | uniq
% cd /srv/obs/repos/bareos:/bareos-17.2/RHEL_7
% rpm -qp --qf '%{NAME} | %{SUMMARY} \\\\\n' */*bareos*.rpm | sed -e 's/&/and/g' -e 's/|/\&/' | grep -v debuginfo | sort | uniq
\begin{center}
\begin{tabular}{ | l | l | }
\hline
\textbf{Package Name} & \textbf{Description} \\
\hline
bareos & Backup Archiving REcovery Open Sourced - metapackage \\
bareos-bconsole & Bareos administration console (CLI) \\
bareos-client & Bareos client Meta-All-In-One package \\
bareos-common & Common files, required by multiple Bareos packages \\
bareos-database-common & Generic abstraction libs and files to connect to a database \\
bareos-database-mysql & Libs and tools for mysql catalog \\
bareos-database-postgresql & Libs and tools for postgresql catalog \\
bareos-database-sqlite3 & Libs and tools for sqlite3 catalog \\
bareos-database-tools & Bareos CLI tools with database dependencies (bareos-dbcheck, bscan) \\
bareos-devel & Devel headers \\
bareos-director & Bareos Director daemon \\
bareos-director-python-plugin & Python plugin for Bareos Director daemon \\
bareos-filedaemon & Bareos File daemon (backup and restore client) \\
bareos-filedaemon-ceph-plugin & CEPH plugin for Bareos File daemon \\
bareos-filedaemon-glusterfs-plugin & GlusterFS plugin for Bareos File daemon \\
bareos-filedaemon-ldap-python-plugin & LDAP Python plugin for Bareos File daemon \\
bareos-filedaemon-python-plugin & Python plugin for Bareos File daemon \\
bareos-regress-config & Required files for bareos-regress \\
bareos-storage & Bareos Storage daemon \\
bareos-storage-ceph & CEPH support for the Bareos Storage daemon \\
bareos-storage-droplet & Object Storage support (through libdroplet) for the Bareos Storage daemon \\
bareos-storage-fifo & FIFO support for the Bareos Storage backend \\
bareos-storage-glusterfs & GlusterFS support for the Bareos Storage daemon \\
bareos-storage-python-plugin & Python plugin for Bareos Storage daemon \\
bareos-storage-tape & Tape support for the Bareos Storage daemon \\
bareos-tools & Bareos CLI tools (bcopy, bextract, bls, bregex, bwild) \\
bareos-traymonitor & Bareos Tray Monitor (QT) \\
bareos-vadp-dumper & VADP Dumper - vStorage APIs for Data Protection Dumper program \\
bareos-vmware-plugin & Bareos VMware plugin \\
bareos-vmware-plugin-compat & Bareos VMware plugin compatibility \\
bareos-vmware-vix-disklib & VMware vix disklib distributable libraries \\
bareos-webui & Bareos Web User Interface \\
python-bareos & Backup Archiving REcovery Open Sourced - Python module \\
\hline
\end{tabular}
\end{center}

Not all packages (especially optional backends and plugins) are available on all platforms.
For details, see \nameref{sec:packages}.

Additionally, packages containing debug information are available.
These are named differently depending on the distribution (\package{bareos-debuginfo} or \package{bareos-dbg} or  $\ldots$).
%bareos-debuginfo & Debug information for package bareos \\
%bareos-debugsource & Debug sources for package bareos \\

Not all packages are required to run Bareos.

\begin{itemize}
    \item For the Bareos Director, the package \package{bareos-director} and one of \package{bareos-database-postgresql}, \package{bareos-database-mysql} or \package{bareos-database-sqlite3} are required. It is recommended to use \package{bareos-database-postgresql}.

    \item For the \bareosSd, the package \package{bareos-storage} is required. If you plan to connect tape drives to the storage director, also install the package \package{bareos-storage-tape}. This is kept separately, because it has additional dependencies  for tape tools.

    \item On a client, only the package \package{bareos-filedaemon} is required. If you run it on a workstation, the packages \package{bareos-traymonitor} gives the user information about running backups.

    \item On a Backup Administration system you need to install at least \package{bareos-bconsole} to have an interactive console to the \bareosDir.

\end{itemize}



\section{Quick Start}

To get Bareos up and running quickly, the author recommends
that you first scan the
Terminology section below, then quickly review the next chapter entitled
\ilink{The Current State of Bareos}{StateChapter}, then the
\ilink{Installing Bareos}{InstallChapter}, 
the \ilink{Getting Started with Bareos}{QuickStartChapter}, which will
give you a quick overview of getting Bareos running. 
After which, you should
proceed to the chapter
\ilink{How to Configure Bareos}{ConfigureChapter}, 
and finally the chapter on
\ilink{Running Bareos}{TutorialChapter}.

\section{Terminology}
\index[general]{Terminology}

\begin{description}

\item [Administrator]
   \index[general]{Administrator}
   The person or persons responsible for administrating the Bareos system.

\item [Backup]
   \index[general]{Backup}
   The term Backup refers to a Bareos Job that saves files.

\item [Bootstrap File]
   \index[general]{Bootstrap File}
   The bootstrap file is an ASCII file containing a compact form of
   commands that allow Bareos or the stand-alone file extraction utility
   (bextract) to restore the contents of one or more Volumes, for
   example, the current state of a system just backed up.  With a bootstrap
   file, Bareos can restore your system without a Catalog.  You can create
   a bootstrap file from a Catalog to extract any file or files you wish.

\item [Catalog]
   \index[general]{Catalog}
   The Catalog is used to store summary information about the Jobs,
   Clients, and Files that were backed up and on what Volume or Volumes.
   The information saved in the Catalog permits the administrator or user
   to determine what jobs were run, their status as well as the important
   characteristics of each file that was backed up, and most importantly,
   it permits you to choose what files to restore.
   The Catalog is an
   online resource, but does not contain the data for the files backed up.
   Most of the information stored in the catalog is also stored on the
   backup volumes (i.e.  tapes).  Of course, the tapes will also have a
   copy of the file data in addition to the File Attributes (see below).

   The catalog feature is one part of Bareos that distinguishes it from
   simple backup and archive programs such as dump and tar.

\item [Client]
   \index[general]{Client}
   \index[general]{File Daemon|see{Client}}
   In Bareos's terminology, the word Client refers to the machine being
   backed up, and it is synonymous with the File services or File daemon,
   and quite often, it is referred to it as the FD. A Client is defined in a
   configuration file resource.

\item [Console]
   \index[general]{Console}
   The program that interfaces to the Director allowing  the user or system
   administrator to control Bareos.

\item [Daemon]
   \index[general]{Daemon}
   Unix terminology for a program that is always present in  the background to
   carry out a designated task. On Windows systems, as well as some Unix
   systems, daemons are called Services.

\item [Directive]
   \index[general]{Directive}
   The term directive is used to refer to a statement or a record within a
   Resource in a configuration file that defines one specific setting.  For
   example, the {\bf Name} directive defines the name of the Resource.

\item [Director]
   \index[general]{Director}
   The main Bareos server daemon that schedules and directs all  Bareos
   operations. Occasionally, the project refers to the Director as DIR.

\item [Differential]
   \index[general]{Differential}
   A backup that includes all files changed since the last  Full save started.
   Note, other backup programs may define this differently.

\item [File Attributes]
   \index[general]{File Attributes}
   The File Attributes are all the information  necessary about a file to
   identify it and all its properties such as  size, creation date, modification
   date, permissions, etc. Normally, the  attributes are handled entirely by
   Bareos so that the user never  needs to be concerned about them. The
   attributes do not include the  file's data.

\item [File daemon]
   \index[general]{File Daemon}
   The daemon running on the client  computer to be backed up. This is also
   referred to as the File  services, and sometimes as the Client services or the
   FD.

\label{FileSetDef}
\item [FileSet]
   A FileSet is a Resource contained in a configuration file that defines
   the files to be backed up.  It consists of a list of included files or
   directories, a list of excluded files, and how the file is to be stored
   (compression, encryption, signatures).  For more details, see the
   \nameref{DirectorResourceFileSet} in the Director
   chapter of this document.

\item [Incremental]
   \index[general]{Incremental}
   A backup that includes all files changed since the  last Full, Differential,
   or Incremental backup started. It is normally  specified on the {\bf Level}
   directive within the Job resource  definition, or in a Schedule resource.

\label{JobDef}
\item [Job]
    \index[general]{Job}
   A Bareos Job is a configuration resource that defines the work that
   Bareos must perform to backup or restore a particular Client.  It
   consists of the {\bf Type} (backup, restore, verify, etc), the {\bf
   Level} (full, differential, incremental, etc.), 
   the {\bf FileSet}, and {\bf Storage} the
   files are to be backed up (Storage device, Media Pool).  For more
   details, see the \nameref{DirectorResourceJob} in the
   Director chapter of this document.

\item [Monitor]
   \index[general]{Monitor}
   The program that interfaces to all the daemons  allowing the user or
   system administrator to monitor Bareos status.

\item [Resource]
   \index[general]{Resource}
   A resource is a part of a configuration file that defines a specific
   unit of information that is available to Bareos.  It consists of several
   directives (individual configuration statements).  For example, the {\bf
   Job} resource defines all the properties of a specific Job: name,
   schedule, Volume pool, backup type, backup level, ...

\item [Restore]
   \index[general]{Restore}
   A restore is a configuration resource that describes the operation of
   recovering a file from backup media.  It is the inverse of a save,
   except that in most cases, a restore will normally have a small set of
   files to restore, while normally a Save backs up all the files on the
   system.  Of course, after a disk crash, Bareos can be called upon to do
   a full Restore of all files that were on the system.

\item [Schedule]
   \index[general]{Schedule}
   A Schedule is a configuration resource that defines when the Bareos Job
   will be scheduled for execution.  To use the Schedule, the Job resource
   will refer to the name of the Schedule.  For more details, see the
   \nameref{DirectorResourceSchedule} in the Director
   chapter of this document.

\item [Service]
   \index[general]{Service}
   This is a program that remains permanently in memory awaiting
   instructions.  In Unix environments, services are also known as
   {\bf daemons}.

\item [Storage Coordinates]
   \index[general]{Storage Coordinates}
   The information returned from the Storage Services that uniquely locates
   a file on a backup medium.  It consists of two parts: one part pertains
   to each file saved, and the other part pertains to the whole Job.
   Normally, this information is saved in the Catalog so that the user
   doesn't need specific knowledge of the Storage Coordinates.  The Storage
   Coordinates include the File Attributes (see above) plus the unique
   location of the information on the backup Volume.

\item [Storage Daemon]
   \index[general]{Storage Daemon}
   The Storage daemon, sometimes referred to as the SD, is the code that
   writes the attributes and data to a storage Volume (usually a tape or
   disk).

\item [Session]
   \index[general]{Session}
   Normally refers to the internal conversation between the File daemon and
   the Storage daemon.  The File daemon opens a {\bf session} with the
   Storage daemon to save a FileSet or to restore it.  A session has a
   one-to-one correspondence to a Bareos Job (see above).

\item [Verify]
   \index[general]{Verify}
   A verify is a job that compares the current file attributes to the
   attributes that have previously been stored in the Bareos Catalog.  This
   feature can be used for detecting changes to critical system files
   similar to what a file integrity checker like Tripwire does.
   One of the major advantages of
   using Bareos to do this is that on the machine you want protected such
   as a server, you can run just the File daemon, and the Director, Storage
   daemon, and Catalog reside on a different machine.  As a consequence, if
   your server is ever compromised, it is unlikely that your verification
   database will be tampered with.

   Verify can also be used to check that the most recent Job data written
   to a Volume agrees with what is stored in the Catalog (i.e.  it compares
   the file attributes), *or it can check the Volume contents against the
   original files on disk.

%\item [*Archive]
%   \index[general]{*Archive}
%   An Archive operation is done after a Save, and it  consists of removing the
%   Volumes on which data is saved from active  use. These Volumes are marked as
%   Archived, and may no longer be  used to save files. All the files contained
%   on an Archived Volume  are removed from the Catalog. NOT YET IMPLEMENTED.

\item [Retention Period]
   \index[general]{Retention Period}
   There are various kinds of retention periods that Bareos recognizes.
   The most important are the {\bf File} Retention Period, {\bf Job}
   Retention Period, and the {\bf Volume} Retention Period.  Each of these
   retention periods applies to the time that specific records will be kept
   in the Catalog database.  This should not be confused with the time that
   the data saved to a Volume is valid.

   The File Retention Period determines the time that File records are kept
   in the catalog database.  This period is important for two reasons: the
   first is that as long as File records remain in the database, you
   can "browse" the database with a console program and restore any
   individual file. Once the File records are removed or pruned from the
   database, the individual files of a backup job can no longer be
   "browsed".  The second reason for carefully choosing the File Retention
   Period is because the volume of
   the database File records use the most storage space in the
   database.  As a consequence, you must ensure that regular "pruning" of
   the database file records is done to keep your database from growing
   too large. (See the Console {\bf prune}
   command for more details on this subject).

   The Job Retention Period is the length of time that Job records will be
   kept in the database.  Note, all the File records are tied to the Job
   that saved those files.  The File records can be purged leaving the Job
   records.  In this case, information will be available about the jobs
   that ran, but not the details of the files that were backed up.
   Normally, when a Job record is purged, all its File records will also be
   purged.

   The Volume Retention Period is the minimum of time that a Volume will be
   kept before it is reused.  Bareos will normally never overwrite a Volume
   that contains the only backup copy of a file.  Under ideal conditions,
   the Catalog would retain entries for all files backed up for all current
   Volumes.  Once a Volume is overwritten, the files that were backed up on
   that Volume are automatically removed from the Catalog.  However, if
   there is a very large pool of Volumes or a Volume is never overwritten,
   the Catalog database may become enormous.  To keep the Catalog to a
   manageable size, the backup information should be removed from the
   Catalog after the defined File Retention Period.  Bareos provides the
   mechanisms for the catalog to be automatically pruned according to the
   retention periods defined.

\item [Scan]
   \index[general]{Scan}
   A Scan operation causes the contents of a Volume or a series of Volumes
   to be scanned.  These Volumes with the information on which files they
   contain are restored to the Bareos Catalog.  Once the information is
   restored to the Catalog, the files contained on those Volumes may be
   easily restored.  This function is particularly useful if certain
   Volumes or Jobs have exceeded their retention period and have been
   pruned or purged from the Catalog.  Scanning data from Volumes into the
   Catalog is done by using the {\bf bscan} program.  See the \ilink{bscan
   section}{bscan} of the Bareos Utilities chapter of this manual for more
   details.

\item [Volume]
   \index[general]{Volume}
   A Volume is an archive unit, normally a tape or a named disk file where
   Bareos stores the data from one or more backup jobs.  All Bareos Volumes
   have a software label written to the Volume by Bareos so that it
   identifies what Volume it is really reading.  (Normally there should be
   no confusion with disk files, but with tapes, it is easy to mount the
   wrong one.)
\end{description}

\section{What Bareos is Not}

Bareos is a backup, restore and verification program and is not a
complete disaster recovery system in itself, but it can be a key part of one
if you plan carefully and follow the instructions included in the
\ilink{Disaster Recovery}{RescueChapter} chapter of this manual.

\section{Interactions Between the Bareos Services}

The following block diagram shows the typical interactions between the Bareos
Services for a backup job. Each block represents in general a separate process
(normally a daemon). In general, the Director oversees the flow of
information. It also maintains the Catalog.

\begin{center}
\includegraphics[width=0.8\textwidth]{\idir flow}
\end{center}
