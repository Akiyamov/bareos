\defDirective{Dir}{Catalog}{Address}{}{}{%
Alias for \linkResourceDirective{Dir}{Catalog}{DB Address}.
}

\defDirective{Dir}{Catalog}{DB Address}{}{}{%
This is the host address  of the database server. Normally, you would specify
this instead  of \linkResourceDirective{Dir}{Catalog}{DB Socket} if the database server is on another machine.
In that case, you will also specify \linkResourceDirective{Dir}{Catalog}{DB Port}. 
This directive  is used
only by MySQL and PostgreSQL and is ignored by SQLite if provided.
}

\defDirective{Dir}{Catalog}{DB Driver}{postgresql {\textbar} mysql {\textbar} sqlite}{}{%
Selects the database type to use.
}

\defDirective{Dir}{Catalog}{DB Name}{}{}{%
This specifies the name of the database.
}

\defDirective{Dir}{Catalog}{DB Password}{}{}{%
This specifies the password to use when login into the database.
}

\defDirective{Dir}{Catalog}{DB Port}{}{}{%
This defines the port to  be used in conjunction with \linkResourceDirective{Dir}{Catalog}{DB Address} to
access the  database if it is on another machine. This directive is used  only
by MySQL and PostgreSQL and is ignored by SQLite if provided.
}

\defDirective{Dir}{Catalog}{DB Socket}{}{}{%
This is the name of  a socket to use on the local host to connect to the
database. This directive is used only by MySQL and is ignored by  SQLite.
Normally, if neither \linkResourceDirective{Dir}{Catalog}{DB Socket} 
or \linkResourceDirective{Dir}{Catalog}{DB Address}  are specified, MySQL
will use the default socket. If the DB Socket is specified, the
MySQL server must reside on the same machine as the Director.
}

\defDirective{Dir}{Catalog}{DB User}{}{}{%
This specifies what user name to use to log into the database.
}

\defDirective{Dir}{Catalog}{Description}{}{}{%
}

\defDirective{Dir}{Catalog}{Disable Batch Insert}{}{}{%
This directive allows you to override at runtime if the Batch insert should
be enabled or disabled. Normally this is determined by querying the database
library if it is thread-safe. If you think that disabling Batch insert will make
your backup run faster you may disable it using this option and set it to 
\parameter{Yes}.
}

\defDirective{Dir}{Catalog}{Idle Timeout}{}{}{%
This directive is used by the experimental database pooling functionality. Only use
this for non production sites. This sets the idle time after which a database pool
should be shrinked.
}

\defDirective{Dir}{Catalog}{Inc Connections}{}{}{%
This directive is used by the experimental database pooling functionality. Only use
this for non production sites. This sets the number of connections to add to a
database pool when not enough connections are available on the pool anymore.
}

\defDirective{Dir}{Catalog}{Max Connections}{}{}{%
This directive is used by the experimental database pooling functionality. Only use
this for non production sites. This sets the maximum number of connections to a
database to keep in this database pool.
}

\defDirective{Dir}{Catalog}{Min Connections}{}{}{%
This directive is used by the experimental database pooling functionality. Only use
this for non production sites. This sets the minimum number of connections to a
database to keep in this database pool.
}

\defDirective{Dir}{Catalog}{Multiple Connections}{}{}{%
%% By default, this  directive is set to no. In that case, each job that uses the
%% same Catalog will use a single connection to the catalog. It will  be shared,
%% and Bareos will allow only one Job at a time to  communicate. If you set this
%% directive to yes, Bareos will  permit multiple connections to the database,
%% and the database  must be multi-thread capable. For SQLite and PostgreSQL,
%% this is  no problem. For MySQL, you must be *very* careful to have the
%% multi-thread version of the client library loaded on your system.  When this
%% directive is set yes, each Job will have a separate  connection to the
%% database, and the database will control the  interaction between the different
%% Jobs. This can significantly  speed up the database operations if you are
%% running multiple  simultaneous jobs. In addition, for SQLite and PostgreSQL,
%% Bareos  will automatically enable transactions. This can significantly  speed
%% up insertion of attributes in the database either for  a single Job or
%% multiple simultaneous Jobs.
%%
%% This directive has not been tested. Please test carefully  before running it
%% in production and report back your results.
Not yet implemented.
}

\defDirective{Dir}{Catalog}{Name}{}{}{%
The name of the Catalog.  No necessary relation to the database server
name.  This name will be specified in the Client resource directive
indicating that all catalog data for that Client is maintained in this
Catalog.
}

\defDirective{Dir}{Catalog}{Password}{}{}{%
Alias for \linkResourceDirective{Dir}{Catalog}{DB Password}.
}

\defDirective{Dir}{Catalog}{User}{}{}{%
Alias for \linkResourceDirective{Dir}{Catalog}{DB User}.
}

\defDirective{Dir}{Catalog}{Validate Timeout}{}{}{%
This directive is used by the experimental database pooling functionality. Only use
this for non production sites. This sets the validation timeout after which the
database connection is polled to see if its still alive.
}

