\chapter{Getting Started with Bareos}
\label{QuickStartChapter}

\section{Understanding Jobs and Schedules}
\index[general]{Schedule!Understanding Schedules}
\label{JobsandSchedules}

In order to make Bareos as flexible as possible, the directions given
to Bareos are specified in several pieces.  The main instruction is the
job resource, which defines a job.  A backup job generally consists of a
FileSet, a Client, a Schedule for one or several levels or times of backups,
a Pool, as well as additional instructions. Another way of looking
at it is the FileSet is what to backup; the Client is who to backup; the
Schedule defines when, and the Pool defines where (i.e. what Volume).

Typically one FileSet/Client combination will have one corresponding job.
Most of the directives, such as FileSets, Pools, Schedules, can be mixed
and matched among the jobs.  So you might have two different Job
definitions (resources) backing up different servers using the same
Schedule, the same Fileset (backing up the same directories on two machines)
and maybe even the same Pools.  The Schedule will define what type of
backup will run when (e.g. Full on Monday, incremental the rest of the
week), and when more than one job uses the same schedule, the job priority
determines which actually runs first.  If you have a lot of jobs, you might
want to use JobDefs, where you can set defaults for the jobs, which can
then be changed in the job resource, but this saves rewriting the
identical parameters for each job.  In addition to the FileSets you want to
back up, you should also have a job that backs up your catalog.

Finally, be aware that in addition to the backup jobs there are
restore, verify, and admin jobs, which have different requirements.

\section{Understanding Pools, Volumes and Labels}
\index[general]{Pools!Understanding}
\index[general]{Volumes!Understanding}
\index[general]{Label!Understanding Labels}
\label{PoolsVolsLabels}

If you have been using a program such as \command{tar} to backup your system,
Pools, Volumes, and labeling may be a bit confusing at first. A Volume is a
single physical tape (or possibly a single file) on which Bareos will write
your backup data. Pools group together Volumes so that a backup is not
restricted to the length of a single Volume (tape). Consequently, rather than
explicitly naming Volumes in your Job, you specify a Pool, and Bareos will
select the next appendable Volume from the Pool and mounts it.

Although the basic Pool options are specified in the Director's \ilink{Pool}{DirectorResourcePool} resource,
the real Pool is maintained in the Bareos Catalog. It contains
information taken from the Pool resource (configuration file) as well as
information on all the Volumes that have been added to the Pool.
% Adding
%Volumes to a Pool is usually done manually with the Console program using the
%\bcommand{label}{} command.

For each Volume, Bareos maintains a fair amount of catalog information such as
the first write date/time, the last write date/time, the number of files on
the Volume, the number of bytes on the Volume, the number of Mounts, etc.

Before Bareos will read or write a Volume, the physical Volume must have a
Bareos software label so that Bareos can be sure the correct Volume is
mounted.
Depending on your configuration,
this is either done automatically by Bareos
or manually using the \bcommand{label}{} command in the Console
program.

The steps for creating a Pool, adding Volumes to it, and writing software
labels to the Volumes, may seem tedious at first, but in fact, they are quite
simple to do, and they allow you to use multiple Volumes (rather than being
limited to the size of a single tape). Pools also give you significant
flexibility in your backup process. For example, you can have a "Daily" Pool
of Volumes for Incremental backups and a "Weekly" Pool of Volumes for Full
backups. By specifying the appropriate Pool in the daily and weekly backup
Jobs, you thereby insure that no daily Job ever writes to a Volume in the
Weekly Pool and vice versa, and Bareos will tell you what tape is needed and
when.

For more on Pools, see the
\nameref{DirectorResourcePool} section of the Director
Configuration chapter, or simply read on, and we will come back to this
subject later.

\section{Setting Up Bareos Configuration Files}
\label{config}
\index[general]{Configuration!Files}

On Unix, Bareos configuration files are usually located in the \path|/etc/bareos/| directory
and are named accordingly to the programs that use it.
Since Bareos \sinceVersion{Dir}{Subdirectory Configuration Scheme used as Default}{16.2.4} the default configuration
is stored as one file per resource in subdirectories under \directory{bareos-dir.d}, \directory{bareos-sd.d} or \directory{bareos-fd.d}.
For details, see \nameref{ConfigureChapter} and \nameref{sec:SubdirectoryConfigurationScheme}.

%For information about Windows configuration files, see the \ilink{Windows chapter}{Windows:Configuration:Files}.


% \subsection{Configuring the bconsole Program}
% \index[general]{Configure!Console}
% 
% The Console program is used by the administrator to interact with the Director
% and to manually start/stop Jobs or to obtain Job status information.
% 
% The Console configuration file is named \file{bconsole.conf}.
% 
% Normally, for first time users, no change is needed to these files.
% Reasonable defaults are set.
% 
% Further details are in the
% \ilink{Console configuration}{ConsoleConfChapter} chapter.

% \subsection{Configuring the Monitor Program}
% \index[general]{Program!Configuring the Monitor}
% \index[general]{Configuring the Monitor Program}
% 
% The Monitor program is typically an icon in the system tray. However, once the
% icon is expanded into a full window, the administrator or user can obtain
% status information about the Director or the backup status on the local
% workstation or any other Bareos daemon that is configured.
% 
% %\addcontentsline{lof}{figure}{Bareos Tray Monitor}
% %\includegraphics{\idir Bareos-tray-monitor}
% 
% % TODO: image may be too wide for 6" wide printed page.
% %The image shows a tray-monitor configured for three daemons. By clicking on
% %the radio buttons in the upper left corner of the image, you can see the
% %status for each of the daemons. The image shows the status for the Storage
% %daemon (MainSD) that is currently selected.
% 
% The Monitor configuration file is named {\bf tray-monitor.conf}. 
% 
% More information is in the
% \ilink{Monitor configuration}{_MonitorChapter} chapter.

% \subsection{Configuring the \bareosFd}
% \index[general]{Configure!Client}
% \index[fd]{Configure}
% 
% The File daemon is a program that runs on each (Client) machine. At the
% request of the \bareosDir,
% finds the files to be backed up and sends them (their
% data) to the \bareosSd.

% The File daemon configuration file is named {\bf
% bareos-fd.conf}.

% Normally, for first time users, no change is needed to this file.
% Reasonable defaults are set.
% 
% However, if you are going to back up more
% than one machine, you will need to install the \bareosFd with a unique
% configuration on each machine to be backed up.
%  The information about each
% \bareosFd must appear in the \bareosDir configuration, see \nameref{sec:AddAClient}.
% 
% Further details about configuration can be found at \nameref{FiledConfChapter}.

% \subsection{Configuring the Director}
% \index[general]{Configure!Director}
% \index[dir]{Configure}
% 
% The Director is the central control program for all the other daemons.
% It schedules and monitors all jobs to be backed up.
% 
% %The Director configuration file is named {\bf bareos-dir.conf}.
% 
% You may want to change the email address for notification from the
% default \email{root@localhost} to your email address.
% 
% If you have multiple systems to be backed up, you will need a
% separate \bareosFd resource for each system, specifying its
% name, address, and password.
% We have found that giving your daemons the same
% name as your system but post fixed with {\bf -fd} helps a lot when reading log messages.
% That is, if your system name is \name{foobaz}, you would give the \bareosFd
% the name \name{foobaz-fd}.
% For the \bareosDir, you should use \name{foobaz-dir},
% and for the storage daemon, you might use \name{foobaz-sd}.
% Each of your Bareos components {\bf must} have a unique name.  If you
% make them all the same, aside from the fact that you will not
% know what daemon is sending what message, if they share the same
% working directory, the daemons temporary file names will not
% be unique, and you will get many strange failures.
% % TODO: why not check for that and not allow sharing working directory?
% 
% More information is in the
% \nameref{DirectorChapter} chapter.
% 
% \subsection{Configuring the Storage daemon}
% \index[general]{Daemon!Configuring the Storage}
% \index[general]{Configuring the Storage daemon}
% 
% The Storage daemon is responsible, at the Director's request, for accepting
% data from a File daemon and placing it on Storage media, or in the case of a
% restore request, to find the data and send it to the File daemon.
% 
% The Storage daemon's configuration file is named {\bf bareos-sd.conf}.
% The default configuration comes with backup to disk only,
% so the Archive device points to a directory in which the
% Volumes will be created as files when you label the Volume.
% %Edit this file to contain the correct Archive device names for any tape
% %devices that you have. 
% %If the configuration process properly detected your
% %system, they will already be correctly set. 
% These Storage resource name and
% Media Type must be the same as the corresponding ones in the Director's
% configuration file {\bf bareos-dir.conf}.
% \label{ConfigTesting}
% 
% Further information is in the
% \ilink{Storage daemon configuration}{StoredConfChapter} chapter.

\section{Testing your Configuration Files}
\index[general]{Testing!Configuration Files}

You can test if your configuration file is syntactically correct by running
the appropriate daemon with the \parameter{-t} option. The daemon will process the
configuration file and print any error messages then terminate.

As the \bareosDir and \bareosSd runs as user \user{bareos},
testing the configuration should be done as \user{bareos}.

This is especially required to test the \bareosDir,
as it also connects to the database and checks if the catalog schema version is correct.
Depending on your database, only the \user{bareos} has permission to access it.

\begin{commands}{Testing Configuration Files}
su bareos -s /bin/sh -c "/usr/sbin/bareos-dir -t"
su bareos -s /bin/sh -c "/usr/sbin/bareos-sd -t"
bareos-fd -t
bconsole -t
bareos-tray-monitor -t
\end{commands}

% \section{Testing Compatibility with Your Tape Drive}
% \index[general]{Drive!Testing Compatibility with Your Tape}
% \index[general]{Testing Compatibility with Your Tape Drive}
% 
% Before spending a lot of time on Bareos only to find that it doesn't work
% with your tape drive, please read the \nameref{TapeTestingChapter} section.
% If you have a modern
% standard SCSI tape drive on a Linux or Solaris, most likely it will work,
% but better test than be sorry.  For FreeBSD (and probably other xBSD
% flavors), reading the above mentioned tape testing chapter is a must.


% \section{Running Bareos}
% \index[general]{Running Bareos}
%
% Probably the most important part of running Bareos is being able to restore
% files. If you haven't tried recovering files at least once, when you actually
% have to do it, you will be under a lot more pressure, and prone to make
% errors, than if you had already tried it once.
% 
% To get a good idea how to use Bareos in a short time, we {\bf strongly}
% recommend that you follow the example in the
% \nameref{TutorialChapter} chapter of this manual where
% you will get detailed instructions on how to run Bareos.

% \section{Log Rotation}
% \index[general]{Rotation!Log}
% \index[general]{Log Rotation}
% If you use the default {\bf bareos-dir.conf} or some variation of it, you will
% note that it logs all the Bareos output to a file. To avoid that this file
% grows without limit, we recommend that you copy the file {\bf logrotate} from
% the {\bf scripts/logrotate} to {\bf /etc/logrotate.d/bareos}. This will cause
% the log file to be rotated once a month and kept for a maximum of five months.
% You may want to edit this file to change the default log rotation preferences.

% \section{Log Watch}
% \index[general]{Watch!Log}
% \index[general]{Log Watch}
% Some systems such as Red Hat and Fedora run the logwatch program
% every night, which does an analysis of your log file and sends an
% email report.  If you wish to include the output from your Bareos
% jobs in that report, please look in the {\bf scripts/logwatch}
% directory.  The {\bf README} file in that directory gives a brief
% explanation on how to install it and what kind of output to expect.


% \section{Disaster Recovery}
% \index[general]{Recovery!Disaster}
% \index[general]{Disaster!Recovery}
% 
% If you intend to use Bareos as a disaster recovery tool rather than simply a
% program to restore lost or damaged files, you will want to read the
% \ilink{Disaster Recovery Using Bareos Chapter}{RescueChapter} of
% this manual.
% 
% In any case, you are strongly urged to carefully test restoring some files
% that you have saved rather than wait until disaster strikes. This way, you
% will be prepared.
