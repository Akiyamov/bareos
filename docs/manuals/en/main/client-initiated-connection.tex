\section{Client Initiated Connection}
\label{sec:ClientInitiatedConnection}

The \bareosDir knows, when it is required to talk to a client (\bareosFd).
Therefore, by defaults, the \bareosDir connects to the clients.

However, there are setups where this can cause problems, as this means that:
\begin{itemize}
    \item The client must be reachable by its configured \linkResourceDirective{Dir}{Client}{Address}. Address can be the DNS name or the IP address. (For completeness: there are potential workarounds by using the \ilink{setip}{bcommandSetIP} command.)
    \item The \bareosDir must be able to connect to the \bareosFd over the network.
\end{itemize}

To circumvent these problems, since Bareos \sinceVersion{Dir}{Client Initiated Connection}{16.2.2} it is possible to let the \bareosFd initiate the network connection to the \bareosDir.

Which address the client connects to reach the \bareosDir is configured in the \linkResourceDirective{Fd}{Director}{Address} directive.

To additional allow this connection direction use:
\begin{itemize}
  \item \linkResourceDirective{Dir}{Client}{Connection From Client To Director} = yes
  \item \linkResourceDirective{Dir}{Client}{Heartbeat Interval} = 60 \verb|#| to keep the network connection established
  \item \linkResourceDirective{Fd}{Director}{Connection From Client To Director} = yes
\end{itemize}

To only allow Connection From the Client to the Director use:
\begin{itemize}
  \item \linkResourceDirective{Dir}{Client}{Connection From Director To Client} = no
  \item \linkResourceDirective{Dir}{Client}{Connection From Client To Director} = yes
  \item \linkResourceDirective{Dir}{Client}{Heartbeat Interval} = 60 \verb|#| to keep the network connection established
  \item \linkResourceDirective{Fd}{Director}{Connection From Director To Client} = no
  \item \linkResourceDirective{Fd}{Director}{Connection From Client To Director} = yes
\end{itemize}

Using Client Initiated Connections has disadvantages.
Without Client Initiated Connections the \bareosDir only establishes a network connection when this is required.
With Client Initiated Connections, the \bareosFd connects to the \bareosDir and the \bareosDir keeps these connections open.
The command \bcommand{status}{dir} will show all waiting connections:
\begin{bconsole}{show waiting client connections}
*<input>status dir</input>
...
Client Initiated Connections (waiting for jobs):
Connect time        Protocol            Authenticated       Name
====================================================================================================
19-Apr-16 21:50     54                  1                   client1.example.com
...
====
\end{bconsole}

When both connection directions are allowed, the \bareosDir 
\begin{enumerate}
  \item checks, if there is a waiting connection from this client.
  \item tries to connect to the client (using the usual timeouts).
  \item waits for a client connection to appear (using the same timeout as when trying to connect to a client).
\end{enumerate}
If none of this worked, the job fails.

When a waiting connection is used for a job, the \bareosFd will detect this and creates an additional connection.
This is required, to keep the client responsive for additional commands, like \bcommand{cancel}{}.

To get feedback in case the \bareosFd fails to connect to the \bareosDir, consider configuring \bareosFd to log in a local file.
This can be archived by adding the line

\configline{Append = "/var/log/bareos/bareos-fd.log" = all, !skipped, !restored}

to the default message resource \resourcename{Fd}{Messages}{Standard}:

\begin{bareosConfigResource}{bareos-fd}{messages}{Standard}
Messages {
  Name = Standard
  Director = bareos-dir = all, !skipped, !restored
  Append = "/var/log/bareos/bareos-fd.log" = all, !skipped, !restored
}
\end{bareosConfigResource}
