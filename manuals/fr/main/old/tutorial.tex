%%
%%

\chapter{Une br\`eve documentation}
\label{_ChapterStart1}
\index[general]{Une br\`eve documentation}
\index[general]{Documentation!br\`eve }

Ce chapitre vous guidera \`a travers les \'etapes n\'ecessaires pour ex\'ecuter Bacula. 
Pour cela, nous supposons que vous avez install\'e Bacula, peut \^etre dans un simple 
r\'epertoire comme le d\'ecrit le chapitre pr\'ec\'edent, auquel cas vous pouvez ex\'ecuter 
Bacula sans \^etre root pour ces tests. Nous supposons d'autre part que vous n'avez 
pas modifi\'e les fichiers de configuration. Dans le cas contraire, nous vous 
recommandons de d\'esinstaller Bacula et de le r\'einstaller sans rien modifier. Les 
exemples de ce chapitre utilisent les fichiers de configuration par d\'efaut, et 
cr\'eent les volumes dans le r\'epertoire {\bf /tmp} de votre disque. De plus, les 
donn\'ees sauvegard\'ees seront celle du r\'epertoire des sources de Bacula o\`u vous 
l'avez compil\'e. Par cons\'equent, tous les {\it daemons} peuvent \^etre ex\'ecut\'es 
sans les droits root pour ces tests. Notez bien qu'en production, vos File 
Daemons devront \^etre ex\'ecut\'es en tant que root. Voyez le chapitre sur la 
s\'ecurit\'e pour plus d'informations sur ce sujet.

Voici les \'etapes que nous suivrons :

\begin{enumerate}
\item cd \lt{}install-directory\gt{}  
\item D\'emarrer la base de donn\'ees (si vous utilisez MySQL ou PostgreSQL)  
\item D\'emarrer les {\it daemons} avec {\bf ./bacula start}  
\item Lancer le programme Console pour interagir avec le Director 
\item Lancer un job
\item Lorsqu'un volume est plein, le d\'emonter, s'il s'agit d'une cartouche, en 
   \'etiqueter une nouvelle et poursuivre. Dans ce chapitre, nous n'\'ecrirons que sur 
   des volumes fichier, aussi vous n'avez pas \`a vous inqui\'eter au sujet des 
   cartouches pour le moment.
\item Tester la restauration de quelques fichiers depuis le volume fraichement \'ecrit 
   pour s'assurer de la validit\'e de la sauvegarde et qu'il est possible de restaurer. 
   Mieux vaut essayer avant qu'un d\'esastre ne survienne...   
\item Ajouter un second client.
   \end{enumerate}

Chacune de ces \'etapes est d\'ecrite en d\'etail ci-dessous.

\section{Avant d'ex\'ecuter Bacula}
\index[general]{Bacula!Avant d'ex\'ecuter}
\index[general]{Avant d'ex\'ecuter Bacula}
\addcontentsline{toc}{section}{Avant d'ex\'ecuter Bacula}

Avant d'utiliser Bacula pour la premi\`ere fois en production, nous vous recommandons 
d'ex\'ecuter la commande {\bf test} du programme {\bf btape} ainsi qu'il est d\'ecrit 
dans le chapitre \ilink{Programmes utilitaires}{btape} de ce manuel. 
Ce programme vous aidera \`a vous assurer que votre lecteur de bandes fonctionne 
correctement avec Bacula. Si vous avez un lecteur moderne de marque HP, Sony, ou 
Quantum DDS ou DLT qui fonctionne sous Linux ou Solaris, vous pouvez probablement 
vous dispenser de faire ce test car Bacula est bien test\'e avec ces lecteurs et ces 
syst\`emes. Dans tous les autres cas, vous \^etes {\bf fortement} encourag\'e \`a 
ex\'ecuter les tests avant de poursuivre. {\bf btape} dispose aussi d'une commande 
{\bf fill} qui tente de reproduire le comportement de Bacula lorsqu'il remplit 
une cartouche et qu'il poursuit son \'ecriture sur la suivante. Vous devriez 
songer \`a faire ce test, sachez cependant qu'il peut \^etre long (environ 
4 heures sur mon lecteur) de remplir une cartouche de haute capacit\'e. 

\section{D\'emarrer la base de donn\'ees}
\label{StartDB}
\index[general]{D\'emarrer la base de donn\'ees}
\index[general]{base de donn\'ees!D\'emarrer la }
\addcontentsline{toc}{section}{D\'emarrer la base de donn\'ees}

Si vous utilisez MySQL ou PostgreSQL pour votre catalogue Bacula, vous devez 
d\'emarrer la base de donn\'ees avant d'essayer de lancer un job pour \'eviter 
d'obtenir des messages d'erreur au d\'emarrage de Bacula. J'utilise les scripts 
{\bf startmysql} et {\bf stopmysql} pour d\'emarrer mon MySQL local. Notez que si 
vous utilisez SQLite, vous n'aurez pas \`a utiliser  {\bf startmysql} ou {\bf stopmysql}. 
Si vous utilisez ceci en production, vous souhaiterez probablement trouver 
un moyen pour d\'emarrer automatiquement MySQL ou PostgreSQL apr\`es chaque 
red\'emarrage du syst\`eme.

Si vous utilisez SQLite (c'est \`a dire, si vous avez sp\'ecifi\'e l'option 
{\bf \verb:--:with-sqlite=xxx} de la commande {\bf ./configure}, vous n'avez rien \`a faire. 
SQLite est d\'emarr\'ee automatiquement par {\bf Bacula}.

\section{D\'emarrer les daemons}
\label{StartDaemon}
\index[general]{D\'emarrer les daemons}
\index[general]{Daemons!D\'emarrer les}
\addcontentsline{toc}{section}{D\'emarrer les daemons}

Que vous ayez compil\'e Bacula depuis les sources ou que vous ayez install\'e les rpms, 
tapez simplement :

./bacula start

dans votre r\'epertoire d'installation pour d\'emarrer les trois {\it daemons}.

Le script {\bf bacula} lance le Storage Daemon, le File Daemon et le Director Daemon, qui 
tournent tous trois en tant que {\it daemons} en t\^ache de fond. Si vous utilisez 
la fonction de d\'emarrage automatique de Bacula, vous pouvez, au choix, lancer les 
trois {\it daemons} lors du d\'emarrage, ou au contraire les lancer individuellement 
avec les scripts  {\bf bacula-dir}, {\bf bacula-fd}, et {\bf bacula-sd} usuellement 
situ\'es dans {\bf /etc/init.d}, bien que leur localisation effective soit d\'ependante du syst\`eme 
d'exploitation.

Notez que seul le File Daemon a \'et\'e port\'e sur les syst\`emes Windows, et qu'il doit \^etre 
d\'emarr\'e diff\'eramment. Veuillez consulter le chapitre  
\ilink{La version Windows de Bacula}{_ChapterStart7} de ce manuel.

Les paquetages rpm configurent les {\it daemons} pour qu'ils s'ex\'ecutent en tant 
qu'utilisateur root et en tant que groupe bacula. Le processus d'installation rpm 
se charge de cr\'eer le groupe bacula s'il n'existe pas sur le syst\`eme. Tout utilisateur 
ajout\'e au groupe bacula h\'erite de l'acc\`es aux fichiers cr\'e\'es par les {\it daemons}. Pour 
modifier ce comportement, \'editez les scripts de d\'emarrage des {\it daemons} :

\begin{itemize}
  \item /etc/bacula/bacula 
  \item /etc/init.d/bacula-dir 
  \item /etc/init.d/bacula-sd 
  \item /etc/init.d/bacula-fd 
\end{itemize}

puis red\'emarrez-les.

Le chapitre 
\ilink{installation}{_ChapterStart17} de ce manuel indique comment installer 
les scripts de d\'emarrage automatique des {\it daemons}.

\section{Interagir avec le Director pour l'interroger sur l'\'etat de Bacula ou lancer des jobs}
\index[general]{Jobs!Interagir avec le Director pour interroger l'\'etat de Bacula ou lancer des}
\index[general]{Interagir avec le Director pour interroger l'\'etat de Bacula ou lancer des jobs}
\addcontentsline{toc}{section}{Interagir avec le Director pour interroger l'\'etat de Bacula ou lancer des jobs}

Pour communiquer avec le Director et pour s'enqu\'erir de l'\'etat de Bacula ou de jobs en 
cours d'ex\'ecution, tapez simplement :

./bconsole

dans le r\'epertoire de plus haut niveau.

Si vous avez install\'e la console GNOME et utilis\'e l'option {\bf \verb:--:enable-gnome} 
de la commande configure, vous pouvez aussi utiliser la console GNOME en tapant : 

./gnome-console

Vous pouvez aussi utiliser le programme wxWidgets {\bf bwx-console}.

Pour simplifier, nous ne d\'ecrirons ici que le programme {\bf ./bconsole}. La plus 
grande partie de ce qui est d\'ecrit ici s'applique aussi aux programmes {\bf ./gnome-console} 
et {\bf bwx-console}.

La commande {\bf ./bconsole} lance le programme Console, qui se connecte au Director. 
Bacula \'etant un programme r\'eseau, vous pouvez utiliser la Console depuis n'importe quelle 
machine de votre r\'eseau. Cependant, la plupart du temps le Console est ex\'ecut\'ee sur la 
m\^eme machine que le Director. En principe, la Console devrait produire un affichage 
similaire \`a :  

\footnotesize
\begin{verbatim}
[kern@polymatou bin]$ ./bconsole
Connecting to Director lpmatou:9101
1000 OK: HeadMan Version: 1.30 (28 April 2003)
*
\end{verbatim}
\normalsize

L'ast\'erisque est l'invite de commande de la console.

Tapez {\bf help} pour obtenir la liste des commandes disponibles :

\footnotesize
\begin{verbatim}
*help
  Command    Description
  =======    ===========
  add        add media to a pool
  autodisplay autodisplay [on/off] -- console messages
  automount  automount [on/off] -- after label
  cancel     cancel job=nnn -- cancel a job
  create     create DB Pool from resource
  delete     delete [pool=<pool-name> | media volume=<volume-name>]
  estimate   performs FileSet estimate debug=1 give full listing
  exit       exit = quit
  help       print this command
  label      label a tape
  list       list [pools | jobs | jobtotals | media <pool> |
             files jobid=<nn>]; from catalog
  llist      full or long list like list command
  messages   messages
  mount      mount <storage-name>
  prune      prune expired records from catalog
  purge      purge records from catalog
  query      query catalog
  quit       quit
  relabel    relabel a tape
  release    release <storage-name>
  restore    restore files
  run        run <job-name>
  setdebug   sets debug level
  show       show (resource records) [jobs | pools | ... | all]
  sqlquery   use SQL to query catalog
  status     status [storage | client]=<name>
  time       print current time
  unmount    unmount <storage-name>
  update     update Volume or Pool
  use        use catalog xxx
  var        does variable expansion
  version    print Director version
  wait       wait until no jobs are running
*
\end{verbatim}
\normalsize

Pour plus de d\'etails sur les commandes de la console, consultez le chapitre 
\ilink{Console}{_ConsoleChapter} de ce manuel. 

\section{ex\'ecuter un job}
\label{Running}
\index[general]{Job!ex\'ecuter un}
\index[general]{ex\'ecuter un job}
\addcontentsline{toc}{section}{Ex\'ecuter un job}

A ce stade, nous supposons que vous avez :

\begin{itemize}
\item Configur\'e Bacula avec la commande {\bf ./configure \verb:--:your-options} 
\item Compil\'e Bacula avec la commande {\bf make} 
\item Install\'e Bacula avec la commande {\bf make install} 
\item Cr\'e\'e votre catalogue avec, par exemple, la commande {\bf
   ./create\_sqlite\_database} 
\item Cr\'e\'e les tables du catalogue avec la commande {\bf
   ./make\_bacula\_tables} 
\item Eventuellement \'edit\'e votre fichier  {\bf bacula-dir.conf} pour le personnaliser 
   quelque peu. ATTENTION ! Si vous modifiez le nom du Director ou son mot de passe, 
   vous devez faire les modifications correspondantes dans les autres fichiers de 
   configuration. Il est sans dout pr\'ef\'erable, pour l'instant, de ne rien changer.
\item D\'emarr\'e Bacula avec la commande {\bf ./bacula start}
\item Invoqu\'e le programme Console avec la commande {\bf ./bconsole}.
\end{itemize}

En outre, nous supposons pour le moment que vous utilisez les fichiers de configuration 
par d\'efaut.

Maintenant, entrez les commandes suivantes :

\footnotesize
\begin{verbatim}
show filesets
\end{verbatim}
\normalsize

Vous devriez obtenir quelque chose comme :

\footnotesize
\begin{verbatim}
FileSet: name=Full Set
      O M
      N
      I /home/kern/bacula/regress/build
      N
      E /proc
      E /tmp
      E /.journal
      E /.fsck
      N
FileSet: name=Catalog
      O M
      N
      I /home/kern/bacula/regress/working/bacula.sql
      N
\end{verbatim}
\normalsize

Il s'agit d'un {\bf FileSet} pr\'ed\'efini qui sauvegardera le r\'epertoire des 
sources de Bacula. Les noms de r\'epertoires qui seront r\'eellement affich\'es 
devraient correspondre \`a votre configuration. Dans une perspective de tests, 
nous avons choisi un r\'epertoire de taille et de complexit\'e mod\'er\'ee (environ 
40 Mo). Le FileSet {\bf Catalog} est utilis\'e pour sauvegarder le catalogue 
Bacula et nous ne nous y attarderons pas pour le moment. Les entr\'ees {\bf I} 
sont les fichiers ou r\'epertoires qui seront inclus dans la sauvegarde, tandis 
que les entr\'ees  {\bf E} sont ceux qui en seront exclus, quand aux entr\'ees {\bf O}, 
ce sont les options sp\'ecifi\'ees pour ce FileSet. Vous pouvez changer ce qui est 
sauvegard\'e en modifiant la ligne {\bf File =} de la ressource {\bf FileSet}.


Il est maintenant temps de lancer votre premi\`ere sauvegarde. Nous allons 
sauvegarder votre r\'epertoire sources de Bacula vers un volume File dans votre 
r\'epertoire {\bf /tmp} afin de vous montrer combien c'est facile. Saisissez :  

\footnotesize
\begin{verbatim}
status dir
\end{verbatim}
\normalsize

Vous devriez obtenir :

\footnotesize
\begin{verbatim}
rufus-dir Version: 1.30 (28 April 2003)
Daemon started 28-Apr-2003 14:03, 0 Jobs run.
Console connected at 28-Apr-2003 14:03
No jobs are running.
Level          Type     Scheduled          Name
=================================================================
Incremental    Backup   29-Apr-2003 01:05  Client1
Full           Backup   29-Apr-2003 01:10  BackupCatalog
====
\end{verbatim}
\normalsize

O\`u les dates et le nom du Director seront diff\'erents et en accord avec votre 
installation. Ceci montre qu'une sauvegarde incr\'ementale est planifi\'ee pour 
le job {\bf Client1} \`a 1h05, et qu'une sauvegarde full est planifi\'ee pour 
le job {\bf BackupCatalog} \`a 1h10. Vous devriez remplacer le nom {\bf Client1} 
par celui de votre machine, sinon vous risquez la confusion lorsque vous 
ajouterez de nouveaux clients. Pour ma machine r\'eelle, j'utilise {\bf Rufus} 
plut\^ot que {\bf Client1}.

A pr\'esent, tapez :

\footnotesize
\begin{verbatim}
status client
\end{verbatim}
\normalsize

Vous devriez obtenir :

\footnotesize
\begin{verbatim}
The defined Client resources are:
     1: rufus-fd
Item 1 selected automatically.
Connecting to Client rufus-fd at rufus:8102
rufus-fd Version: 1.30 (28 April 2003)
Daemon started 28-Apr-2003 14:03, 0 Jobs run.
Director connected at: 28-Apr-2003 14:14
No jobs running.
====
\end{verbatim}
\normalsize

Dans ce cas, le client se nomme {\bf rufus-fd}, votre nom sera diff\'erent, mais la 
ligne qui d\'ebute par {\bf rufus-fd Version...} est produite par votre File Daemon, 
nous sommes donc maintenant surs qu'il fonctionne.

Finalement, faites de m\^eme pour votre Storage Daemon :

\footnotesize
\begin{verbatim}
status storage
\end{verbatim}
\normalsize

Vous devriez obtenir :

\footnotesize
\begin{verbatim}
The defined Storage resources are:
     1: File
Item 1 selected automatically.
Connecting to Storage daemon File at rufus:8103
rufus-sd Version: 1.30 (28 April 2003)
Daemon started 28-Apr-2003 14:03, 0 Jobs run.
Device /tmp is not open.
No jobs running.
====
\end{verbatim}
\normalsize

Vous noterez que le p\'eriph\'erique du Storage Daemon par d\'efaut est nomm\'e {\bf File} 
et qu'il utilise le p\'eriph\'erique {\bf /tmp}, qui n'est actuellement pas ouvert. 

Maintenant, lancez un job :

\footnotesize
\begin{verbatim}
run
\end{verbatim}
\normalsize
 
Vous devriez obtenir :

\footnotesize
\begin{verbatim}
Using default Catalog name=MyCatalog DB=bacula
A job name must be specified.
The defined Job resources are:
     1: Client1
     2: BackupCatalog
     3: RestoreFiles
Select Job resource (1-3):
\end{verbatim}
\normalsize

Ici, Bacula affiche la liste des trois diff\'erents jobs que vous pouvez ex\'ecuter. 
Choisissez le num\'ero {\bf 1} et validez (entr\'ee).

Vous devriez obtenir :

\footnotesize
\begin{verbatim}
Run Backup job
JobName:  Client1
FileSet:  Full Set
Level:    Incremental
Client:   rufus-fd
Storage:  File
Pool:     Default
When:     2003-04-28 14:18:57
OK to run? (yes/mod/no):
\end{verbatim}
\normalsize

Prenez un peu de temps pour examiner cet affichage et le comprendre. Il vous 
est demand\'e de valider, modifier ou annuler l'ex\'ecution d'un job nomm\'e 
{\bf Client1} avec le FileSet {\bf Full Set} que nous avons affich\'e plus haut 
en incr\'emental sur votre client rufus, utilisant le p\'eriph\'erique de stockage 
{\bf File} et le pool {\bf Default} \`a la date indiqu\'ee sur la ligne "When".

Nous avons le choix de valider ({\bf yes}), modifier un ou plusieurs des 
param\`etres ci-dessus ({\bf mod}), ou de ne pas ex\'ecuter le job ({\bf no}). 

Validez l'ex\'ecution du job ({\bf yes}), vous devriez imm\'ediatement obtenir 
l'invite de commande de la console (un ast\'erisque). Apr\`es quelques minutes, 
la commande {\bf messages} devrait produire un r\'esultat tel que :

\footnotesize
\begin{verbatim}
28-Apr-2003 14:22 rufus-dir: Last FULL backup time not found. Doing
                  FULL backup.
28-Apr-2003 14:22 rufus-dir: Start Backup JobId 1,
                  Job=Client1.2003-04-28_14.22.33
28-Apr-2003 14:22 rufus-sd: Job Client1.2003-04-28_14.22.33 waiting.
                  Cannot find any appendable volumes.
Please use the "label"  command to create a new Volume for:
    Storage:      FileStorage
    Media type:   File
    Pool:         Default
\end{verbatim}
\normalsize

Le premier message signale qu'aucune sauvegarde full n'a jamais \'et\'e faite, et 
que par cons\'equent Bacula \'el\`eve votre incr\'ementale en une Full (ce comportement 
est normal). Le second message indique que le job a d\'emarr\'e avec le JobId 1 et le 
troisi\`eme message vous informe que Bacula ne peut trouver aucun volume dans le 
pool Default sur lequel \'ecrire les donn\'ees du job. Ceci est normal, car nous 
n'avons encore cr\'e\'e (ou \'etiquet\'e) aucun volume. Bacula vous fournit tous les d\'etails 
concernant le volume dont il a besoin. 

A ce point, le job est bloqu\'e en attente d'un volume. Vous pouvez le v\'erifier 
en utilisant la commande {\bf status dir}. Pour continuer, vous devez cr\'eer un 
volume sur lequel Bacula pourra \'ecrire. Voici la manipulation :

\footnotesize
\begin{verbatim}
label
\end{verbatim}
\normalsize

Bacula devrait afficher :

\footnotesize
\begin{verbatim}
The defined Storage resources are:
     1: File
Item 1 selected automatically.
Enter new Volume name:
\end{verbatim}
\normalsize

Entrez un nom commen\c {c}ant par une lettre et ne contenant que des chiffres et des lettres 
(p\'eriodes, tirets et soulign\'e "\_" sont aussi autoris\'es). Par exemple entrez {\bf TestVolume001}, 
vous devriez obtenir : 

\footnotesize
\begin{verbatim}
Defined Pools:
     1: Default
Item 1 selected automatically.
Connecting to Storage daemon File at rufus:8103 ...
Sending label command for Volume "TestVolume001" Slot 0 ...
3000 OK label. Volume=TestVolume001 Device=/tmp
Catalog record for Volume "TestVolume002", Slot 0  successfully created.
Requesting mount FileStorage ...
3001 OK mount. Device=/tmp
\end{verbatim}
\normalsize

Finalement, tapez la commande {\bf messages}, vous devriez obtenir quelque chose comme :

\footnotesize
\begin{verbatim}
28-Apr-2003 14:30 rufus-sd: Wrote label to prelabeled Volume
   "TestVolume001" on device /tmp
28-Apr-2003 14:30 rufus-dir: Bacula 1.30 (28Apr03): 28-Apr-2003 14:30
JobId:                  1
Job:                    Client1.2003-04-28_14.22.33
FileSet:                Full Set
Backup Level:           Full
Client:                 rufus-fd
Start time:             28-Apr-2003 14:22
End time:               28-Apr-2003 14:30
Files Written:          1,444
Bytes Written:          38,988,877
Rate:                   81.2 KB/s
Software Compression:   None
Volume names(s):        TestVolume001
Volume Session Id:      1
Volume Session Time:    1051531381
Last Volume Bytes:      39,072,359
FD termination status:  OK
SD termination status:  OK
Termination:            Backup OK
28-Apr-2003 14:30 rufus-dir: Begin pruning Jobs.
28-Apr-2003 14:30 rufus-dir: No Jobs found to prune.
28-Apr-2003 14:30 rufus-dir: Begin pruning Files.
28-Apr-2003 14:30 rufus-dir: No Files found to prune.
28-Apr-2003 14:30 rufus-dir: End auto prune.
\end{verbatim}
\normalsize

Si rien ne se passe dans l'imm\'ediat, vous pouvez continuer de rentrer la 
commande {\bf messages} jusqu'\`a ce que le job se termine, ou utiliser la 
commande {\bf autodisplay on} afin que les messages soient affich\'es d\`es-qu'ils 
sont disponibles.

si vous faites {\bf ls -l} dans votre r\'epertoire {\bf /tmp}, vous verrez 
l'\'el\'ement suivant : 

\footnotesize
\begin{verbatim}
-rw-r-----    1 kern     kern     39072153 Apr 28 14:30 TestVolume001
\end{verbatim}
\normalsize

Il s'agit du volume File que vous venez juste d'\'ecrire, et qui contient toutes 
les donn\'ees du job que vous venez d'ex\'ecuter. Si vous ex\'ecutez d'autres jobs, 
il seront ajout\'es \`a la suite de ce volume, \`a moins que vous n'ayez sp\'ecifi\'e 
un autre comportement.

Vous vous demandez peut-\^etre s'il va vous falloir \'etiqueter vous m\^eme chaque 
volume que Bacula sera amen\'e \`a utiliser. La r\'eponse, en ce qui concerne les 
volumes disque tels que celui que nous avons utilis\'e, est non. Il est possible 
de param\'etrer Bacula pour qu'il cr\'e\'ee lui m\^eme les volumes. En revanche, 
pour les volumes de type cartouche, il vous faudra tr\`es probablement 
\'etiqueter chaque volume que vous voulez utiliser.

Si vous souhaitez en rester l\`a, saisissez simplement {\bf quit} dans la 
console, puis stoppez Bacula avec {\bf ./bacula stop}. Pour nettoyer 
votre installation des r\'esultats de vos tests, supprimez le fichier 
 {\bf /tmp/TestVolume001}, et r\'einitialisez votre catalogue en utilisant :

\footnotesize
\begin{verbatim}
./drop_bacula_tables
./make_bacula_tables
\end{verbatim}
\normalsize

Notez bien que ceci supprimera toutes les informations concernant les jobs pr\'ec\'edemment 
ex\'ecut\'es et que, si c'est sans doute ce que vous souhaitez faire en fin de phase de 
test, ce n'est g\'en\'eralement pas une op\'eration souhaitable en utilisation normale.

Si vous souhaitez essayer de restaurer les fichiers que vous venez de sauvegarder, 
lisez la section suivante.

\label{restoring}

\section{Restaurer vos fichiers}
\index[general]{Fichiers!Restaurer vos}
\index[general]{Restaurer vos fichiers}
\addcontentsline{toc}{section}{Restaurer vos fichiers}

Si vous avez utilis\'e la configuration par d\'efaut et sauvegard\'e les sources de Bacula 
comme dans la d\'emonstration ci-dessus, vous pouvez restaurer les fichiers sauvegard\'es 
en saisissant les commandes suivantes dans la Console :  

\footnotesize
\begin{verbatim}
restore all
\end{verbatim}
\normalsize

Vous obtiendrez :

\footnotesize
\begin{verbatim}
First you select one or more JobIds that contain files
to be restored. You will be presented several methods
of specifying the JobIds. Then you will be allowed to
select which files from those JobIds are to be restored.

To select the JobIds, you have the following choices:
     1: List last 20 Jobs run
     2: List Jobs where a given File is saved
     3: Enter list of comma separated JobIds to select
     4: Enter SQL list command
     5: Select the most recent backup for a client
     6: Select backup for a client before a specified time
     7: Enter a list of files to restore
     8: Enter a list of files to restore before a specified time
     9: Find the JobIds of the most recent backup for a client
    10: Find the JobIds for a backup for a client before a specified time
    11: Enter a list of directories to restore for found JobIds
    12: Cancel
Select item:  (1-12): 
\end{verbatim}
\normalsize

Comme vous pouvez le constater, les options sont nombreuses, mais pour l'instant, 
choisissez l'option {\bf 5} afin de s\'electionner la derni\`ere sauvegarde effectu\'ee. 
Vous obtiendrez :  

\footnotesize
\begin{verbatim}
Defined Clients:
     1: rufus-fd
Item 1 selected automatically.
The defined FileSet resources are:
     1: 1  Full Set  2003-04-28 14:22:33
Item 1 selected automatically.
+-------+-------+----------+---------------------+---------------+
| JobId | Level | JobFiles | StartTime           | VolumeName    |
+-------+-------+----------+---------------------+---------------+
| 1     | F     | 1444     | 2003-04-28 14:22:33 | TestVolume002 |
+-------+-------+----------+---------------------+---------------+
You have selected the following JobId: 1
Building directory tree for JobId 1 ...
1 Job inserted into the tree and marked for extraction.
The defined Storage resources are:
     1: File
Item 1 selected automatically.
You are now entering file selection mode where you add and
remove files to be restored. All files are initially added.
Enter "done" to leave this mode.
cwd is: /
$
\end{verbatim}
\normalsize

(J'ai tronqu\'e l'affichage \`a droite par soucis de lisibilit\'e.)
Comme vous pouvez le constater au d\'ebut de cet affichage, Bacula conna\^it 
vos clients, et puisque vous n'en avez qu'un, il est automatiquement 
s\'electionn\'e. Il en va de m\^eme pour le FileSet. Bacula produit alors une 
liste de tous les jobs qui constituent la sauvegarde courante. Dans le cas 
pr\'esent, il n'y en a qu'un. Notez que le Storage Daemon est aussi 
s\'electionn\'e automatiquement. Bacula est maintenant en mesure de produire 
une {\bf arborescence} \`a partir de tous les fichiers qui ont \'et\'e 
sauvegard\'es (il s'agit d'une repr\'esentation en m\'emoire de votre syst\`eme de 
fichiers). A ce stade, vous pouvez utiliser les commandes {\bf cd },  {\bf ls} 
et {\bf dir} pour naviguer dans l'arborescence et voir quels fichiers 
peuvent \^etre restaur\'es. Par exemple, si je saisis {\bf cd /home/kern/bacula/bacula-1.30} 
suivi de {\bf dir}, j'obtiens la liste de tous les fichiers du r\'epertoire source de 
Bacula. Pour plus d'information sur ce sujet, veuillez consulter le chapitre 
\ilink{La commande Restore}{_ChapterStart13}.

Pour quitter, tapez simplement :

\footnotesize
\begin{verbatim}
done
\end{verbatim}
\normalsize

Vous obtiendrez :

\footnotesize
\begin{verbatim}
Bootstrap records written to
   /home/kern/bacula/testbin/working/restore.bsr
The restore job will require the following Volumes:
   
   TestVolume001
1444 files selected to restore.
Run Restore job
JobName:    RestoreFiles
Bootstrap:  /home/kern/bacula/testbin/working/restore.bsr
Where:      /tmp/bacula-restores
Replace:    always
FileSet:    Full Set
Client:     rufus-fd
Storage:    File
JobId:      *None*
When:       2005-04-28 14:53:54
OK to run? (yes/mod/no):
\end{verbatim}
\normalsize

Si vous acceptez ({\bf yes}), vos fichiers seront restaur\'es vers le r\'epertoire 
{\bf /tmp/bacula-restores}. Si vous pr\'ef\'erez restaurer les fichiers \`a leurs 
emplacements d'origine, vous devez utiliser l'option {\bf mod} et r\'egler 
explicitement le param\`etre {\bf Where} \`a vide ou "/". Nous vous conseillons de 
poursuivre avec  {\bf yes}. Apr\`es quelques instants, la commande {\bf messages} 
devrait produire la liste des fichiers restaur\'es, ainsi qu'un r\'esum\'e du job 
qui devrait ressembler \`a ceci :

\footnotesize
\begin{verbatim}
28-Apr-2005 14:56 rufus-dir: Bacula 1.30 (28Apr03): 28-Apr-2003 14:56
JobId:                  2
Job:                    RestoreFiles.2005-04-28_14.56.06
Client:                 rufus-fd
Start time:             28-Apr-2005 14:56
End time:               28-Apr-2005 14:56
Files Restored:         1,444
Bytes Restored:         38,816,381
Rate:                   9704.1 KB/s
FD termination status:  OK
Termination:            Restore OK
28-Apr-2005 14:56 rufus-dir: Begin pruning Jobs.
28-Apr-2005 14:56 rufus-dir: No Jobs found to prune.
28-Apr-2005 14:56 rufus-dir: Begin pruning Files.
28-Apr-2005 14:56 rufus-dir: No Files found to prune.
28-Apr-2005 14:56 rufus-dir: End auto prune.
\end{verbatim}
\normalsize

Apr\`es avoir quitt\'e la Console, vous pouvez examiner les fichiers dans le 
r\'epertoire {\bf /tmp/bacula-restores}, il contient l'arborescence avec tous 
vos fichiers. Supprimez-le apr\`es avoir v\'erifi\'e :

\footnotesize
\begin{verbatim}
rm -rf /tmp/bacula-restore
\end{verbatim}
\normalsize

\section{Quitter le programme Console}
\index[general]{Programme!Quitter Console }
\index[general]{Quitter le programme Console}
\addcontentsline{toc}{section}{Quitter le programme Console}

Saisissez simplement la commande {\bf quit}. 
\label{SecondClient}

\section{Ajouter un client}
\index[general]{Client!Ajouter }
\index[general]{Ajouter un client }
\addcontentsline{toc}{section}{Ajouter un client}

Si vous \^etes parvenus \`a faire fonctionner tous les exemples ci-dessus, vous \^etes 
sans doute pr\`et \`a ajouter un nouveau client (File Daemon), c'est \`a dire une seconde 
machine que vous souhaitez sauvegarder. La seule chose \`a installer sur la nouvelle 
machine est le binaire {\bf bacula-fd} (ou {\bf bacula-fd.exe} pour Windows) et son 
fichier de configuration {\bf bacula-fd.conf}. Vous pouvez d\'emarrer en copiant le fichier 
pr\'ec\'edemment cr\'e\'e moyennant une modification mineure pour l'adapter au nouveau client : 
changez le nom de File Daemon ({\bf rufus-fd} dans l'exemple ci-dessus) en le nom 
que vous avez choisi pour le nouveau client. Le mieux est d'utiliser le nom de 
la machine. Par exemple :

\footnotesize
\begin{verbatim}
...
#
# "Global" File daemon configuration specifications
#
FileDaemon {                          # this is me
  Name = rufus-fd
  FDport = 9102                  # where we listen for the director
  WorkingDirectory = /home/kern/bacula/working
  Pid Directory = /var/run
}
...
\end{verbatim}
\normalsize

devient :

\footnotesize
\begin{verbatim}
...
#
# "Global" File daemon configuration specifications
#
FileDaemon {                          # this is me
  Name = matou-fd
  FDport = 9102                  # where we listen for the director
  WorkingDirectory = /home/kern/bacula/working
  Pid Directory = /var/run
}
...
\end{verbatim}
\normalsize

O\`u {\bf rufus-fd} est devenu {\bf matou-fd} (je ne montre qu'une partie du fichier). 
Le choix des noms vous appartient. Pour l'instant, je vous recommande de ne rien changer 
d'autre. Plus tard, vous changerez le mot de passe.

Installez cette configuration sur votre seconde machine. Il vous faut maintenant 
ajouter quelques lignes \`a votre {\bf bacula-dir.conf} pour d\'efinir le nouveau 
File Daemon. En vous basant sur l'exemple initial qui devrait \^etre install\'e 
sur votre syst\`eme, ajoutez les lignes suivantes (essentiellement, une copie des lignes 
existantes avec seulement les noms modifi\'es) \`a votre {\bf bacula-dir.conf} :

\footnotesize
\begin{verbatim}
#
# Define the main nightly save backup job
#   By default, this job will back up to disk in /tmp
Job {
  Name = "Matou"
  Type = Backup
  Client = matou-fd
  FileSet = "Full Set"
  Schedule = "WeeklyCycle"
  Storage = File
  Messages = Standard
  Pool = Default
  Write Bootstrap = "/home/kern/bacula/working/matou.bsr"
}
# Client (File Services) to backup
Client {
  Name = matou-fd
  Address = matou
  FDPort = 9102
  Catalog = MyCatalog
  Password = "xxxxx"                  # password for
  File Retention = 30d                # 30 days
  Job Retention = 180d                # six months
  AutoPrune = yes                     # Prune expired Jobs/Files
}
\end{verbatim}
\normalsize

Assurez-vous que le param\`etre Address de la ressource Storage a pour valeur 
le nom pleinement qualifi\'e et non quelque chose comme "localhost". L'adresse 
sp\'ecifi\'ee est envoy\'ee au client et doit \^etre un nom pleinement qualifi\'e. Si vous 
utilisez "localhost", l'adresse du Storage Daemon ne sera pas r\'esolue 
correctement, il en r\'esultera un {\it timeout} lorsque le File Daemon 
\'echouera \`a connecter le Storage Daemon.

Il n'y a rien d'autre \`a faire. J'ai copi\'e les ressources existantes pour cr\'eer 
un second job (Matou) pour sauvegarder le second client (matou-fd). le client 
se nomme {\bf matou-fd} et le job {\bf Matou}, le fichier bootstrap est modifi\'e 
mais tout le reste est inchang\'e. Ceci signifie que Matou sera sauvegard\'e 
avec la m\^eme planification sur les m\^emes cartouches. Vous pourrez changer ceci 
plus tard, pour le moment, restons simples.

La seconde modification consiste en l'ajout d'une nouvelle ressource Client 
qui d\'efinit {\bf matou-fd} et qui a l'adresse correcte {\bf matou} (mais dans 
la vraie vie, vous pouvez avoir besoin d'un nom pleinement qualifi\'e ou d'une 
adresse IP. J'ai aussi conserv\'e le m\^eme mot de passe (xxxxx dans l'exemple).

A ce stade, il suffit de red\'emarrer Bacula pour qu'il prenne en compte vos 
modifications. L'invite que vous avez vu plus haut devrait maintenant 
inclure la nouvelle machine.

Pour une utilisation en production vous voudrez probablement utiliser 
plusieurs pools et diff\'erentes planifications. Il vous appartient de faire les 
adaptations qui seyent \`a vos besoins. Dans tous les cas, n'oubliez pas de 
changer les mots de passe dans les fichiers de configuration du Director et 
du Client pour des raisons de s\'ecurit\'e.

Vous trouverez des astuces importantes concernant le changement des noms et mots de 
passe, ainsi qu'un diagramme d\'ecrivant leurs correspondances dans la section 
\ilink{Erreurs d'authentification}{AuthorizationErrors} du chapitre FAQ de ce manuel.

\section{Lorsque la cartouche est pleine}
\label{FullTape}
\index[general]{pleine!Lorsque la cartouche }
\index[general]{Lorsque la cartouche est pleine}
\addcontentsline{toc}{section}{Lorsque la cartouche est pleine}
Si vous avez planifi\'e votre job, il viendra un moment o\`u la cartouche sera pleine 
et o\`u {\bf Bacula} ne pourra continuer. Dans ce cas, Bacula vous enverra un message 
tel que :

\begin{verbatim}
rufus-sd: block.c:337 === Write error errno=28: ERR=No space left
          on device
\end{verbatim}
\normalsize

Ceci indique que Bacula a re\c {c}u une erreur d'\'ecriture \`a cause de la carouche pleine.
Bacula va maintenant rechercher une cartouche utilisable dans le pool sp\'ecifi\'e 
pour le job. Dans la situation id\'eale, vous avez r\'egl\'e correctement vos r\'etentions 
et sp\'ecifi\'e que vos cartouches peuvent \^etre recycl\'ees automatiquement. Dans ce cas, 
Bacula recycle automatiquement vos cartouches sorties de r\'etention et est en mesure 
de r\'e\'ecrire dessus. Pour plus d'informations sur le recyclage, veuillez consulter 
le chapitre \ilink{Recyclage}{_ChapterStart22} de ce manuel. Si vous constatez que 
vos cartouches ne sont pas recycl\'ees correctement, consultez la section sur le 
\ilink{Recyclage manuel}{manualrecycling} du chapitre Recyclage.

Si comme moi, vous avez un tr\`es grand nombre de cartouches que vous \'etiquetez 
avec la date de premi\`ere \'ecriture, si vous n'avez pas r\'egl\'e vos p\'eriodes de 
r\'etention, Bacula ne trouvera pas de cartouche dans le pool et il vous enverra 
un message tel que :

\footnotesize
\begin{verbatim}
rufus-sd: Job kernsave.2002-09-19.10:50:48 waiting. Cannot find any
          appendable volumes.
Please use the "label"  command to create a new Volume for:
    Storage:      SDT-10000
    Media type:   DDS-4
    Pool:         Default
\end{verbatim}
\normalsize

Ce message sera r\'ep\'et\'e une heure plus tard, puis deux heures plus tard et 
ainsi de suite en doublant \`a chaque fois l'intervalle \`a concurrence d'un jour 
jusqu'\`a ce que vous cr\'eiez un volume.

Que faire dans cette situation ?

La r\'eponse est simple : d'abord, fermez le lecteur \`a l'aide de la commande 
{\bf unmount} du programme Console. Si vous n'avez qu'un lecteur, il sera 
s\'electionn\'e automatiquement, sinon assurez-vous de d\'emonter celui sp\'ecifi\'e 
dans le message (dans ce cas {\bf STD-10000}).

Ensuite, retirez la cartouche du lecteur et ins\'erez-en une vierge. Notez que 
sur certains lecteurs anciens, il peut \^etre n\'ecessaire d'\'ecrire une marque de 
fin de fichier ({\bf mt \ -f \ /dev/nst0 \ weof}) pour \'eviter que le lecteur 
ne d\'eroule toute la cartouche lorsque Bacula tente de lire le label. (NDT : j'ai un doute, la vo dit : "to prevent the drive from running away when Bacula attempts to read the label.") 

Finalement, utilisez la commande {\bf label} dans la console pour \'ecrire un 
label sur le nouveau volume. la commande {\bf label} va contacter le Storage 
Daemon pour qu'il \'ecrive l'\'etiquette logicielle. Si cette op\'eration se termine 
correctement, le nouveau volume est ajout\'e au pool et la commande {\bf mount} est 
envoy\'ee au Storage Daemon. Voyez les sections pr\'ec\'edentes de ce chapitre pour plus 
de d\'etails sur l'\'etiquetage des cartouches.

Bacula peut maintenant poursuivre le job et continuer d'\'ecrire les donn\'ees 
sauvegard\'ees sur le nouveau volume.

Si Bacula cycle sur un pool de volumes, au lieu du message ci-dessus 
 "Cannot find any appendable volumes.", Bacula peut vous demander de 
monter un volume particulier. Dans ce cas, essayez de le satisfaire. Si, pour 
quelque raison, vous n'avez plus le volume, vous pouvez monter n'importe quel 
autre volume du pool, pourvu qu'il soit utilisable, Bacula l'utilisera. 
La commande  {\bf list volumes} du programme Console permet de d\'eterminer 
les volumes utilisables et ceux qui ne le sont pas.

Si, comme moi, vous avez param\'etr\'e correctement vos p\'eriodes de r\'etention, mais 
n'avez plus aucun volume libre, vous pouvez r\'e-\'etiqueter et r\'e-utiliser un volume 
comme suit :

\begin{itemize}
\item Saisissez {\bf list volumes} dans la console et s\'electionnez le volume le plus 
anciens pour le r\'e-\'etiqueter.
\item Si vos p\'eriodes de r\'etention sont judicieusement choisies, le volume devrait 
avoir le statut {\bf Purged}.
\item Si le statut n'est pas {\bf Purged}, il vous faut purger le catalogue des jobs \'ecrits 
sur ce volume. Ceci peut \^etre fait avec la commande {\bf purge jobs volume} dans 
la console. Si vous avez plusieurs pools, vous serez invit\'e \`a choisir lequel avant 
de devoir saisir le VolumeName (ou MediaId).
\item Enfin, utilisez simplement la commande  {\bf relabel} pour r\'e-\'etiqueter le 
volume.
   \end{itemize}

Pour r\'e-\'etiqueter manuellement le volume, suivez les \'etapes suppl\'ementaire ci-dessous :

\begin{itemize}
\item Effacez le volume du catalogue avec la commande {\bf delete volume} dans la 
console (s\'electionnez le VolumeName ou le MediaId lorsque vous y \^etes invit\'e).
\item Utilisez la commande  {\bf unmount} pour d\'emonter l'ancienne cartouche.
\item R\'e-\'etiquetez physiquement l'ancienne cartouche de sorte qu'elle puisse 
\^etre r\'eutilis\'ee.
\item Ins\'erez l'ancienne cartouche dans le lecteur. 
\item Depuis la ligne de commande, saississez : {\bf mt \ -f \ /dev/st0 \ rewind} et 
{\bf mt \ -f \ /dev/st0 \ weof}, o\`u vous prendrez soin de substituer la cha\^ine d\'esignant 
 votre lecteur \`a {\bf /dev/st0}.  
\item Utilisez la commande {\bf label} dans la console pour \'ecrire une nouvelle 
\'etiquette Bacula sur votre cartouche.
\item Utilisez la commande  {\bf mount}, si ce n'est pas r\'ealis\'e automatiquement, afin 
que Bacula commence \`a utiliser la cartouche fraichement \'etiquet\'ee.
\end{itemize}

\section{D'autres commandes utiles de la console Bacula}
\index[general]{Commands!autres commandes utiles de la console Bacula}
\index[general]{autres commandes utiles de la console Bacula}
\addcontentsline{toc}{section}{D'autres commandes utiles de la console Bacula}

\begin{description}

\item [status dir]
   \index[console]{status dir }
   Affiche un \'etat de tous les jobs en cours d'ex\'ecution ainsi que tous les 
   jobs programm\'es dans les prochine 24 heures

\item [status]
   \index[console]{status }
   Le programme Console vous invite \`a s\'electionner un {\it daemon}, puis 
   il s'enquiert de l'\'etat de ce  {\it daemon}. 

\item [status jobid=nn]
   \index[console]{status jobid }
   Affiche un \'etat du JobId nn s'il est en cours d'ex\'ecution. Le Storage 
   Daemon est aussi contact\'e pour produire un \'etat du job.

\item [list pools]
   \index[console]{list pools }
   Affiche la liste des pools d\'efinis dans le catalogue.

\item [list media]
   \index[console]{list media }
   Affiche la liste des m\'edia d\'efinis dans le catalogue.

\item [list jobs]
   \index[console]{list jobs }
   Affiche la liste de tous les jobs enregistr\'es dans le catalogue et squi ont \'et\'e 
   ex\'ecut\'es.

\item [list jobid=nn]
   \index[console]{list jobid }
   Affiche le JobId nn depuis le catalogue.

\item [list jobtotals]
   \index[console]{list jobtotals }
   Affiche les totaux pour tous le jobs du catalogue.

\item [list files jobid=nn]
   \index[console]{list files jobid }
   Affiche la liste des fichiers sauvegard\'es pour le JobId nn. 

\item [list jobmedia]
   \index[console]{list jobmedia }
   Affiche des informations relatives aux m\'edia utilis\'es pour chaque job ex\'ecut\'e.

\item [messages]
   \index[console]{messages }
   Affiche tous les messages redirig\'es vers la console.

\item [unmount storage=storage-name]
   \index[console]{unmount storage }
   D\'emonte le lecteur associ\'e au p\'eriph\'erique de stockage d\'esign\'e par 
   {\bf storage-name} s'il n'est pas en cours d'utilisation. Cette commande 
   est utile si vous souhaitez que Bacula lib\`ere le lecteur. 

\item [mount storage=storage-name]
   \index[sd]{mount storage }
   Le lecteur associ\'e au p\'eriph\'erique de stockage est mont\'e \`a nouveau. Lorsque 
   Bacula atteint la fin d'un volume et vous demande d'en monter un nouveau, 
   vous devez utiliser cette commande apr\`es avoir introduit une nouvelle 
   cartouche dans le lecteur. En effet, c'est le signal qui indique \`a Bacula 
   qu'il peut commencer \`a lire ou \'ecrire sur la cartouche.

\item [quit]
   \index[sd]{quit }
   Permet de quitter le programme Console.
\end{description}

La plupart des commandes cit\'ees ci-dessus, \`a l'exception de  {\bf list}, 
vous invitent \`a compl\'eter la liste des arguments fournis si vous 
vous contentez d'entrer le nom de la commande.

\section{D\'ebugger la sortie des daemons}
\index[general]{D\'ebugger sortie daemons}
\index[general]{Output!D\'ebugger daemons}
\addcontentsline{toc}{section}{D\'ebugger la sortie des daemons}

Si vous voulez d\'ebugger la sortie des {\it daemons} en cours d'ex\'ecution, 
lancez-les, depuis le r\'epertoire d'installation, comme suit :

\footnotesize
\begin{verbatim}
./bacula start -d100
\end{verbatim}
\normalsize

Cette possibilit\'e peut vous fournir une aide pr\'ecieuse si vos {\it daemons} 
ne d\'emarrent pas correctement. Normalement, la sortie des {\it daemons} est 
dirig\'ee vers le p\'eriph\'erique NULL, avec un niveau de d\'ebuggage sup\'erieur \`a 
z\'ero, elle est dirig\'ee vers le terminal de lancement.   

Pour stopper les trois {\it daemons}, tapez simplement :

\footnotesize
\begin{verbatim}
./bacula stop
\end{verbatim}
\normalsize

dans le r\'epertoire d'installation.

L'ex\'ecution de {\bf bacula stop} peut signaler des pids non trouv\'es. C'est Ok, 
sp\'ecialement si l'un des {\bf bacula stop} est mort, ce qui est tr\`es rare.

Pour faire une sauvegarde compl\`ete (Full) du syst\`eme, chaque File Daemon doit 
\^etre ex\'ecut\'e en tant que root afin d'avoir les permissions requises pour acc\'eder 
\`a tous les fichiers. Les autres {\it daemons} n'ont pas besoin des privil\`eges 
root. Cependant, le Storage Daemon doit \^etre capable d'acc\'eder aux lecteurs, ce qui  
Sur beaucoup de syst\`emes, n'est possible que pour root. Vous pouvez, au choix, 
ex\'ecuter le Storage Daemon en tant que root, ou changer les permissions sur les 
lecteurs pour autoriser les acc\`es non-root. MySQL et PostgreSQL peuvent \^etre 
install\'es et ex\'ecut\'es avec un userid quelconque, les privil\`eges root ne sont pas 
requis. 

\section{Soyez patient lorsque vous d\'emarrez les {\it daemons} ou montez des 
cartouches vierges}
\index[general]{Soyez patient lorsque vous d\'emarrez les {\it daemons} ou montez des 
cartouches vierges}
\index[general]{Cartouches!Soyez patient lorsque vous d\'emarrez les {\it daemon}s ou montez}
\addcontentsline{toc}{section}{Soyez patient lorsque vous d\'emarrez les {\it daemon}s 
ou montez des cartouches vierges}

Lorsque vous lancez les {\it daemons} Bacula, le Storage Daemon tente d'ouvrir 
tous les p\'eriph\'eriques de stockage d\'efinis et de v\'erifier le volumes courrament 
mont\'es. Il n'accepte aucune connection de la console tant que tous les p\'eriph\'eriques 
n'ont pas \'et\'e v\'erifi\'es. Une cartouche qui a \'et\'e utilis\'e pr\'ec\'edemment doit \^etre 
rembobin\'ee, ce qui, sur certain lecteurs, peut prendre plusieurs minutes. 
Par cons\'equent, vous devriez faire preuve d'un peu de patience lorsue vous 
tentez de contacter le Storage Daemon pour la premi\`ere fois apr\`es le 
lancement de Bacula. Si vous avez un acc\`es visuel \`a votre lecteur, celui-ci 
devrait \^etre pr\`et \`a l'emploi lorsque son t\'emoin lumineux cesse de clignoter. 

Les m\^emes consid\'erations s'appliquent si vous avez mont\'e une cartouche vierge 
dans un lecteur tels qu'un HP DLT. Il peut s'\'ecouler une \`a deux minutes avant 
que le lecteur se rende compte que la cartouche est vierge. Si vous tentez 
de la monter pendant cette p\'eriode, il est probable que vous aller geler votre 
pilote SCSI (c'est le cas sur mon syst\`eme RedHat). Par cons\'equent, nous vous 
enjoignons une fois encore \`a \^etre patient lors de l'insertion de cartouches vierges. 
Laissez le lecteur s'initialiser avant de tenter d'y acc\'eder.

\section{Probl\`emes de connection du FD vers le SD}
\index[general]{Probl\`emes de connection du FD vers le SD }
\index[general]{SD!Probl\`emes de connection du FD vers le}
\addcontentsline{toc}{section}{Probl\`emes de connection du FD vers le SD}

Si l'un ou plusieurs de vos File Daemons rencontre des difficult\'es \`a se connecter 
au Storage Daemon, c'est tr\`es probablement que vous n'avez pas utilis\'e un nom 
pleinement qualifi\'e pour la directive {\bf Address} de la ressource Storage 
du fichier de configuration du Director. Le r\'esolveur de la machine cliente 
(celle qui ex\'ecute le FD) doit \^etre capable de r\'esoudre le nom que vous avez 
sp\'ecifi\'e dans cette directive en une adresse IP. Un exemple d'adresse ne 
fonctionnant pas est {\bf localhost}. Un exemple qui pourrait fonctionner : 
{\bf megalon}. Un exemple qui a encore plus de chances de fonctionner : 
{\bf magalon.mydomain.com}. Sur les syst\`emes Win32, si vous ne disposez pas d'un 
bon r\'esolveur (c'est souvent le cas sur Win98), vous pouvez essayer en utilisant 
une adresse IP plut\^ot qu'un nom.

Si votre adresse est correcte, assurez vous qu'aucun autre programme n'utilise 
le port 9103 sur la machine qui h\'eberge le Storage Daemon. Les num\'eros de ports 
de Bacula sont autoris\'es par l'IANA, et ne devraient donc pas \^etre utilis\'es par 
d'autres programmes, mais il semble que certaines imprimantes HP les utilisent.
Ex\'ecutez la commande {\bf netstat -a} sur la machine qui h\'eberge le Storage 
Daemon pour d\'eterminer qui utilise le port 9103 (utilis\'e pour les communications 
du FD vers le SD).

\section{Options en ligne de commande des Daemons}
\index[general]{Options en ligne de commande des Daemons}
\index[general]{Options!en ligne de commande des Daemons}
\addcontentsline{toc}{section}{Options en ligne de commande des Daemons}

Chacun des trois {\it daemons} (Director, File, Storage) acceptent quelques options 
sur la ligne de commande. En g\'en\'eral, chacun d'entre eux, de m\^eme que le 
programme Console, admet les otpions suivantes :


\begin{description}

\item [-c \lt{}file\gt{}]
   \index[sd]{-c \lt{}file\gt{} }
   D\'efinit le fichier de configuration \`a utiliser. La valeur par d\'efaut est le 
nom du {\it daemon} suivi de {\bf conf}, par exemple {\bf bacula -dir.conf} pour 
le Director, {\bf bacula-fd} pour le File Daemon, et {\bf bacula-sd.conf} pour 
le Storage Daemon.   

\item [-d nn]
   \index[sd]{-d nn }
   Fixe le niveau de d\'ebuggage \`a la valeur {\bf nn}. Les niveaux les plus \'elev\'e 
permettent d'afficher plus d'information sur STDOUT concernant ce que le {\it daemon} est 
en train de faire.

\item [-f]
   Ex\'ecute le {\it daemon} en arri\`ere plan. Cette option est requise pour ex\'ecuter les 
{\it daemon}s avec le debugger.

\item [-s]
   Ne pas capturer les signaux. Cette option est requise pour ex\'ecuter les 
{\it daemon}s avec le debugger.

\item [-t]
   Lire les fichiers de configuration et afficher les messages d'erreur, et quitter 
imm\'ediatement. Tr\`es utile pour tester la syntaxe de nouveaux fichiers de configuration.

\item [-v]
   Mode verbeux. Utile pour rendre les messages d'erreur et d'information plus complets.

\item [-?]
   Affiche la version et la liste des options.
   \end{description}

Le Director a les options sp\'ecifiques suivantes :

\begin{description}

\item [-r \lt{}job\gt{}]
   \index[fd]{-r \lt{}job\gt{} }
   Ex\'ecute le job d\'esign\'e imm\'ediatement. Ceci ne devrait servir qu'\`a des fins 
de d\'ebuggage.
\end{description}

Le File Daemon les options sp\'ecifiques suivantes :

\begin{description}

\item [-i]
   Suppose que le {\it daemon} est appel\'e par {\bf inetd}  ou {\bf xinetd}. Dans ce cas, 
le {\it daemon} suppose qu'une connection est d\'ej\`a \'etablie et qu'elle est pass\'ee en tant que 
STDIN. Le {\it daemon} s'arr\`ete d\`es que la connection se termine.
\end{description}

Le Storage Daemon n'a pas d'options sp\'ecifiques.

Le programme Console n'a pas d'options sp\'ecifiques.

\section{Cr\'eer un Pool}
\label{Pool}
\index[general]{Pool!Cr\'eer un }
\index[general]{Cr\'eer un Pool }
\addcontentsline{toc}{section}{Cr\'eer un Pool}

La cr\'eation de pool est automatique au d\'emarrage de Bacula, aussi si vous 
comprenez d\'ej\`a le concept de pools et leur fonctionnement, vous pouvez passer 
\`a la section suivante.

Lorsque vous ex\'ecutez un job, Bacula doit d\'eterminer quel volume utiliser pour 
sauvegarder le FileSet. Plut\^ot que de sp\'ecifier un volume directement, vous 
sp\'ecifiez l'ensemble de volumes dans lequel vous autorisez Bacula \`a puiser 
lorsqu'il lui faut un volume pour \'ecrire les donn\'ees sauvegard\'ees. D\`es lors, Bacula 
se charge de s\'electionner le premier volume utilisable dans le pool appropri\'e 
au p\'eriph\'erique que vous avez sp\'ecifi\'e pour le job ex\'ecut\'e. Lorsqu'un volume est 
plein, Bacula change son VolStatus de {\bf Append} en {\bf Full}, et utilise le 
volume suivant, et ainsi de de suite. S'il n'y a pas de volume utilisable, 
Bacula envoie un message \`a l'op\'erateur pour r\'eclamer la cr\'eation d'un  
volume appropri\'e.

{\bf Bacula} garde trace des noms de pools, des volumes contenus dans les pools, 
et de plusieurs caract\'eristiques de chacun de ces volumes. 

Lorsque Bacula d\'emarre, il s'assure que toutes les d\'efinitions de ressources Pool 
ont \'et\'e enregistr\'ees dans le catalogue. Vous pouvez le v\'erifier avec la commande :

\footnotesize
\begin{verbatim}
list pools
\end{verbatim}
\normalsize

du programme Console, qui devrait produire quelque chose comme :

\footnotesize
\begin{verbatim}
*list pools
Using default Catalog name=MySQL DB=bacula
+--------+---------+---------+---------+----------+-------------+
| PoolId | Name    | NumVols | MaxVols | PoolType | LabelFormat |
+--------+---------+---------+---------+----------+-------------+
| 1      | Default | 3       | 0       | Backup   | *           |
| 2      | File    | 12      | 12      | Backup   | File        |
+--------+---------+---------+---------+----------+-------------+
*
\end{verbatim}
\normalsize

Si vous tentez de cr\'eer un pool existant, Bacula affiche :

\footnotesize
\begin{verbatim}
Error: Pool Default already exists.
Once created, you may use the {\bf update} command to
modify many of the values in the Pool record.
\end{verbatim}
\normalsize

\label{Labeling}

\section{Etiqueter vos Volumes}
\index[general]{Volumes!Etiqueter vos}
\index[general]{Etiqueter vos Volumes}
\addcontentsline{toc}{section}{Etiqueter vos Volumes}

Bacula exige que chaque volume comporte une \'etiquette (NDT : label) logicielle. 
Il existe plusieurs strat\'egies pour \'etiqueter les volumes. Celle que j'utilise 
consiste \`a les \'etiqueter \`a l'aide du programme Console au fur et \`a mesure qu'ils 
sont requis par Bacula. Ainsi, lorsqu'il a besoin d'un volume qu'il ne trouve pas 
dans son catalogue, Bacula m'envoie un e-mail pour m'enjoindre \`a ajouter un 
volume au pool. J'utilise alors la commande  {\bf label} dans la console pour 
\'etiqueter un nouveau volume et le d\'efinir dans le catalogue, apr\`es quoi Bacula 
est en mesure de l'utiliser. Alternativement, je peux utiliser la commande 
{\bf relabel} pour r\'e-\'etiquter un volume qui n'est plus utilis\'e, pourvu qu'il ait 
le VolStatus {\bf Purged}.

Une autre strat\'egie consiste \`a \'etiqueter un ensemble de volumes, et \`a les 
utiliser au fur et \`a mesure que Bacula les r\'eclame. C'est le plus souvent ce qui 
est fait lorsque vous cyclez sur un groupe de volumes, par exemple avec une 
librairie. Pour plus de d\'etails sur le recyclage, veuillez consulter le 
chapitre  \ilink{Recyclage automatique des volumes}{_ChapterStart22} de ce 
manuel.

Si vous ex\'ecutez un job Bacula alors que vous n'avez pas de volumes 
\'etiquet\'es dans le pool concern\'e, Bacula vous en informe, et vous pouvez les 
cr\'eer "\`a la vol\'ee". Dans mon cas, j'\'etiquette mes cartouches avec la date, 
par exemple :  {\bf DLT-18April02}. Voyez ci-dessous pour plus de d\'etails 
sur l'usage de la commande {\bf label}.

\section{Etiquetage des volumes dans la console}
\index[general]{Etiquetage des volumes dans la console}
\index[general]{Console!Etiquetage des volumes dans la}
\addcontentsline{toc}{section}{Etiquetage des volumes dans la console}

L'\'etiquetage des volumes se fait, en principe, avec le programme Console.

\begin{enumerate}
\item ./bconsole  
\item label 
   \end{enumerate}

Si Bacula annonce que vous ne pouvez \'etiqueter une cartouche au motif qu'elle 
porte d\'ej\`a une \'etiquette, d\'emontez-la avec la commande {\bf unmount}, puis 
recommencez avec une cartouche vierge.  

Etand donn\'e que le support de stockage physique est diff\'erent pour chaque 
p\'eriph\'erique, la commande {\bf label} vous propose une liste de ressources 
Storage d\'efinies telle que celle-ci :

\footnotesize
\begin{verbatim}
The defined Storage resources are:
     1: File
     2: 8mmDrive
     3: DLTDrive
     4: SDT-10000
Select Storage resource (1-4):
\end{verbatim}
\normalsize

A ce stade, vous devriez avoir une cartouche vierge dans votre lecteur 
d'un type correspondant \`a la ressource Storage que vous avez s\'electionn\'e. 

Bacula vous demande le nom du volume :

\footnotesize
\begin{verbatim}
Enter new Volume name:
\end{verbatim}
\normalsize

S'il proteste :

\footnotesize
\begin{verbatim}
Media record for Volume xxxx already exists.
\end{verbatim}
\normalsize

Cela signifie que le nom de volume {\bf xxxx} que vous avez entr\'e existe d\`ej\`a 
dans le catalogue. Vous pouvez afficher la liste des m\'edia d\'efinis avec la 
commande {\bf list media}. Notez que la colonne LastWritten a ici \'et\'e 
tronqu\'ee pour permettre un affichage propre.

\footnotesize
\begin{verbatim}
+---------------+---------+--------+----------------+-----/~/-+------------+-----+
| VolumeName    | MediaTyp| VolStat| VolBytes       | LastWri | VolReten   | Recy|
+---------------+---------+--------+----------------+---------+------------+-----+
| DLTVol0002    | DLT8000 | Purged | 56,128,042,217 | 2001-10 | 31,536,000 |   0 |
| DLT-07Oct2001 | DLT8000 | Full   | 56,172,030,586 | 2001-11 | 31,536,000 |   0 |
| DLT-08Nov2001 | DLT8000 | Full   | 55,691,684,216 | 2001-12 | 31,536,000 |   0 |
| DLT-01Dec2001 | DLT8000 | Full   | 55,162,215,866 | 2001-12 | 31,536,000 |   0 |
| DLT-28Dec2001 | DLT8000 | Full   | 57,888,007,042 | 2002-01 | 31,536,000 |   0 |
| DLT-20Jan2002 | DLT8000 | Full   | 57,003,507,308 | 2002-02 | 31,536,000 |   0 |
| DLT-16Feb2002 | DLT8000 | Full   | 55,772,630,824 | 2002-03 | 31,536,000 |   0 |
| DLT-12Mar2002 | DLT8000 | Full   | 50,666,320,453 | 1970-01 | 31,536,000 |   0 |
| DLT-27Mar2002 | DLT8000 | Full   | 57,592,952,309 | 2002-04 | 31,536,000 |   0 |
| DLT-15Apr2002 | DLT8000 | Full   | 57,190,864,185 | 2002-05 | 31,536,000 |   0 |
| DLT-04May2002 | DLT8000 | Full   | 60,486,677,724 | 2002-05 | 31,536,000 |   0 |
| DLT-26May02   | DLT8000 | Append |  1,336,699,620 | 2002-05 | 31,536,000 |   1 |
+---------------+---------+--------+----------------+-----/~/-+------------+-----+
\end{verbatim}
\normalsize

Une fois que Bacula a v\'erifi\'e que le volume n'existe pas encore, il vous 
demande le pool dans lequel vous souhaitez que le volume soit cr\'e\'e. S'il 
n'existe qu'un pool, il est s\'electionn\'e automatiquement.

Si la cartouche est \'etiquet\'ee correctement, un enregistrement de volume est 
aussi cr\'e\'e dans le pool. Ainsi, le nom du volume et tous ses attributs 
appara\^itront lorque vous afficherez les volumes du pool. De plus, le volume 
est disponible pour les sauvegardes, pourvu que le MediaType co\"�incide avec 
celui requis par le Storage Daemon.

Lorsque vous avez \'etiquet\'e la cartouche, vous n'avez r\'epondu qu'\`a quelques 
questions la concernant -- principalement son nom, et \'eventuellement le {\it Slot}. 
Cependant, un enregistrement de volume dans le catalogue (connu au niveau interne 
en tant qu'enregistrement Media) contient un certain nombre d'attributs. 
La plupart d'entre eux sont renseign\'es selon les valeurs par d\'efaut qui ont \'et\'e 
d\'efinies lors de la cr\'eation du pool (au trement dit, le pool comporte la plupart des 
attributs par d\'efaut utilis\'es lors de la cr\'eation d'un volume).  

Il est aussi possible d'ajouter des media aux pools sans les \'etiqueter 
physiquement. C'est la fonction de la commande {\bf add}. Pour plus 
d'informations, veuillez consulterle chapitre \ilink{Console}{_ConsoleChapter} 
de ce manuel. 
