%%
%%

\chapter{Caract\'eristiques syst\`eme g\'en\'erales indispensables \`a Bacula}
\label{_ChapterStart51}
\index[general]{Caract\'eristiques syst\`eme g\'en\'erales indispensables \`a Bacula }
\index[general]{Bacula!Caract\'eristiques syst\`eme g\'en\'erales indispensables \`a }
\addcontentsline{toc}{section}{Caract\'eristiques syst\`eme g\'en\'erales indispensables \`a Bacula}

\label{SysReqs}

\section{Caract\'eristiques syst\`eme g\'en\'erales indispensables \`a Bacula}
\index[general]{Caract\'eristiques syst\`eme g\'en\'erales indispensables \`a Bacula }
\index[general]{Bacula!Caract\'eristiques syst\`eme g\'en\'erales indispensables \`a }
\addcontentsline{toc}{section}{Caract\'eristiques syst\`eme g\'en\'erales indispensables \`a Bacula}

\begin{itemize}
\item {\bf Bacula} a \'et\'e compil\'e et ex\'ecut\'e sur  les syst\`emes
   Linux RedHat, Mandriva, SUSE, Debian et Gentoo, sur FreeBSD, et Solaris. 
\item Il requiert GNU C++ version 2.95 ou sup\'erieur pour compiler. Vous 
   pouvez essayer avec d'autres compilateurs et des versions plus anciennes,  mais
   vous serez seuls.  Nous avons compil\'e et utilis\'e avec succ\`es Bacula sur 
RH8.0/RH9/RHEL 3.0 avec GCC 3.2. Note, en g\'en\'eral GNU C++ est un  paquet
s\'epar\'e (e.g. RPM) de GNU C, et vous devrez avoir les deux. Sur les
syst\`emes RedHat, le compilateur C++ fait partie du paquet RPM {\bf
gcc-c++}. 
\item Certains paquets tiers sont n\'ecessaires \`a {\bf Bacula}. 
   Except\'e pour MySQL et PostgreSQL, ils peuvent tous \^etre  trouv\'es dans
   les distributions {\bf depkgs} et {\bf depkgs1}. 
\item Si vous voulez construire les binaires Win32, vous aurez besoin du 
   compilateur Microsoft Visual C++ (ou Visual Studio).  Bien que tous les
   composants compilent (la console produit quelques messages  d'alertes), seul
le File Daemon a \'et\'e  test\'e. 
\item {\bf Bacula} requiert une bonne impl\'ementation fonctionnelle des
   pthreads.  Ce n'est pas le cas sur certains syst\`emes BSD. 
\item Le code source a \'et\'e \'ecrit dans un esprit de  portabilit\'e et est
   le plus souvent compatible POSIX.  Ainsi le portage sur chaque syst\`eme
   d'exploitation  compatible POSIX est relativement ais\'e. 
\item Le programme GNOME Console est developp\'e et test\'e sous GNOME  2.X.
   Il s'ex\'ecute aussi sous GNOME 1.4 mais cette version est  d\'epr\'eci\'ee et
   n'est plus maintenue. 
\item Le programme wxWidgets Console est developp\'e et test\'e avec la 
   derni\`ere version stable de 
   \elink{ wxWidgets}{http://www.wxwidgets.org/} (2.4.2). Il fonctionne  bien
avec la version Windows et GTK+-1.x de wxWidgets, ainsi que  sur les autres
plateformes support\'ees par wxWidgets. 
\item Le programme Tray Monitor est developp\'e pour GTK+-2.x.  Il n\'ecessite
   Gnome \gt{}=2.2, KDE \gt{}=3.1 ou un  gestionnaire de fen\^etre supportant le
   standard 
\elink{systemtray}{http://www.freedesktop.org/Standards/systemtray-spec} de 
FreeDesktop. 
\item Si vous voulez permettre l'\'edition en ligne de  commande et
   l'historique, il vous faudra  /usr/include/termcap.h et l'une des
   biblioth\`eques  termcap ou ncurses charg\'ee (libtermcap-devel ou 
ncurses-devel).
\item Si vous voulez utiliser des DVD en guise de media de sauvegarde, vous devrez 
   t\'el\'echarger les \elink{dvd+rw-tools 5.21.4.10.8}{http://fy.chalmers.se/~appro/linux/DVD+RW/}, 
   appliquer le \elink{patch}{http://cvs.sourceforge.net/viewcvs.py/*checkout*/bacula/bacula/patches/dvd+rw-tools-5.21.4.10.8.bacula.patch} 
   pour rendre ces outils compatibles avec Bacula, puis les compiler et installer. 
   N'utilisez pas les dvd+rw-tools fournis par votre distribution, ils ne 
   fonctionneront pas avec Bacula.

\end{itemize}
