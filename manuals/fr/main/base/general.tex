%%
%%

\chapter{What is Bacula?}
\label{GeneralChapter}
\index[general]{Bacula!What is }
\index[general]{What is Bacula? }

Bacula is a set of computer programs that permits the system
administrator to manage backup, recovery, and verification of computer data
across a network of computers of different kinds. Bacula can also run entirely
upon a single computer and can backup to various types of media, including tape
and disk.

In technical terms, it is a
network Client/Server based backup program. Bacula is relatively easy to use
and efficient, while offering many advanced storage management features that
make it easy to find and recover lost or damaged files. Due to its modular
design, Bacula is scalable from small single computer systems to systems
consisting of hundreds of computers located over a large network. 

\section{Who Needs Bacula?}
\index[general]{Who Needs Bacula? }
\index[general]{Bacula!Who Needs }

If you are currently using a program such as tar, dump, or
bru to backup your computer data, and you would like a network solution, more
flexibility, or catalog services, Bacula will most likely provide the
additional features you want. However, if you are new to Unix systems or do
not have offsetting experience with a sophisticated backup package, the Bacula project does not
recommend using Bacula as it is much more difficult to setup and use than 
tar or dump. 

If you want Bacula to behave like the above mentioned simple
programs and write over any tape that you put in the drive, then you will find
working with Bacula difficult. Bacula is designed to protect your data
following the rules you specify, and this means reusing a tape only
as the last resort. It is possible to "force" Bacula to write
over any tape in the drive, but it is easier and more efficient to use a
simpler program for that kind of operation.

If you would like a backup program that can write
to multiple volumes (i.e. is not limited by your tape drive capacity), Bacula
can most likely fill your needs. In addition, quite a number of Bacula users
report that Bacula is simpler to setup and use than other equivalent programs.

If you are currently using a sophisticated commercial package such as Legato
Networker. ARCserveIT, Arkeia, or PerfectBackup+, you may be interested in
Bacula, which provides many of the same features and is free software
available under the GNU Version 2 software license. 

\section{Bacula Components or Services}
\index[general]{Bacula Components or Services }
\index[general]{Services!Bacula Components or }

Bacula is made up of the following five major components or services:
Director, Console, File, Storage, and Monitor services.


\addcontentsline{lof}{figure}{Bacula Applications}
\includegraphics{\idir bacula-applications.eps} 
(thanks to Aristedes Maniatis for this graphic and the one below) 
% TODO: move the thanks to Credits section in preface

\subsection*{Bacula Director}
   \label{DirDef}
   The Bacula Director service is the program that supervises
   all the backup, restore, verify and archive operations.  The system
   administrator uses the Bacula Director to schedule backups and to
   recover files.  For more details see the Director Services Daemon Design
   Document in the Bacula Developer's Guide.  The Director runs as a daemon
   (or service) in the background.
% TODO: tell reader where this Developer's Guide is at?
   \label{UADef}

\subsection*{Bacula Console}

   The Bacula Console service is the program that allows the
   administrator or user to communicate with the Bacula Director
   Currently, the Bacula Console is available in three versions:
   text-based console interface, QT-based interface, and a
   wxWidgets graphical interface.
   The first and simplest is to run the Console program in a shell window
   (i.e.  TTY interface).  Most system administrators will find this
   completely adequate.  The second version is a GNOME GUI interface that
   is far from complete, but quite functional as it has most the
   capabilities of the shell Console.  The third version is a wxWidgets GUI
   with an interactive file restore.  It also has most of the capabilities
   of the shell console, allows command completion with tabulation, and
   gives you instant help about the command you are typing.  For more
   details see the \ilink{Bacula Console Design Document}{_ConsoleChapter}.

\subsection*{Bacula File}
   \label{FDDef}
   The Bacula File service (also known as the Client program) is the software
   program that is installed on the machine to be backed up.
   It is specific to the
   operating system on which it runs and is responsible for providing the
   file attributes and data when requested by the Director.  The File
   services are also responsible for the file system dependent part of
   restoring the file attributes and data during a recovery operation.  For
   more details see the File Services Daemon Design Document in the Bacula
   Developer's Guide.  This program runs as a daemon on the machine to be
   backed up.
   In addition to Unix/Linux File daemons, there is a Windows File daemon
   (normally distributed in binary format).  The Windows File daemon runs
   on current Windows versions (NT, 2000, XP, 2003, and possibly Me and
   98).
% TODO: maybe do not list Windows here as that is listed elsewhere
% TODO: remove "possibly"?
% TODO: mention Vista?

\subsection*{Bacula Storage}
   \label{SDDef}
   The Bacula Storage services consist of the software programs that
   perform the storage and recovery of the file attributes and data to the
   physical backup media or volumes.  In other words, the Storage daemon is
   responsible for reading and writing your tapes (or other storage media,
   e.g.  files).  For more details see the Storage Services Daemon Design
   Document in the Bacula Developer's Guide.  The Storage services runs as
   a daemon on the machine that has the backup device (usually a tape
   drive).
% TODO: may switch e.g. to "for example" or "such as" as appropriate
% TODO: is "usually" correct? Maybe "such as" instead?

\subsection*{Catalog}
   \label{DBDefinition}
   The Catalog services are comprised of the software programs
   responsible for maintaining the file indexes and volume databases for
   all files backed up.  The Catalog services permit the system
   administrator or user to quickly locate and restore any desired file.
   The Catalog services sets Bacula apart from simple backup programs like
   tar and bru, because the catalog maintains a record of all Volumes used,
   all Jobs run, and all Files saved, permitting efficient restoration and
   Volume management.  Bacula currently supports three different databases,
   MySQL, PostgreSQL, and SQLite, one of which must be chosen when building
   Bacula.

   The three SQL databases currently supported (MySQL, PostgreSQL or
   SQLite) provide quite a number of features, including rapid indexing,
   arbitrary queries, and security.  Although the Bacula project plans to support other
   major SQL databases, the current Bacula implementation interfaces only
   to MySQL, PostgreSQL and SQLite.  For the technical and porting details
   see the Catalog Services Design Document in the developer's documented.

   The packages for MySQL and PostgreSQL are available for several operating
   systems.
   Alternatively, installing from the
   source is quite easy, see the \ilink{ Installing and Configuring
   MySQL}{MySqlChapter} chapter of this document for the details.  For
   more information on MySQL, please see:
   \elink{www.mysql.com}{http://www.mysql.com}.  Or see the \ilink{
   Installing and Configuring PostgreSQL}{PostgreSqlChapter} chapter of this
   document for the details.  For more information on PostgreSQL, please
   see: \elink{www.postgresql.org}{http://www.postgresql.org}.

   Configuring and building SQLite is even easier.  For the details of
   configuring SQLite, please see the \ilink{ Installing and Configuring
   SQLite}{SqlLiteChapter} chapter of this document.

\subsection*{Bacula Monitor} 
   \label{MonDef}
   A Bacula Monitor service is the program that allows the
   administrator or user to watch current status of Bacula Directors,
   Bacula File Daemons and Bacula Storage Daemons.
   Currently, only a GTK+ version is available, which works with GNOME,
   KDE, or any window manager that supports the FreeDesktop.org system tray
   standard. 

   To perform a successful save or restore, the following four daemons must be
   configured and running: the Director daemon, the File daemon, the Storage
   daemon, and the Catalog service (MySQL, PostgreSQL or SQLite). 

\section{Bacula Configuration}
\index[general]{Configuration!Bacula }
\index[general]{Bacula Configuration }

In order for Bacula to understand your system, what clients you want backed
up and how, you must create a number of configuration files containing
resources (or objects). The following presents an overall picture of this: 

\addcontentsline{lof}{figure}{Bacula Objects}
\includegraphics{\idir bacula-objects.eps} 

\section{Conventions Used in this Document}
\index[general]{Conventions Used in this Document }
\index[general]{Document!Conventions Used in this }

Bacula is in a state of evolution, and as a consequence, this manual
will not always agree with the code. If an item in this manual is preceded by
an asterisk (*), it indicates that the particular feature is not implemented.
If it is preceded by a plus sign (+), it indicates that the feature may be
partially implemented. 
% TODO: search for plus sign and asterisk and "IMPLEMENTED" and fix for printed book

If you are reading this manual as supplied in a released version of the
software, the above paragraph holds true. If you are reading the online
version of the manual, 
\elink{ www.bacula.org}{http://www.bacula.org}, please bear in
mind that this version describes the current version in development (in the
CVS) that may contain features not in the released version. Just the same, it
generally lags behind the code a bit. 
% TODO: is this still true? there are separate websites

\section{Quick Start}
\index[general]{Quick Start }
\index[general]{Start!Quick }

To get Bacula up and running quickly, the author recommends
that you first scan the
Terminology section below, then quickly review the next chapter entitled 
\ilink{The Current State of Bacula}{StateChapter}, then the 
\ilink{Getting Started with Bacula}{QuickStartChapter}, which will
give you a quick overview of getting Bacula running. After which, you should
proceed to the chapter on 
\ilink{Installing Bacula}{InstallChapter}, then 
\ilink{How to Configure Bacula}{ConfigureChapter}, and finally the
chapter on 
\ilink{ Running Bacula}{TutorialChapter}. 

\section{Terminology}
\index[general]{Terminology }

\begin{description}

\item [Administrator]
   \index[fd]{Administrator }
   The person or persons responsible for administrating the Bacula system. 

\item [Backup]
   \index[fd]{Backup }
   The term Backup refers to a Bacula Job that saves files. 

\item [Bootstrap File]
   \index[fd]{Bootstrap File }
   The bootstrap file is an ASCII file containing a compact form of
   commands that allow Bacula or the stand-alone file extraction utility
   (bextract) to restore the contents of one or more Volumes, for
   example, the current state of a system just backed up.  With a bootstrap
   file, Bacula can restore your system without a Catalog.  You can create
   a bootstrap file from a Catalog to extract any file or files you wish.

\item [Catalog]
   \index[fd]{Catalog }
   The Catalog is used to store summary information about the Jobs,
   Clients, and Files that were backed up and on what Volume or Volumes.
   The information saved in the Catalog permits the administrator or user
   to determine what jobs were run, their status as well as the important
   characteristics of each file that was backed up, and most importantly,
   it permits you to choose what files to restore.
   The Catalog is an
   online resource, but does not contain the data for the files backed up.
   Most of the information stored in the catalog is also stored on the
   backup volumes (i.e.  tapes).  Of course, the tapes will also have a
   copy of the file data in addition to the File Attributes (see below).

   The catalog feature is one part of Bacula that distinguishes it from
   simple backup and archive programs such as dump and tar.

\item [Client]
   \index[fd]{Client }
   In Bacula's terminology, the word Client refers to the machine being
   backed up, and it is synonymous with the File services or File daemon,
   and quite often, it is referred to it as the FD. A Client is defined in a
   configuration file resource.

\item [Console]
   \index[fd]{Console }
   The program that interfaces to the Director allowing  the user or system
   administrator to control Bacula. 

\item [Daemon]
   \index[fd]{Daemon }
   Unix terminology for a program that is always present in  the background to
   carry out a designated task. On Windows systems, as well as some Unix
   systems, daemons are called Services. 

\item [Directive]
   \index[fd]{Directive }
   The term directive is used to refer to a statement or a record within a
   Resource in a configuration file that defines one specific setting.  For
   example, the {\bf Name} directive defines the name of the Resource.

\item [Director]
   \index[fd]{Director }
   The main Bacula server daemon that schedules and directs all  Bacula
   operations. Occasionally, the project refers to the Director as DIR. 

\item [Differential]
   \index[fd]{Differential }
   A backup that includes all files changed since the last  Full save started.
   Note, other backup programs may define this differently. 

\item [File Attributes]
   \index[fd]{File Attributes }
   The File Attributes are all the information  necessary about a file to
   identify it and all its properties such as  size, creation date, modification
   date, permissions, etc. Normally, the  attributes are handled entirely by
   Bacula so that the user never  needs to be concerned about them. The
   attributes do not include the  file's data. 

\item [File Daemon]
   \index[fd]{File Daemon }
   The daemon running on the client  computer to be backed up. This is also
   referred to as the File  services, and sometimes as the Client services or the
   FD. 

\label{FileSetDef}
\item [FileSet]
\index[fd]{a name }
   A FileSet is a Resource contained in a configuration file that defines
   the files to be backed up.  It consists of a list of included files or
   directories, a list of excluded files, and how the file is to be stored
   (compression, encryption, signatures).  For more details, see the
   \ilink{FileSet Resource definition}{FileSetResource} in the Director
   chapter of this document.

\item [Incremental]
   \index[fd]{Incremental }
   A backup that includes all files changed since the  last Full, Differential,
   or Incremental backup started. It is normally  specified on the {\bf Level}
   directive within the Job resource  definition, or in a Schedule resource. 

\label{JobDef}
\item [Job]
\index[fd]{a name }
   A Bacula Job is a configuration resource that defines the work that
   Bacula must perform to backup or restore a particular Client.  It
   consists of the {\bf Type} (backup, restore, verify, etc), the {\bf
   Level} (full, incremental,...), the {\bf FileSet}, and {\bf Storage} the
   files are to be backed up (Storage device, Media Pool).  For more
   details, see the \ilink{Job Resource definition}{JobResource} in the
   Director chapter of this document.
% TODO: clean up "..." for book

\item [Monitor]
   \index[fd]{Monitor }
   The program that interfaces to all the daemons  allowing the user or
   system administrator to monitor Bacula status. 

\item [Resource]
   \index[fd]{Resource }
   A resource is a part of a configuration file that defines a specific
   unit of information that is available to Bacula.  It consists of several
   directives (individual configuration statements).  For example, the {\bf
   Job} resource defines all the properties of a specific Job: name,
   schedule, Volume pool, backup type, backup level, ...
% TODO: clean up "..." for book

\item [Restore]
   \index[fd]{Restore }
   A restore is a configuration resource that describes the operation of
   recovering a file from backup media.  It is the inverse of a save,
   except that in most cases, a restore will normally have a small set of
   files to restore, while normally a Save backs up all the files on the
   system.  Of course, after a disk crash, Bacula can be called upon to do
   a full Restore of all files that were on the system.
% TODO: Why? Why say "Of course"??

% TODO: define "Save"
% TODO: define "Full"

\item [Schedule]
   \index[fd]{Schedule }
   A Schedule is a configuration resource that defines when the Bacula Job
   will be scheduled for execution.  To use the Schedule, the Job resource
   will refer to the name of the Schedule.  For more details, see the
   \ilink{Schedule Resource definition}{ScheduleResource} in the Director
   chapter of this document.

\item [Service]
   \index[fd]{Service }
   This is a program that remains permanently in memory awaiting
   instructions.  In Unix environments, services are also known as
   {\bf daemons}. 

\item [Storage Coordinates]
   \index[fd]{Storage Coordinates }
   The information returned from the Storage Services that uniquely locates
   a file on a backup medium.  It consists of two parts: one part pertains
   to each file saved, and the other part pertains to the whole Job.
   Normally, this information is saved in the Catalog so that the user
   doesn't need specific knowledge of the Storage Coordinates.  The Storage
   Coordinates include the File Attributes (see above) plus the unique
   location of the information on the backup Volume.

\item [Storage Daemon]
   \index[fd]{Storage Daemon }
   The Storage daemon, sometimes referred to as the SD, is the code that
   writes the attributes and data to a storage Volume (usually a tape or
   disk).

\item [Session]
   \index[sd]{Session }
   Normally refers to the internal conversation between the File daemon and
   the Storage daemon.  The File daemon opens a {\bf session} with the
   Storage daemon to save a FileSet or to restore it.  A session has a
   one-to-one correspondence to a Bacula Job (see above).

\item [Verify]
   \index[sd]{Verify }
   A verify is a job that compares the current file attributes to the
   attributes that have previously been stored in the Bacula Catalog.  This
   feature can be used for detecting changes to critical system files
   similar to what a file integrity checker like Tripwire does.
   One of the major advantages of
   using Bacula to do this is that on the machine you want protected such
   as a server, you can run just the File daemon, and the Director, Storage
   daemon, and Catalog reside on a different machine.  As a consequence, if
   your server is ever compromised, it is unlikely that your verification
   database will be tampered with.

   Verify can also be used to check that the most recent Job data written
   to a Volume agrees with what is stored in the Catalog (i.e.  it compares
   the file attributes), *or it can check the Volume contents against the
   original files on disk.

\item [*Archive]
   \index[fd]{*Archive }
   An Archive operation is done after a Save, and it  consists of removing the
   Volumes on which data is saved from active  use. These Volumes are marked as
   Archived, and may no longer be  used to save files. All the files contained
   on an Archived Volume  are removed from the Catalog. NOT YET IMPLEMENTED. 

\item [Retention Period]
   \index[fd]{Retention Period }
   There are various kinds of retention periods that Bacula recognizes.
   The most important are the {\bf File} Retention Period, {\bf Job}
   Retention Period, and the {\bf Volume} Retention Period.  Each of these
   retention periods applies to the time that specific records will be kept
   in the Catalog database.  This should not be confused with the time that
   the data saved to a Volume is valid.

   The File Retention Period determines the time that File records are kept
   in the catalog database.  This period is important for two reasons: the
   first is that as long as File records remain in the database, you
   can "browse" the database with a console program and restore any
   individual file. Once the File records are removed or pruned from the
   database, the individual files of a backup job can no longer be
   "browsed".  The second reason for carefully choosing the File Retention
   Period is because the volume of
   the database File records use the most storage space in the
   database.  As a consequence, you must ensure that regular "pruning" of
   the database file records is done to keep your database from growing 
   too large. (See the Console {\bf prune}
   command for more details on this subject).

   The Job Retention Period is the length of time that Job records will be
   kept in the database.  Note, all the File records are tied to the Job
   that saved those files.  The File records can be purged leaving the Job
   records.  In this case, information will be available about the jobs
   that ran, but not the details of the files that were backed up.
   Normally, when a Job record is purged, all its File records will also be
   purged.

   The Volume Retention Period is the minimum of time that a Volume will be
   kept before it is reused.  Bacula will normally never overwrite a Volume
   that contains the only backup copy of a file.  Under ideal conditions,
   the Catalog would retain entries for all files backed up for all current
   Volumes.  Once a Volume is overwritten, the files that were backed up on
   that Volume are automatically removed from the Catalog.  However, if
   there is a very large pool of Volumes or a Volume is never overwritten,
   the Catalog database may become enormous.  To keep the Catalog to a
   manageable size, the backup information should be removed from the
   Catalog after the defined File Retention Period.  Bacula provides the
   mechanisms for the catalog to be automatically pruned according to the
   retention periods defined.

\item [Scan]
   \index[sd]{Scan }
   A Scan operation causes the contents of a Volume or a series of Volumes
   to be scanned.  These Volumes with the information on which files they
   contain are restored to the Bacula Catalog.  Once the information is
   restored to the Catalog, the files contained on those Volumes may be
   easily restored.  This function is particularly useful if certain
   Volumes or Jobs have exceeded their retention period and have been
   pruned or purged from the Catalog.  Scanning data from Volumes into the
   Catalog is done by using the {\bf bscan} program.  See the \ilink{ bscan
   section}{bscan} of the Bacula Utilities Chapter of this manual for more
   details.

\item [Volume]
   \index[sd]{Volume }
   A Volume is an archive unit, normally a tape or a named disk file where
   Bacula stores the data from one or more backup jobs.  All Bacula Volumes
   have a software label written to the Volume by Bacula so that it
   identifies what Volume it is really reading.  (Normally there should be
   no confusion with disk files, but with tapes, it is easy to mount the
   wrong one.)
\end{description}

\section{What Bacula is Not}
\index[general]{What Bacula is Not}

Bacula is a backup, restore and verification program and is not a
complete disaster recovery system in itself, but it can be a key part of one
if you plan carefully and follow the instructions included in the 
\ilink{ Disaster Recovery}{RescueChapter} Chapter of this manual. 

With proper planning, as mentioned in the Disaster Recovery chapter,
Bacula can be a central component of your disaster recovery system. For
example, if you have created an emergency boot disk, and/or a Bacula Rescue disk to
save the current partitioning information of your hard disk, and maintain a
complete Bacula backup, it is possible to completely recover your system from
"bare metal" that is starting from an empty disk. 

If you have used the {\bf WriteBootstrap} record in your job or some other
means to save a valid bootstrap file, you will be able to use it to extract
the necessary files (without using the catalog or manually searching for the
files to restore). 

\section{Interactions Between the Bacula Services}
\index[general]{Interactions Between the Bacula Services}
\index[general]{Services!Interactions Between the Bacula}

The following block diagram shows the typical interactions between the Bacula
Services for a backup job. Each block represents in general a separate process
(normally a daemon). In general, the Director oversees the flow of
information. It also maintains the Catalog. 

\addcontentsline{lof}{figure}{Interactions between Bacula Services}
\includegraphics{\idir flow.eps} 
