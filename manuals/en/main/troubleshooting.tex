
\section{Tape Drive}
\label{TapeTestingChapter}
\index[general]{Problem!Tape}

%%
%%

\chapter{Testing Your Tape Drive With Bacula}
\label{TapeTestingChapter}
\index[general]{Testing Your Tape Drive With Bacula}

This chapter is concerned with testing and configuring your tape drive to make
sure that it will work properly with Bacula using the {\bf btape} program.
\label{summary}

\section{Get Your Tape Drive Working}

In general, you should follow the following steps to get your tape drive to
work with Bacula. Start with a tape mounted in your drive. If you have an
autochanger, load a tape into the drive. We use {\bf /dev/nst0} as the tape
drive name, you will need to adapt it according to your system.

Do not proceed to the next item until you have succeeded with the previous
one.

\begin{enumerate}
\item Make sure that Bacula (the Storage daemon) is not running
  or that you have {\bf unmount}ed the drive you will use
  for testing.

\item Use tar to write to, then read from your drive:

   \footnotesize
\begin{verbatim}
   mt -f /dev/nst0 rewind
   tar cvf /dev/nst0 .
   mt -f /dev/nst0 rewind
   tar tvf /dev/nst0

\end{verbatim}
\normalsize

\item Make sure you have a valid and correct Device resource corresponding
   to your drive.  For Linux users, generally, the default one works.  For
   FreeBSD users, there are two possible Device configurations (see below).
   For other drives and/or OSes, you will need to first ensure that your
   system tape modes are properly setup (see below), then possibly modify
   you Device resource depending on the output from the btape program (next
   item). When doing this, you should consult the \ilink{Storage Daemon
   Configuration}{StoredConfChapter} of this manual.

\item If you are using a Fibre Channel to connect your tape drive to
   Bacula, please be sure to disable any caching in the NSR (network
   storage router, which is a Fibre Channel to SCSI converter).

\item Run the btape {\bf test} command:

   \footnotesize
\begin{verbatim}
   ./btape -c bacula-sd.conf /dev/nst0
   test

\end{verbatim}
\normalsize

   It isn't necessary to run the autochanger part of the test at this time,
   but do not go past this point until the basic test succeeds.  If you do
   have an autochanger, please be sure to read the \ilink{Autochanger
   chapter}{AutochangersChapter} of this manual.

\item Run the btape {\bf fill} command, preferably with two volumes.  This
   can take a long time. If you have an autochanger and it  is configured, Bacula
   will automatically use it. If you do  not have it configured, you can manually
   issue the appropriate  {\bf mtx} command, or press the autochanger buttons to
   change  the tape when requested to do so.

\item FreeBSD users, if you have a pre-5.0 system run the {\bf tapetest}
   program, and make sure your system is patched if necessary. The tapetest
   program can be found in the platform/freebsd directory. The instructions
   for its use are at the top of the file.

\item Run Bacula, and backup a reasonably small directory, say 60
   Megabytes.  Do three successive backups of this directory.

\item Stop Bacula, then restart it.  Do another full backup of the same
   directory.  Then stop and restart Bacula.

\item Do a restore of the directory backed up, by entering the  following
   restore command, being careful to restore it to  an alternate location:


\footnotesize
\begin{verbatim}
   restore select all done
   yes

\end{verbatim}
\normalsize

   Do a {\bf diff} on the restored directory to ensure it is identical  to the
   original directory. If you are going to backup multiple different systems
   (Linux, Windows, Mac, Solaris, FreeBSD, ...), be sure you test the restore
   on each system type.

\item If you have an autochanger, you should now go back to the  btape program
   and run the autochanger test:

\footnotesize
\begin{verbatim}
     ./btape -c bacula-sd.conf /dev/nst0
     auto

\end{verbatim}
\normalsize

   Adjust your autochanger as necessary to ensure that it works  correctly. See
   the Autochanger chapter of this manual  for a complete discussion of testing
   your autochanger.

\item We strongly recommend that you use a dedicated SCSI
   controller for your tape drives. Scanners are known to induce
   serious problems with the SCSI bus, causing it to reset. If the
   SCSI bus is reset while Bacula has the tape drive open, it will
   most likely be fatal to your tape since the drive will rewind.
   These kinds of problems show up in the system log. For example,
   the following was most likely caused by a scanner:

\footnotesize
\begin{verbatim}
Feb 14 17:29:55 epohost kernel: (scsi0:A:2:0): No or incomplete CDB sent to device.
Feb 14 17:29:55 epohost kernel: scsi0: Issued Channel A Bus Reset. 1 SCBs aborted
\end{verbatim}
\normalsize

\end{enumerate}

If you have reached this point, you stand a good chance of having everything
work. If you get into trouble at any point, {\bf carefully} read the
documentation given below. If you cannot get past some point, ask the {\bf
bacula-users} email list, but specify which of the steps you have successfully
completed. In particular, you may want to look at the
\ilink{ Tips for Resolving Problems}{problems1} section below.


\label{NoTapeInDrive}
\subsection{Problems When no Tape in Drive}
\index[general]{Problems When no Tape in Drive}
When Bacula was first written the Linux 2.4 kernel permitted opening the
drive whether or not there was a tape in the drive. Thus the Bacula code is
based on the concept that if the drive cannot be opened, there is a serious
problem, and the job is failed.

With version 2.6 of the Linux kernel, if there is no tape in the drive, the
OS will wait two minutes (default) and then return a failure, and consequently,
Bacula version 1.36 and below will fail the job.  This is important to keep
in mind, because if you use an option such as {\bf Offline on Unmount =
yes}, there will be a point when there is no tape in the drive, and if
another job starts or if Bacula asks the operator to mount a tape, when
Bacula attempts to open the drive (about a 20 minute delay), it will fail
and Bacula will fail the job.

In version 1.38.x, the Bacula code partially gets around this problem -- at
least in the initial open of the drive.  However, functions like Polling
the drive do not work correctly if there is no tape in the drive.
Providing you do not use {\bf Offline on Unmount = yes}, you should not
experience job failures as mentioned above.  If you do experience such
failures, you can also increase the {\bf Maximum Open Wait} time interval,
which will give you more time to mount the next tape before the job is
failed.

\subsection{Specifying the Configuration File}
\index[general]{File!Specifying the Configuration}
\index[general]{Specifying the Configuration File}

Starting with version 1.27, each of the tape utility programs including the
{\bf btape} program requires a valid Storage daemon configuration file
(actually, the only part of the configuration file that {\bf btape} needs is
the {\bf Device} resource definitions). This permits {\bf btape} to find the
configuration parameters for your archive device (generally a tape drive).
Without those parameters, the testing and utility programs do not know how to
properly read and write your drive. By default, they use {\bf bacula-sd.conf}
in the current directory, but you may specify a different configuration file
using the {\bf -c} option.

\subsection{Specifying a Device Name For a Tape}
\index[general]{Tape!Specifying a Device Name For a}
\index[general]{Specifying a Device Name For a Tape}

{\bf btape} {\bf device-name} where the Volume can be found. In the case of a
tape, this is the physical device name such as {\bf /dev/nst0} or {\bf
/dev/rmt/0ubn} depending on your system that you specify on the Archive Device
directive. For the program to work, it must find the identical name in the
Device resource of the configuration file. If the name is not found in the
list of physical names, the utility program will compare the name you entered
to the Device names (rather than the Archive device names).

When specifying a tape device, it is preferable that the "non-rewind"
variant of the device file name be given.  In addition, on systems such as
Sun, which have multiple tape access methods, you must be sure to specify
to use Berkeley I/O conventions with the device.  The
{\bf b} in the Solaris (Sun) archive specification {\bf /dev/rmt/0mbn} is
what is needed in this case.  Bacula does not support SysV tape drive
behavior.

See below for specifying Volume names.

\subsection{Specifying a Device Name For a File}
\index[general]{File!Specifying a Device Name For a}
\index[general]{Specifying a Device Name For a File}

If you are attempting to read or write an archive file rather than a tape, the
{\bf device-name} should be the full path to the archive location including
the filename. The filename (last part of the specification) will be stripped
and used as the Volume name, and the path (first part before the filename)
must have the same entry in the configuration file. So, the path is equivalent
to the archive device name, and the filename is equivalent to the volume name.


\section{btape}
\label{btape1}
\index[general]{Btape}

This program permits a number of elementary tape operations via a tty command
interface. The {\bf test} command, described below, can be very useful for
testing tape drive compatibility problems. Aside from initial testing of tape
drive compatibility with {\bf Bacula}, {\bf btape} will be mostly used by
developers writing new tape drivers.

{\bf btape} can be dangerous to use with existing {\bf Bacula} tapes because
it will relabel a tape or write on the tape if so requested regardless of
whether or not the tape contains valuable data, so please be careful and use
it only on blank tapes.

To work properly, {\bf btape} needs to read the Storage daemon's configuration
file. As a default, it will look for {\bf bacula-sd.conf} in the current
directory. If your configuration file is elsewhere, please use the {\bf -c}
option to specify where.

The physical device name or the Device resource name must be specified on the
command line, and this same device name must be present in the Storage
daemon's configuration file read by {\bf btape}

\footnotesize
\begin{verbatim}
Usage: btape [options] device_name
       -b <file>   specify bootstrap file
       -c <file>   set configuration file to file
       -d <nn>     set debug level to nn
       -p          proceed inspite of I/O errors
       -s          turn off signals
       -v          be verbose
       -?          print this message.
\end{verbatim}
\normalsize

\subsection{Using btape to Verify your Tape Drive}
\index[general]{Using btape to Verify your Tape Drive}
\index[general]{Drive!Using btape to Verify your Tape}

An important reason for this program is to ensure that a Storage daemon
configuration file is defined so that Bacula will correctly read and write
tapes.

It is highly recommended that you run the {\bf test} command before running
your first Bacula job to ensure that the parameters you have defined for your
storage device (tape drive) will permit {\bf Bacula} to function properly. You
only need to mount a blank tape, enter the command, and the output should be
reasonably self explanatory. For example:

\footnotesize
\begin{verbatim}
(ensure that Bacula is not running)
./btape -c /usr/bin/bacula/bacula-sd.conf /dev/nst0
\end{verbatim}
\normalsize

The output will be:

\footnotesize
\begin{verbatim}
Tape block granularity is 1024 bytes.
btape: btape.c:376 Using device: /dev/nst0
*
\end{verbatim}
\normalsize

Enter the test command:

\footnotesize
\begin{verbatim}
test
\end{verbatim}
\normalsize

The output produced should be something similar to the following: I've cut the
listing short because it is frequently updated to have new tests.

\footnotesize
\begin{verbatim}
=== Append files test ===
This test is essential to Bacula.
I'm going to write one record  in file 0,
                   two records in file 1,
             and three records in file 2
btape: btape.c:387 Rewound /dev/nst0
btape: btape.c:855 Wrote one record of 64412 bytes.
btape: btape.c:857 Wrote block to device.
btape: btape.c:410 Wrote EOF to /dev/nst0
btape: btape.c:855 Wrote one record of 64412 bytes.
btape: btape.c:857 Wrote block to device.
btape: btape.c:855 Wrote one record of 64412 bytes.
btape: btape.c:857 Wrote block to device.
btape: btape.c:410 Wrote EOF to /dev/nst0
btape: btape.c:855 Wrote one record of 64412 bytes.
btape: btape.c:857 Wrote block to device.
btape: btape.c:855 Wrote one record of 64412 bytes.
btape: btape.c:857 Wrote block to device.
btape: btape.c:855 Wrote one record of 64412 bytes.
btape: btape.c:857 Wrote block to device.
btape: btape.c:410 Wrote EOF to /dev/nst0
btape: btape.c:387 Rewound /dev/nst0
btape: btape.c:693 Now moving to end of media.
btape: btape.c:427 Moved to end of media
We should be in file 3. I am at file 3. This is correct!
Now the important part, I am going to attempt to append to the tape.
...
=== End Append files test ===
\end{verbatim}
\normalsize

If you do not successfully complete the above test, please resolve the
problem(s) before attempting to use {\bf Bacula}. Depending on your tape
drive, the test may recommend that you add certain records to your
configuration. We strongly recommend that you do so and then re-run the above
test to insure it works the first time.

Some of the suggestions it provides for resolving the problems may or may not
be useful. If at all possible avoid using fixed blocking. If the test suddenly
starts to print a long series of:

\footnotesize
\begin{verbatim}
Got EOF on tape.
Got EOF on tape.
...
\end{verbatim}
\normalsize

then almost certainly, you are running your drive in fixed block mode rather
than variable block mode. See below for more help of resolving fix
versus variable block problems.

It is also possible that you have your drive
set in SysV tape drive mode. The drive must use BSD tape conventions.
See the section above on setting your {\bf Archive device} correctly.

For FreeBSD users, please see the notes below for doing further testing of
your tape drive.

\subsection{Testing tape drive speed}
\label{sec:btapespeed}

To determine the best configuration of your tape drive, you can run the
\texttt{speed} command available in the \texttt{btape} program.

This command can have the following arguments:
\begin{itemize}
\item[\texttt{file\_size=n}] Specify the Maximum File Size for this test
  (between 1 and 5GB). This counter is in GB.
\item[\texttt{nb\_file=n}] Specify the number of file to be written. The amount
  of data should be greater than your memory ($file\_size*nb\_file$).
\item[\texttt{skip\_zero}] This flag permits to skip tests with constant
  data.
\item[\texttt{skip\_random}] This flag permits to skip tests with random
  data.
\item[\texttt{skip\_raw}] This flag permits to skip tests with raw access.
\item[\texttt{skip\_block}] This flag permits to skip tests with Bacula block
  access.
\end{itemize}

\begin{verbatim}
*speed file_size=3 skip_raw
btape.c:1078 Test with zero data and bacula block structure.
btape.c:956 Begin writing 3 files of 3.221 GB with blocks of 129024 bytes.
++++++++++++++++++++++++++++++++++++++++++
btape.c:604 Wrote 1 EOF to "Drive-0" (/dev/nst0)
btape.c:406 Volume bytes=3.221 GB. Write rate = 44.128 MB/s
...
btape.c:383 Total Volume bytes=9.664 GB. Total Write rate = 43.531 MB/s

btape.c:1090 Test with random data, should give the minimum throughput.
btape.c:956 Begin writing 3 files of 3.221 GB with blocks of 129024 bytes.
+++++++++++++++++++++++++++++++++++++++++++
btape.c:604 Wrote 1 EOF to "Drive-0" (/dev/nst0)
btape.c:406 Volume bytes=3.221 GB. Write rate = 7.271 MB/s
+++++++++++++++++++++++++++++++++++++++++++
...
btape.c:383 Total Volume bytes=9.664 GB. Total Write rate = 7.365 MB/s

\end{verbatim}

When using compression, the random test will give your the minimum throughput
of your drive . The test using constant string will give you the maximum speed
of your hardware chain. (cpu, memory, scsi card, cable, drive, tape).

You can change the block size in the Storage Daemon configuration file.

\label{SCSITricks}
\subsection{Linux SCSI Tricks}
\index[general]{Tricks!Linux SCSI}
\index[general]{Linux SCSI Tricks}

You can find out what SCSI devices you have by doing:

\footnotesize
\begin{verbatim}
lsscsi
\end{verbatim}
\normalsize

Typical output is:

\footnotesize
\begin{verbatim}
[0:0:0:0]    disk    ATA      ST3160812AS      3.AD  /dev/sda
[2:0:4:0]    tape    HP       Ultrium 2-SCSI   F6CH  /dev/st0
[2:0:5:0]    tape    HP       Ultrium 2-SCSI   F6CH  /dev/st1
[2:0:6:0]    mediumx OVERLAND LXB              0107  -
[2:0:9:0]    tape    HP       Ultrium 1-SCSI   E50H  /dev/st2
[2:0:10:0]   mediumx OVERLAND LXB              0107  -
\end{verbatim}
\normalsize

There are two drives in one autochanger: /dev/st0 and /dev/st1
and a third tape drive at /dev/st2.  For using them with Bacula, one
would normally reference them as /dev/nst0 ... /dev/nst2.  Not also,
there are two different autochangers identified as "mediumx OVERLAND LXB".
They can be addressed via their /dev/sgN designation, which can be
obtained by counting from the beginning as 0 to each changer.  In the
above case, the two changers are located on /dev/sg3 and /dev/sg5. The one
at /dev/sg3, controls drives /dev/nst0 and /dev/nst1; and the one at
/dev/sg5 controles drive /dev/nst2.

If you do not have the {\bf lsscsi}  command, you can obtain the same
information as follows:

\footnotesize
\begin{verbatim}
cat /proc/scsi/scsi
\end{verbatim}
\normalsize

For the above example with the three drives and two autochangers,
I get:

\footnotesize
\begin{verbatim}
Attached devices:
Host: scsi0 Channel: 00 Id: 00 Lun: 00
  Vendor: ATA      Model: ST3160812AS      Rev: 3.AD
  Type:   Direct-Access                    ANSI SCSI revision: 05
Host: scsi2 Channel: 00 Id: 04 Lun: 00
  Vendor: HP       Model: Ultrium 2-SCSI   Rev: F6CH
  Type:   Sequential-Access                ANSI SCSI revision: 03
Host: scsi2 Channel: 00 Id: 05 Lun: 00
  Vendor: HP       Model: Ultrium 2-SCSI   Rev: F6CH
  Type:   Sequential-Access                ANSI SCSI revision: 03
Host: scsi2 Channel: 00 Id: 06 Lun: 00
  Vendor: OVERLAND Model: LXB              Rev: 0107
  Type:   Medium Changer                   ANSI SCSI revision: 02
Host: scsi2 Channel: 00 Id: 09 Lun: 00
  Vendor: HP       Model: Ultrium 1-SCSI   Rev: E50H
  Type:   Sequential-Access                ANSI SCSI revision: 03
Host: scsi2 Channel: 00 Id: 10 Lun: 00
  Vendor: OVERLAND Model: LXB              Rev: 0107
  Type:   Medium Changer                   ANSI SCSI revision: 02
\end{verbatim}
\normalsize


As an additional example, I get the following (on a different machine from the
above example):

\footnotesize
\begin{verbatim}
Attached devices:
Host: scsi2 Channel: 00 Id: 01 Lun: 00
  Vendor: HP       Model: C5713A           Rev: H107
  Type:   Sequential-Access                ANSI SCSI revision: 02
Host: scsi2 Channel: 00 Id: 04 Lun: 00
  Vendor: SONY     Model: SDT-10000        Rev: 0110
  Type:   Sequential-Access                ANSI SCSI revision: 02
\end{verbatim}
\normalsize

The above represents first an autochanger and second a simple
tape drive. The HP changer (the first entry) uses the same SCSI channel
for data and for control, so in Bacula, you would use:
\footnotesize
\begin{verbatim}
Archive Device = /dev/nst0
Changer Device = /dev/sg0
\end{verbatim}
\normalsize

If you want to remove the SDT-10000 device, you can do so as root with:

\footnotesize
\begin{verbatim}
echo "scsi remove-single-device 2 0 4 0">/proc/scsi/scsi
\end{verbatim}
\normalsize

and you can put add it back with:

\footnotesize
\begin{verbatim}
echo "scsi add-single-device 2 0 4 0">/proc/scsi/scsi
\end{verbatim}
\normalsize

where the 2 0 4 0 are the Host, Channel, Id, and Lun as seen on the output
from {\bf cat /proc/scsi/scsi}. Note, the Channel must be specified as
numeric.

Below is a slightly more complicated output, which is a single autochanger
with two drives, and which operates the changer on a different channel
from from the drives:

\footnotesize
\begin{verbatim}
Attached devices:
Host: scsi0 Channel: 00 Id: 00 Lun: 00
  Vendor: ATA      Model: WDC WD1600JD-75H Rev: 08.0
  Type:   Direct-Access                    ANSI SCSI revision: 05
Host: scsi2 Channel: 00 Id: 04 Lun: 00
  Vendor: HP       Model: Ultrium 2-SCSI   Rev: F6CH
  Type:   Sequential-Access                ANSI SCSI revision: 03
Host: scsi2 Channel: 00 Id: 05 Lun: 00
  Vendor: HP       Model: Ultrium 2-SCSI   Rev: F6CH
  Type:   Sequential-Access                ANSI SCSI revision: 03
Host: scsi2 Channel: 00 Id: 06 Lun: 00
  Vendor: OVERLAND Model: LXB              Rev: 0106
  Type:   Medium Changer                   ANSI SCSI revision: 02
\end{verbatim}
\normalsize

The above tape drives are accessed on /dev/nst0 and /dev/nst1, while
the control channel for those two drives is /dev/sg3.



\label{problems1}
\section{Tips for Resolving Problems}
\index[general]{Problems!Tips for Resolving}
\index[general]{Tips for Resolving Problems}

\label{CannotRestore}
\subsection{Bacula Saves But Cannot Restore Files}
\index[general]{Files!Bacula Saves But Cannot Restore}
\index[general]{Bacula Saves But Cannot Restore Files}

If you are getting error messages such as:

\footnotesize
\begin{verbatim}
Volume data error at 0:1! Wanted block-id: "BB02", got "". Buffer discarded
\end{verbatim}
\normalsize

It is very likely that Bacula has tried to do block positioning and ended up
at an invalid block. This can happen if your tape drive is in fixed block mode
while Bacula's default is variable blocks. Note that in such cases, Bacula is
perfectly able to write to your Volumes (tapes), but cannot position to read
them.

There are two possible solutions.

\begin{enumerate}
\item The first and  best is to always ensure that your drive is in  variable
   block mode. Note, it can switch back to  fixed block mode on a reboot or if
   another program  uses the drive. So on such systems you  need to modify the
   Bacula startup files  to explicitly set:

\footnotesize
\begin{verbatim}
mt -f /dev/nst0 defblksize 0
\end{verbatim}
\normalsize

or whatever is appropriate on your system. Note, if you are running a Linux
system, and the above command does not work, it is most likely because you
have not loaded the appropriate {\bf mt} package, which is often called
{\bf mt\_st}, but may differ according to your distribution.

\item The second possibility, especially, if Bacula wrote  while the drive was
   in fixed block mode, is to turn  off block positioning in Bacula. This is done
   by  adding:

\footnotesize
\begin{verbatim}
Block Positioning = no
\end{verbatim}
\normalsize

to the Device resource. This is not the recommended  procedure because it can
enormously slow down  recovery of files, but it may help where all else
fails. This directive is available in version 1.35.5  or later (and not yet
tested).
\end{enumerate}

If you are getting error messages such as:
\footnotesize
\begin{verbatim}
Volume data error at 0:0!
Block checksum mismatch in block=0 len=32625 calc=345678 blk=123456
\end{verbatim}
\normalsize

You are getting tape read errors, and this is most likely due to
one of the following things:
\begin{enumerate}
\item An old or bad tape.
\item A dirty drive that needs cleaning (particularly for DDS drives).
\item A loose SCSI cable.
\item Old firmware in your drive. Make sure you have the latest firmware
      loaded.
\item Computer memory errors.
\item Over-clocking your CPU.
\item A bad SCSI card.
\end{enumerate}


\label{opendevice}
\subsection{Bacula Cannot Open the Device}
\index[general]{Device!Bacula Cannot Open the}
\index[general]{Bacula Cannot Open the Device}

If you get an error message such as:

\footnotesize
\begin{verbatim}
dev open failed: dev.c:265 stored: unable to open
device /dev/nst0:> ERR=No such device or address
\end{verbatim}
\normalsize

the first time you run a job, it is most likely due to the fact that you
specified the incorrect device name on your {\bf Archive Device}.

If Bacula works fine with your drive, then all off a sudden you get error
messages similar to the one shown above, it is quite possible that your driver
module is being removed because the kernel deems it idle. This is done via
{\bf crontab} with the use of {\bf rmmod -a}. To fix the problem, you can
remove this entry from {\bf crontab}, or you can manually {\bf modprob} your
driver module (or add it to the local startup script). Thanks to Alan Brown
for this tip.
\label{IncorrectFiles}

\subsection{Incorrect File Number}
\index[general]{Number!Incorrect File}
\index[general]{Incorrect File Number}

When Bacula moves to the end of the medium, it normally uses the {\bf
ioctl(MTEOM)} function. Then Bacula uses the {\bf ioctl(MTIOCGET)} function to
retrieve the current file position from the {\bf mt\_fileno} field. Some SCSI
tape drivers will use a fast means of seeking to the end of the medium and in
doing so, they will not know the current file position and hence return a {\bf
-1}. As a consequence, if you get {\bf "This is NOT correct!"} in the
positioning tests, this may be the cause. You must correct this condition in
order for Bacula to work.

There are two possible solutions to the above problem of incorrect file
number:

\begin{itemize}
\item Figure out how to configure your SCSI driver to  keep track of the file
   position during the MTEOM  request. This is the preferred solution.
\item Modify the {\bf Device} resource of your {\bf bacula-sd.conf} file  to
   include:

\footnotesize
\begin{verbatim}
Hardware End of File = no
\end{verbatim}
\normalsize

This will cause Bacula to use the MTFSF request to  seek to the end of the
medium, and Bacula will keep  track of the file number itself.
\end{itemize}

\label{IncorrectBlocks}
\subsection{Incorrect Number of Blocks or Positioning Errors}
\index[general]{Testing!Incorrect Number of Blocks or Positioning Errors}
\index[general]{Incorrect Number of Blocks or Positioning Errors}

{\bf Bacula's} preferred method of working with tape drives (sequential
devices) is to run in variable block mode, and this is what is set by default.
You should first ensure that your tape drive is set for variable block mode
(see below).

If your tape drive is in fixed block mode and you have told Bacula to use
different fixed block sizes or variable block sizes (default), you will get
errors when Bacula attempts to forward space to the correct block (the kernel
driver's idea of tape blocks will not correspond to Bacula's).

All modern tape drives support variable tape blocks, but some older drives (in
particular the QIC drives) as well as the ATAPI ide-scsi driver run only in
fixed block mode. The Travan tape drives also apparently must run in fixed
block mode (to be confirmed).

Even in variable block mode, with the exception of the first record on the
second or subsequent volume of a multi-volume backup, Bacula will write blocks
of a fixed size. However, in reading a tape, Bacula will assume that for each
read request, exactly one block from the tape will be transferred. This the
most common way that tape drives work and is well supported by {\bf Bacula}.

Drives that run in fixed block mode can cause serious problems for Bacula if
the drive's block size does not correspond exactly to {\bf Bacula's} block
size. In fixed block size mode, drivers may transmit a partial block or
multiple blocks for a single read request. From {\bf Bacula's} point of view,
this destroys the concept of tape blocks. It is much better to run in variable
block mode, and almost all modern drives (the OnStream is an exception) run in
variable block mode. In order for Bacula to run in fixed block mode, you must
include the following records in the Storage daemon's Device resource
definition:

\footnotesize
\begin{verbatim}
Minimum Block Size = nnn
Maximum Block Size = nnn
\end{verbatim}
\normalsize

where {\bf nnn} must be the same for both records and must be identical to the
driver's fixed block size.

We recommend that you avoid this configuration if at all possible by using
variable block sizes.

If you must run with fixed size blocks, make sure they are not 512 bytes. This
is too small and the overhead that Bacula has with each record will become
excessive. If at all possible set any fixed block size to something like
64,512 bytes or possibly 32,768 if 64,512 is too large for your drive. See
below for the details on checking and setting the default drive block size.

To recover files from tapes written in fixed block mode, see below.

\label{TapeModes}
\subsection{Ensuring that the Tape Modes Are Properly Set -- {\bf Linux
Only}}
\index[general]{Ensuring that the Tape Modes Are Properly Set -- Linux Only}

If you have a modern SCSI tape drive and you are having problems with the {\bf
test} command as noted above, it may be that some program has set one or more
of your SCSI driver's options to non-default values. For example, if your
driver is set to work in SysV manner, Bacula will not work correctly because
it expects BSD behavior. To reset your tape drive to the default values, you
can try the following, but {\bf ONLY} if you have a SCSI tape drive on a {\bf
Linux} system:

\footnotesize
\begin{verbatim}
become super user
mt -f /dev/nst0 rewind
mt -f /dev/nst0 stoptions buffer-writes async-writes read-ahead
\end{verbatim}
\normalsize

The above commands will clear all options and then set those specified. None
of the specified options are required by Bacula, but a number of other options
such as SysV behavior must not be set. Bacula does not support SysV tape
behavior. On systems other than Linux, you will need to consult your {\bf mt}
man pages or documentation to figure out how to do the same thing. This should
not really be necessary though -- for example, on both Linux and Solaris
systems, the default tape driver options are compatible with Bacula.
On Solaris systems, you must take care to specify the correct device
name on the {\bf Archive device} directive. See above for more details.

You may also want to ensure that no prior program has set the default block
size, as happened to one user, by explicitly turning it off with:

\footnotesize
\begin{verbatim}
mt -f /dev/nst0 defblksize 0
\end{verbatim}
\normalsize

If you are running a Linux
system, and the above command does not work, it is most likely because you
have not loaded the appropriate {\bf mt} package, which is often called
{\bf mt\_st}, but may differ according to your distribution.

If you would like to know what options you have set before making any of the
changes noted above, you can now view them on Linux systems, thanks to a tip
provided by Willem Riede. Do the following:

\footnotesize
\begin{verbatim}
become super user
mt -f /dev/nst0 stsetoptions 0
grep st0 /var/log/messages
\end{verbatim}
\normalsize

and you will get output that looks something like the following:

\footnotesize
\begin{verbatim}
kernel: st0: Mode 0 options: buffer writes: 1, async writes: 1, read ahead: 1
kernel: st0:    can bsr: 0, two FMs: 0, fast mteom: 0, auto lock: 0,
kernel: st0:    defs for wr: 0, no block limits: 0, partitions: 0, s2 log: 0
kernel: st0:    sysv: 0 nowait: 0
\end{verbatim}
\normalsize

Note, I have chopped off the beginning of the line with the date and machine
name for presentation purposes.

Some people find that the above settings only last until the next reboot, so
please check this otherwise you may have unexpected problems.

Beginning with Bacula version 1.35.8, if Bacula detects that you are running
in variable block mode, it will attempt to set your drive appropriately. All
OSes permit setting variable block mode, but some OSes do not permit setting
the other modes that Bacula needs to function properly.

\label{compression}
\subsection{Tape Hardware Compression and Blocking Size}
\index[general]{Tape Hardware Compression and Blocking Size}
\index[general]{Size!Tape Hardware Compression and Blocking Size}

You should be able to verify the tape compression status with sysfs on Linux.
\begin{verbatim}
cat /sys/class/scsi_tape/nst0/default_compression
\end{verbatim}

You can, turn it on by using (on Linux):

\footnotesize
\begin{verbatim}
become super user
mt -f /dev/nst0 defcompression 1
\end{verbatim}
\normalsize

and of course, if you use a zero instead of the one at the end, you will turn
it off.

If you have built the {\bf mtx} program in the {\bf depkgs} package, you can
use tapeinfo to get quite a bit of information about your tape drive even if
it is not an autochanger. This program is called using the SCSI control
device. On Linux for tape drive /dev/nst0, this is usually /dev/sg0, while on
FreeBSD for /dev/nsa0, the control device is often /dev/pass2. For example on
my DDS-4 drive (/dev/nst0), I get the following:

\footnotesize
\begin{verbatim}
tapeinfo -f /dev/sg0
Product Type: Tape Drive
Vendor ID: 'HP      '
Product ID: 'C5713A          '
Revision: 'H107'
Attached Changer: No
MinBlock:1
MaxBlock:16777215
SCSI ID: 5
SCSI LUN: 0
Ready: yes
BufferedMode: yes
Medium Type: Not Loaded
Density Code: 0x26
BlockSize: 0
\end{verbatim}
\normalsize

where the {\bf DataCompEnabled: yes} means that tape hardware compression is
turned on. You can turn it on and off (yes|no) by using the {\bf mt}
commands given above. Also, this output will tell you if the {\bf BlockSize}
is non-zero and hence set for a particular block size. Bacula is not likely to
work in such a situation because it will normally attempt to write blocks of
64,512 bytes, except the last block of the job which will generally be
shorter. The first thing to try is setting the default block size to zero
using the {\bf mt -f /dev/nst0 defblksize 0} command as shown above.
On FreeBSD, this would be something like: {\bf mt -f /dev/nsa0 blocksize 0}.

On some operating systems with some tape drives, the amount of data that
can be written to the tape and whether or not compression is enabled is
determined by the density usually the {\bf mt -f /dev/nst0 setdensity xxx} command.
Often  {\bf mt -f /dev/nst0 status} will print out the current
density code that is used with the drive.  Most systems, but unfortunately
not all, set the density to the maximum by default. On some systems, you
can also get a list of all available density codes with:
{\bf mt -f /dev/nst0 densities} or a similar {\bf mt} command.
Note, for DLT and SDLT devices, no-compression versus compression is very
often controlled by the density code.  On FreeBSD systems, the compression
mode is set using {\bf mt -f /dev/nsa0 comp xxx} where xxx is the
mode you want.  In general, see {\bf man mt}  for the options available on
your system.

Note, some of the above {\bf mt} commands may not be persistent depending
on your system configuration. That is they may be reset if a program
other than Bacula uses the drive or, as is frequently the case, on reboot
of your system.

If your tape drive requires fixed block sizes (very unusual), you can use the
following records:

\footnotesize
\begin{verbatim}
Minimum Block Size = nnn
Maximum Block Size = nnn
\end{verbatim}
\normalsize

in your Storage daemon's Device resource to force Bacula to write fixed size
blocks (where you sent nnn to be the same for both of the above records). This
should be done only if your drive does not support variable block sizes, or
you have some other strong reasons for using fixed block sizes. As mentioned
above, a small fixed block size of 512 or 1024 bytes will be very inefficient.
Try to set any fixed block size to something like 64,512 bytes or larger if
your drive will support it.

Also, note that the {\bf Medium Type} field of the output of {\bf tapeinfo}
reports {\bf Not Loaded}, which is not correct. As a consequence, you should
ignore that field as well as the {\bf Attached Changer} field.

To recover files from tapes written in fixed block mode, see below.
\label{FreeBSDTapes}

\subsection{Tape Modes on FreeBSD}
\index[general]{FreeBSD!Tape Modes on}
\index[general]{Tape Modes on FreeBSD}

On most FreeBSD systems such as 4.9 and most tape drives, Bacula should run
with:

\footnotesize
\begin{verbatim}
mt  -f  /dev/nsa0  seteotmodel  2
mt  -f  /dev/nsa0  blocksize   0
mt  -f  /dev/nsa0  comp  enable
\end{verbatim}
\normalsize

You might want to put those commands in a startup script to make sure your
tape driver is properly initialized before running Bacula, because
depending on your system configuration, these modes may be reset if a
program other than Bacula uses the drive or when your system is rebooted.

Then according to what the {\bf btape test} command returns, you will probably
need to set the following (see below for an alternative):

\footnotesize
\begin{verbatim}
  Hardware End of Medium = no
  BSF at EOM = yes
  Backward Space Record = no
  Backward Space File = no
  Fast Forward Space File = no
  TWO EOF = yes
\end{verbatim}
\normalsize

Then be sure to run some append tests with Bacula where you start and stop
Bacula between appending to the tape, or use {\bf btape} version 1.35.1 or
greater, which includes simulation of stopping/restarting Bacula.

Please see the file {\bf platforms/freebsd/pthreads-fix.txt} in the main
Bacula directory concerning {\bf important} information concerning
compatibility of Bacula and your system. A much more optimal Device
configuration is shown below, but does not work with all tape drives. Please
test carefully before putting either into production.

Note, for FreeBSD 4.10-RELEASE, using a Sony TSL11000 L100 DDS4 with an
autochanger set to variable block size and DCLZ compression, Brian McDonald
reports that to get Bacula to append correctly between Bacula executions,
the correct values to use are:

\footnotesize
\begin{verbatim}
mt  -f  /dev/nsa0  seteotmodel  1
mt  -f  /dev/nsa0  blocksize  0
mt  -f /dev/nsa0  comp  enable
\end{verbatim}
\normalsize

and

\footnotesize
\begin{verbatim}
  Hardware End of Medium = no
  BSF at EOM = no
  Backward Space Record = no
  Backward Space File = no
  Fast Forward Space File = yes
  TWO EOF = no
\end{verbatim}
\normalsize

This has been confirmed by several other people using different hardware. This
configuration is the preferred one because it uses one EOF and no backspacing
at the end of the tape, which works much more efficiently and reliably with
modern tape drives.

Finally, here is a Device configuration that Danny Butroyd reports to work
correctly with the Overland Powerloader tape library using LT0-2 and
FreeBSD 5.4-Stable:

\footnotesize
\begin{verbatim}
# Overland Powerloader LT02 - 17 slots single drive
Device {
  Name = Powerloader
  Media Type = LT0-2
  Archive Device = /dev/nsa0
  AutomaticMount = yes;
  AlwaysOpen = yes;
  RemovableMedia = yes;
  RandomAccess = no;
  Changer Command = "/usr/local/sbin/mtx-changer %c %o %S %a %d"
  Changer Device = /dev/pass2
  AutoChanger = yes
  Alert Command = "sh -c 'tapeinfo -f %c |grep TapeAlert|cat'"

  # FreeBSD Specific Settings
  Offline On Unmount = no
  Hardware End of Medium = no
  BSF at EOM = yes
  Backward Space Record = no
  Fast Forward Space File = no
  TWO EOF = yes
}

The following Device resource works fine with Dell PowerVault 110T and
120T devices on both FreeBSD 5.3 and on NetBSD 3.0.  It also works
with Sony AIT-2 drives on FreeBSD.
\footnotesize
\begin{verbatim}
Device {
  ...
  # FreeBSD/NetBSD Specific Settings
  Hardware End of Medium = no
  BSF at EOM = yes
  Backward Space Record = no
  Fast Forward Space File = yes
  TWO EOF = yes
}
\end{verbatim}
\normalsize

On FreeBSD version 6.0, it is reported that you can even set
Backward Space Record = yes.



\subsection{Finding your Tape Drives and Autochangers on FreeBSD}
\index[general]{FreeBSD!Finding Tape Drives and Autochangers}
\index[general]{Finding Tape Drives and Autochangers on FreeBSD}

On FreeBSD, you can do a {\bf camcontrol devlist} as root to determine what
drives and autochangers you have. For example,

\footnotesize
\begin{verbatim}
undef# camcontrol devlist
    at scbus0 target 2 lun 0 (pass0,sa0)
    at scbus0 target 4 lun 0 (pass1,sa1)
    at scbus0 target 4 lun 1 (pass2)
\end{verbatim}
\normalsize

from the above, you can determine that there is a tape drive on {\bf /dev/sa0}
and another on {\bf /dev/sa1} in addition since there is a second line for the
drive on {\bf /dev/sa1}, you know can assume that it is the control device for
the autochanger (i.e. {\bf /dev/pass2}). It is also the control device name to
use when invoking the tapeinfo program. E.g.

\footnotesize
\begin{verbatim}
tapeinfo -f /dev/pass2
\end{verbatim}
\normalsize

\label{onstream}

\subsection{Using the OnStream driver on Linux Systems}
\index[general]{Using the OnStream driver on Linux Systems}
\index[general]{Systems!Using the OnStream driver on Linux}

Bacula version 1.33 (not 1.32x) is now working and ready for testing with the
OnStream kernel osst driver version 0.9.14 or above. Osst is available from:
\elink{http://sourceforge.net/projects/osst/}
{http://sourceforge.net/projects/osst/}.

To make Bacula work you must first load the new driver then, as root, do:

\footnotesize
\begin{verbatim}
  mt -f /dev/nosst0 defblksize 32768
\end{verbatim}
\normalsize

Also you must add the following to your Device resource in your Storage
daemon's conf file:

\footnotesize
\begin{verbatim}
 Minimum Block Size = 32768
 Maximum Block Size = 32768
\end{verbatim}
\normalsize

Here is a Device specification provided by Michel Meyers that is known to
work:

\footnotesize
\begin{verbatim}
Device {
  Name = "Onstream DI-30"
  Media Type = "ADR-30"
  Archive Device = /dev/nosst0
  Minimum Block Size = 32768
  Maximum Block Size = 32768
  Hardware End of Medium = yes
  BSF at EOM = no
  Backward Space File = yes
  Fast Forward Space File = yes
  Two EOF = no
  AutomaticMount = yes
  AlwaysOpen = yes
  Removable Media = yes
}
\end{verbatim}
\normalsize

\section{Hardware Compression on EXB-8900}
\index[general]{Hardware Compression on EXB-8900}
\index[general]{EXB-8900!Hardware Compression}

To active, check, or disable the hardware compression feature
on an EXB-8900, use the exabyte MammothTool. You can get it here:
\elink{http://www.exabyte.com/support/online/downloads/index.cfm}
{http://www.exabyte.com/support/online/downloads/index.cfm}.
There is a Solaris version of this tool. With option -C 0 or 1 you
can disable or activate compression. Start this tool without any
options for a small reference.

\label{fill}
\subsection{Using btape to Simulate Filling a Tape}
\index[general]{Using btape to Simulate Filling a Tape}
\index[general]{Tape!Using btape to Simulate Filling}

Because there are often problems with certain tape drives or systems when end
of tape conditions occur, {\bf btape} has a special command {\bf fill} that
causes it to write random data to a tape until the tape fills. It then writes
at least one more Bacula block to a second tape. Finally, it reads back both
tapes to ensure that the data has been written in a way that Bacula can
recover it. Note, there is also a single tape option as noted below, which you
should use rather than the two tape test. See below for more details.

This can be an extremely time consuming process (here it is about 6 hours) to
fill a full tape. Note, that btape writes random data to the tape when it is
filling it. This has two consequences: 1. it takes a bit longer to generate
the data, especially on slow CPUs. 2. the total amount of data is
approximately the real physical capacity of your tape, regardless of whether
or not the tape drive compression is on or off. This is because random data
does not compress very much.

To begin this test, you enter the {\bf fill} command and follow the
instructions. There are two options: the simple single tape option and the
multiple tape option. Please use only the simple single tape option because
the multiple tape option still doesn't work totally correctly. If the single
tape option does not succeed, you should correct the problem before using
Bacula.
\label{RecoveringFiles}

\section{Recovering Files Written With Fixed Block Sizes}
\index[general]{Recovering Files Written With Fixed Block Sizes}

If you have been previously running your tape drive in fixed block mode
(default 512) and Bacula with variable blocks (default), then in version
1.32f-x and 1.34 and above, Bacula will fail to recover files because it does
block spacing, and because the block sizes don't agree between your tape drive
and Bacula it will not work.

The long term solution is to run your drive in variable block mode as
described above. However, if you have written tapes using fixed block sizes,
this can be a bit of a pain. The solution to the problem is: while you are
doing a restore command using a tape written in fixed block size, ensure that
your drive is set to the fixed block size used while the tape was written.
Then when doing the {\bf restore} command in the Console program, do not
answer the prompt {\bf yes/mod/no}. Instead, edit the bootstrap file (the
location is listed in the prompt) using any ASCII editor. Remove all {\bf
VolBlock} lines in the file. When the file is re-written, answer the question,
and Bacula will run without using block positioning, and it should recover
your files.

\label{BlockModes}
\section{Tape Blocking Modes}
\index[general]{Modes!Tape Blocking}
\index[general]{Tape Blocking Modes}

SCSI tapes may either be written in {\bf variable} or {\bf fixed} block sizes.
Newer drives support both modes, but some drives such as the QIC devices
always use fixed block sizes. Bacula attempts to fill and write complete
blocks (default 65K), so that in normal mode (variable block size), Bacula
will always write blocks of the same size except the last block of a Job. If
Bacula is configured to write fixed block sizes, it will pad the last block of
the Job to the correct size. Bacula expects variable tape block size drives to
behave as follows: Each write to the drive results in a single record being
written to the tape. Each read returns a single record. If you request less
bytes than are in the record, only those number of bytes will be returned, but
the entire logical record will have been read (the next read will retrieve the
next record). Thus data from a single write is always returned in a single
read, and sequentially written records are returned by sequential reads.

Bacula expects fixed block size tape drives to behave as follows: If a write
length is greater than the physical block size of the drive, the write will be
written as two blocks each of the fixed physical size. This single write may
become multiple physical records on the tape. (This is not a good situation).
According to the documentation, one may never write an amount of data that is
not the exact multiple of the blocksize (it is not specified if an error
occurs or if the the last record is padded). When reading, it is my
understanding that each read request reads one physical record from the tape.
Due to the complications of fixed block size tape drives, you should avoid
them if possible with Bacula, or you must be ABSOLUTELY certain that you use
fixed block sizes within Bacula that correspond to the physical block size of
the tape drive. This will ensure that Bacula has a one to one correspondence
between what it writes and the physical record on the tape.

Please note that Bacula will not function correctly if it writes a block and
that block is split into two or more physical records on the tape. Bacula
assumes that each write causes a single record to be written, and that it can
sequentially recover each of the blocks it has written by using the same
number of sequential reads as it had written.

\section{Details of Tape Modes}
\index[general]{Modes!Details}
\index[general]{Details of Tape Modes}
Rudolf Cejka has provided the following information concerning
certain tape modes and MTEOM.

\begin{description}
\item[Tape level]
  It is always possible to position filemarks or blocks, whereas
  positioning to the end-of-data is only optional feature, however it is
  implemented very often.  SCSI specification also talks about optional
  sequential filemarks, setmarks and sequential setmarks, but these are not
  implemented so often.  Modern tape drives keep track of file positions in
  built-in chip (AIT, LTO) or at the beginning of the tape (SDLT), so there
  is not any speed difference, if end-of-data or filemarks is used (I have
  heard, that LTO-1 from all 3 manufacturers do not use its chip for file
  locations, but a tape as in SDLT case, and I'm not sure about LTO-2 and
  LTO-3 case).  However there is a big difference, that end-of-data ignores
  file position, whereas filemarks returns the real number of skipped
  files, so OS can track current file number just in filemarks case.

\item[OS level]
  Solaris does use just SCSI SPACE Filemarks, it does not support SCSI
  SPACE End-of-data.  When MTEOM is called, Solaris does use SCSI SPACE
  Filemarks with count = 1048576 for fast mode, and combination of SCSI
  SPACE Filemarks with count = 1 with SCSI SPACE Blocks with count = 1 for
  slow mode, so EOD mark on the tape on some older tape drives is not
  skipped.  File number is always tracked for MTEOM.

  Linux does support both SCSI SPACE Filemarks and End-of-data: When MTEOM
  is called in MT\_ST\_FAST\_MTEOM mode, SCSI SPACE End-of-data is used.
  In the other case, SCSI SPACE Filemarks with count =
  8388607 is used.
  There is no real slow mode like in Solaris - I just expect, that for
  older tape drives Filemarks may be slower than End-of-data, but not so
  much as in Solaris slow mode.  File number is tracked for MTEOM just
  without MT\_ST\_FAST\_MTEOM - when MT\_ST\_FAST\_MTEOM is used, it is not.

  FreeBSD does support both SCSI SPACE Filemarks and End-of-data, but when
  MTEOD (MTEOM) is called, SCSI SPACE End-of-data is always used.  FreeBSD
  never use SCSI SPACE Filemarks for MTEOD. File number is never tracked
  for MTEOD.

\item[Bacula level]
  When {\bf Hardware End of Medium = Yes} is used, MTEOM is called, but it
  does not mean, that hardware End-of-data must be used.  When Hardware End
  of Medium = No, if Fast Forward Space File = Yes, MTFSF with count =
  32767 is used, else Block Read with count = 1 with Forward Space File
  with count = 1 is used, which is really very slow.

\item [Hardware End of Medium = Yes|No]
  The name of this option is misleading and is the source of confusion,
  because it is not the hardware EOM, what is really switched here.

  If I use Yes, OS must not use SCSI SPACE End-of-data, because Bacula
  expects, that there is tracked file number, which is not supported by
  SCSI specification.  Instead, the OS have to use SCSI SPACE Filemarks.

  If I use No, an action depends on Fast Forward Space File.

  When I set {\bf Hardware End of Medium = no}
  and {\bf Fast Forward Space File = no}
  file positioning was very slow
  on my LTO-3 (about ten to 100 minutes), but

  with {\bf Hardware End of Medium = no} and
{\bf Fast Forward Space File = yes}, the time is ten to
100 times faster (about one to two minutes).

\end{description}

\section{Tape Performance Problems}
\index[general]{Tape Performance}
If you have LTO-3 or LTO-4 drives, you should be able to
fairly good transfer rates; from 60 to 150 MB/second, providing
you have fast disks; GigaBit Ethernet connections (probably 2); you are
running multiple simultaneous jobs; you have Bacula data spooling
enabled; your tape block size is set to 131072 or 262144; and
you have set {\bf Maximum File Size = 5G}.

If you are not getting good performance, consider some of the following
suggestions from the Allen Balck on the Bacula Users email list:

\begin{enumerate}
\item You are using an old HBA (i.e. SCSI-1, which only does 5 MB/s)

\item There are other, slower, devices on the SCSI bus. The HBA will
   negotiate the speed of every device down to the speed of the
   slowest.

\item There is a termination problem on the bus (either too much or
   too little termination). The HBA will drop the bus speed in an
   attempt to increase the reliability of the bus.

\item Loose or damaged cabling - this will probably make the HBA "think"
   you have a termination problem and it will react as in 3 above.
\end{enumerate}

See if /var/adm/messages (or /var/log/messages) tells you what the sync
rate of the SCSI devices/bus are. Also, the next time you reboot, the
BIOS may be able to tell you what the rate of each device is.


\section{Autochanger Errors}
\index[general]{Errors!Autochanger}
\index[general]{Autochanger Errors}

If you are getting errors such as:

\footnotesize
\begin{verbatim}
3992 Bad autochanger "load slot 1, drive 1": ERR=Child exited with code 1.
\end{verbatim}
\normalsize

and you are running your Storage daemon as non-root, then most likely
you are having permissions problems with the control channel. Running
as root, set permissions on /dev/sgX so that the userid and group of
your Storage daemon can access the device. You need to ensure that you
all access to the proper control device, and if you don't have any
SCSI disk drives (including SATA drives), you might want to change
the permissions on /dev/sg*.

\section{Syslog Errors}
\index[general]{Errors!Syslog}
\index[general]{Syslog Errors}

If you are getting errors such as:

\footnotesize
\begin{verbatim}
: kernel: st0: MTSETDRVBUFFER only allowed for root
\end{verbatim}
\normalsize

you are most likely running your Storage daemon as non-root, and
Bacula is attempting to set the correct OS buffering to correspond
to your Device resource. Most OSes allow only root to issue this
ioctl command. In general, the message can be ignored providing
you are sure that your OS parameters are properly configured as
described earlier in this manual.  If you are running your Storage daemon
as root, you should not be getting these system log messages, and if
you are, something is probably wrong.


\section{Autochanger}

\subsection{Testing Autochanger and Adapting mtx-changer script}
\label{AutochangerTesting}
\index[general]{Autochanger!Testing}
\index[general]{Autochanger!mtx-changer}
\index[general]{Command!mtx-changer}
\index[general]{Problem!Autochanger}
\index[general]{Problem!mtx-changer}

In case, Bareos does not work well with the Autochanger,
it is preferable to
"hand-test" that the changer works. To do so, we suggest you do the
following commands:

Make sure Bareos is not running.

\command{/usr/lib/bareos/scripts/mtx-changer /dev/sg0 list 0 /dev/nst0 0}
\index[sd]{mtx-changer list}

This command should print:

\footnotesize
\begin{verbatim}
   1:
   2:
   3:
   ...

\end{verbatim}
\normalsize

or one number per line for each slot that is  occupied in your changer, and
the number should be  terminated by a colon ({\bf :}). If your changer has
barcodes, the barcode will follow the colon.  If an error message is printed,
you must resolve the  problem (e.g. try a different SCSI control device name
if {\bf /dev/sg0}  is incorrect). For example, on FreeBSD systems, the
autochanger  SCSI control device is generally {\bf /dev/pass2}.

\command{/usr/lib/bareos/scripts/mtx-changer /dev/sg0 listall 0 /dev/nst0 0}
\index[sd]{mtx-changer listall}

This command should print:

\footnotesize
\begin{verbatim}
 Drive content:         D:Drive num:F:Slot loaded:Volume Name
 D:0:F:2:vol2        or D:Drive num:E
 D:1:F:42:vol42
 D:3:E

 Slot content:
 S:1:F:vol1             S:Slot num:F:Volume Name
 S:2:E               or S:Slot num:E
 S:3:F:vol4

 Import/Export tray slots:
 I:10:F:vol10           I:Slot num:F:Volume Name
 I:11:E              or I:Slot num:E
 I:12:F:vol40

\end{verbatim}
\normalsize

\command{/usr/lib/bareos/scripts/mtx-changer /dev/sg0 transfer 1 2}
\index[sd]{mtx-changer listall}

This command should transfer a volume from source (1) to destination (2)

\command{/usr/lib/bareos/scripts/mtx-changer /dev/sg0 slots}
\index[sd]{mtx-changer slots}

This command should return the number of slots in your autochanger.

\command{/usr/lib/bareos/scripts/mtx-changer /dev/sg0 unload 1 /dev/nst0 0}
\index[sd]{mtx-changer unload}

   If a tape is loaded from slot 1, this should cause it to be unloaded.

\command{/usr/lib/bareos/scripts/mtx-changer /dev/sg0 load 3 /dev/nst0 0}
\index[sd]{mtx-changer load}

Assuming you have a tape in slot 3,  it will be loaded into drive (0).


\command{/usr/lib/bareos/scripts/mtx-changer /dev/sg0 loaded 0 /dev/nst0 0}
\index[sd]{mtx-changer loaded}

It should print "3"
Note, we have used an "illegal" slot number 0. In this case, it is simply
ignored because the slot number is not used.  However, it must be specified
because the drive parameter at the end of the command is needed to select
the correct drive.

\command{/usr/lib/bareos/scripts/mtx-changer /dev/sg0 unload 3 /dev/nst0 0}
\index[sd]{mtx-changer unload}

will unload the tape into slot 3.


Once all the above commands work correctly, assuming that you have the right
{\bf Changer Command} in your configuration, Bareos should be able to operate
the changer. The only remaining area of problems will be if your autoloader
needs some time to get the tape loaded after issuing the command. After the
{\bf mtx-changer} script returns, Bareos will immediately rewind and read the
tape. If Bareos gets rewind I/O errors after a tape change, you will probably
need to configure the \parameter{load_sleep} paramenter in the config file \file{/etc/bareos/mtx-changer.conf}.
You can test whether or not you need a {\bf sleep} by putting the following
commands into a file and running it as a script:

\footnotesize
\begin{verbatim}
#!/bin/sh
/usr/lib/bareos/scripts/mtx-changer /dev/sg0 unload 1 /dev/nst0 0
/usr/lib/bareos/scripts/mtx-changer /dev/sg0 load 3 /dev/nst0 0
mt -f /dev/st0 rewind
mt -f /dev/st0 weof
\end{verbatim}
\normalsize

If the above script runs, you probably have no timing problems. If it does not
run, start by putting a {\bf sleep 30} or possibly a {\bf sleep 60} in the
script just after the mtx-changer load command. If that works, then you should
configure the \parameter{load_sleep} paramenter in the config file \file{/etc/bareos/mtx-changer.conf} to the specified value  so that it will be
effective when Bareos runs.

A second problem that comes up with a small number of autochangers is that
they need to have the cartridge ejected before it can be removed. If this is
the case, the {\bf load 3} will never succeed regardless of how long you wait.
If this seems to be your problem, you can insert an eject just after the
unload so that the script looks like:

\footnotesize
\begin{verbatim}
#!/bin/sh
/usr/lib/bareos/scripts/mtx-changer /dev/sg0 unload 1 /dev/nst0 0
mt -f /dev/st0 offline
/usr/lib/bareos/scripts/mtx-changer /dev/sg0 load 3 /dev/nst0 0
mt -f /dev/st0 rewind
mt -f /dev/st0 weof
\end{verbatim}
\normalsize

If this solves your problems, set the parameter \parameter{offline} in the config file \file{/etc/bareos/mtx-changer.conf} to "1".

% \TODO{provide a test script, distributed with Bareos}

%As noted earlier, there are several scripts in {\bf
%{\textless}bareos-source{\textgreater}/examples/devices} that implement the above features,
%so they may be a help to you in getting your script to work.

%If Bareos complains "Rewind error on /dev/nst0. ERR=Input/output error." you
%most likely need more sleep time in your {\bf mtx-changer} before returning to
%Bareos after a load command has been completed.


\section{Restore}

\subsection{Restore a pruned job using a pattern}
\index[general]{Restore!pruned job}
\index[general]{Problem!Restore!pruned job}
\index[general]{Regex}

  It is possible to configure Bareos in a way,
  that job information are still stored in the Bareos catalog,
  while the individual file information are already pruned.

  If all File records are pruned from the catalog
  for a Job, normally Bareos can restore only all files saved. That
  is there is no way using the catalog to select individual files.
  With this new feature, Bareos will ask if you want to specify a Regex
  expression for extracting only a part of the full backup.

\begin{verbatim}
  Building directory tree for JobId(s) 1,3 ...
  There were no files inserted into the tree, so file selection
  is not possible.Most likely your retention policy pruned the files

  Do you want to restore all the files? (yes|no): no

  Regexp matching files to restore? (empty to abort): /etc/.*
  Bootstrap records written to /tmp/regress/working/zog4-dir.restore.1.bsr
\end{verbatim}

  See also \ilink{FileRegex bsr option}{FileRegex} for more information.


\subsection{Problems Restoring Files}
\index[general]{Restore!Files!Problem}
\index[general]{Problem!Restoring Files}
\index[general]{Problem!Tape!fixed mode}
\index[general]{Problem!Tape!variable mode}


The most frequent problems users have restoring files are error messages such
as:

\footnotesize
\begin{verbatim}
04-Jan 00:33 z217-sd: RestoreFiles.2005-01-04_00.31.04 Error:
block.c:868 Volume data error at 20:0! Short block of 512 bytes on
device /dev/tape discarded.
\end{verbatim}
\normalsize

or

\footnotesize
\begin{verbatim}
04-Jan 00:33 z217-sd: RestoreFiles.2005-01-04_00.31.04 Error:
block.c:264 Volume data error at 20:0! Wanted ID: "BB02", got ".".
Buffer discarded.
\end{verbatim}
\normalsize

Both these kinds of messages indicate that you were probably running your tape
drive in fixed block mode rather than variable block mode. Fixed block mode
will work with any program that reads tapes sequentially such as tar, but
Bareos repositions the tape on a block basis when restoring files because this
will speed up the restore by orders of magnitude when only a few files are being
restored. There are several ways that you can attempt to recover from this
unfortunate situation.

Try the following things, each separately, and reset your Device resource to
what it is now after each individual test:

\begin{enumerate}
\item Set "Block Positioning = no" in your Device resource  and try the
   restore. This is a new directive and untested.

\item Set "Minimum Block Size = 512" and "Maximum  Block Size = 512" and
   try the restore.  If you are able to determine the block size your drive
   was previously using, you should try that size if 512 does not work.
   This is a really horrible solution, and it is not at all recommended
   to continue backing up your data without correcting this condition.
   Please see the Tape Testing chapter for more on this.

\item Try editing the restore.bsr file at the Run xxx yes/mod/no prompt
   before starting the restore job and remove all the VolBlock statements.
   These are what causes Bareos to reposition the tape, and where problems
   occur if you have a fixed block size set for your drive.  The VolFile
   commands also cause repositioning, but this will work regardless of the
   block size.

\item Use bextract to extract the files you want -- it reads the  Volume
   sequentially if you use the include list feature, or if you use a .bsr
   file, but remove all the VolBlock statements after the .bsr file is
   created (at the Run yes/mod/no) prompt but before you start the restore.
\end{enumerate}


\subsection{Restoring Files Can Be Slow}
\index[general]{Restore!slow}
\index[general]{Problem!Restore!slow}


Restoring files is generally {\bf much} slower than backing them up for several
reasons. The first is that during a backup the tape is normally already
positioned and Bareos only needs to write. On the other hand, because restoring
files is done so rarely, Bareos keeps only the start file and block on the
tape for the whole job rather than on a file by file basis which would use
quite a lot of space in the catalog.

Bareos will forward space to the correct file mark on the tape for the Job,
then forward space to the correct block, and finally sequentially read each
record until it gets to the correct one(s) for the file or files you want to
restore. Once the desired files are restored, Bareos will stop reading the
tape.

Finally, instead of just reading a file for backup, during the restore, Bareos
must create the file, and the operating system must allocate disk space for
the file as Bareos is restoring it.

For all the above reasons the restore process is generally much slower than
backing up (sometimes it takes three times as long).


\subsection{Restoring When Things Go Wrong}
\label{database_restore}
\index[general]{Catalog!Restore}
\index[general]{Restore!Catalog}
\index[general]{Disaster!Recovery!Catalog}
\index[general]{Problem!Repair Catalog}


This and the following sections will try to present a few of the kinds of
problems that can come up making restoring more difficult. We will try to
provide a few ideas how to get out of these problem situations.
In addition to what is presented here, there is more specific information
on restoring a \ilink{Client}{restore_client} and your
\ilink{Server}{restore_server} in the \ilink{Disaster Recovery Using
Bareos}{RescueChapter} chapter of this manual.

\begin{description}
\item[Problem]
   My database is broken.
\item[Solution]
   For SQLite, use the vacuum command to try to fix the database. For either
   MySQL or PostgreSQL, see the vendor's documentation. They have specific tools
   that check and repair databases, see the \ilink{database
   repair}{DatabaseRepair} sections of this manual for links to vendor
   information.

   Assuming the above does not resolve the problem, you will need to restore
   or rebuild your catalog.  Note, if it is a matter of some
   inconsistencies in the Bareos tables rather than a broken database, then
   running \ilink{bareos-dbcheck}{bareos-dbcheck} might help, but you will need to ensure
   that your database indexes are properly setup.

\item[Problem]
   How do I restore my catalog?
\item[Solution with a Catalog backup]
   If you have backed up your database nightly (as you should) and you
   have made a bootstrap file, you can immediately load back your
   database (or the ASCII SQL output).  Make a copy of your current
   database, then re-initialize it, by running the following scripts:
\begin{verbatim}
   ./drop_bareos_tables
   ./make_bareos_tables
\end{verbatim}
   After re-initializing the database, you should be able to run
   Bareos. If you now try to use the restore command, it will not
   work because the database will be empty. However, you can manually
   run a restore job and specify your bootstrap file. You do so
   by entering the {\bf run} command in the console and selecting the
   restore job.  If you are using the default bareos-dir.conf, this
   Job will be named {\bf RestoreFiles}. Most likely it will prompt
   you with something such as:

\footnotesize
\begin{verbatim}
Run Restore job
JobName:    RestoreFiles
Bootstrap:  /home/user/bareos/working/restore.bsr
Where:      /tmp/bareos-restores
Replace:    always
FileSet:    Full Set
Client:     rufus-fd
Storage:    File
When:       2005-07-10 17:33:40
Catalog:    MyCatalog
Priority:   10
OK to run? (yes/mod/no):
\end{verbatim}
\normalsize

   A number of the items will be different in your case.  What you want to
   do is: to use the mod option to change the Bootstrap to point to your
   saved bootstrap file; and to make sure all the other items such as
   Client, Storage, Catalog, and Where are correct.  The FileSet is not
   used when you specify a bootstrap file.  Once you have set all the
   correct values, run the Job and it will restore the backup of your
   database, which is most likely an ASCII dump.

   You will then need to follow the instructions for your
   database type to recreate the database from the ASCII backup file.
   See the \ilink {Catalog Maintenance}{CatMaintenanceChapter} chapter of
   this manual for examples of the command needed to restore a
   database from an ASCII dump (they are shown in the Compacting Your
   XXX Database sections).

   Also, please note that after you restore your database from an ASCII
   backup, you do NOT want to do a {\bf make\_bareos\_tables}  command, or
   you will probably erase your newly restored database tables.


\item[Solution with a Job listing]
   If you did save your database but did not make a bootstrap file, then
   recovering the database is more difficult.  You will probably need to
   use \command{bextract} to extract the backup copy.  First you should locate the
   listing of the job report from the last catalog backup.  It has
   important information that will allow you to quickly find your database
   file.  For example, in the job report for the CatalogBackup shown below,
   the critical items are the Volume name(s), the Volume Session Id and the
   Volume Session Time.  If you know those, you can easily restore your
   Catalog.

\footnotesize
\begin{verbatim}
22-Apr 10:22 HeadMan: Start Backup JobId 7510,
Job=CatalogBackup.2005-04-22_01.10.0
22-Apr 10:23 HeadMan: Bareos 1.37.14 (21Apr05): 22-Apr-2005 10:23:06
  JobId:                  7510
  Job:                    CatalogBackup.2005-04-22_01.10.00
  Backup Level:           Full
  Client:                 Polymatou
  FileSet:                "CatalogFile" 2003-04-10 01:24:01
  Pool:                   "Default"
  Storage:                "DLTDrive"
  Start time:             22-Apr-2005 10:21:00
  End time:               22-Apr-2005 10:23:06
  FD Files Written:       1
  SD Files Written:       1
  FD Bytes Written:       210,739,395
  SD Bytes Written:       210,739,521
  Rate:                   1672.5 KB/s
  Software Compression:   None
  Volume name(s):         DLT-22Apr05
  Volume Session Id:      11
  Volume Session Time:    1114075126
  Last Volume Bytes:      1,428,240,465
  Non-fatal FD errors:    0
  SD Errors:              0
  FD termination status:  OK
  SD termination status:  OK
  Termination:            Backup OK
\end{verbatim}
\normalsize

  From the above information, you can manually create a bootstrap file,
  and then follow the instructions given above for restoring your database.
  A reconstructed bootstrap file for the above backup Job would look
  like the following:

\footnotesize
\begin{verbatim}
Volume="DLT-22Apr05"
VolSessionId=11
VolSessionTime=1114075126
FileIndex=1-1
\end{verbatim}
\normalsize

  Where we have inserted the Volume name, Volume Session Id, and Volume
  Session Time that correspond to the values in the job report.  We've also
  used a FileIndex of one, which will always be the case providing that
  there was only one file backed up in the job.

  The disadvantage of this bootstrap file compared to what is created when
  you ask for one to be written, is that there is no File and Block
  specified, so the restore code must search all data in the Volume to find
  the requested file.  A fully specified bootstrap file would have the File
  and Blocks specified as follows:

\footnotesize
\begin{verbatim}
Volume="DLT-22Apr05"
VolSessionId=11
VolSessionTime=1114075126
VolFile=118-118
VolBlock=0-4053
FileIndex=1-1
\end{verbatim}
\normalsize

   Once you have restored the ASCII dump of the database,
   you will then to follow the instructions for your
   database type to recreate the database from the ASCII backup file.
   See the \ilink {Catalog Maintenance}{CatMaintenanceChapter} chapter of
   this manual for examples of the command needed to restore a
   database from an ASCII dump (they are shown in the Compacting Your
   XXX Database sections).

   Also, please note that after you restore your database from an ASCII
   backup, you do NOT want to do a {\bf make\_bareos\_tables}  command, or
   you will probably erase your newly restored database tables.

\item [Solution without a Job Listing]
   If you do not have a job listing, then it is a bit more difficult.
   Either you use the \ilink{bscan}{bscan} program to scan the contents
   of your tape into a database, which can be very time consuming
   depending on the size of the tape, or you can use the \ilink{bls}{bls}
   program to list everything on the tape, and reconstruct a bootstrap
   file from the bls listing for the file or files you want following
   the instructions given above.

   There is a specific example of how to use {\bf bls} below.

\item [Problem]
   Trying to restore the last known good full backup by specifying
   item 3 on the restore menu then the JobId to restore, but Bareos
   then reports:

\footnotesize
\begin{verbatim}
   1 Job 0 Files
\end{verbatim}
\normalsize
   and restores nothing.

\item[Solution]
   Most likely the File records were pruned from the database either due
   to the File Retention period expiring or by explicitly purging the
   Job. By using the "llist jobid=nn" command, you can obtain all the
   important information about the job:

\footnotesize
\begin{verbatim}
llist jobid=120
           JobId: 120
             Job: save.2005-12-05_18.27.33
        Job.Name: save
     PurgedFiles: 0
            Type: B
           Level: F
    Job.ClientId: 1
     Client.Name: Rufus
       JobStatus: T
       SchedTime: 2005-12-05 18:27:32
       StartTime: 2005-12-05 18:27:35
         EndTime: 2005-12-05 18:27:37
        JobTDate: 1133803657
    VolSessionId: 1
  VolSessionTime: 1133803624
        JobFiles: 236
       JobErrors: 0
 JobMissingFiles: 0
      Job.PoolId: 4
       Pool.Name: Full
   Job.FileSetId: 1
 FileSet.FileSet: BackupSet
\end{verbatim}
\normalsize

   Then you can find the Volume(s) used by doing:

\footnotesize
\begin{verbatim}
sql
select VolumeName from JobMedia,Media where JobId=1 and JobMedia.MediaId=Media.MediaId;
\end{verbatim}
\normalsize

   Finally, you can create a bootstrap file as described in the previous
   problem above using this information.

   Bareos will ask you if
   you would like to restore all the files in the job, and it will
   collect the above information and write the bootstrap file for
   you.

\item [Problem]
  You don't have a bootstrap file, and you don't have the Job report for
  the backup of your database, but you did backup the database, and you
  know the Volume to which it was backed up.

\item [Solution]
  Either \command{bscan} the tape (see below for bscanning), or better use \command{bls}
  to find where it is on the tape, then use \command{bextract} to
  restore the database.  For example,


\footnotesize
\begin{verbatim}
./bls -j -V DLT-22Apr05 /dev/nst0
\end{verbatim}
\normalsize
  Might produce the following output:
\footnotesize
\begin{verbatim}
bls: butil.c:258 Using device: "/dev/nst0" for reading.
21-Jul 18:34 bls: Ready to read from volume "DLT-22Apr05" on device "DLTDrive"
(/dev/nst0).
Volume Record: File:blk=0:0 SessId=11 SessTime=1114075126 JobId=0 DataLen=164
...
Begin Job Session Record: File:blk=118:0 SessId=11 SessTime=1114075126
JobId=7510
   Job=CatalogBackup.2005-04-22_01.10.0 Date=22-Apr-2005 10:21:00 Level=F Type=B
End Job Session Record: File:blk=118:4053 SessId=11 SessTime=1114075126
JobId=7510
   Date=22-Apr-2005 10:23:06 Level=F Type=B Files=1 Bytes=210,739,395 Errors=0
Status=T
...
21-Jul 18:34 bls: End of Volume at file 201 on device "DLTDrive" (/dev/nst0),
Volume "DLT-22Apr05"
21-Jul 18:34 bls: End of all volumes.
\end{verbatim}
\normalsize
  Of course, there will be many more records printed, but we have indicated
  the essential lines of output. From the information on the Begin Job and End
  Job Session Records, you can reconstruct a bootstrap file such as the one
  shown above.

\item[Problem]
  How can I find where a file is stored?
\item[Solution]
  Normally, it is not necessary, you just use the {\bf restore} command to
  restore the most recently saved version (menu option 5), or a version
  saved before a given date (menu option 8).  If you know the JobId of the
  job in which it was saved, you can use menu option 3 to enter that JobId.

  If you would like to know the JobId where a file was saved, select
  restore menu option 2.

  You can also use the {\bf query} command to find information such as:
\footnotesize
\begin{verbatim}
*query
Available queries:
     1: List up to 20 places where a File is saved regardless of the
directory
     2: List where the most recent copies of a file are saved
     3: List last 20 Full Backups for a Client
     4: List all backups for a Client after a specified time
     5: List all backups for a Client
     6: List Volume Attributes for a selected Volume
     7: List Volumes used by selected JobId
     8: List Volumes to Restore All Files
     9: List Pool Attributes for a selected Pool
    10: List total files/bytes by Job
    11: List total files/bytes by Volume
    12: List Files for a selected JobId
    13: List Jobs stored on a selected MediaId
    14: List Jobs stored for a given Volume name
    15: List Volumes Bareos thinks are in changer
    16: List Volumes likely to need replacement from age or errors
Choose a query (1-16):
\end{verbatim}
\normalsize

\item[Problem]
  I didn't backup my database. What do I do now?
\item[Solution]
  This is probably the worst of all cases, and you will probably have
  to re-create your database from scratch and then bscan in all your
  volumes, which is a very long, painful, and inexact process.

There are basically three steps to take:

\begin{enumerate}
\item Ensure that your SQL server is running (MySQL or PostgreSQL)
   and that the Bareos database (normally bareos) exists.  See the
   \ilink{Installation}{CreateDatabase} chapter of the manual.
\item Ensure that the Bareos databases are created. This is also
   described at the above link.
\item Start and stop the Bareos Director using the propriate
   bareos-dir.conf file so that it can create the Client and
   Storage records which are not stored on the Volumes.  Without these
   records, scanning is unable to connect the Job records to the proper
  client.
\end{enumerate}

When the above is complete, you can begin bscanning your Volumes. Please
see the \ilink{bscan}{bscan} section of the Volume Utility Tools of this
chapter for more details.

\end{description}


\TODO{add the debug chapter?}
