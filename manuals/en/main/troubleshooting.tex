
\section{Tape Drive}
\label{TapeTestingChapter}

\TODO{Tape Testing Chapter is to be written}

\section{Autochanger}

\subsection{Testing Autochanger and Adapting mtx-changer script}
\label{AutochangerTesting}
\index[general]{Autochanger!Testing}
\index[general]{Autochanger!mtx-changer}
\index[general]{Command!mtx-changer}
\index[general]{Problem!Autochanger}
\index[general]{Problem!mtx-changer}

In case, Bareos does not work well with the Autochanger,
it is preferable to
"hand-test" that the changer works. To do so, we suggest you do the
following commands:

Make sure Bareos is not running.

\command{/usr/lib/bareos/scripts/mtx-changer /dev/sg0 list 0 /dev/nst0 0}
\index[sd]{mtx-changer list}

This command should print:

\footnotesize
\begin{verbatim}
   1:
   2:
   3:
   ...

\end{verbatim}
\normalsize

or one number per line for each slot that is  occupied in your changer, and
the number should be  terminated by a colon ({\bf :}). If your changer has
barcodes, the barcode will follow the colon.  If an error message is printed,
you must resolve the  problem (e.g. try a different SCSI control device name
if {\bf /dev/sg0}  is incorrect). For example, on FreeBSD systems, the
autochanger  SCSI control device is generally {\bf /dev/pass2}.

\command{/usr/lib/bareos/scripts/mtx-changer /dev/sg0 listall 0 /dev/nst0 0}
\index[sd]{mtx-changer listall}

This command should print:

\footnotesize
\begin{verbatim}
 Drive content:         D:Drive num:F:Slot loaded:Volume Name
 D:0:F:2:vol2        or D:Drive num:E
 D:1:F:42:vol42
 D:3:E

 Slot content:
 S:1:F:vol1             S:Slot num:F:Volume Name
 S:2:E               or S:Slot num:E
 S:3:F:vol4

 Import/Export tray slots:
 I:10:F:vol10           I:Slot num:F:Volume Name
 I:11:E              or I:Slot num:E
 I:12:F:vol40

\end{verbatim}
\normalsize

\command{/usr/lib/bareos/scripts/mtx-changer /dev/sg0 transfer 1 2}
\index[sd]{mtx-changer listall}

This command should transfer a volume from source (1) to destination (2)

\command{/usr/lib/bareos/scripts/mtx-changer /dev/sg0 slots}
\index[sd]{mtx-changer slots}

This command should return the number of slots in your autochanger.

\command{/usr/lib/bareos/scripts/mtx-changer /dev/sg0 unload 1 /dev/nst0 0}
\index[sd]{mtx-changer unload}

   If a tape is loaded from slot 1, this should cause it to be unloaded.

\command{/usr/lib/bareos/scripts/mtx-changer /dev/sg0 load 3 /dev/nst0 0}
\index[sd]{mtx-changer load}

Assuming you have a tape in slot 3,  it will be loaded into drive (0).


\command{/usr/lib/bareos/scripts/mtx-changer /dev/sg0 loaded 0 /dev/nst0 0}
\index[sd]{mtx-changer loaded}

It should print "3"
Note, we have used an "illegal" slot number 0. In this case, it is simply
ignored because the slot number is not used.  However, it must be specified
because the drive parameter at the end of the command is needed to select
the correct drive.

\command{/usr/lib/bareos/scripts/mtx-changer /dev/sg0 unload 3 /dev/nst0 0}
\index[sd]{mtx-changer unload}

will unload the tape into slot 3.


Once all the above commands work correctly, assuming that you have the right
{\bf Changer Command} in your configuration, Bareos should be able to operate
the changer. The only remaining area of problems will be if your autoloader
needs some time to get the tape loaded after issuing the command. After the
{\bf mtx-changer} script returns, Bareos will immediately rewind and read the
tape. If Bareos gets rewind I/O errors after a tape change, you will probably
need to configure the \parameter{load_sleep} paramenter in the config file \file{/etc/bareos/mtx-changer.conf}.
You can test whether or not you need a {\bf sleep} by putting the following
commands into a file and running it as a script:

\footnotesize
\begin{verbatim}
#!/bin/sh
/usr/lib/bareos/scripts/mtx-changer /dev/sg0 unload 1 /dev/nst0 0
/usr/lib/bareos/scripts/mtx-changer /dev/sg0 load 3 /dev/nst0 0
mt -f /dev/st0 rewind
mt -f /dev/st0 weof
\end{verbatim}
\normalsize

If the above script runs, you probably have no timing problems. If it does not
run, start by putting a {\bf sleep 30} or possibly a {\bf sleep 60} in the
script just after the mtx-changer load command. If that works, then you should
configure the \parameter{load_sleep} paramenter in the config file \file{/etc/bareos/mtx-changer.conf} to the specified value  so that it will be
effective when Bareos runs.

A second problem that comes up with a small number of autochangers is that
they need to have the cartridge ejected before it can be removed. If this is
the case, the {\bf load 3} will never succeed regardless of how long you wait.
If this seems to be your problem, you can insert an eject just after the
unload so that the script looks like:

\footnotesize
\begin{verbatim}
#!/bin/sh
/usr/lib/bareos/scripts/mtx-changer /dev/sg0 unload 1 /dev/nst0 0
mt -f /dev/st0 offline
/usr/lib/bareos/scripts/mtx-changer /dev/sg0 load 3 /dev/nst0 0
mt -f /dev/st0 rewind
mt -f /dev/st0 weof
\end{verbatim}
\normalsize

If this solves your problems, set the parameter \parameter{offline} in the config file \file{/etc/bareos/mtx-changer.conf} to "1".

% \TODO{provide a test script, distributed with Bareos}

%As noted earlier, there are several scripts in {\bf
%{\textless}bareos-source{\textgreater}/examples/devices} that implement the above features,
%so they may be a help to you in getting your script to work.

%If Bareos complains "Rewind error on /dev/nst0. ERR=Input/output error." you
%most likely need more sleep time in your {\bf mtx-changer} before returning to
%Bareos after a load command has been completed.


\section{Restore}

\subsection{Restore a pruned job using a pattern}
\index[general]{Restore!pruned job}
\index[general]{Problem!Restore!pruned job}
\index[general]{Regex}

  It is possible to configure Bareos in a way,
  that job information are still stored in the Bareos catalog,
  while the individual file information are already pruned.

  If all File records are pruned from the catalog
  for a Job, normally Bareos can restore only all files saved. That
  is there is no way using the catalog to select individual files.
  With this new feature, Bareos will ask if you want to specify a Regex
  expression for extracting only a part of the full backup.

\begin{verbatim}
  Building directory tree for JobId(s) 1,3 ...
  There were no files inserted into the tree, so file selection
  is not possible.Most likely your retention policy pruned the files

  Do you want to restore all the files? (yes|no): no

  Regexp matching files to restore? (empty to abort): /etc/.*
  Bootstrap records written to /tmp/regress/working/zog4-dir.restore.1.bsr
\end{verbatim}

  See also \ilink{FileRegex bsr option}{FileRegex} for more information.


\subsection{Problems Restoring Files}
\index[general]{Restore!Files!Problem}
\index[general]{Problem!Restoring Files}
\index[general]{Problem!Tape!fixed mode}
\index[general]{Problem!Tape!variable mode}


The most frequent problems users have restoring files are error messages such
as:

\footnotesize
\begin{verbatim}
04-Jan 00:33 z217-sd: RestoreFiles.2005-01-04_00.31.04 Error:
block.c:868 Volume data error at 20:0! Short block of 512 bytes on
device /dev/tape discarded.
\end{verbatim}
\normalsize

or

\footnotesize
\begin{verbatim}
04-Jan 00:33 z217-sd: RestoreFiles.2005-01-04_00.31.04 Error:
block.c:264 Volume data error at 20:0! Wanted ID: "BB02", got ".".
Buffer discarded.
\end{verbatim}
\normalsize

Both these kinds of messages indicate that you were probably running your tape
drive in fixed block mode rather than variable block mode. Fixed block mode
will work with any program that reads tapes sequentially such as tar, but
Bareos repositions the tape on a block basis when restoring files because this
will speed up the restore by orders of magnitude when only a few files are being
restored. There are several ways that you can attempt to recover from this
unfortunate situation.

Try the following things, each separately, and reset your Device resource to
what it is now after each individual test:

\begin{enumerate}
\item Set "Block Positioning = no" in your Device resource  and try the
   restore. This is a new directive and untested.

\item Set "Minimum Block Size = 512" and "Maximum  Block Size = 512" and
   try the restore.  If you are able to determine the block size your drive
   was previously using, you should try that size if 512 does not work.
   This is a really horrible solution, and it is not at all recommended
   to continue backing up your data without correcting this condition.
   Please see the Tape Testing chapter for more on this.

\item Try editing the restore.bsr file at the Run xxx yes/mod/no prompt
   before starting the restore job and remove all the VolBlock statements.
   These are what causes Bareos to reposition the tape, and where problems
   occur if you have a fixed block size set for your drive.  The VolFile
   commands also cause repositioning, but this will work regardless of the
   block size.

\item Use bextract to extract the files you want -- it reads the  Volume
   sequentially if you use the include list feature, or if you use a .bsr
   file, but remove all the VolBlock statements after the .bsr file is
   created (at the Run yes/mod/no) prompt but before you start the restore.
\end{enumerate}


\subsection{Restoring Files Can Be Slow}
\index[general]{Restore!slow}
\index[general]{Problem!Restore!slow}


Restoring files is generally {\bf much} slower than backing them up for several
reasons. The first is that during a backup the tape is normally already
positioned and Bareos only needs to write. On the other hand, because restoring
files is done so rarely, Bareos keeps only the start file and block on the
tape for the whole job rather than on a file by file basis which would use
quite a lot of space in the catalog.

Bareos will forward space to the correct file mark on the tape for the Job,
then forward space to the correct block, and finally sequentially read each
record until it gets to the correct one(s) for the file or files you want to
restore. Once the desired files are restored, Bareos will stop reading the
tape.

Finally, instead of just reading a file for backup, during the restore, Bareos
must create the file, and the operating system must allocate disk space for
the file as Bareos is restoring it.

For all the above reasons the restore process is generally much slower than
backing up (sometimes it takes three times as long).


\subsection{Restoring When Things Go Wrong}
\label{database_restore}
\index[general]{Catalog!Restore}
\index[general]{Restore!Catalog}
\index[general]{Disaster!Recovery!Catalog}
\index[general]{Problem!Repair Catalog}


This and the following sections will try to present a few of the kinds of
problems that can come up making restoring more difficult. We will try to
provide a few ideas how to get out of these problem situations.
In addition to what is presented here, there is more specific information
on restoring a \ilink{Client}{restore_client} and your
\ilink{Server}{restore_server} in the \ilink{Disaster Recovery Using
Bareos}{RescueChapter} chapter of this manual.

\begin{description}
\item[Problem]
   My database is broken.
\item[Solution]
   For SQLite, use the vacuum command to try to fix the database. For either
   MySQL or PostgreSQL, see the vendor's documentation. They have specific tools
   that check and repair databases, see the \ilink{database
   repair}{DatabaseRepair} sections of this manual for links to vendor
   information.

   Assuming the above does not resolve the problem, you will need to restore
   or rebuild your catalog.  Note, if it is a matter of some
   inconsistencies in the Bareos tables rather than a broken database, then
   running \ilink{bareos-dbcheck}{bareos-dbcheck} might help, but you will need to ensure
   that your database indexes are properly setup.

\item[Problem]
   How do I restore my catalog?
\item[Solution with a Catalog backup]
   If you have backed up your database nightly (as you should) and you
   have made a bootstrap file, you can immediately load back your
   database (or the ASCII SQL output).  Make a copy of your current
   database, then re-initialize it, by running the following scripts:
\begin{verbatim}
   ./drop_bareos_tables
   ./make_bareos_tables
\end{verbatim}
   After re-initializing the database, you should be able to run
   Bareos. If you now try to use the restore command, it will not
   work because the database will be empty. However, you can manually
   run a restore job and specify your bootstrap file. You do so
   by entering the {\bf run} command in the console and selecting the
   restore job.  If you are using the default bareos-dir.conf, this
   Job will be named {\bf RestoreFiles}. Most likely it will prompt
   you with something such as:

\footnotesize
\begin{verbatim}
Run Restore job
JobName:    RestoreFiles
Bootstrap:  /home/kern/bareos/working/restore.bsr
Where:      /tmp/bareos-restores
Replace:    always
FileSet:    Full Set
Client:     rufus-fd
Storage:    File
When:       2005-07-10 17:33:40
Catalog:    MyCatalog
Priority:   10
OK to run? (yes/mod/no):
\end{verbatim}
\normalsize

   A number of the items will be different in your case.  What you want to
   do is: to use the mod option to change the Bootstrap to point to your
   saved bootstrap file; and to make sure all the other items such as
   Client, Storage, Catalog, and Where are correct.  The FileSet is not
   used when you specify a bootstrap file.  Once you have set all the
   correct values, run the Job and it will restore the backup of your
   database, which is most likely an ASCII dump.

   You will then need to follow the instructions for your
   database type to recreate the database from the ASCII backup file.
   See the \ilink {Catalog Maintenance}{CatMaintenanceChapter} chapter of
   this manual for examples of the command needed to restore a
   database from an ASCII dump (they are shown in the Compacting Your
   XXX Database sections).

   Also, please note that after you restore your database from an ASCII
   backup, you do NOT want to do a {\bf make\_bareos\_tables}  command, or
   you will probably erase your newly restored database tables.


\item[Solution with a Job listing]
   If you did save your database but did not make a bootstrap file, then
   recovering the database is more difficult.  You will probably need to
   use \command{bextract} to extract the backup copy.  First you should locate the
   listing of the job report from the last catalog backup.  It has
   important information that will allow you to quickly find your database
   file.  For example, in the job report for the CatalogBackup shown below,
   the critical items are the Volume name(s), the Volume Session Id and the
   Volume Session Time.  If you know those, you can easily restore your
   Catalog.

\footnotesize
\begin{verbatim}
22-Apr 10:22 HeadMan: Start Backup JobId 7510,
Job=CatalogBackup.2005-04-22_01.10.0
22-Apr 10:23 HeadMan: Bareos 1.37.14 (21Apr05): 22-Apr-2005 10:23:06
  JobId:                  7510
  Job:                    CatalogBackup.2005-04-22_01.10.00
  Backup Level:           Full
  Client:                 Polymatou
  FileSet:                "CatalogFile" 2003-04-10 01:24:01
  Pool:                   "Default"
  Storage:                "DLTDrive"
  Start time:             22-Apr-2005 10:21:00
  End time:               22-Apr-2005 10:23:06
  FD Files Written:       1
  SD Files Written:       1
  FD Bytes Written:       210,739,395
  SD Bytes Written:       210,739,521
  Rate:                   1672.5 KB/s
  Software Compression:   None
  Volume name(s):         DLT-22Apr05
  Volume Session Id:      11
  Volume Session Time:    1114075126
  Last Volume Bytes:      1,428,240,465
  Non-fatal FD errors:    0
  SD Errors:              0
  FD termination status:  OK
  SD termination status:  OK
  Termination:            Backup OK
\end{verbatim}
\normalsize

  From the above information, you can manually create a bootstrap file,
  and then follow the instructions given above for restoring your database.
  A reconstructed bootstrap file for the above backup Job would look
  like the following:

\footnotesize
\begin{verbatim}
Volume="DLT-22Apr05"
VolSessionId=11
VolSessionTime=1114075126
FileIndex=1-1
\end{verbatim}
\normalsize

  Where we have inserted the Volume name, Volume Session Id, and Volume
  Session Time that correspond to the values in the job report.  We've also
  used a FileIndex of one, which will always be the case providing that
  there was only one file backed up in the job.

  The disadvantage of this bootstrap file compared to what is created when
  you ask for one to be written, is that there is no File and Block
  specified, so the restore code must search all data in the Volume to find
  the requested file.  A fully specified bootstrap file would have the File
  and Blocks specified as follows:

\footnotesize
\begin{verbatim}
Volume="DLT-22Apr05"
VolSessionId=11
VolSessionTime=1114075126
VolFile=118-118
VolBlock=0-4053
FileIndex=1-1
\end{verbatim}
\normalsize

   Once you have restored the ASCII dump of the database,
   you will then to follow the instructions for your
   database type to recreate the database from the ASCII backup file.
   See the \ilink {Catalog Maintenance}{CatMaintenanceChapter} chapter of
   this manual for examples of the command needed to restore a
   database from an ASCII dump (they are shown in the Compacting Your
   XXX Database sections).

   Also, please note that after you restore your database from an ASCII
   backup, you do NOT want to do a {\bf make\_bareos\_tables}  command, or
   you will probably erase your newly restored database tables.

\item [Solution without a Job Listing]
   If you do not have a job listing, then it is a bit more difficult.
   Either you use the \ilink{bscan}{bscan} program to scan the contents
   of your tape into a database, which can be very time consuming
   depending on the size of the tape, or you can use the \ilink{bls}{bls}
   program to list everything on the tape, and reconstruct a bootstrap
   file from the bls listing for the file or files you want following
   the instructions given above.

   There is a specific example of how to use {\bf bls} below.

\item [Problem]
   Trying to restore the last known good full backup by specifying
   item 3 on the restore menu then the JobId to restore, but Bareos
   then reports:

\footnotesize
\begin{verbatim}
   1 Job 0 Files
\end{verbatim}
\normalsize
   and restores nothing.

\item[Solution]
   Most likely the File records were pruned from the database either due
   to the File Retention period expiring or by explicitly purging the
   Job. By using the "llist jobid=nn" command, you can obtain all the
   important information about the job:

\footnotesize
\begin{verbatim}
llist jobid=120
           JobId: 120
             Job: save.2005-12-05_18.27.33
        Job.Name: save
     PurgedFiles: 0
            Type: B
           Level: F
    Job.ClientId: 1
     Client.Name: Rufus
       JobStatus: T
       SchedTime: 2005-12-05 18:27:32
       StartTime: 2005-12-05 18:27:35
         EndTime: 2005-12-05 18:27:37
        JobTDate: 1133803657
    VolSessionId: 1
  VolSessionTime: 1133803624
        JobFiles: 236
       JobErrors: 0
 JobMissingFiles: 0
      Job.PoolId: 4
       Pool.Name: Full
   Job.FileSetId: 1
 FileSet.FileSet: BackupSet
\end{verbatim}
\normalsize

   Then you can find the Volume(s) used by doing:

\footnotesize
\begin{verbatim}
sql
select VolumeName from JobMedia,Media where JobId=1 and JobMedia.MediaId=Media.MediaId;
\end{verbatim}
\normalsize

   Finally, you can create a bootstrap file as described in the previous
   problem above using this information.

   Bareos will ask you if
   you would like to restore all the files in the job, and it will
   collect the above information and write the bootstrap file for
   you.

\item [Problem]
  You don't have a bootstrap file, and you don't have the Job report for
  the backup of your database, but you did backup the database, and you
  know the Volume to which it was backed up.

\item [Solution]
  Either \command{bscan} the tape (see below for bscanning), or better use \command{bls}
  to find where it is on the tape, then use \command{bextract} to
  restore the database.  For example,


\footnotesize
\begin{verbatim}
./bls -j -V DLT-22Apr05 /dev/nst0
\end{verbatim}
\normalsize
  Might produce the following output:
\footnotesize
\begin{verbatim}
bls: butil.c:258 Using device: "/dev/nst0" for reading.
21-Jul 18:34 bls: Ready to read from volume "DLT-22Apr05" on device "DLTDrive"
(/dev/nst0).
Volume Record: File:blk=0:0 SessId=11 SessTime=1114075126 JobId=0 DataLen=164
...
Begin Job Session Record: File:blk=118:0 SessId=11 SessTime=1114075126
JobId=7510
   Job=CatalogBackup.2005-04-22_01.10.0 Date=22-Apr-2005 10:21:00 Level=F Type=B
End Job Session Record: File:blk=118:4053 SessId=11 SessTime=1114075126
JobId=7510
   Date=22-Apr-2005 10:23:06 Level=F Type=B Files=1 Bytes=210,739,395 Errors=0
Status=T
...
21-Jul 18:34 bls: End of Volume at file 201 on device "DLTDrive" (/dev/nst0),
Volume "DLT-22Apr05"
21-Jul 18:34 bls: End of all volumes.
\end{verbatim}
\normalsize
  Of course, there will be many more records printed, but we have indicated
  the essential lines of output. From the information on the Begin Job and End
  Job Session Records, you can reconstruct a bootstrap file such as the one
  shown above.

\item[Problem]
  How can I find where a file is stored?
\item[Solution]
  Normally, it is not necessary, you just use the {\bf restore} command to
  restore the most recently saved version (menu option 5), or a version
  saved before a given date (menu option 8).  If you know the JobId of the
  job in which it was saved, you can use menu option 3 to enter that JobId.

  If you would like to know the JobId where a file was saved, select
  restore menu option 2.

  You can also use the {\bf query} command to find information such as:
\footnotesize
\begin{verbatim}
*query
Available queries:
     1: List up to 20 places where a File is saved regardless of the
directory
     2: List where the most recent copies of a file are saved
     3: List last 20 Full Backups for a Client
     4: List all backups for a Client after a specified time
     5: List all backups for a Client
     6: List Volume Attributes for a selected Volume
     7: List Volumes used by selected JobId
     8: List Volumes to Restore All Files
     9: List Pool Attributes for a selected Pool
    10: List total files/bytes by Job
    11: List total files/bytes by Volume
    12: List Files for a selected JobId
    13: List Jobs stored on a selected MediaId
    14: List Jobs stored for a given Volume name
    15: List Volumes Bareos thinks are in changer
    16: List Volumes likely to need replacement from age or errors
Choose a query (1-16):
\end{verbatim}
\normalsize

\item[Problem]
  I didn't backup my database. What do I do now?
\item[Solution]
  This is probably the worst of all cases, and you will probably have
  to re-create your database from scratch and then bscan in all your
  volumes, which is a very long, painful, and inexact process.

There are basically three steps to take:

\begin{enumerate}
\item Ensure that your SQL server is running (MySQL or PostgreSQL)
   and that the Bareos database (normally bareos) exists.  See the
   \ilink{Installation}{CreateDatabase} chapter of the manual.
\item Ensure that the Bareos databases are created. This is also
   described at the above link.
\item Start and stop the Bareos Director using the propriate
   bareos-dir.conf file so that it can create the Client and
   Storage records which are not stored on the Volumes.  Without these
   records, scanning is unable to connect the Job records to the proper
  client.
\end{enumerate}

When the above is complete, you can begin bscanning your Volumes. Please
see the \ilink{bscan}{bscan} section of the Volume Utility Tools of this
chapter for more details.

\end{description}
