
\section{Use a dummy device to test the backup}
\label{testbackup}
If your are testing your configuration, but don't want to store the backup data,
it is possible to use a dummy FIFO device to test your configuration, see \ilink{Stored configuration}{SetupFifo}.

Obviously, it can not be used to do a restore.

\footnotesize
\begin{verbatim}
Device {
  Name = NULL
  Media Type = NULL
  Device Type = Fifo
  Archive Device = /dev/null
  LabelMedia = yes
  Random Access = no
  AutomaticMount = no
  RemovableMedia = no
  MaximumOpenWait = 60
  AlwaysOpen = no
}
\end{verbatim}
\normalsize



\section{Backing Up Third Party Databases}
\index[general]{Backing Up Third Party Databases}
\index[general]{Database!Backing Up Third Party}
\label{BackingUPOtherDBs}

If you are running a database in production mode on your machine, Bareos will
happily backup the files, but if the database is in use while Bareos is
reading it, you may back it up in an unstable state.

The best solution is to shutdown your database before backing it up, or use
some tool specific to your database to make a valid live copy perhaps by
dumping the database in ASCII format.

\subsection{Backing Up MSSQL Databases with Bareos Plugin}
\index[general]{MSSQL Backup}
\index[general]{Database!MSSQL}
\label{MSSQL}

\sinceVersion{general}{MSSQL}{13.2}

\subsubsection {Preparation}
If you like to use the MSSQL-Plugin to backing up your Databases you need to consider some things:

\begin{itemize}

\item {\bf Database Mode}\\
The database need to run in {\bf Full Recovery Mode}. Otherwise you are not able to use differential and incremental backups or to use point in time recovery.\\
\warning{If you set the databases into the mentionend mode you have to consider some maintance facts. The database doesn't shrink or delete the logs unanttended, so you have to shrink them manual once a week and you have to truncate the logs once in a month.}

\item {\bf Security and Access}\\
For connections you can use a SQL-User or a integrated systemaccount (Windows NT user). Both connection types are supported.
	\begin{itemize}
        \item Standard Security \\
You have to provide user credentials within your options which do belong to user with the sufficent right performing restores and backups from the database. This way stands for an extra backup/restore user.
	\item Trusted Security \\
	You use a systemaccount which have the sufficent rights to performing backups and restores on a database. This systemaccount have to be same account like the {\bf bareos-filedeamon} runs on.
	\end{itemize}
\item{\bf Permissions and Roles}
	\begin{itemize}
	\item Server-Role\\
	The user should be in the groups {\bf sysadmin} and {\bf dbcreator}.
	\item Database permissions\\
	The user have to be a {\bf backupoperator} and {\bf dbowner} of the database which you like to backup.
	\end{itemize}
\end{itemize}
There is no difference for the rights and roles between using a systemaccount (trusted security method) or a extra backup user (standard security method).
Please keep in mind if you use the trusted security method you have to use the same system account like the bareos-filedeamon runs on.

\subsubsection{MSSQL Plugin Installation}
\label{MssqlPluginInstallation}

\paragraph{Bareos Windows-Installer}

Install the bareos fileadaemon on the MSSQL server computer and install also the component "Bareos FileDameon Plugins".
Make sure, that you install the file daemon {\bf without the "compatible" option}.\\

\paragraph{Manual install}

After downloading the plugin you need to copy it into \path|C:\Program Files\Bareos\Plugins|.
Then you need to define the plugin directory and which plugin the {\bf bareos-filedaemon} should use.
You have to edit the {\bf bareos-filedaemon} ressource in  \path|C:\Program Data\bareos-fd.conf| as followed.\\

\begin{bconfig}{MSSQL plugin configuration}
FileDaemon {
  Name = mssqlserver-fd
  Maximum Concurrent Jobs = 20

  # remove comment in next line to load plugins from specified directory
  Plugin Directory = "C:/Program Files/Bareos/Plugins"

  # if compatible is set to yes, we are compatible with bacula
  # if you want to use new bareos features, please set compatible to no
  Plugin Names = "mssqlvdi"
  compatible = no
}
\end{bconfig}

%\warning{The Bareos filedaemon and Bareos director have to be the same version.}

\subsubsection{Plugin Test}

\begin{bconsole}{status client=mssqlserver-fd}
*<input>status client=mssqlserver-fd</input>
Connecting to Client mssqlserver-fd at 192.168.10.101:9102

mssqlserver-fd Version: 13.2.2 (12 November 2013)  VSS Linux Cross-compile Win64
Daemon started 18-Nov-13 11:51. Jobs: run=0 running=0.
Microsoft Windows Server 2012 Standard Edition (build 9200), 64-bit
 Heap: heap=0 smbytes=20,320 max_bytes=20,522 bufs=71 max_bufs=73
 Sizeof: boffset_t=8 size_t=8 debug=0 trace=1 bwlimit=0kB/s
Plugin Info:
 Plugin     : mssqlvdi-fd.dll
 Description: Bareos MSSQL VDI Windows File Daemon Plugin
 Version    : 1, Date: July 2013
 Author     : Zilvinas Krapavickas
 License    : Bareos AGPLv3
 Usage      :
  mssqlvdi:
  instance=<instance name>:
  database=<database name>:
  username=<database username>:
  password=<database password>:
  norecovery=<yes|no>:
  replace=<yes|no>:
  recoverafterrestore=<yes|no>:
  stopbeforemark=<log sequence number specification>:
  stopatmark=<log sequence number specification>:
  stopat=<timestamp>

 examples:
  timestamp: 'Apr 15, 2020 12:00 AM'
  log sequence number: 'lsn:15000000040000037'
\end{bconsole}

\subsubsection{Configure the FileSet}
\index[general]{Backupconfiguration}
\label{filesetconfig}
To use the plugin you need to configure it in the fileset as a plugin ressource. For each database instance you need to define a exclusive backup job and fileset.

\begin{bconfig}{MSSQL FileSet}
Fileset {
  Name = "Mssql"
  Enable VSS = no
  Include {
    Options {
      Signature = MD5
    }
    Plugin = "mssqlvdi:instance=default:database=myDatabase:username=bareos:password=bareos"
  }
}
\end{bconfig}

In this example we use the standard security method for the connection.

Used options in the plugin string are:

\begin{description}
  \item[mssqlvdi]
  This is the reference to the MSSQL plugin.
  \item[instance] Defines the instance within the database server.
  \item[database] Defines the database that should get backuped.
  \item[username and password] Username and Password are required, when the connection is done using a MSSQL user. If the systemaccount the bareos-fd runs with has succifient permissions, this is not required.
\end{description}

It is recommend to define an additional restore job. 

For every database seperat job and FileSet are required.

\subsubsection{Run Backups}

Here you can see an example for a backup:

\begin{bconsole}{run MSSQL backup job}
*<input>run job=MSSQLBak</input>
Using Catalog "MyCatalog"
Run Backup job
JobName:  MSSQLBak
Level:    Full
Client:   mssqlserver-fd
Format:   Native
FileSet:  Mssql
Pool:     File (From Job resource)
Storage:  File (From Job resource)
When:     2013-11-21 09:48:27
Priority: 10
OK to run? (yes/mod/no): <input>yes</input>
Job queued. JobId=7
You have no messages.
*<input>mess</input>
21-Nov 09:48 bareos-dir JobId 7: Start Backup JobId 7, Job=MSSQLBak.2013-11-21_09.48.30_04
21-Nov 09:48 bareos-dir JobId 7: Using Device "FileStorage" to write.
21-Nov 09:49 bareos-sd JobId 7: Volume "test1" previously written, moving to end of data.
21-Nov 09:49 bareos-sd JobId 7: Ready to append to end of Volume "test1" size=2300114868
21-Nov 09:49 bareos-sd JobId 7: Elapsed time=00:00:27, Transfer rate=7.364 M Bytes/second

21-Nov 09:49 bareos-dir JobId 7: Bareos bareos-dir 13.4.0 (01Oct13):
  Build OS:               x86_64-pc-linux-gnu debian Debian GNU/Linux 7.0 (wheezy)
  JobId:                  7
  Job:                    MSSQLBak.2013-11-21_09.48.30_04
  Backup Level:           Full
  Client:                 "mssqlserver-fd" 13.2.2 (12Nov13) Microsoft Windows Server 2012 Standard Edition (build 9200), 64-bit,Cross-compile,Win64
  FileSet:                "Mssql" 2013-11-04 23:00:01
  Pool:                   "File" (From Job resource)
  Catalog:                "MyCatalog" (From Client resource)
  Storage:                "File" (From Job resource)
  Scheduled time:         21-Nov-2013 09:48:27
  Start time:             21-Nov-2013 09:49:13
  End time:               21-Nov-2013 09:49:41
  Elapsed time:           28 secs
  Priority:               10
  FD Files Written:       1
  SD Files Written:       1
  FD Bytes Written:       198,836,224 (198.8 MB)
  SD Bytes Written:       198,836,435 (198.8 MB)
  Rate:                   7101.3 KB/s
  Software Compression:   None
  VSS:                    no
  Encryption:             no
  Accurate:               no
  Volume name(s):         test1
  Volume Session Id:      1
  Volume Session Time:    1384961357
  Last Volume Bytes:      2,499,099,145 (2.499 GB)
  Non-fatal FD errors:    0
  SD Errors:              0
  FD termination status:  OK
  SD termination status:  OK
  Termination:            Backup OK
\end{bconsole}

At least you gain a full backup which contains the follow:

\footnotesize
\begin{verbatim}
@MSSQL/
@MSSQL/default/
@MSSQL/default/myDatabase/
@MSSQL/default/myDatabase/db-full
\end{verbatim}
\normalsize

So if you perform your first full backup your are capable to perfom differntial and incremental backups.\\

Differntial FileSet example:

\footnotesize
\begin{verbatim}
/@MSSQL/
/@MSSQL/default/
/@MSSQL/default/myDatabase/
/@MSSQL/default/myDatabase/db-full
/@MSSQL/default/myDatabase/db-diff
\end{verbatim}
\normalsize

Incremental FileSet example:

\footnotesize
\begin{verbatim}
*@MSSQL/
  *default/
    *myDatabase/
      *db-diff
      *db-full
      *log-2013-11-21 17:32:20
\end{verbatim}
\normalsize

\subsubsection{Restores}

If you want to perfom a restore of a full backup without differentials or incrementals you have some options which helps you to restore even the corrupted database still exist. But you have to specifiy the options like plugin, instance and database during every backup.

\begin{description}
  \item[replace=\textless yes|no\textgreater]
  With this option you can replace the database if it still exist.
  \item[instance]
  Defines the server instance whithin the database is running.
  \item[database]
  Defines the database you want to backup.
\end{description}

Example for a full restore:

\begin{bconsole}{restore MSSQL database}
*<input>restore client=mssqlserver-fd</input>
Using Catalog "MyCatalog"

First you select one or more JobIds that contain files
to be restored. You will be presented several methods
of specifying the JobIds. Then you will be allowed to
select which files from those JobIds are to be restored.

To select the JobIds, you have the following choices:
     1: List last 20 Jobs run
     2: List Jobs where a given File is saved
     3: Enter list of comma separated JobIds to select
     4: Enter SQL list command
     5: Select the most recent backup for a client
     6: Select backup for a client before a specified time
     7: Enter a list of files to restore
     8: Enter a list of files to restore before a specified time
     9: Find the JobIds of the most recent backup for a client
    10: Find the JobIds for a backup for a client before a specified time
    11: Enter a list of directories to restore for found JobIds
    12: Select full restore to a specified Job date
    13: Cancel
Select item:  (1-13): <input>5</input>
Automatically selected FileSet: Mssql
+-------+-------+----------+-------------+---------------------+------------+
| JobId | Level | JobFiles | JobBytes    | StartTime           | VolumeName |
+-------+-------+----------+-------------+---------------------+------------+
|     8 | F     |        1 | 198,836,224 | 2013-11-21 09:52:28 | test1      |
+-------+-------+----------+-------------+---------------------+------------+
You have selected the following JobId: 8

Building directory tree for JobId(s) 8 ...
1 files inserted into the tree.

You are now entering file selection mode where you add (mark) and
remove (unmark) files to be restored. No files are initially added, unless
you used the "all" keyword on the command line.
Enter "done" to leave this mode.

cwd is: /
$ <input>mark *</input>
1 file marked.
$ <input>done</input>
Bootstrap records written to /var/lib/bareos/bareos-dir.restore.4.bsr

The job will require the following
   Volume(s)                 Storage(s)                SD Device(s)
===========================================================================

    test1                     File                      FileStorage

Volumes marked with "*" are online.


1 file selected to be restored.

The defined Restore Job resources are:
     1: RestoreMSSQL
     2: RestoreFiles
Select Restore Job (1-2): <input>1</input>
Using Catalog "MyCatalog"
Run Restore job
JobName:         RestoreMSSQL
Bootstrap:       /var/lib/bareos/bareos-dir.restore.4.bsr
Where:           /tmp
Replace:         Always
FileSet:         Mssql
Backup Client:   mssqlserver-fd
Restore Client:  mssqlserver-fd
Format:          Native
Storage:         File
When:            2013-11-21 17:12:05
Catalog:         MyCatalog
Priority:        10
Plugin Options:  *None*
OK to run? (yes/mod/no): <input>mod</input>
Parameters to modify:
     1: Level
     2: Storage
     3: Job
     4: FileSet
     5: Restore Client
     6: Backup Format
     7: When
     8: Priority
     9: Bootstrap
    10: Where
    11: File Relocation
    12: Replace
    13: JobId
    14: Plugin Options
Select parameter to modify (1-14): <input>14</input>
Please enter Plugin Options string: <input>mssqlvdi:instance=default:database=myDatabase:replace=yes</input>
Run Restore job
JobName:         RestoreMSSQL
Bootstrap:       /var/lib/bareos/bareos-dir.restore.4.bsr
Where:           /tmp
Replace:         Always
FileSet:         Mssql
Backup Client:   mssqlserver-fd
Restore Client:  mssqlserver-fd
Format:          Native
Storage:         File
When:            2013-11-21 17:12:05
Catalog:         MyCatalog
Priority:        10
Plugin Options:  mssqlvdi:instance=default:database=myDatabase:replace=yes
OK to run? (yes/mod/no): <input>yes</input>
Job queued. JobId=10
You have messages.
*<input>mess</input>
21-Nov 17:12 bareos-dir JobId 10: Start Restore Job RestoreMSSQL.2013-11-21_17.12.26_11
21-Nov 17:12 bareos-dir JobId 10: Using Device "FileStorage" to read.
21-Nov 17:13 damorgan-sd JobId 10: Ready to read from volume "test1" on device "FileStorage" (/storage).
21-Nov 17:13 damorgan-sd JobId 10: Forward spacing Volume "test1" to file:block 0:2499099145.
21-Nov 17:13 damorgan-sd JobId 10: End of Volume at file 0 on device "FileStorage" (/storage), Volume "test1"
21-Nov 17:13 damorgan-sd JobId 10: End of all volumes.
21-Nov 17:13 bareos-dir JobId 10: Bareos bareos-dir 13.4.0 (01Oct13):
  Build OS:               x86_64-pc-linux-gnu debian Debian GNU/Linux 7.0 (wheezy)
  JobId:                  10
  Job:                    RestoreMSSQL.2013-11-21_17.12.26_11
  Restore Client:         mssqlserver-fd
  Start time:             21-Nov-2013 17:12:28
  End time:               21-Nov-2013 17:13:21
  Files Expected:         1
  Files Restored:         1
  Bytes Restored:         198,836,224
  Rate:                   3751.6 KB/s
  FD Errors:              0
  FD termination status:  OK
  SD termination status:  OK
  Termination:            Restore OK
\end{bconsole}

\paragraph{Restore a Backup Chain}

If you like to restore a specific state or a whole chain consists of full, incremental and differential backups you need to use the "norecovery=yes" option. After this the database is in "recovery mode". You can also use a option which put the database right after the restore back into the right mode. If you like to restore certains point with protocols or
"LSN" it it not recommend to work with this option.

\begin{description}
  \item[norecovery=\textless yes|no\textgreater]
  This option must be set to yes, if the database server should not do a automatic recovery after the backup. Instead, additional manual maintenace operations are possible.
  \item[recoverafterrestore=\textless yes|no\textgreater]
  With this command the database is right after backup in the correct mode. If you not use this you have to use the followed tsql statement:
  \footnotesize
  \begin{verbatim}
    Restore DATABASE yourDatabase WITH RECOVERY
    GO
  \end{verbatim}
  \normalsize
  \item[stopbeforemark=\textless log sequence number specification\textgreater]
  used for point in time recovery.
  \item[stopatmark=\textless log sequence number specification\textgreater]
  used for point in time recovery.
  \item[stopat=\textless timestamp\textgreater]
  used for point in time recovery.
\end{description}

Followed is a example for a restore of full, differential and incremental backup with a replace of the original database:

\begin{bconsole}{restore MSSQL database chain}
*<input>restore client=mssqlserver-fd</input>

First you select one or more JobIds that contain files
to be restored. You will be presented several methods
of specifying the JobIds. Then you will be allowed to
select which files from those JobIds are to be restored.

To select the JobIds, you have the following choices:
     1: List last 20 Jobs run
     2: List Jobs where a given File is saved
     3: Enter list of comma separated JobIds to select
     4: Enter SQL list command
     5: Select the most recent backup for a client
     6: Select backup for a client before a specified time
     7: Enter a list of files to restore
     8: Enter a list of files to restore before a specified time
     9: Find the JobIds of the most recent backup for a client
    10: Find the JobIds for a backup for a client before a specified time
    11: Enter a list of directories to restore for found JobIds
    12: Select full restore to a specified Job date
    13: Cancel
Select item:  (1-13): <input>5</input>
Automatically selected FileSet: Mssql
+-------+-------+----------+-------------+---------------------+------------+
| JobId | Level | JobFiles | JobBytes    | StartTime           | VolumeName |
+-------+-------+----------+-------------+---------------------+------------+
|     8 | F     |        1 | 198,836,224 | 2013-11-21 09:52:28 | test1      |
|    11 | D     |        1 |   2,555,904 | 2013-11-21 17:19:45 | test1      |
|    12 | I     |        1 |     720,896 | 2013-11-21 17:29:39 | test1      |
+-------+-------+----------+-------------+---------------------+------------+
You have selected the following JobIds: 8,11,12

Building directory tree for JobId(s) 8,11,12 ...
3 files inserted into the tree.

You are now entering file selection mode where you add (mark) and
remove (unmark) files to be restored. No files are initially added, unless
you used the "all" keyword on the command line.
Enter "done" to leave this mode.

cwd is: /
$ <input>mark *</input>
3 files marked.
$ <input>lsmark</input>
*@MSSQL/
  *default/
    *myDatabase/
      *db-diff
      *db-full
      *log-2013-11-21 17:32:20
$ <input>done</input>
Bootstrap records written to /var/lib/bareos/bareos-dir.restore.6.bsr

The job will require the following
   Volume(s)                 Storage(s)                SD Device(s)
===========================================================================

    test1                     File                      FileStorage

Volumes marked with "*" are online.


1 file selected to be restored.

The defined Restore Job resources are:
     1: RestoreMSSQL
     2: RestoreFiles
Select Restore Job (1-2): <input>1</input>
Run Restore job
JobName:         RestoreMSSQL
Bootstrap:       /var/lib/bareos/bareos-dir.restore.6.bsr
Where:           /tmp
Replace:         Always
FileSet:         Mssql
Backup Client:   mssqlserver-fd
Restore Client:  mssqlserver-fd
Format:          Native
Storage:         File
When:            2013-11-21 17:34:23
Catalog:         MyCatalog
Priority:        10
Plugin Options:  *None*
OK to run? (yes/mod/no): <input>mod</input>
Parameters to modify:
     1: Level
     2: Storage
     3: Job
     4: FileSet
     5: Restore Client
     6: Backup Format
     7: When
     8: Priority
     9: Bootstrap
    10: Where
    11: File Relocation
    12: Replace
    13: JobId
    14: Plugin Options
Select parameter to modify (1-14): <input>14</input>
Please enter Plugin Options string: <input>mssqlvdi:instance=default:database=myDatabase:replace=yes:norecovery=yes</input>
Run Restore job
JobName:         RestoreMSSQL
Bootstrap:       /var/lib/bareos/bareos-dir.restore.6.bsr
Where:           /tmp
Replace:         Always
FileSet:         Mssql
Backup Client:   mssqlserver-fd
Restore Client:  mssqlserver-fd
Format:          Native
Storage:         File
When:            2013-11-21 17:34:23
Catalog:         MyCatalog
Priority:        10
Plugin Options:  mssqlvdi:instance=default:database=myDatabase:replace=yes:norecovery=yes
OK to run? (yes/mod/no): <input>yes</input>
Job queued. JobId=14
21-Nov 17:34 bareos-dir JobId 14: Start Restore Job RestoreMSSQL.2013-11-21_17.34.40_16
21-Nov 17:34 bareos-dir JobId 14: Using Device "FileStorage" to read.
21-Nov 17:35 damorgan-sd JobId 14: Ready to read from volume "test1" on device "FileStorage" (/storage).
21-Nov 17:35 damorgan-sd JobId 14: Forward spacing Volume "test1" to file:block 0:2499099145.
21-Nov 17:35 damorgan-sd JobId 14: End of Volume at file 0 on device "FileStorage" (/storage), Volume "test1"
21-Nov 17:35 damorgan-sd JobId 14: End of all volumes.
21-Nov 17:35 bareos-dir JobId 14: Bareos bareos-dir 13.4.0 (01Oct13):
  Build OS:               x86_64-pc-linux-gnu debian Debian GNU/Linux 7.0 (wheezy)
  JobId:                  14
  Job:                    RestoreMSSQL.2013-11-21_17.34.40_16
  Restore Client:         mssqlserver-fd
  Start time:             21-Nov-2013 17:34:42
  End time:               21-Nov-2013 17:35:36
  Files Expected:         1
  Files Restored:         3
  Bytes Restored:         202,113,024
  Rate:                   3742.8 KB/s
  FD Errors:              0
  FD termination status:  OK
  SD termination status:  OK
  Termination:            Restore OK
\end{bconsole}

\section{The bpipe Plugin}
\label{bpipe}

The bpipe plugin is a generic pipe program, that simply transmits the data from a specified program to Bareos for backup, and
from Bareos to a specified program for restore. The purpose of the plugin is to provide an interface to any system program
for backup and restore. That allows you, for example, to do database backups without a local dump. By using different command 
lines to bpipe, you can backup any kind of data (ASCII or binary) depending on the program called.

The bpipe plugin is provided in the directory \path|src/plugins/filed/bpipe-fd.c| of the Bareos source distribution or by the 
bareos-common binary package of your distributions repository and has to be installed on the client where the FileDaemon is running. 

The following line has to be present in your bareos-fd.conf, so the FileDaemon knows the path where to look for the plugin, 
when it's called.

\begin{bconfig}{bpipe fdconfig}
PluginDirectory = /usr/lib64/bareos/plugins
\end{bconfig}

Please note that this is a very simple plugin that was written for demonstration and test purposes. It is and can be used 
in production, but that was never intended. The bpipe plugin is so simple and flexible, you may call it the 
"Swiss Army Knife" of the current existing plugins for Bareos.

The bpipe plugin is specified in the Include section of your Job's FileSet resource in your bareos-dir.conf.

\begin{bconfig}{bpipe fileset}
FileSet {
  Name = "MyFileSet"
  Include {
    Options {
      signature = MD5
      compression = gzip
    }
    Plugin = "bpipe:..."
  }
}
\end{bconfig}

The syntax and semantics of the Plugin directive require the first part of the string up to the colon to be the name of the plugin.
Everything after the first colon is ignored by the File daemon but is passed to the plugin. Thus the plugin writer may define the 
meaning of the rest of the string as he wishes. The full syntax of the plugin directive as interpreted by the bpipe plugin is:

\begin{bconfig}{bpipe directive}
Plugin = "<plugin>:<path>:<command1>:<command2>"
\end{bconfig}

\begin{description}
\item[plugin] is the name of the plugin with the trailing -fd.so stripped off, so in this case, we would put bpipe in the field.
\item[path] specifies the namespace, which for bpipe is the pseudo path and filename under which the backup will be saved. This
pseudo path and filename will be seen by the user in the restore file tree. For example, if the value is /MySQL/mydump.sql, the data
backed up by the plugin will be put under that "pseudo" path and filename. You must be careful to choose a naming convention that is unique
to avoid a conflict with a path and filename that actually exists on your system.
\item[command1] for the bpipe plugin specifies the "reader" program that is called by the plugin during backup to read the data. bpipe
will call this program by doing a popen on it.
\item[command2] for the bpipe plugin specifies the "writer" program that is called by the plugin during restore to write the data back 
to the filesystem.
\end{description}

Please note that the two items above describing the "reader" and "writer", these programs are "executed" by Bareos, which means 
there is no shell inerpretation of any command line arguments you might use. If you want to use shell characters (redirection of input 
or output, ...), then we recommend that you put your command or commands in a shell script and execute the script. In addition if you
backup a file with reader program, when running the writer program during the restore, Bareos will not automatically create the path
to the file. Either the path must exist, or you must explicitly do so with your command or in a shell script.

Putting it all together, the full plugin directive line might look like the following:

\begin{bconfig}{bpipe directive PostgreSQL dump}
Plugin = "bpipe:/POSTGRESQL/bareos.sql:pg_dumpall -U postgres:psql -U postgres"
\end{bconfig}

This causes the File daemon to call bpipe plugin, which will write its data into the "pseudo" file \path|/POSTGRESQL/bareos.sql| by
calling the program \command{pg_dumpall -U postgres} to read the data during backup. The \command{pg_dumpall} command outputs all 
the data for the database, which will be read by the plugin and stored in the backup. During restore, the data that was backed up will
be sent to the program specified in the last field, which in this case is psql. When psql is called, it will read the data sent to it by 
the plugin then write it back to the same database from which it came from.

Here is a complete FileSet example, where first your \path|/home| is backed up and then a mysqldump is done.

\begin{bconfig}{bpipe fileset MySQL dump}
FileSet {
  Name = "<YourFileSetName>"
  Include = {
    File = /home
    Plugin = "bpipe:/MYSQL/<dumpname.sql>:mysqldump -u <user> --password=<password> --opt --all-databases:mysql -u <user> -p <password>"
    Options {
      signature = MD5
      compression = gzip
    }
  }
  Exclude {
  }
}
\end{bconfig}

Note: It is also possible to define more than one plugin directive in a FileSet to do several database dumps at once.
