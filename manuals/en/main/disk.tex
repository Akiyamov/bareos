%%
%%

\chapter{Basic Volume Management}
\label{DiskChapter}
\index[general]{Basic Volume Management}
\index[general]{Management!Basic Volume}
\index[general]{Disk Volumes}

This chapter presents most all the features needed to do Volume management.
Most of the concepts apply equally well to both tape and disk Volumes.
However, the chapter was originally written to explain backing up to disk, so
you will see it is slanted in that direction, but all the directives
presented here apply equally well whether your volume is disk or tape.

If you have a lot of hard disk storage or you absolutely must have your
backups run within a small time window, you may want to direct Bareos to
backup to disk Volumes rather than tape Volumes. This chapter is intended to
give you some of the options that are available to you so that you can manage
either disk or tape volumes.

\label{Concepts}
\section{Key Concepts and Resource Records}
\index[general]{Key Concepts and Resource Records}
\index[general]{Records!Key Concepts and Resource}

Getting Bareos to write to disk rather than tape in the simplest case is
rather easy. In the Storage daemon's configuration file, you simply define an
{\bf Archive Device} to be a directory.
The default directory to store backups on disk is \path|/var/lib/bareos/storage|:

\footnotesize
\begin{verbatim}
Device {
  Name = FileBackup
  Media Type = File
  Archive Device = /var/lib/bareos/storage
  Random Access = Yes;
  AutomaticMount = yes;
  RemovableMedia = no;
  AlwaysOpen = no;
}
\end{verbatim}
\normalsize

Assuming you have the appropriate {\bf Storage} resource in your Director's
configuration file that references the above Device resource,

\footnotesize
\begin{verbatim}
Storage {
  Name = FileStorage
  Address = ...
  Password = ...
  Device = FileBackup
  Media Type = File
}
\end{verbatim}
\normalsize

Bareos will then write the archive to the file {\bf
/var/lib/bareos/storage/{\textless}volume-name{\textgreater}}
where {\textless}volume-name{\textgreater} is the
volume name of a Volume defined in the Pool. For example, if you have labeled
a Volume named {\bf Vol001}, Bareos will write to the file {\bf
/var/lib/bareos/storage/Vol001}. Although you can later move the archive file to
another directory, you should not rename it or it will become unreadable by
Bareos. This is because each archive has the filename as part of the internal
label, and the internal label must agree with the system filename before
Bareos will use it.

Although this is quite simple, there are a number of problems. The first is
that unless you specify otherwise, Bareos will always write to the same volume
until you run out of disk space. This problem is addressed below.

In addition, if you want to use concurrent jobs that write to several
different volumes at the same time, you will need to understand a number
of other details. An example of such a configuration is given
at the end of this chapter under \ilink{Concurrent Disk
Jobs}{ConcurrentDiskJobs}.

\subsection{Pool Options to Limit the Volume Usage}
\index[general]{Usage!Pool Options to Limit the Volume}
\index[general]{Pool Options to Limit the Volume Usage}

Some of the options you have, all of which are specified in the Pool record,
are:

\begin{itemize}
\item To write each Volume only once (i.e. one Job per Volume or file  in this
   case), use:

{\bf UseVolumeOnce = yes}.

\item To write nnn Jobs to each Volume, use:

   {\bf Maximum Volume Jobs = nnn}.

\item To limit the maximum size of each Volume, use:

   {\bf Maximum Volume Bytes = mmmm}.

   Note, if you use disk volumes you should probably limit the Volume size to some reasonable
   value such as say 5GB. This is because during a restore, Bareos is
   currently unable to seek to the proper place in a disk volume to restore
   a file, which means that it must read all records up to where the
   restore begins. If your Volumes are 50GB, reading half or more of the
   volume could take quite a bit of time.  Also, if you ever have a partial
   hard disk failure, you are more likely to be able to recover more data
   if they are in smaller Volumes.

\item To limit the use time (i.e. write the Volume for a maximum of five days),
   use:

{\bf Volume Use Duration = ttt}.
\end{itemize}

Note that although you probably would not want to limit the number of bytes on
a tape as you would on a disk Volume, the other options can be very useful in
limiting the time Bareos will use a particular Volume (be it tape or disk).
For example, the above directives can allow you to ensure that you rotate
through a set of daily Volumes if you wish.

As mentioned above, each of those directives is specified in the Pool or
Pools that you use for your Volumes. In the case of {\bf Maximum Volume Job},
{\bf Maximum Volume Bytes}, and {\bf Volume Use Duration}, you can actually
specify the desired value on a Volume by Volume basis. The value specified in
the Pool record becomes the default when labeling new Volumes. Once a Volume
has been created, it gets its own copy of the Pool defaults, and subsequently
changing the Pool will have no effect on existing Volumes. You can either
manually change the Volume values, or refresh them from the Pool defaults using
the {\bf update volume} command in the Console. As an example
of the use of one of the above, suppose your Pool resource contains:

\footnotesize
\begin{verbatim}
Pool {
  Name = File
  Pool Type = Backup
  Volume Use Duration = 23h
}
\end{verbatim}
\normalsize

then if you run a backup once a day (every 24 hours), Bareos will use a new
Volume for each backup, because each Volume it writes can only be used for 23 hours
after the first write. Note, setting the use duration to 23 hours is not a very
good solution for tapes unless you have someone on-site during the weekends,
because Bareos will want a new Volume and no one will be present to mount it,
so no weekend backups will be done until Monday morning.

\label{AutomaticLabeling}
\subsection{Automatic Volume Labeling}
\index[general]{Automatic Volume Labeling}
\index[general]{Labeling!Automatic Volume}

Use of the above records brings up another problem -- that of labeling your
Volumes. For automated disk backup, you can either manually label each of your
Volumes, or you can have Bareos automatically label new Volumes when they are
needed.

Please note that automatic Volume labeling can also be used with tapes, but
it is not nearly so practical since the tapes must be pre-mounted.  This
requires some user interaction.  Automatic labeling from templates does NOT
work with autochangers since Bareos will not access unknown slots.  There
are several methods of labeling all volumes in an autochanger magazine.
For more information on this, please see the \ilink{Autochanger}{AutochangersChapter} chapter of this manual.

Automatic Volume labeling is enabled by making a change to both the Pool
resource (Director) and to the Device resource (Storage daemon) shown above.
In the case of the Pool resource, you must provide Bareos with a label format
that it will use to create new names. In the simplest form, the label format
is simply the Volume name, to which Bareos will append a four digit number.
This number starts at 0001 and is incremented for each Volume the catalog
contains. Thus if you modify your Pool resource to be:

\footnotesize
\begin{verbatim}
Pool {
  Name = File
  Pool Type = Backup
  Volume Use Duration = 23h
  LabelFormat = "Vol"
}
\end{verbatim}
\normalsize

Bareos will create Volume names Vol0001, Vol0002, and so on when new Volumes
are needed. Much more complex and elaborate labels can be created using
variable expansion defined in the
\ilink{Variable Expansion}{VarsChapter} chapter of this manual.

The second change that is necessary to make automatic labeling work is to give
the Storage daemon permission to automatically label Volumes. Do so by adding
{\bf LabelMedia = yes} to the Device resource as follows:

\footnotesize
\begin{verbatim}
Device {
  Name = File
  Media Type = File
  Archive Device = /home/bareos/backups
  Random Access = Yes;
  AutomaticMount = yes;
  RemovableMedia = no;
  AlwaysOpen = no;
  LabelMedia = yes
}
\end{verbatim}
\normalsize

You can find more details of the {\bf Label Format} Pool record in
\ilink{Label Format}{Label} description of the Pool resource
records.

\label{Recycling1}
\subsection{Restricting the Number of Volumes and Recycling}
\index[general]{Recycling!Restricting the Number of Volumes and Recycling}
\index[general]{Restricting the Number of Volumes and Recycling}

Automatic labeling discussed above brings up the problem of Volume management.
With the above scheme, a new Volume will be created every day. If you have not
specified Retention periods, your Catalog will continue to fill keeping track
of all the files Bareos has backed up, and this procedure will create one new
archive file (Volume) every day.

The tools Bareos gives you to help automatically manage these problems are the
following:

\begin{enumerate}
\item Catalog file record retention periods, the
   \ilink{File Retention = ttt}{FileRetention}  record in the Client
   resource.
\item Catalog job record retention periods, the
   \ilink{Job Retention = ttt}{JobRetention}  record in the Client
   resource.
\item The
   \ilink{ AutoPrune = yes}{AutoPrune} record in the Client resource
   to permit  application of the above two retention periods.
\item The
   \ilink{ Volume Retention = ttt}{VolRetention} record in the Pool
   resource.
\item The
   \ilink{ AutoPrune = yes}{PoolAutoPrune} record in the Pool
   resource to permit  application of the Volume retention period.
\item The
   \ilink{ Recycle = yes}{PoolRecycle} record in the Pool resource
   to permit  automatic recycling of Volumes whose Volume retention period has
   expired.
\item The
   \ilink{ Recycle Oldest Volume = yes}{RecycleOldest} record in the
   Pool resource tells Bareos  to Prune the oldest volume in the Pool, and if all
   files  were pruned to recycle this volume and use it.
\item The
   \ilink{ Recycle Current Volume = yes}{RecycleCurrent} record in
   the Pool resource tells Bareos  to Prune the currently mounted volume in the
   Pool, and if all files  were pruned to recycle this volume and use it.
\item The
   \ilink{ Purge Oldest Volume = yes}{PurgeOldest} record in the
   Pool resource  permits a forced recycling of the oldest Volume when a new one
   is  needed.  {\bf N.B. This record ignores retention periods! We highly
   recommend  not to use this record, but instead use Recycle Oldest Volume}
\item The
   \ilink{ Maximum Volumes = nnn}{MaxVolumes} record in the Pool
   resource to limit  the number of Volumes that can be created.
\end{enumerate}

The first three records (File Retention, Job Retention, and AutoPrune)
determine the amount of time that Job and File records will remain in your
Catalog, and they are discussed in detail in the
\ilink{Automatic Volume Recycling}{RecyclingChapter} chapter of
this manual.

Volume Retention, AutoPrune, and Recycle determine how long Bareos will keep
your Volumes before reusing them, and they are also discussed in detail in the
\ilink{Automatic Volume Recycling}{RecyclingChapter} chapter of
this manual.

The Maximum Volumes record can also be used in conjunction with the Volume
Retention period to limit the total number of archive Volumes (files) that
Bareos will create. By setting an appropriate Volume Retention period, a
Volume will be purged just before it is needed and thus Bareos can cycle
through a fixed set of Volumes. Cycling through a fixed set of Volumes can
also be done by setting {\bf Recycle Oldest Volume = yes} or {\bf Recycle
Current Volume = yes}. In this case, when Bareos needs a new Volume, it will
prune the specified volume.

\label{ConcurrentDiskJobs}
\section{Concurrent Disk Jobs}
\index[general]{Concurrent Disk Jobs}
Above, we discussed how you could have a single device named {\bf
FileBackup} that writes to volumes in {\bf /home/bareos/backups}.
You can, in fact, run multiple concurrent jobs using the
Storage definition given with this example, and all the jobs will
simultaneously write into the Volume that is being written.

Now suppose you want to use multiple Pools, which means multiple
Volumes, or suppose you want each client to have its own Volume
and perhaps its own directory such as {\bf /home/bareos/client1}
and {\bf /home/bareos/client2} ... With the single Storage and Device
definition above, neither of these two is possible.  Why?  Because
Bareos disk storage follows the same rules as tape devices. Only
one Volume can be mounted on any Device at any time. If you want
to simultaneously write multiple Volumes, you will need multiple
Device resources in your bareos-sd.conf file, and thus multiple
Storage resources in your bareos-dir.conf.

OK, so now you should understand that you need multiple Device definitions
in the case of different directories or different Pools, but you also
need to know that the catalog data that Bareos keeps contains only
the Media Type and not the specific storage device.  This permits a tape
for example to be re-read on any compatible tape drive.  The compatibility
being determined by the Media Type.  The same applies to disk storage.
Since a volume that is written by a Device in say directory {\bf
/home/bareos/backups}  cannot be read by a Device with an Archive Device
definition of {\bf /home/bareos/client1}, you will not be able to
restore all your files if you give both those devices
{\bf Media Type = File}. During the restore, Bareos will simply choose
the first available device, which may not be the correct one.  If this
is confusing, just remember that the Directory has only the Media Type
and the Volume name.  It does not know the {\bf Archive Device} (or the
full path) that is specified in the Storage daemon.  Thus you must
explicitly tie your Volumes to the correct Device by using the Media Type.

The example shown below shows a case where there are two clients, each
using its own Pool and storing their Volumes in different directories.


\label{Example2}
\section{An Example}
\index[general]{Example}

The following example is not very practical, but can be used to demonstrate
the proof of concept in a relatively short period of time. The example
consists of a two clients that are backed up to a set of 12 archive files
(Volumes) for each client into different directories on the Storage
machine.  Each Volume is used (written) only once, and there are four Full
saves done every hour (so the whole thing cycles around after three hours).

What is key here is that each physical device on the Storage daemon
has a different Media Type. This allows the Director to choose the
correct device for restores ...

The Director's configuration file is as follows:

\footnotesize
\begin{verbatim}
Director {
  Name = my-dir
  QueryFile = "~/bareos/bin/query.sql"
  PidDirectory = "~/bareos/working"
  WorkingDirectory = "~/bareos/working"
  Password = dir_password
}
Schedule {
  Name = "FourPerHour"
  Run = Level=Full hourly at 0:05
  Run = Level=Full hourly at 0:20
  Run = Level=Full hourly at 0:35
  Run = Level=Full hourly at 0:50
}
Job {
  Name = "RecycleExample"
  Type = Backup
  Level = Full
  Client = Rufus
  FileSet= "Example FileSet"
  Messages = Standard
  Storage = FileStorage
  Pool = Recycle
  Schedule = FourPerHour
}

Job {
  Name = "RecycleExample2"
  Type = Backup
  Level = Full
  Client = Roxie
  FileSet= "Example FileSet"
  Messages = Standard
  Storage = FileStorage1
  Pool = Recycle1
  Schedule = FourPerHour
}

FileSet {
  Name = "Example FileSet"
  Include {
        Options {
        compression=GZIP
        signature=SHA1
                }
    File = /home/kern/bareos/bin
          }
}

Client {
  Name = Rufus
  Address = rufus
  Catalog = BackupDB
  Password = client_password
}

Client {
  Name = Roxie
  Address = roxie
  Catalog = BackupDB
  Password = client1_password
}

Storage {
  Name = FileStorage
  Address = rufus
  Password = local_storage_password
  Device = RecycleDir
  Media Type = File
}

Storage {
  Name = FileStorage1
  Address = rufus
  Password = local_storage_password
  Device = RecycleDir1
  Media Type = File1
}

Catalog {
  Name = BackupDB
  dbname = bareos; user = bareos; password = ""
}
Messages {
  Name = Standard
  ...
}
Pool {
  Name = Recycle
  Use Volume Once = yes
  Pool Type = Backup
  LabelFormat = "Recycle-"
  AutoPrune = yes
  VolumeRetention = 2h
  Maximum Volumes = 12
  Recycle = yes
}

Pool {
  Name = Recycle1
  Use Volume Once = yes
  Pool Type = Backup
  LabelFormat = "Recycle1-"
  AutoPrune = yes
  VolumeRetention = 2h
  Maximum Volumes = 12
  Recycle = yes
}

\end{verbatim}
\normalsize

and the Storage daemon's configuration file is:

\footnotesize
\begin{verbatim}
Storage {
  Name = my-sd
  WorkingDirectory = "~/bareos/working"
  Pid Directory = "~/bareos/working"
  MaximumConcurrentJobs = 10
}
Director {
  Name = my-dir
  Password = local_storage_password
}
Device {
  Name = RecycleDir
  Media Type = File
  Archive Device = /home/bareos/backups
  LabelMedia = yes;
  Random Access = Yes;
  AutomaticMount = yes;
  RemovableMedia = no;
  AlwaysOpen = no;
}

Device {
  Name = RecycleDir1
  Media Type = File1
  Archive Device = /home/bareos/backups1
  LabelMedia = yes;
  Random Access = Yes;
  AutomaticMount = yes;
  RemovableMedia = no;
  AlwaysOpen = no;
}

Messages {
  Name = Standard
  director = my-dir = all
}
\end{verbatim}
\normalsize

With a little bit of work, you can change the above example into a weekly or
monthly cycle (take care about the amount of archive disk space used).

\label{MultipleDisks}
\section{Backing up to Multiple Disks}
\index[general]{Disks!Backing up to Multiple}
\index[general]{Backing up to Multiple Disks}

Bareos can, of course, use multiple disks, but in general, each disk must be a
separate Device specification in the Storage daemon's conf file, and you must
then select what clients to backup to each disk. You will also want to
give each Device specification a different Media Type so that during
a restore, Bareos will be able to find the appropriate drive.

The situation is a bit more complicated if you want to treat two different
physical disk drives (or partitions) logically as a single drive, which
Bareos does not directly support.  However, it is possible to back up your
data to multiple disks as if they were a single drive by linking the
Volumes from the first disk to the second disk.

For example, assume that you have two disks named {\bf /disk1} and {\bf
/disk2}. If you then create a standard Storage daemon Device resource for
backing up to the first disk, it will look like the following:

\footnotesize
\begin{verbatim}
Device {
  Name = client1
  Media Type = File
  Archive Device = /disk1
  LabelMedia = yes;
  Random Access = Yes;
  AutomaticMount = yes;
  RemovableMedia = no;
  AlwaysOpen = no;
}
\end{verbatim}
\normalsize

Since there is no way to get the above Device resource to reference both {\bf
/disk1} and {\bf /disk2} we do it by pre-creating Volumes on /disk2 with the
following:

\footnotesize
\begin{verbatim}
ln -s /disk2/Disk2-vol001 /disk1/Disk2-vol001
ln -s /disk2/Disk2-vol002 /disk1/Disk2-vol002
ln -s /disk2/Disk2-vol003 /disk1/Disk2-vol003
...
\end{verbatim}
\normalsize

At this point, you can label the Volumes as Volume {\bf Disk2-vol001}, {\bf
Disk2-vol002}, ... and Bareos will use them as if they were on /disk1 but
actually write the data to /disk2. The only minor inconvenience with this
method is that you must explicitly name the disks and cannot use automatic
labeling unless you arrange to have the labels exactly match the links you
have created.

An important thing to know is that Bareos treats disks like tape drives
as much as it can. This means that you can only have a single Volume
mounted at one time on a disk as defined in your Device resource in
the Storage daemon's conf file.  You can have multiple concurrent
jobs running that all write to the one Volume that is being used, but
if you want to have multiple concurrent jobs that are writing to
separate disks drives (or partitions), you will need to define
separate Device resources for each one, exactly as you would do for
two different tape drives.  There is one fundamental difference, however.
The Volumes that you create on the two drives cannot be easily exchanged
as they can for a tape drive, because they are physically resident (already
mounted in a sense) on the particular drive.  As a consequence, you will
probably want to give them different Media Types so that Bareos can
distinguish what Device resource to use during a restore.
An example would be the following:

\footnotesize
\begin{verbatim}
Device {
  Name = Disk1
  Media Type = File1
  Archive Device = /disk1
  LabelMedia = yes;
  Random Access = Yes;
  AutomaticMount = yes;
  RemovableMedia = no;
  AlwaysOpen = no;
}

Device {
  Name = Disk2
  Media Type = File2
  Archive Device = /disk2
  LabelMedia = yes;
  Random Access = Yes;
  AutomaticMount = yes;
  RemovableMedia = no;
  AlwaysOpen = no;
}
\end{verbatim}
\normalsize

With the above device definitions, you can run two concurrent
jobs each writing at the same time, one to {\bf /disk1} and the
other to {\bf /disk2}.  The fact that you have given them different
Media Types will allow Bareos to quickly choose the correct
Storage resource in the Director when doing a restore.

\label{MultipleClients}
\section{Considerations for Multiple Clients}
\index[general]{Clients!Considerations for Multiple}
\index[general]{Multiple Clients}

If we take the above example and add a second Client, here are a few
considerations:

\begin{itemize}
\item Although the second client can write to the same set of  Volumes, you
   will probably want to write to a different  set.
\item You can write to a different set of Volumes by defining  a second Pool,
   which has a different name and a different  {\bf LabelFormat}.
\item If you wish the Volumes for the second client to go into  a different
   directory (perhaps even on a different filesystem  to spread the load), you
   would do so by defining a second  Device resource in the Storage daemon. The
{\bf Name}  must be different, and the {\bf Archive Device} could  be
different. To ensure that Volumes are never mixed from  one pool to another,
you might also define a different  MediaType (e.g. {\bf File1}).
\end{itemize}

In this example, we have two clients, each with a different Pool and a
different number of archive files retained. They also write to different
directories with different Volume labeling.

The Director's configuration file is as follows:

\footnotesize
\begin{verbatim}
Director {
  Name = my-dir
  QueryFile = "~/bareos/bin/query.sql"
  PidDirectory = "~/bareos/working"
  WorkingDirectory = "~/bareos/working"
  Password = dir_password
}
# Basic weekly schedule
Schedule {
  Name = "WeeklySchedule"
  Run = Level=Full fri at 1:30
  Run = Level=Incremental sat-thu at 1:30
}
FileSet {
  Name = "Example FileSet"
  Include {
	Options {
	compression=GZIP
	signature=SHA1
		}
    File = /home/kern/bareos/bin
	  }
}
Job {
  Name = "Backup-client1"
  Type = Backup
  Level = Full
  Client = client1
  FileSet= "Example FileSet"
  Messages = Standard
  Storage = File1
  Pool = client1
  Schedule = "WeeklySchedule"
}
Job {
  Name = "Backup-client2"
  Type = Backup
  Level = Full
  Client = client2
  FileSet= "Example FileSet"
  Messages = Standard
  Storage = File2
  Pool = client2
  Schedule = "WeeklySchedule"
}
Client {
  Name = client1
  Address = client1
  Catalog = BackupDB
  Password = client1_password
  File Retention = 7d
}
Client {
  Name = client2
  Address = client2
  Catalog = BackupDB
  Password = client2_password
}
# Two Storage definitions with different Media Types
#  permits different directories
Storage {
  Name = File1
  Address = rufus
  Password = local_storage_password
  Device = client1
  Media Type = File1
}
Storage {
  Name = File2
  Address = rufus
  Password = local_storage_password
  Device = client2
  Media Type = File2
}
Catalog {
  Name = BackupDB
  dbname = bareos; user = bareos; password = ""
}
Messages {
  Name = Standard
  ...
}
# Two pools permits different cycling periods and Volume names
# Cycle through 15 Volumes (two weeks)
Pool {
  Name = client1
  Use Volume Once = yes
  Pool Type = Backup
  LabelFormat = "Client1-"
  AutoPrune = yes
  VolumeRetention = 13d
  Maximum Volumes = 15
  Recycle = yes
}
# Cycle through 8 Volumes (1 week)
Pool {
  Name = client2
  Use Volume Once = yes
  Pool Type = Backup
  LabelFormat = "Client2-"
  AutoPrune = yes
  VolumeRetention = 6d
  Maximum Volumes = 8
  Recycle = yes
}
\end{verbatim}
\normalsize

and the Storage daemon's configuration file is:

\footnotesize
\begin{verbatim}
Storage {
  Name = my-sd
  WorkingDirectory = "~/bareos/working"
  Pid Directory = "~/bareos/working"
  MaximumConcurrentJobs = 10
}
Director {
  Name = my-dir
  Password = local_storage_password
}
# Archive directory for Client1
Device {
  Name = client1
  Media Type = File1
  Archive Device = /home/bareos/client1
  LabelMedia = yes;
  Random Access = Yes;
  AutomaticMount = yes;
  RemovableMedia = no;
  AlwaysOpen = no;
}
# Archive directory for Client2
Device {
  Name = client2
  Media Type = File2
  Archive Device = /home/bareos/client2
  LabelMedia = yes;
  Random Access = Yes;
  AutomaticMount = yes;
  RemovableMedia = no;
  AlwaysOpen = no;
}
Messages {
  Name = Standard
  director = my-dir = all
}
\end{verbatim}
\normalsize
