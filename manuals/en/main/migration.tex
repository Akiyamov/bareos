
\chapter{Migration and Copy}
\label{MigrationChapter}
\index[general]{Migration}
\index[general]{Copy}

The term Migration, as used in the context of Bareos, means moving data from
one Volume to another.  In particular it refers to a Job (similar to a backup
job) that reads data that was previously backed up to a Volume and writes
it to another Volume.  As part of this process, the File catalog records
associated with the first backup job are purged.  In other words, Migration
moves Bareos Job data from one Volume to another by reading the Job data
from the Volume it is stored on, writing it to a different Volume in a
different Pool, and then purging the database records for the first Job.

The Copy process is essentially identical to the Migration feature with the
exception that the Job that is copied is left unchanged.  This essentially
creates two identical copies of the same backup. However, the copy is treated
as a copy rather than a backup job, and hence is not directly available for
restore. If Bareos finds a copy when a job record is purged (deleted) from the
catalog, it will promote the copy as \textsl{real} backup and will make it
available for automatic restore.

Copy and Migration jobs do not involve the File daemon.

The selection process for which Job or Jobs are migrated
can be based on quite a number of different criteria such as:
\begin{itemize}
\item a single previous Job
\item a Volume
\item a Client
\item a regular expression matching a Job, Volume, or Client name
\item the time a Job has been on a Volume
\item high and low water marks (usage or occupation) of a Pool
\item Volume size
\end{itemize}

The details of these selection criteria will be defined below.

To run a Migration job, you must first define a Job resource very similar
to a Backup Job but with \linkResourceDirective{Dir}{Job}{Type} = Migrate 
instead of \linkResourceDirective{Dir}{Job}{Type} = Backup.
One of the key points to remember is that the Pool that is
specified for the migration job is the only pool from which jobs will
be migrated, with one exception noted below. In addition, the Pool to
which the selected Job or Jobs will be migrated is defined by the
\linkResourceDirective{Dir}{Pool}{Next Pool} = ... in the Pool resource 
specified for the Migration Job.

Bareos permits Pools to contain Volumes with different Media Types.
However, when doing migration, this is a very undesirable condition.  For
migration to work properly, you should use Pools containing only Volumes of
the same Media Type for all migration jobs.

The migration job normally is either manually started or starts
from a Schedule much like a backup job. It searches
for a previous backup Job or Jobs that match the parameters you have
specified in the migration Job resource, primarily a 
\linkResourceDirective{Dir}{Job}{Selection Type}.  
Then for
each previous backup JobId found, the Migration Job will run a new Job which
copies the old Job data from the previous Volume to a new Volume in
the Migration Pool.  It is possible that no prior Jobs are found for
migration, in which case, the Migration job will simply terminate having
done nothing, but normally at a minimum, three jobs are involved during a
migration:

\begin{itemize}
\item The currently running Migration control Job. This is only
      a control job for starting the migration child jobs.
\item The previous Backup Job (already run). The File records
      for this Job are purged if the Migration job successfully
      terminates.  The original data remains on the Volume until
      it is recycled and rewritten.
\item A new Migration Backup Job that moves the data from the
      previous Backup job to the new Volume.  If you subsequently
      do a restore, the data will be read from this Job.
\end{itemize}

If the Migration control job finds a number of JobIds to migrate (e.g.
it is asked to migrate one or more Volumes), it will start one new
migration backup job for each JobId found on the specified Volumes.
Please note that Migration doesn't scale too well since Migrations are
done on a Job by Job basis. This if you select a very large volume or
a number of volumes for migration, you may have a large number of
Jobs that start. Because each job must read the same Volume, they will
run consecutively (not simultaneously).



\section{Important Migration Considerations}
\index[general]{Important Migration Considerations}
\begin{itemize}
\item Each Pool into which you migrate Jobs or Volumes {\bf must}
      contain Volumes of only one \linkResourceDirective{Dir}{Storage}{Media Type}.

\item Migration takes place on a JobId by JobId basis. That is
      each JobId is migrated in its entirety and independently
      of other JobIds. Once the Job is migrated, it will be
      on the new medium in the new Pool, but for the most part,
      aside from having a new JobId, it will appear with all the
      same characteristics of the original job (start, end time, ...).
      The column RealEndTime in the catalog Job table will contain the
      time and date that the Migration terminated, and by comparing
      it with the EndTime column you can tell whether or not the
      job was migrated.  The original job is purged of its File
      records, and its Type field is changed from "B" to "M" to
      indicate that the job was migrated.

\item Jobs on Volumes will be Migration only if the Volume is
      marked, Full, Used, or Error.  Volumes that are still
      marked Append will not be considered for migration. This
      prevents Bareos from attempting to read the Volume at
      the same time it is writing it. It also reduces other deadlock
      situations, as well as avoids the problem that you migrate a
      Volume and later find new files appended to that Volume.

\item As noted above, for the Migration High Bytes, the calculation
      of the bytes to migrate is somewhat approximate.

\item If you keep Volumes of different Media Types in the same Pool,
      it is not clear how well migration will work.  We recommend only
      one \linkResourceDirective{Dir}{Storage}{Media Type} per pool.

\item It is possible to get into a resource deadlock where Bareos does
      not find enough drives to simultaneously read and write all the
      Volumes needed to do Migrations. For the moment, you must take
      care as all the resource deadlock algorithms are not yet implemented.

\item Migration is done only when you run a Migration job. If you set a
      Migration High Bytes and that number of bytes is exceeded in the Pool
      no migration job will automatically start.  You must schedule the
      migration jobs, and they must run for any migration to take place.

\item If you migrate a number of Volumes, a very large number of Migration
      jobs may start.

\item Figuring out what jobs will actually be migrated can be a bit complicated
      due to the flexibility provided by the regex patterns and the number of
      different options.  Turning on a debug level of 100 or more will provide
      a limited amount of debug information about the migration selection
      process.

\item Bareos currently does only minimal Storage conflict resolution, so you
      must take care to ensure that you don't try to read and write to the
      same device or Bareos may block waiting to reserve a drive that it
      will never find. In general, ensure that all your migration
      pools contain only one \linkResourceDirective{Dir}{Storage}{Media Type},
      and that you always
      migrate to pools with different Media Types.

\item The \linkResourceDirective{Dir}{Pool}{Next Pool} = ... directive must be defined in the Pool
     referenced in the Migration Job to define the Pool into which the
     data will be migrated.

\item Pay particular attention to the fact that data is migrated on a Job
     by Job basis, and for any particular Volume, only one Job can read
     that Volume at a time (no simultaneous read), so migration jobs that
     all reference the same Volume will run sequentially.  This can be a
     potential bottle neck and does not scale very well to large numbers
     of jobs.

\item Only migration of Selection Types of Job and Volume have
     been carefully tested. All the other migration methods (time,
     occupancy, smallest, oldest, ...) need additional testing.
\end{itemize}

\section{Configure Copy or Migration Jobs}

The following directives can be used to define a Copy or Migration job:

\paragraph{Job Resource}

\begin{itemize}
    \item \linkResourceDirective{Dir}{Job}{Type} = Migrate{\textbar}Copy
    \item \linkResourceDirective{Dir}{Job}{Selection Type}
        \item \linkResourceDirective{Dir}{Job}{Selection Pattern}
    \item \linkResourceDirective{Dir}{Job}{Pool} \\
        For \linkResourceDirective{Dir}{Job}{Selection Type} other than SQLQuery, 
        this defines what Pool will be examined for finding JobIds to migrate
    \item \linkResourceDirective{Dir}{Job}{Purge Migration Job}
\end{itemize}

\paragraph{Pool Resource}

\begin{itemize}
    \item \linkResourceDirective{Dir}{Pool}{Next Pool} \\
        to what pool Jobs will be migrated
    \item \linkResourceDirective{Dir}{Pool}{Migration Time} \\
        if \linkResourceDirective{Dir}{Job}{Selection Type} = PoolTime
    \item \linkResourceDirective{Dir}{Pool}{Migration High Bytes} \\
        if \linkResourceDirective{Dir}{Job}{Selection Type} = PoolOccupancy
    \item \linkResourceDirective{Dir}{Pool}{Migration Low Bytes} \\
        optional if \linkResourceDirective{Dir}{Job}{Selection Type} = PoolOccupancy is used
    \item \linkResourceDirective{Dir}{Pool}{Storage} \\
        if Copy/Migration involves multiple Storage Daemon, see \nameref{sec:CopyMigrationJobsMultipleStorageDaemons}
\end{itemize}


\subsection{Example Migration Jobs}
\index[general]{Example!Migration Jobs}

When you specify a Migration Job, you must specify all the standard
directives as for a Job.  However, certain such as the Level, Client, and
FileSet, though they must be defined, are ignored by the Migration job
because the values from the original job used instead.

As an example, suppose you have the following Job that
you run every night. To note: there is no Storage directive in the
Job resource; there is a Storage directive in each of the Pool
resources; the Pool to be migrated (File) contains a Next Pool
directive that defines the output Pool (where the data is written
by the migration job).

\begin{bconfig}{Backup Job}
# Define the backup Job
Job {
  Name = "NightlySave"
  Type = Backup
  Level = Incremental                 # default
  Client=rufus-fd
  FileSet="Full Set"
  Schedule = "WeeklyCycle"
  Messages = Standard
  Pool = Default
}

# Default pool definition
Pool {
  Name = Default
  Pool Type = Backup
  AutoPrune = yes
  Recycle = yes
  Next Pool = Tape
  Storage = File
  LabelFormat = "File"
}

# Tape pool definition
Pool {
  Name = Tape
  Pool Type = Backup
  AutoPrune = yes
  Recycle = yes
  Storage = DLTDrive
}

# Definition of File storage device
Storage {
  Name = File
  Address = rufus
  Password = "secret"
  Device = "File"          # same as Device in Storage daemon
  Media Type = File        # same as MediaType in Storage daemon
}

# Definition of DLT tape storage device
Storage {
  Name = DLTDrive
  Address = rufus
  Password = "secret"
  Device = "HP DLT 80"      # same as Device in Storage daemon
  Media Type = DLT8000      # same as MediaType in Storage daemon
}
\end{bconfig}

Where we have included only the essential information -- i.e. the
Director, FileSet, Catalog, Client, Schedule, and Messages resources are
omitted.

As you can see, by running the NightlySave Job, the data will be backed up
to File storage using the Default pool to specify the Storage as File.

Now, if we add the following Job resource to this conf file.

\begin{bconfig}{Migrate Job: migrate all Volumes named File}
Job {
  Name = "migrate-volume"
  Type = Migrate
  Level = Full
  Messages = Standard
  Pool = Default
  Maximum Concurrent Jobs = 4
  Selection Type = Volume
  Selection Pattern = "File"
}
\end{bconfig}

and then run the job named {\bf migrate-volume}, all volumes in the Pool
named Default (as specified in the migrate-volume Job that match the
regular expression pattern {\bf File} will be migrated to tape storage
DLTDrive because the \linkResourceDirective{Dir}{Pool}{Next Pool} in the Default Pool specifies that
Migrations should go to the pool named {\bf Tape}, which uses
Storage {\bf DLTDrive}.

If instead, we use a Job resource as follows:

\begin{bconfig}{Migrate Job: migrate all jobs named *Save}
Job {
  Name = "migrate"
  Type = Migrate
  Level = Full
  Messages = Standard
  Pool = Default
  Maximum Concurrent Jobs = 4
  Selection Type = Job
  Selection Pattern = ".*Save"
}
\end{bconfig}

All jobs ending with the name Save will be migrated from the File Default to
the Tape Pool, or from File storage to Tape storage.

\subsubsection{Multiple Storage Daemons}
    \label{sec:CopyMigrationJobsMultipleStorageDaemons}

Beginning from Bareos \sinceVersion{dir}{Copy and Migration Jobs between different Storage Daemons}{13.2.0}, 
Migration and Copy jobs are also possible from one Storage daemon to another Storage Daemon.

Please note:
\begin{itemize}
 \item the director must have two different storage resources configured (e.g. storage1 and storage2)
    \item each storage needs an own device and an individual pool (e.g. pool1, pool2)
    \item each pool is linked to its own storage via the storage directive in the pool resource
    \item to configure the migration from pool1 to pool2, the \linkResourceDirective{Dir}{Pool}{Next Pool} directive of pool1 has to point to pool2
    \item the copy job itself has to be of type copy/migrate (exactly as already known in copy- and migration jobs)
\end{itemize}

Example:

\begin{bconfig}{bareos-dir.conf: Copy Job between different Storage Daemons}
#bareos-dir.conf

# Fake fileset for copy jobs
Fileset {
  Name = None
  Include {
    Options {
      signature = MD5
    }
  }
}

# Fake client for copy jobs
Client {
  Name = None
  Address = localhost
  Password = "NoNe"
  Catalog = MyCatalog
}

# Source storage for migration
Storage {
   Name = storage1
   Address = sd1.example.com
   Password = "secret1"
   Device = File1
   Media Type = File
}

# Target storage for migration
Storage {
   Name = storage2
   Address = sd2.example.com
   Password = "secret2"
   Device = File2
   Media Type = File2   # Has to be different than in storage1
}

Pool {
   Name = pool1
   Storage = storage1
   Next Pool = pool2    # This points to the target storage
}

Pool {
   Name = pool2
   Storage = storage2
}

Job {
   Name = CopyToRemote
   Type = Copy
   Messages = Standard
   Selection Type = PoolUncopiedJobs
   Spool Data = Yes
   Pool = pool1
}
\end{bconfig}
