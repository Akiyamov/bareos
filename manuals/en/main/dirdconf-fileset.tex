
\subsection{FileSet Include Ressource}
\label{fileset-include}

The Include resource must contain a list of directories and/or files to be
processed in the backup job.  

Normally, all files found in all
subdirectories of any directory in the Include File list will be backed up.
Note, see below for the definition of {\textless}file-list{\textgreater}.
The Include resource may also contain one or more Options resources that
specify options such as compression to be applied to all or any subset of
the files found when processing the file-list for backup. Please see
below for more details concerning Options resources.

There can be any number of {\bf Include} resources within the FileSet, each
having its own list of directories or files to be backed up and the backup
options defined by one or more Options resources.  

Please take note of the following items in the FileSet syntax:

\begin{enumerate}
\item There is no equal sign (=) after the Include and before the opening
   brace (\{). The same is true for the Exclude.
\item Each directory (or filename) to be included or excluded is preceded by a {\bf File
   =}.  Previously they were simply listed on separate lines.
\item The Exclude resource does not accept Options.
\item When using wild-cards or regular expressions, directory names are
   always terminated with a slash (/) and filenames have no trailing slash.
\end{enumerate}

\begin{description}
\directive{dir}{File}{ filename \textbar\ dirname \textbar\ \textbar command \textbar\ \textbackslash\textless includefile-client \textbar\ \textless includefile-server }{}{}{}
    The file list
    consists of one file or directory name per line.  Directory names should be
    specified without a trailing slash with Unix path notation.

    Windows users, please take note to specify directories (even c:/...) in
    Unix path notation. If you use Windows conventions, you will most likely
    not be able to restore your files due to the fact that the Windows
    path separator was defined as an escape character long before Windows
    existed, and Bareos adheres to that convention (i.e. means the next character
    appears as itself).

    You should always specify a full path for every directory and file that you
    list in the FileSet.  In addition, on Windows machines, you should {\bf
    always} prefix the directory or filename with the drive specification
    (e.g.  {\bf c:/xxx}) using Unix directory name separators
    (forward slash).  The drive letter itself can be upper or lower case (e.g.
    c:/xxx or C:/xxx).

    Bareos's default for processing directories is to recursively descend in
    the directory saving all files and subdirectories.  Bareos will not by
    default cross filesystems (or mount points in Unix parlance).  This means
    that if you specify the root partition (e.g.  {\bf /}), Bareos will save
    only the root partition and not any of the other mounted filesystems.
    Similarly on Windows systems, you must explicitly specify each of the
    drives you want saved (e.g.
    {\bf c:/} and {\bf d:/} ...). In addition, at least for Windows systems, you
    will most likely want to enclose each specification within double quotes
    particularly if the directory (or file) name contains spaces. The {\bf df}
    command on Unix systems will show you which mount points you must specify to
    save everything. See below for an example.

Take special care not to include a directory twice or Bareos will backup
the same files two times wasting a lot of space on your archive device.
Including a directory twice is very easy to do.  For example:

\begin{bconfig}{File Set}
  Include {
    Options {
      compression=GZIP
    }
    File = /
    File = /usr
  }
\end{bconfig}
on a Unix system where /usr is a subdirectory (rather than a mounted
filesystem) will cause /usr to be backed up twice.

{\bf {\textless}file-list{\textgreater}} is a list of directory and/or filename names
specified with a {\bf File =} directive. To include names containing spaces,
enclose the name between double-quotes. Wild-cards are not interpreted
in file-lists. They can only be specified in Options resources.

There are a number of special cases when specifying directories and files in a
{\bf file-list}. They are:

\begin{itemize}
\item Any name preceded by an at-sign (@) is assumed to be the  name of a
   file, which contains a list of files each preceded by a "File =".  The
   named file is read once when the configuration file is parsed during the
   Director startup.  Note, that the file is read on the Director's machine
   and not on the Client's.  In fact, the @filename can appear anywhere
   within the conf file where a token would be read, and the contents of
   the named file will be logically inserted in the place of the @filename.
   What must be in the file depends on the location the @filename is
   specified in the conf file.  For example:

\begin{bconfig}{File Set with Include File}
Include {
  Options {
    compression=GZIP
  }
  @/home/files/my-files
}
\end{bconfig}


\item Any name beginning with a vertical bar ({\textbar}) is  assumed to
   be the name of a program.  This program will be executed on the Director's
   machine at the time the Job starts (not when the Director reads the
   configuration file), and any output from that program will be assumed to
   be a list of files or directories, one per line, to be included. Before
   submitting the specified command Bareos will performe
   \ilink{character substitution}{character substitution}.

   This allows you to have a job that, for example, includes all the local
   partitions even if you change the partitioning by adding a disk.  The
   examples below show you how to do this.  However, please note two
   things: \\
   1.  if you want the local filesystems, you probably should be
   using the {\bf fstype} directive and set {\bf onefs=no}.
   \\

   2.  the exact syntax of the command needed in the examples below is very
   system dependent.  For example, on recent Linux systems, you may need to
   add the -P option, on FreeBSD systems, the options will be different as
   well.

   In general, you will need to prefix your command or commands with a {\bf
   sh -c} so that they are invoked by a shell.  This will not be the case
   if you are invoking a script as in the second example below.  Also, you
   must take care to escape (precede with a \textbackslash{}) wild-cards,
   shell character, and to ensure that any spaces in your command are
   escaped as well.  If you use a single quotes (') within a double quote
   ("), Bareos will treat everything between the single quotes as one field
   so it will not be necessary to escape the spaces.  In general, getting
   all the quotes and escapes correct is a real pain as you can see by the
   next example.  As a consequence, it is often easier to put everything in
   a file and simply use the file name within Bareos.  In that case the
   {\bf sh -c} will not be necessary providing the first line of the file
   is {\bf \#!/bin/sh}.

   As an  example:

\begin{bconfig}{File Set with inline script}
Include {
   Options {
     signature = SHA1
   }
   File = "|sh -c 'df -l | grep \"^/dev/hd[ab]\" | grep -v \".*/tmp\" | awk \"{print \\$6}\"'"
}
\end{bconfig}
% workaround for kile editor
\hide{$}
   will produce a list of all the local partitions on a Linux system.
   Quoting is a real problem because you must quote for Bareos  which consists of
   preceding every \textbackslash{} and every " with a \textbackslash{}, and
   you must also quote for the shell command. In the end, it is probably  easier
   just to execute a script file with:

\begin{bconfig}{File Set with external script}
Include {
  Options {
    signature=MD5
  }
  File = "|my_partitions"
}
\end{bconfig}

   where \command{my_partitions} has:

\footnotesize
\begin{verbatim}
#!/bin/sh
df -l | grep "^/dev/hd[ab]" | grep -v ".*/tmp" \
      | awk "{print \$6}"
\end{verbatim}
\normalsize

   If the vertical bar (\verb+|+) in front of \command{my_partitions} is preceded by a
   backslash as in \textbackslash{}\verb+|+, the program will be executed on the
   Client's machine instead of on the Director's machine.
   Please note that if the filename is given within quotes, you
   will need to use two slashes.  An example, provided by John Donagher,
   that backs up all the local UFS partitions on a remote system is:

\begin{bconfig}{File Set with inline script in quotes}
FileSet {
  Name = "All local partitions"
  Include {
    Options {
      signature=SHA1
      onefs=yes
    }
    File = "\\|bash -c \"df -klF ufs | tail +2 | awk '{print \$6}'\""
  }
}
\end{bconfig}

   The above requires two backslash characters after the double quote (one
   preserves  the next one). If you are a Linux user, just change the {\bf ufs}
   to  {\bf ext3} (or your preferred filesystem type), and you will be in
   business.

   If you know what filesystems you have mounted on your system, e.g.
   for Linux only using ext2, ext3 or ext4, you can backup
   all local filesystems using something like:

\begin{bconfig}{File Set to backup all extfs partions}
Include {
   Options {
     signature = SHA1
     onfs=no
     fstype=ext2
   }
   File = /
}
\end{bconfig}

\item Any file-list item preceded by a less-than sign ({\textless})  will be taken
   to be a file. This file will be read on the Director's machine (see
   below for doing it on the Client machine) at the time
   the Job starts, and the  data will be assumed to be a list of directories or
   files,  one per line, to be included. The names should start in  column 1 and
   should not be quoted even if they contain  spaces. This feature allows you to
   modify the external  file and change what will be saved without stopping and
   restarting Bareos as would be necessary if using the @  modifier noted above.
   For example:

\footnotesize
\begin{verbatim}
Include {
  Options {
    signature = SHA1
  }
  File = "</home/files/local-filelist"
}
\end{verbatim}
\normalsize

   If you precede the less-than sign ({\textless}) with a backslash as in
   \textbackslash{}{\textless}, the file-list will be read on the Client machine
   instead of on the Director's machine.  Please note that if the filename
   is given within quotes, you will need to use two slashes.

\footnotesize
\begin{verbatim}
Include {
  Options {
    signature = SHA1
  }
  File = "\\</home/xxx/filelist-on-client"
}
\end{verbatim}
\normalsize

\item     
    \index[general]{Backing up!Partitions}
    \index[general]{Backing up Raw Partitions}
    If you explicitly specify a block device such as {\bf /dev/hda1},  then
   Bareos will assume that this  is a raw partition
   to be backed up. In this case, you are strongly  urged to specify a {\bf
   sparse=yes} include option, otherwise, you  will save the whole partition
   rather than just the actual data that  the partition contains. For example:

\begin{bconfig}{Backup Raw Partitions}
Include {
  Options {
    signature=MD5
    sparse=yes
  }
  File = /dev/hd6
}
\end{bconfig}

   will backup the data in device /dev/hd6. Note, the {bf /dev/hd6} must be
   the raw partition itself. Bareos will not back it up as a raw device if
   you specify a symbolic link to a raw device such as my be created by the
   LVM Snapshot utilities.


\item A file-list may not contain wild-cards. Use directives in the
   Options resource if you wish to specify wild-cards or regular expression
   matching.

\end{itemize}



\directive{dir}{Exclude Dir Containing}{filename}{}{}{}
    This directive can be added to the Include section of the FileSet resource.  If the specified
    filename ({\bf filename-string}) is found on the Client in any directory to be
    backed up, the whole directory will be ignored (not backed up).
    We recommend to use the filename \file{.nobackup}, as it is a hidden file on unix
    systems, and explains what is the purpose of the file.

    For example:

    \begin{bconfig}{Exlude Directories containing the file .nobackup}
    # List of files to be backed up
    FileSet {
        Name = "MyFileSet"
        Include {
            Options {
                signature = MD5
            }
            File = /home
            Exclude Dir Containing = .nobackup
        }
    }
    \end{bconfig}

    But in /home, there may be hundreds of directories of users and some
    people want to indicate that they don't want to have certain
    directories backed up. For example, with the above FileSet, if
    the user or sysadmin creates a file named {\bf .nobackup} in
    specific directories, such as

    \begin{verbatim}
    /home/user/www/cache/.nobackup
    /home/user/temp/.nobackup
    \end{verbatim}

    then Bareos will not backup the two directories named:

    \begin{verbatim}
    /home/user/www/cache
    /home/user/temp
    \end{verbatim}

    NOTE: subdirectories will not be backed up.  That is, the directive
    applies to the two directories in question and any children (be they
    files, directories, etc).


\directive{dir}{Plugin}{plugin-name:plugin-parameter1:plugin-parameter2:$\ldots$}{}{}{}
\label{directive-fileset-plugin}
        Instead of only specifying files, a file set can also use plugins.
        Plugins are additional libraries that handle specific requirements.
        The purpose of plugins is to provide an interface to any system program
        for backup and restore. That allows you, for example, to do database backups without a local dump.

        The syntax and semantics of the Plugin directive require 
        the first part of the string up to the colon to be the name of the plugin.
        Everything after the first colon is ignored by the File daemon but is passed to the plugin. 
        Thus the plugin writer may define the 
        meaning of the rest of the string as he wishes.

        The program \nameref{bpluginfo} can be used, to retreive information about a specific plugin.

        Examples about the bpipe- and the mssql-plugin can be found in the sections about the \nameref{bpipe} and the \nameref{MSSQL}.

        Note: It is also possible to define more than one plugin directive in a FileSet to do several database dumps at once.

\directive{dir}{Options}{$\ldots$}{}{}{}
    See the \nameref{fileset-options} section.

\end{description}


\subsubsection{FileSet Options Ressource}
\label{fileset-options}

The Options resource is optional, but when specified, it will contain a
list of {\bf keyword=value} options to be applied to the file-list.
See below for the definition of file-list.
Multiple Options resources may be specified one after another.  As the
files are found in the specified directories, the Options will applied to
the filenames to determine if and how the file should be backed up.  The
wildcard and regular expression pattern matching parts of the
Options resources are checked in the order they are specified in the
FileSet until the first one that matches. Once one matches, the
compression and other flags within the Options specification will
apply to the pattern matched.

A key point is that in the absence of an Option or no other Option is
matched, every file is accepted for backing up. This means that if
you want to exclude something, you must explicitly specify an Option
with an {\bf exclude = yes} and some pattern matching.

Once Bareos determines that the Options resource matches the file under
consideration, that file will be saved without looking at any other Options
resources that may be present.  This means that any wild cards must appear
before an Options resource without wild cards.

If for some reason, Bareos checks all the Options resources to a file under
consideration for backup, but there are no matches (generally because of wild
cards that don't match), Bareos as a default will then backup the file.  This
is quite logical if you consider the case of no Options clause is specified,
where you want everything to be backed up, and it is important to keep in mind
when excluding as mentioned above.

However, one additional point is that in the case that no match was found,
Bareos will use the options found in the last Options resource.  As a
consequence, if you want a particular set of "default" options, you should put
them in an Options resource after any other Options.

It is a good idea to put all your wild-card and regex expressions inside
double quotes to prevent conf file scanning problems.

This is perhaps a bit overwhelming, so there are a number of examples included
below to illustrate how this works.

You find yourself using a lot of Regex statements, which will cost quite a lot
of CPU time, we recommend you simplify them if you can, or better yet
convert them to Wild statements which are much more efficient.

%\input{autogenerated/datatype-options-table.tex}

The directives within an Options resource may be one of the following:

\begin{description}
    \xdirective{Dir}{}{AutoExclude}{\dtYesNo}{}{yes}{14.2.2}{%
        Automatically exclude files not intended for backup.
        Currently only used for Windows, to exclude files defined in the registry key \registrykey{HKEY_LOCAL_MACHINE\SYSTEM\CurrentControlSet\Control\BackupRestore\FilesNotToBackup}, see section \nameref{FilesNotToBackup}.
    }

    \item [compression={\textless}GZIP{\textbar}LZO{\textbar}LZFAST{\textbar}LZ4{\textbar}LZ4HC{\textgreater}] \hfill \\
    \begin{description}
        \item [compression=GZIP] \hfill \\
        \index[dir]{compression}
        \index[dir]{Directive!compression}
        All files saved will be software compressed using the GNU ZIP
        compression format.  The compression is done on a file by file basis by
        the File daemon.  If there is a problem reading the tape in a single
        record of a file, it will at most affect that file and none of the other
        files on the tape.  Normally this option is {\bf not} needed if you have
        a modern tape drive as the drive will do its own compression.  In fact,
        if you specify software compression at the same time you have hardware
        compression turned on, your files may actually take more space on the
        volume.

        Software compression is very important if you are writing your Volumes
        to a file, and it can also be helpful if you have a fast computer but a
        slow network, otherwise it is generally better to rely your tape drive's
        hardware compression.  As noted above, it is not generally a good idea
        to do both software and hardware compression.

        Specifying {\bf GZIP} uses the default compression level 6 (i.e.  {\bf
        GZIP} is identical to {\bf GZIP6}).  If you want a different compression
        level (1 through 9), you can specify it by appending the level number
        with no intervening spaces to {\bf GZIP}.  Thus {\bf compression=GZIP1}
        would give minimum compression but the fastest algorithm, and {\bf
        compression=GZIP9} would give the highest level of compression, but
        requires more computation.  According to the GZIP documentation,
        compression levels greater than six generally give very little extra
        compression and are rather CPU intensive.

        You can overwrite this option per Storage resource using the \linkResourceDirective{Dir}{Storage}{Allow Compression} option.

        \item [compression=LZO] \hfill \\
        \index[dir]{compression}
        \index[dir]{Directive!compression}
        All files saved will be software compressed using the LZO
        compression format. The compression is done on a file by file basis by
        the File daemon. Everything else about GZIP is true for LZO.

        LZO provides much faster compression and decompression speed but lower
        compression ratio than GZIP. If your CPU is fast enough you should be able
        to compress your data without making the backup duration longer.

        Note that Bareos only use one compression level LZO1X-1 specified by LZO.

        You can overwrite this option per Storage resource using the \linkResourceDirective{Dir}{Storage}{Allow Compression} option.

        \item [compression=LZFAST] \hfill \\
        \index[dir]{compression}
        \index[dir]{Directive!compression}
        All files saved will be software compressed using the LZFAST
        compression format. The compression is done on a file by file basis by
        the File daemon. Everything else about GZIP is true for LZFAST.

        LZFAST provides much faster compression and decompression speed but lower
        compression ratio than GZIP. If your CPU is fast enough you should be able
        to compress your data without making the backup duration longer.

        You can overwrite this option per Storage resource using the \linkResourceDirective{Dir}{Storage}{Allow Compression} option.

        \item [compression=LZ4] \hfill \\
        \index[dir]{compression}
        \index[dir]{Directive!compression}
        All files saved will be software compressed using the LZ4
        compression format. The compression is done on a file by file basis by
        the File daemon. Everything else about GZIP is true for LZ4.

        LZ4 provides much faster compression and decompression speed but lower
        compression ratio than GZIP. If your CPU is fast enough you should be able
        to compress your data without making the backup duration longer.

        Both LZ4 and LZ4HC have the same decompression speed which is about twice
        the speed of the LZO compression. So for a restore both LZ4 and LZ4HC are
        good candidates.

        You can overwrite this option per Storage resource using the \linkResourceDirective{Dir}{Storage}{Allow Compression} option.

        \warning{As LZ4 compression is not supported by Bacula, you need to set \linkResourceDirective{Fd}{Client}{Compatible} = no.}

        \item [compression=LZ4HC] \hfill \\
        \index[dir]{compression}
        \index[dir]{Directive!compression}
        All files saved will be software compressed using the LZ4HC
        compression format. The compression is done on a file by file basis by
        the File daemon. Everything else about GZIP is true for LZ4.

        LZ4HC is the High Compression version of the LZ4 compression. It has
        a higher compression ratio than LZ4 and is more comparable to GZIP-6
        in both compression rate and cpu usage.

        Both LZ4 and LZ4HC have the same decompression speed which is about twice
        the speed of the LZO compression. So for a restore both LZ4 and LZ4HC are
        good candidates.

        You can overwrite this option per Storage resource using the \linkResourceDirective{Dir}{Storage}{Allow Compression} option.
    \end{description}

 \item [signature={\textless}SHA1{\textbar}MD5{\textgreater}] \hfill \\
    \begin{description}
        \item [signature=SHA1] \hfill \\
        \index[dir]{signature}
        \index[dir]{SHA1}
        \index[dir]{Directive!signature}
        An SHA1 signature will be computed for all The SHA1 algorithm is
        purported to be some what slower than the MD5 algorithm, but at the same
        time is significantly better from a cryptographic point of view (i.e.
        much fewer collisions, much lower probability of being hacked.) It adds
        four more bytes than the MD5 signature.  We strongly recommend that
        either this option or MD5 be specified as a default for all files.
        Note, only one of the two options MD5 or SHA1 can be computed for any
        file.

        \item [signature=MD5] \hfill \\
        \index[dir]{signature}
        \index[dir]{MD5}
        \index[dir]{Directive!signature}
        An MD5 signature will be computed for all files saved.  Adding this
        option generates about 5\% extra overhead for each file saved.  In
        addition to the additional CPU time, the MD5 signature adds 16 more
        bytes per file to your catalog.  We strongly recommend that this option
        or the SHA1 option be specified as a default for all files.
    \end{description}


\item[basejob={\textless}options{\textgreater}]
\index[dir]{basejob}
\index[dir]{Directive!basejob}

The options letters specified are used when running a {\bf Backup Level=Full}
with BaseJobs. The options letters are the same than in the \textbf{verify=}
option below.

\item[accurate={\textless}options{\textgreater}] \index[dir]{accurate}
  \index[dir]{Directive!accurate} The options letters specified are used when
  running a {\bf Backup Level=Incremental/Differential} in Accurate mode. The
  options letters are the same than in the \textbf{verify=} option below.

\item [verify={\textless}options{\textgreater}] \hfill \\
\index[dir]{verify}
\index[dir]{Directive!verify}
   The options letters specified are used  when running a {\bf Verify
   Level=Catalog} as well as the  {\bf DiskToCatalog} level job. The options
   letters may be any  combination of the following:

      \begin{description}

      \item {\bf i}
      compare the inodes

      \item {\bf p}
      compare the permission bits

      \item {\bf n}
      compare the number of links

      \item {\bf u}
      compare the user id

      \item {\bf g}
      compare the group id

      \item {\bf s}
      compare the size

      \item {\bf a}
      compare the access time

      \item {\bf m}
      compare the modification time (st\_mtime)

      \item {\bf c}
      compare the change time (st\_ctime)

      \item {\bf d}
      report file size decreases

      \item {\bf 5}
      compare the MD5 signature

      \item {\bf 1}
      compare the SHA1 signature

      \item {\bf A}
      Only for Accurate option, it allows to always backup the file

      \end{description}

   A useful set of general options on the {\bf Level=Catalog}  or {\bf
   Level=DiskToCatalog}  verify is {\bf pins5} i.e. compare permission bits,
   inodes, number  of links, size, and MD5 changes.

\item [onefs=yes{\textbar}no] \hfill \\
\index[dir]{onefs}
\index[dir]{Directive!onefs}
   If set to {\bf yes} (the default), {\bf Bareos} will remain on a single
   file system.  That is it will not backup file systems that are mounted
   on a subdirectory.  If you are using a *nix system, you may not even be
   aware that there are several different filesystems as they are often
   automatically mounted by the OS (e.g.  /dev, /net, /sys, /proc, ...).
   Bareos will inform you when it decides not to
   traverse into another filesystem.  This can be very useful if you forgot
   to backup a particular partition.  An example of the informational
   message in the job report is:

\footnotesize
\begin{verbatim}
rufus-fd: /misc is a different filesystem. Will not descend from / into /misc
rufus-fd: /net is a different filesystem. Will not descend from / into /net
rufus-fd: /var/lib/nfs/rpc_pipefs is a different filesystem. Will not descend from /var/lib/nfs into /var/lib/nfs/rpc_pipefs
rufus-fd: /selinux is a different filesystem. Will not descend from / into /selinux
rufus-fd: /sys is a different filesystem. Will not descend from / into /sys
rufus-fd: /dev is a different filesystem. Will not descend from / into /dev
rufus-fd: /home is a different filesystem. Will not descend from / into /home
\end{verbatim}
\normalsize

   If you wish to backup multiple filesystems, you can  explicitly
   list each filesystem you want saved.  Otherwise, if you set the onefs option
   to {\bf no}, Bareos will backup  all mounted file systems (i.e. traverse mount
   points) that  are found within the {\bf FileSet}. Thus if  you have NFS or
   Samba file systems mounted on a directory listed  in your FileSet, they will
   also be backed up. Normally, it is  preferable to set {\bf onefs=yes} and to
   explicitly name  each filesystem you want backed up. Explicitly naming  the
   filesystems you want backed up avoids the possibility  of getting into a
   infinite loop recursing filesystems.  Another possibility is to
   use {\bf onefs=no} and to set {\bf fstype=ext2, ...}.
   See the example below for more details.

   If you think that Bareos should be backing up a particular directory
   and it is not, and you have {\bf onefs=no} set, before you complain,
   please do:

\footnotesize
\begin{verbatim}
  stat /
  stat <filesystem>
\end{verbatim}
\normalsize

where you replace {\bf filesystem} with the one in question.  If the
{\bf Device:} number is different for / and for your filesystem, then they
are on different filesystems.  E.g.
\footnotesize
\begin{verbatim}
stat /
  File: `/'
  Size: 4096            Blocks: 16         IO Block: 4096   directory
Device: 302h/770d       Inode: 2           Links: 26
Access: (0755/drwxr-xr-x)  Uid: (    0/    root)   Gid: (    0/    root)
Access: 2005-11-10 12:28:01.000000000 +0100
Modify: 2005-09-27 17:52:32.000000000 +0200
Change: 2005-09-27 17:52:32.000000000 +0200

stat /net
  File: `/home'
  Size: 4096            Blocks: 16         IO Block: 4096   directory
Device: 308h/776d       Inode: 2           Links: 7
Access: (0755/drwxr-xr-x)  Uid: (    0/    root)   Gid: (    0/    root)
Access: 2005-11-10 12:28:02.000000000 +0100
Modify: 2005-11-06 12:36:48.000000000 +0100
Change: 2005-11-06 12:36:48.000000000 +0100
\end{verbatim}
\normalsize

   Also be aware that even if you include {\bf /home} in your list
   of files to backup, as you most likely should, you will get the
   informational message that  "/home is a different filesystem" when
   Bareos is processing the {\bf /} directory.  This message does not
   indicate an error. This message means that while examining the
   {\bf File =} referred to in the second part of the message, Bareos will
   not descend into the directory mentioned in the first part of the message.
   However, it is possible that the separate filesystem will be backed up
   despite the message. For example, consider the following FileSet:

\footnotesize
\begin{verbatim}
  File = /
  File = /var
\end{verbatim}
\normalsize

   where {\bf /var} is a separate filesystem.  In this example, you will get a
   message saying that Bareos will not decend from {\bf /} into {\bf /var}.  But
   it is important to realise that Bareos will descend into {\bf /var} from the
   second File directive shown above.  In effect, the warning is bogus,
   but it is supplied to alert you to possible omissions from your FileSet. In
   this example, {\bf /var} will be backed up.  If you changed the FileSet such
   that it did not specify {\bf /var}, then {\bf /var} will not be backed up.

\item [honor nodump flag={\textless}yes{\textbar}no{\textgreater}] \hfill \\
\index[dir]{honornodumpflag}
\index[dir]{Directive!honornodumpflag}
   If your file system supports the {\bf nodump} flag (e. g. most
   BSD-derived systems) Bareos will honor the setting of the flag
   when this option is set to {\bf yes}. Files having this flag set
   will not be included in the backup and will not show up in the
   catalog. For directories with the {\bf nodump} flag set recursion
   is turned off and the directory will be listed in the catalog.
   If the {\bf honor nodump flag} option is not defined
   or set to {\bf no} every file and directory will be eligible for
   backup.

\item [portable=yes{\textbar}no] \hfill \\
\index[dir]{portable}
\index[dir]{Directive!portable}
\label{portable}
   If set to {\bf yes} (default is {\bf no}), the Bareos File daemon will
   backup Win32 files in a portable format, but not all Win32 file
   attributes will be saved and restored.  By default, this option is set
   to {\bf no}, which means that on Win32 systems, the data will be backed
   up using Windows API calls and on WinNT/2K/XP, all the security and
   ownership attributes will be properly backed up (and restored).  However
   this format is not portable to other systems -- e.g.  Unix, Win95/98/Me.
   When backing up Unix systems, this option is ignored, and unless you
   have a specific need to have portable backups, we recommend accept the
   default ({\bf no}) so that the maximum information concerning your files
   is saved.

\item [recurse=yes{\textbar}no] \hfill \\
\index[dir]{recurse}
\index[dir]{Directive!recurse}
   If set to {\bf yes} (the default), Bareos will recurse (or descend) into
   all subdirectories found unless the directory is explicitly excluded
   using an {\bf exclude} definition.  If you set {\bf recurse=no}, Bareos
   will save the subdirectory entries, but not descend into the
   subdirectories, and thus will not save the files or directories
   contained in the subdirectories.  Normally, you will want the default
   ({\bf yes}).

\item [sparse=yes{\textbar}no] \hfill \\
\index[dir]{sparse}
\index[dir]{Directive!sparse}
   Enable special code that checks for sparse files such as created by
   ndbm.  The default is {\bf no}, so no checks are made for sparse files.
   You may specify {\bf sparse=yes} even on files that are not sparse file.
   No harm will be done, but there will be a small additional overhead to
   check for buffers of all zero, and if there is a 32K block of all zeros
   (see below), that block will become a hole in the file, which
   may not be desirable if the original file was not a sparse file.

   {\bf Restrictions:} Bareos reads files in 32K buffers.  If the whole
   buffer is zero, it will be treated as a sparse block and not written to
   tape.  However, if any part of the buffer is non-zero, the whole buffer
   will be written to tape, possibly including some disk sectors (generally
   4098 bytes) that are all zero.  As a consequence, Bareos's detection of
   sparse blocks is in 32K increments rather than the system block size.
   If anyone considers this to be a real problem, please send in a request
   for change with the reason.

   If you are not familiar with sparse files, an example is say a file
   where you wrote 512 bytes at address zero, then 512 bytes at address 1
   million.  The operating system will allocate only two blocks, and the
   empty space or hole will have nothing allocated.  However, when you read
   the sparse file and read the addresses where nothing was written, the OS
   will return all zeros as if the space were allocated, and if you backup
   such a file, a lot of space will be used to write zeros to the volume.
   Worse yet, when you restore the file, all the previously empty space
   will now be allocated using much more disk space.  By turning on the
   {\bf sparse} option, Bareos will specifically look for empty space in
   the file, and any empty space will not be written to the Volume, nor
   will it be restored.  The price to pay for this is that Bareos must
   search each block it reads before writing it.  On a slow system, this
   may be important.  If you suspect you have sparse files, you should
   benchmark the difference or set sparse for only those files that are
   really sparse.

   You probably should not use this option on files or raw disk devices
   that are not really sparse files (i.e. have holes in them).

\item [readfifo=yes{\textbar}no] \hfill \\
\index[dir]{readfifo}
\index[dir]{Directive!readfifo}
\label{readfifo}
   If enabled, tells the Client to read the data on a backup and write the
   data on a restore to any FIFO (pipe) that is explicitly mentioned in the
   FileSet.  In this case, you must have a program already running that
   writes into the FIFO for a backup or reads from the FIFO on a restore.
   This can be accomplished with the {\bf RunBeforeJob} directive.  If this
   is not the case, Bareos will hang indefinitely on reading/writing the
   FIFO. When this is not enabled (default), the Client simply saves the
   directory entry for the FIFO.

   Normally, when Bareos runs a RunBeforeJob, it waits until that
   script terminates, and if the script accesses the FIFO to write
   into it, the Bareos job will block and everything will stall.
   However, Vladimir Stavrinov as supplied tip that allows this feature
   to work correctly.  He simply adds the following to the beginning
   of the RunBeforeJob script:

\begin{verbatim}
   exec > /dev/null
\end{verbatim}


\begin{bconfig}{FileSet with Fifo}
Include {
  Options {
    signature=SHA1
    readfifo=yes
  }
  File = /home/abc/fifo
}
\end{bconfig}

   This feature can be used to do a "hot" database backup.  
   You can use the {\bf RunBeforeJob} to create the fifo
   and to start a program that dynamically reads your database and writes
   it to the fifo.  Bareos will then write it to the Volume. 

   During the restore operation, the inverse is true, after Bareos creates
   the fifo if there was any data stored with it (no need to explicitly
   list it or add any options), that data will be written back to the fifo.
   As a consequence, if any such FIFOs exist in the fileset to be restored,
   you must ensure that there is a reader program or Bareos will block, and
   after one minute, Bareos will time out the write to the fifo and move on
   to the next file.

    If you are planing to use a Fifo for backup, better take a look to the \nameref{bpipe} section.


\item [noatime=yes{\textbar}no] \hfill \\
\index[dir]{noatime}
\index[dir]{Directive!noatime}
   If enabled, and if your Operating System supports the O\_NOATIME file
   open flag, Bareos will open all files to be backed up with this option.
   It makes it possible to read a file without updating the inode atime
   (and also without the inode ctime update which happens if you try to set
   the atime back to its previous value).  It also prevents a race
   condition when two programs are reading the same file, but only one does
   not want to change the atime.  It's most useful for backup programs and
   file integrity checkers (and Bareos can fit on both categories).

   This option is particularly useful for sites where users are sensitive
   to their MailBox file access time.  It replaces both the {\bf keepatime}
   option without the inconveniences of that option (see below).

   If your Operating System does not support this option, it will be
   silently ignored by Bareos.


\item [mtimeonly=yes{\textbar}no] \hfill \\
\index[dir]{mtimeonly}
\index[dir]{Directive!mtimeonly}
   If enabled, tells the Client that the selection of files during
   Incremental and Differential backups should based only on the st\_mtime
   value in the stat() packet.  The default is {\bf no} which means that
   the selection of files to be backed up will be based on both the
   st\_mtime and the st\_ctime values.  In general, it is not recommended
   to use this option.

\item [keepatime=yes{\textbar}no] \hfill \\
\index[dir]{keepatime}
\index[dir]{Directive!keepatime}
   The default is {\bf no}.  When enabled, Bareos will reset the st\_atime
   (access time) field of files that it backs up to their value prior to
   the backup.  This option is not generally recommended as there are very
   few programs that use st\_atime, and the backup overhead is increased
   because of the additional system call necessary to reset the times.
   However, for some files, such as mailboxes, when Bareos backs up the
   file, the user will notice that someone (Bareos) has accessed the
   file. In this, case keepatime can be useful.
   (I'm not sure this works on Win32).

   Note, if you use this feature, when Bareos resets the access time, the
   change time (st\_ctime) will automatically be modified by the system,
   so on the next incremental job, the file will be backed up even if
   it has not changed. As a consequence, you will probably also want
   to use {\bf mtimeonly = yes} as well as keepatime (thanks to
   Rudolf Cejka for this tip).

\item [checkfilechanges=yes{\textbar}no] \hfill \\
\index[dir]{checkfilechanges}
\index[dir]{Directive!checkfilechanges}
   If enabled, the Client will check size, age of each file after
   their backup to see if they have changed during backup. If time
   or size mismatch, an error will raise.

\begin{verbatim}
 zog-fd: Client1.2007-03-31_09.46.21 Error: /tmp/test mtime changed during backup.
\end{verbatim}

   In general, it is recommended to use this option.

\item [hardlinks=yes{\textbar}no] \hfill \\
\index[dir]{hardlinks}
\index[dir]{Directive!hardlinks}
   When enabled (default), this directive will cause hard links to be
   backed up. However, the File daemon keeps track of hard linked files and
   will backup the data only once. The process of keeping track of the
   hard links can be quite expensive if you have lots of them (tens of
   thousands or more). This doesn't occur on normal Unix systems, but if
   you use a program like BackupPC, it can create hundreds of thousands, or
   even millions of hard links. Backups become very long and the File daemon
   will consume a lot of CPU power checking hard links.  In such a case,
   set {\bf hardlinks=no} and hard links will not be backed up.  Note, using
   this option will most likely backup more data and on a restore the file
   system will not be restored identically to the original.

\item [wild={\textless}string{\textgreater}] \hfill \\
\index[dir]{wild}
\index[dir]{Directive!wild}
   Specifies a wild-card string to be applied to the filenames and
   directory names.  Note, if {\bf Exclude} is not enabled, the wild-card
   will select which files are to be included.  If {\bf Exclude=yes} is
   specified, the wild-card will select which files are to be excluded.
   Multiple wild-card directives may be specified, and they will be applied
   in turn until the first one that matches.  Note, if you exclude a
   directory, no files or directories below it will be matched.

   You may want to test your expressions prior to running your
   backup by using the bwild program. Please see the
   \ilink{Utilities}{bwild} chapter of this manual for
   more. You can also test your full FileSet definition by using
   the \ilink{estimate}{estimate} command in the Console
   chapter of this manual.
   It is recommended to enclose the string in double quotes.

\item [wilddir={\textless}string{\textgreater}] \hfill \\
\index[dir]{wilddir}
\index[dir]{Directive!wilddir}
   Specifies a wild-card string to be applied to directory names only.  No
   filenames will be matched by this directive.  Note, if {\bf Exclude} is
   not enabled, the wild-card will select directories to be
   included.  If {\bf Exclude=yes} is specified, the wild-card will select
   which directories are to be excluded.  Multiple wild-card directives may be
   specified, and they will be applied in turn until the first one that
   matches.  Note, if you exclude a directory, no files or directories
   below it will be matched.

   It is recommended to enclose the string in double quotes.

   You may want to test your expressions prior to running your
   backup by using the bwild program. Please see the
   \ilink{Utilities}{bwild} chapter of this manual for
   more. You can also test your full FileSet definition by using
   the \ilink{estimate}{estimate} command in the Console
   chapter of this manual.
   An example of excluding with the WildDir option on Win32 machines is
   presented below.

\item [wildfile={\textless}string{\textgreater}] \hfill \\
\index[dir]{wildfile}
\index[dir]{Directive!wildfile}
   Specifies a wild-card string to be applied to non-directories. That
   is no directory entries will be matched by this directive.
   However, note that the match is done against the full path and filename,
   so your wild-card string must take into account that filenames
   are preceded by the full path.
   If {\bf Exclude}
   is not enabled, the wild-card will select which files are to be
   included.  If {\bf Exclude=yes} is specified, the wild-card will select
   which files are to be excluded.  Multiple wild-card directives may be
   specified, and they will be applied in turn until the first one that
   matches.

   It is recommended to enclose the string in double quotes.

   You may want to test your expressions prior to running your
   backup by using the bwild program. Please see the
   \ilink{Utilities}{bwild} chapter of this manual for
   more. You can also test your full FileSet definition by using
   the \ilink{estimate}{estimate} command in the Console
   chapter of this manual.
   An example of excluding with the WildFile option on Win32 machines is
   presented below.


\item [regex={\textless}string{\textgreater}] \hfill \\
\index[dir]{regex}
\index[dir]{Directive!regex}

\label{FileRegex}

   Specifies a POSIX extended regular expression to be applied to the
   filenames and directory names, which include the full path.  If {\bf
   Exclude} is not enabled, the regex will select which files are to be
   included.  If {\bf Exclude=yes} is specified, the regex will select
   which files are to be excluded.  Multiple regex directives may be
   specified within an Options resource, and they will be applied in turn
   until the first one that matches.  Note, if you exclude a directory, no
   files or directories below it will be matched.

   It is recommended to enclose the string in double quotes.

   The regex libraries differ from one operating system to
   another, and in addition, regular expressions are complicated,
   so you may want to test your expressions prior to running your
   backup by using the bregex program. Please see the
   \ilink{Utilities}{bwild} chapter of this manual for
   more. You can also test your full FileSet definition by using
   the \ilink{estimate}{estimate} command in the Console
   chapter of this manual.

   You find yourself using a lot of Regex statements, which will cost quite a lot
   of CPU time, we recommend you simplify them if you can, or better yet
   convert them to Wild statements which are much more efficient.


\item [regexfile={\textless}string{\textgreater}] \hfill \\
\index[dir]{regexfile}
\index[dir]{Directive!regexfile}
   Specifies a POSIX extended regular expression to be applied to
   non-directories. No directories will be matched by this directive.
   However, note that the match is done against the full path and
   filename, so your regex string must take into account that filenames
   are preceded by the full path.
   If {\bf Exclude} is not enabled, the regex will select which files are
   to be included.  If {\bf Exclude=yes} is specified, the regex will
   select which files are to be excluded.  Multiple regex directives may be
   specified, and they will be applied in turn until the first one that
   matches.

   It is recommended to enclose the string in double quotes.

   The regex libraries differ from one operating system to
   another, and in addition, regular expressions are complicated,
   so you may want to test your expressions prior to running your
   backup by using the bregex program. Please see the
   \ilink{Utilities}{bregex} chapter of this manual for
   more.


\item [regexdir={\textless}string{\textgreater}] \hfill \\
\index[dir]{regexdir}
\index[dir]{Directive!regexdir}
   Specifies a POSIX extended regular expression to be applied to directory
   names only.  No filenames will be matched by this directive.  Note, if
   {\bf Exclude} is not enabled, the regex will select directories
   files are to be included.  If {\bf Exclude=yes} is specified, the
   regex will select which files are to be excluded.  Multiple
   regex directives may be specified, and they will be applied in turn
   until the first one that matches.  Note, if you exclude a directory, no
   files or directories below it will be matched.

   It is recommended to enclose the string in double quotes.

   The regex libraries differ from one operating system to
   another, and in addition, regular expressions are complicated,
   so you may want to test your expressions prior to running your
   backup by using the bregex program. Please see the
   \ilink{Utilities}{bregex} chapter of this manual for
   more.


\xdirective{dir}{}{Exclude}{\dtYesNo}{}{no}{}{%
   When enabled, any files matched within the
   Options will be excluded from the backup.
}

\item [aclsupport=yes{\textbar}no] \hfill \\
\index[dir]{aclsupport}
\index[dir]{Directive!aclsupport}
\label{ACLSupport}
   The default is {\bf no}.  If this option is set to yes, and you have the
   POSIX {\bf libacl} installed on your Linux system, Bareos will backup the
   file and directory Unix Access Control Lists (ACL) as defined in IEEE Std
   1003.1e draft 17 and "POSIX.1e" (abandoned).  This feature is
   available on Unix systems only and requires the Linux ACL library. Bareos is
   automatically compiled with ACL support if the {\bf libacl} library is
   installed on your Linux system (shown in config.out).  While restoring the
   files Bareos will try to restore the ACLs, if there is no ACL support
   available on the system, Bareos restores the files and directories but
   not the ACL information.  Please note, if you backup an EXT3 or XFS
   filesystem with ACLs, then you restore them to a different filesystem
   (perhaps reiserfs) that does not have ACLs, the ACLs will be ignored.

   For other operating systems there is support for either POSIX ACLs or
   the more extensible NFSv4 ACLs.

   The ACL stream format between Operation Systems is \textbf{not}
   compatible so for example an ACL saved on Linux cannot be restored on
   Solaris.

   The following Operating Systems are currently supported:

   \begin{enumerate}
   \item AIX (pre-5.3 (POSIX) and post 5.3 (POSIX and NFSv4) ACLs)
   \item Darwin
   \item FreeBSD (POSIX and NFSv4/ZFS ACLs)
   \item HPUX
   \item IRIX
   \item Linux
   \item Solaris (POSIX and NFSv4/ZFS ACLs)
   \item Tru64
   \end{enumerate}

\label{XattrSupport}
\item [xattrsupport=yes{\textbar}no] \hfill \\
\index[dir]{xattrsupport}
\index[dir]{Directive!xattrsupport}
   The default is {\bf no}.  If this option is set to yes, and your
   operating system support either so called Extended Attributes or
   Extensible Attributes Bareos will backup the file and directory
   XATTR data. This feature is available on UNIX only and depends on
   support of some specific library calls in libc.

   The XATTR stream format between Operating Systems is {\bf not}
   compatible so an XATTR saved on Linux cannot for example be restored
   on Solaris.

   On some operating systems ACLs are also stored as Extended Attributes
   (Linux, Darwin, FreeBSD) Bareos checks if you have the aclsupport
   option enabled and if so will not save the same info when saving
   extended attribute information. Thus ACLs are only saved once.

   The following Operating Systems are currently supported:

   \begin{enumerate}
   \item AIX (Extended Attributes)
   \item Darwin (Extended Attributes)
   \item FreeBSD (Extended Attributes)
   \item IRIX (Extended Attributes)
   \item Linux (Extended Attributes)
   \item NetBSD (Extended Attributes)
   \item Solaris (Extended Attributes and Extensible Attributes)
   \item Tru64 (Extended Attributes)
   \end{enumerate}

\item [ignore case=yes{\textbar}no] \hfill \\
\index[dir]{ignore case}
\index[dir]{Directive!ignore case}
   The default is {\bf no}.  On Windows systems, you will almost surely
   want to set this to {\bf yes}.  When this directive is set to {\bf yes}
   all the case of character will be ignored in wild-card and regex
   comparisons.  That is an uppercase A will match a lowercase a.

\item [fstype=filesystem-type] \hfill \\
\index[dir]{fstype}
\index[dir]{Directive!fstype}
   This option allows you to select files and directories by the
   filesystem type.  The permitted filesystem-type names are:

   ext2, jfs, ntfs, proc, reiserfs, xfs, usbdevfs, sysfs, smbfs,
   iso9660.

   You may have multiple Fstype directives, and thus permit matching
   of multiple filesystem types within a single Options resource.  If
   the type specified on the fstype directive does not match the
   filesystem for a particular directive, that directory will not be
   backed up.  This directive can be used to prevent backing up
   non-local filesystems. Normally, when you use this directive, you
   would also set {\bf onefs=no} so that Bareos will traverse filesystems.

   This option is not implemented in Win32 systems.

\item [DriveType=Windows-drive-type] \hfill \\
\index[dir]{DriveType}
\index[dir]{Directive!DriveType}
   This option is effective only on Windows machines and is
   somewhat similar to the Unix/Linux {\bf fstype} described
   above, except that it allows you to select what Windows
   drive types you want to allow.  By default all drive
   types are accepted.

   The permitted drivetype names are:

   removable, fixed, remote, cdrom, ramdisk

   You may have multiple Driveype directives, and thus permit matching
   of multiple drive types within a single Options resource.  If
   the type specified on the drivetype directive does not match the
   filesystem for a particular directive, that directory will not be
   backed up.  This directive can be used to prevent backing up
   non-local filesystems. Normally, when you use this directive, you
   would also set {\bf onefs=no} so that Bareos will traverse filesystems.

   This option is not implemented in Unix/Linux systems.

\item [hfsplussupport=yes{\textbar}no] \hfill \\
\index[dir]{hfsplussupport}
\index[dir]{Directive!hfsplussupport}
   This option allows you to turn on support for Mac OSX HFS plus
   finder information.

\item [strippath={\textless}integer{\textgreater}] \hfill \\
\index[dir]{strippath}
\index[dir]{Directive!strippath}
   This option will cause {\bf integer} paths to be stripped from
   the front of the full path/filename being backed up. This can
   be useful if you are migrating data from another vendor or if
   you have taken a snapshot into some subdirectory.  This directive
   can cause your filenames to be overlayed with regular backup data,
   so should be used only by experts and with great care.

\item [size=sizeoption] \hfill \\
\index[dir]{size}
\index[dir]{Directive!size}
   This option will allow you to select files by their actual size.
   You can select either files smaller than a certain size or bigger
   then a certain size, files of a size in a certain range or files
   of a size which is within 1 \% of its actual size.

   The following settings can be used:

   \begin{enumerate}
   \item {\bf {\textless}size{\textgreater}-{\textless}size{\textgreater}} - Select file in range size - size.
   \item {\bf {\textless}size} - Select files smaller than size.
   \item {\bf {\textgreater}size} - Select files bigger than size.
   \item {\bf size} - Select files which are within 1 \% of size.
   \end{enumerate}

\item [shadowing=none{\textbar}localwarn{\textbar}localremove{\textbar}globalwarn{\textbar}globalremove] \hfill \\
\index[dir]{shadowing}
\index[dir]{Directive!shadowing}
   The default is {\bf none}. This option performs a check within the
   fileset for any file-list entries which are shadowing each other.
   Lets say you specify / and /usr but /usr is not a separate filesystem.
   Then in the normal situation both / and /usr would lead to data being
   backed up twice.

   The following settings can be used:

   \begin{enumerate}
   \item none - Do NO shadowing check
   \item localwarn - Do shadowing check within one include block and warn
   \item localremove - Do shadowing check within one include block and remove duplicates
   \item globalwarn - Do shadowing check between all include blocks and warn
   \item globalremove - Do shadowing check between all include blocks and remove duplicates
   \end{enumerate}

   The local and global part of the setting relate to the fact if the check
   should be performed only within one include block (local) or between multiple
   include blocks of the same fileset (global). The warn and remove part of the
   keyword sets the action e.g. warn the user about shadowing or remove
   the entry shadowing the other.

   Example for a fileset resource with fileset shadow warning enabled:

\begin{bconfig}{FileSet resource with fileset shadow warning enabled}
FileSet {
   Name = "Test Set"
   Include {
   Options {
    signature = MD5
    shadowing = localwarn
   }
   File = /
   File = /usr
  }
}
\end{bconfig}


\item [meta=tag] \hfill \\
\index[dir]{meta}
\index[dir]{Directive!meta}
   This option will add a meta tag to a fileset. These meta tags are used
   by the Native NDMP protocol to pass NDMP backup or restore environment
   variables via the Data Management Agent (DMA) in Bareos to the remote
   NDMP Data Agent. You can have zero or more metatags which are all passed
   to the remote NDMP Data Agent.

\end{description}



\subsection{FileSet Exclude Ressource}
\label{fileset-exclude}
\index[general]{Excluding Files and Directories}

FileSet Exclude-Ressources very similar to Include-Ressources, except that they only allow following directives:

\begin{description}
% file | directoy | |command | \<includefile-client | <includefile-server
\xdirective{dir}{}{File}{ 
  filename \textbar\ 
  directory \textbar\ 
  \textbar command \textbar\ 
  \textbackslash\textless includefile-client \textbar\ 
  \textless includefile-server 
  }{}{}{}{%
    Files to exclude are descripted in the same way as at the \nameref{fileset-include}.
}
\end{description}

For example:

\begin{bconfig}{FileSet using Exclude}
FileSet {
  Name = Exclusion_example
  Include {
    Options {
      Signature = SHA1
    }
    File = /
    File = /boot
    File = /home
    File = /rescue
    File = /usr
  }
  Exclude {
    File = /proc
    File = /tmp                          # Don't add trailing /
    File = .journal
    File = .autofsck
  }
}
\end{bconfig}

Another way to exclude files and directories is to use the \configdirective{Exclude} option from the Include section.


\subsection{FileSet Examples}
\index[general]{Example!FileSet}
\index[general]{FileSet!Example}

The following is an example of a valid FileSet resource definition.  Note,
the first Include pulls in the contents of the file \file{/etc/backup.list}
when Bareos is started (i.e.  the @), and that file must have each filename
to be backed up preceded by a {\bf File =} and on a separate line.

\begin{bconfig}{FileSet using import}
FileSet {
  Name = "Full Set"
  Include {
    Options {
      Compression=GZIP
      signature=SHA1
      Sparse = yes
    }
    @/etc/backup.list
  }
  Include {
     Options {
        wildfile = "*.o"
        wildfile = "*.exe"
        Exclude = yes
     }
     File = /root/myfile
     File = /usr/lib/another_file
  }
}
\end{bconfig}

In the above example, all the files contained in \file{/etc/backup.list} will
be compressed with GZIP compression, an SHA1 signature will be computed on the
file's contents (its data), and sparse file handling will apply.

The two directories \file{/root/myfile} and \file{/usr/lib/another_file} will also be saved
without any options, but all files in those directories with the extensions
\file{.o} and \file{.exe} will be excluded.

Let's say that you now want to exclude the directory \file{/tmp}. The simplest way
to do so is to add an exclude directive that lists \file{/tmp}.  The example
above would then become:

\begin{bconfig}{extended FileSet excluding /tmp}
FileSet {
  Name = "Full Set"
  Include {
    Options {
      Compression=GZIP
      signature=SHA1
      Sparse = yes
    }
    @/etc/backup.list
  }
  Include {
     Options {
        wildfile = "*.o"
        wildfile = "*.exe"
        Exclude = yes
     }
     File = /root/myfile
     File = /usr/lib/another_file
  }
  Exclude {
     File = /tmp                          # don't add trailing /
  }
}
\end{bconfig}

You can add wild-cards to the File directives listed in the Exclude
directory, but you need to take care because if you exclude a directory,
it and all files and directories below it will also be excluded.

Now lets take a slight variation on the above and suppose
you want to save all your whole filesystem except \file{/tmp}.
The problem that comes up is that Bareos will not normally
cross from one filesystem to another.
Doing a \command{df} command, you get the following output:

\begin{commands}{df}
<command>df</command>
Filesystem      1k-blocks      Used Available Use% Mounted on
/dev/hda5         5044156    439232   4348692  10% /
/dev/hda1           62193      4935     54047   9% /boot
/dev/hda9        20161172   5524660  13612372  29% /home
/dev/hda2           62217      6843     52161  12% /rescue
/dev/hda8         5044156     42548   4745376   1% /tmp
/dev/hda6         5044156   2613132   2174792  55% /usr
none               127708         0    127708   0% /dev/shm
//minimatou/c$   14099200   9895424   4203776  71% /mnt/mmatou
lmatou:/          1554264    215884   1258056  15% /mnt/matou
lmatou:/home      2478140   1589952    760072  68% /mnt/matou/home
lmatou:/usr       1981000   1199960    678628  64% /mnt/matou/usr
lpmatou:/          995116    484112    459596  52% /mnt/pmatou
lpmatou:/home    19222656   2787880  15458228  16% /mnt/pmatou/home
lpmatou:/usr      2478140   2038764    311260  87% /mnt/pmatou/usr
deuter:/          4806936     97684   4465064   3% /mnt/deuter
deuter:/home      4806904    280100   4282620   7% /mnt/deuter/home
deuter:/files    44133352  27652876  14238608  67% /mnt/deuter/files
\end{commands}
\hide{$}

And we see that there are a number of separate filesystems (/ /boot
/home /rescue /tmp and /usr not to mention mounted systems).
If you specify only {\bf /} in your Include list, Bareos will only save the
Filesystem {\bf /dev/hda5}. To save all filesystems except {\bf /tmp} with
out including any of the Samba or NFS mounted systems, and explicitly
excluding a /tmp, /proc, .journal, and .autofsck, which you will not want to
be saved and restored, you can use the following:

\begin{bconfig}{FileSet mount points}
FileSet {
  Name = Include_example
  Include {
    Options {
       wilddir = /proc
       wilddir = /tmp
       wildfile = "/.journal"
       wildfile = "/.autofsck"
       exclude = yes
    }
    File = /
    File = /boot
    File = /home
    File = /rescue
    File = /usr
  }
}
\end{bconfig}

Since \file{/tmp} is on its own filesystem and it was not explicitly named in the
Include list, it is not really needed in the exclude list. It is better to
list it in the Exclude list for clarity, and in case the disks are changed so
that it is no longer in its own partition.

Now, lets assume you only want to backup .Z and .gz files and nothing
else. This is a bit trickier because Bareos by default will select
everything to backup, so we must exclude everything but .Z and .gz files.
If we take the first example above and make the obvious modifications
to it, we might come up with a FileSet that looks like this:

\begin{bconfig}{Non-working example}
FileSet {
  Name = "Full Set"
  Include {                    !!!!!!!!!!!!
     Options {                    This
        wildfile = "*.Z"          example
        wildfile = "*.gz"         doesn't
                                  work
     }                          !!!!!!!!!!!!
     File = /myfile
  }
}
\end{bconfig}

The *.Z and *.gz files will indeed be backed up, but all other files
that are not matched by the Options directives will automatically
be backed up too (i.e. that is the default rule).

To accomplish what we want, we must explicitly exclude all other files.
We do this with the following:

\begin{bconfig}{Exclude all except specific wildcards}
FileSet {
  Name = "Full Set"
  Include {
     Options {
        wildfile = "*.Z"
        wildfile = "*.gz"
     }
     Options {
        Exclude = yes
        RegexFile = ".*"
     }
     File = /myfile
  }
}
\end{bconfig}

The "trick" here was to add a RegexFile expression that matches
all files. It does not match directory names, so all directories in
/myfile will be backed up (the directory entry) and any *.Z and *.gz
files contained in them. If you know that certain directories do
not contain any *.Z or *.gz files and you do not want the directory
entries backed up, you will need to explicitly exclude those directories.
Backing up a directory entries is not very expensive.

Bareos uses the system regex library and some of them are
different on different OSes. The above has been reported not to work
on FreeBSD. This can be tested by using the {\bf estimate job=job-name
listing} command in the console and adapting the RegexFile expression
appropriately.

Please be aware that allowing Bareos to traverse or change file systems can be
{\bf very} dangerous. For example, with the following:

\begin{bconfig}{backup all filesystem below /mnt/matou (use with care)}
FileSet {
  Name = "Bad example"
  Include {
    Options {
      onefs=no
    }
    File = /mnt/matou
  }
}
\end{bconfig}

you will be backing up an NFS mounted partition ({\bf /mnt/matou}), and since
{\bf onefs} is set to {\bf no}, Bareos will traverse file systems. Now if {\bf
/mnt/matou} has the current machine's file systems mounted, as is often the
case, you will get yourself into a recursive loop and the backup will never
end.

As a final example, let's say that you have only one or two
subdirectories of /home that you want to backup.  For example,
you want to backup only subdirectories beginning with the letter
a and the letter b -- i.e. \file{/home/a*} and \file{/home/b*}. 
Now, you might first try:
\begin{bconfig}{Non-working example}
FileSet {
  Name = "Full Set"
  Include {
     Options {
        wilddir = "/home/a*"
        wilddir = "/home/b*"
     }
     File = /home
  }
}
\end{bconfig}

The problem is that the above will include everything in /home.  To get
things to work correctly, you need to start with the idea of exclusion
instead of inclusion.  So, you could simply exclude all directories
except the two you want to use:
\begin{bconfig}{Exclude by regex}
FileSet {
  Name = "Full Set"
  Include {
     Options {
        RegexDir = "^/home/[c-z]"
        exclude = yes
     }
     File = /home
  }
}
\end{bconfig}

And assuming that all subdirectories start with a lowercase letter, this
would work.

An alternative would be to include the two subdirectories desired and
exclude everything else:
\begin{bconfig}{Include and Exclude}
FileSet {
  Name = "Full Set"
  Include {
     Options {
        wilddir = "/home/a*"
        wilddir = "/home/b*"
     }
     Options {
        RegexDir = ".*"
        exclude = yes
     }
     File = /home
  }
}
\end{bconfig}


The following example shows how to back up only the My Pictures directory inside
the My Documents directory for all users in C:/Documents and Settings, i.e.
everything matching the pattern:

\file{C:/Documents and Settings/*/My Documents/My Pictures/*}

To understand how this can be achieved, there are two important points to
remember:

Firstly, Bareos walks over the filesystem depth-first starting from the File =
lines.  It stops descending when a directory is excluded, so you must include
all ancestor directories of each directory containing files to be included.

Secondly, each directory and file is compared to the Options clauses in the
order they appear in the FileSet.  When a match is found, no further clauses
are compared and the directory or file is either included or excluded.

The FileSet resource definition below implements this by including specifc
directories and files and excluding everything else.

\begin{bconfig}{Include/Exclude example}
FileSet {
  Name = "AllPictures"

  Include {

    File  = "C:/Documents and Settings"

    Options {
      signature = SHA1
      verify = s1
      IgnoreCase = yes

      # Include all users' directories so we reach the inner ones.  Unlike a
      # WildDir pattern ending in *, this RegExDir only matches the top-level
      # directories and not any inner ones.
      RegExDir = "^C:/Documents and Settings/[^/]+$"

      # Ditto all users' My Documents directories.
      WildDir = "C:/Documents and Settings/*/My Documents"

      # Ditto all users' My Documents/My Pictures directories.
      WildDir = "C:/Documents and Settings/*/My Documents/My Pictures"

      # Include the contents of the My Documents/My Pictures directories and
      # any subdirectories.
      Wild = "C:/Documents and Settings/*/My Documents/My Pictures/*"
    }

    Options {
      Exclude = yes
      IgnoreCase = yes

      # Exclude everything else, in particular any files at the top level and
      # any other directories or files in the users' directories.
      Wild = "C:/Documents and Settings/*"
    }
  }
}
\end{bconfig}
\hide{$}



\subsection{Windows FileSets}
\index[general]{Windows!FileSet}
\index[general]{FileSet!Windows}
\label{win32}
If you are entering Windows file names, the directory path may be preceded by
the drive and a colon (as in c:). However, the path separators must be
specified in Unix convention (i.e. forward slash (/)). If you wish to include
a quote in a file name, precede the quote with a backslash
(\textbackslash{}). For example you might use the following
for a Windows machine to backup the "My Documents" directory:

\begin{bconfig}{Windows FileSet}
FileSet {
  Name = "Windows Set"
  Include {
    Options {
       WildFile = "*.obj"
       WildFile = "*.exe"
       exclude = yes
     }
     File = "c:/My Documents"
  }
}
\end{bconfig}

For exclude lists to work correctly on Windows, you must observe the following
rules:

\begin{itemize}
\item Filenames are case sensitive, so you must use the correct case.
\item To exclude a directory, you must not have a trailing slash on the
   directory name.
\item If you have spaces in your filename, you must enclose the entire name
   in double-quote characters ("). Trying to use a backslash before  the space
   will not work.
\item If you are using the old Exclude syntax (noted below), you may not
   specify a drive letter in the exclude.  The new syntax noted above
   should work fine including driver letters.
\end{itemize}

Thanks to Thiago Lima for summarizing the above items for us. If you are
having difficulties getting includes or excludes to work, you might want to
try using the {\bf estimate job=xxx listing} command documented in the
\ilink{Console chapter}{estimate} of this manual.

On Win32 systems, if you move a directory or file or rename a file into the
set of files being backed up, and a Full backup has already been made, Bareos
will not know there are new files to be saved during an Incremental or
Differential backup (blame Microsoft, not us). To avoid this problem, please
{\bf copy} any new directory or files into the backup area. If you do not have
enough disk to copy the directory or files, move them, but then initiate a
Full backup.


\paragraph*{Example Fileset for Windows}
\index[general]{FileSet!Windows Example}
\index[general]{Windows!FileSet!Example}

The following example demostrates a Windows FileSet.
It backups all data from all fixed drives and only excludes some Windows temporary data.

\begin{bconfig}{Windows All Drives FileSet}
FileSet {
  Name = "Windows All Drives"
  Enable VSS = yes
  Include {
    Options {
      Signature = MD5
      Drive Type = fixed
      IgnoreCase = yes
      WildFile = "[A-Z]:/pagefile.sys"
      WildDir = "[A-Z]:/RECYCLER"
      WildDir = "[A-Z]:/$RECYCLE.BIN"
      WildDir = "[A-Z]:/System Volume Information"
      Exclude = yes
    }
    File = /
  }
}
\end{bconfig}

% workaround for kile editor
\hide{$}

\variable{File = /} includes all Windows drives.
Using \variable{Drive Type = fixed} excludes drives like USB-Stick or CD-ROM Drive.
Using \variable{WildDir = "[A-Z]:/RECYCLER"} excludes the backup of the directory \path|RECYCLER| from all drives.


\subsection{Testing Your FileSet}
\index[general]{FileSet!Testing Your}
\index[general]{Testing Your FileSet}

If you wish to get an idea of what your FileSet will really backup or if your
exclusion rules will work correctly, you can test it by using the
{\bf estimate} command in the Console program. See the
\ilink{estimate}{estimate} in the Console chapter of this
manual.

As an example, suppose you add the following test FileSet:

\begin{bconfig}{FileSet for all *.c files}
FileSet {
  Name = Test
  Include {
    File = /home/xxx/test
    Options {
       regex = ".*\\.c$"
    }
  }
}
\end{bconfig}
\hide{$}

You could then add some test files to the directory {\bf /home/xxx/test}
and use the following command in the console:

\begin{bconsole}{estimate}
estimate job=<any-job-name> listing client=<desired-client> fileset=Test
\end{bconsole}

to give you a listing of all files that match.  In the above
example, it should be only files with names ending in  {\bf .c}.
