%%
%%


When each of the Bareos programs starts, it reads a configuration file
specified on the command line or the default \file{bareos-dir.conf}, 
\file{bareos-fd.conf}, \file{bareos-sd.conf}, or \file{console.conf} for the Director
daemon, the File daemon, the Storage daemon, and the Console program
respectively.

Each service (Director, Client, Storage, Console) has its own configuration
file containing a set of Resource definitions. These resources are very
similar from one service to another, but may contain different directives
(records) depending on the service. For example, in the Director's resource
file, the \nameref{DirectorResourceDirector} defines the name of the Director, a number
of global Director parameters and his password. In the File daemon
configuration file, the \nameref{ClientResourceDirector} specifies which Directors are
permitted to use the File daemon.

If you install a full Bareos system (Director, Storage and File Daemon) to one system,
the Bareos packages tries there best to generate a working configuration as a basis for your individual configuration.

The details of each resource and the directives permitted therein are
described in the following chapters.

The following configuration files must be defined:

\begin{itemize}
\item
   \nameref{DirectorChapter} -- to define the resources
   necessary for the Director. You define all the Clients  and Storage daemons
   that you use in this configuration file.
\item
   \nameref{StoredConfChapter} -- to define the resources to
   be used by each Storage daemon. Normally, you will have  a single Storage
   daemon that controls your tape drive or tape  drives. However, if you have
   tape drives on several machines,  you will have at least one Storage daemon
   per machine.
\item
   \nameref{FiledConfChapter} -- to define the resources for
   each client to be backed up. That is, you will have a separate  Client
   resource file on each machine that runs a File daemon.
\item
   \nameref{ConsoleConfChapter} -- to define the resources for
   the Console program (user interface to the Director).  It defines which
Directors are  available so that you may interact with them.
\end{itemize}



\section{Character Sets}
\index[general]{Character Sets}
Bareos is designed to handle most character sets of the world,
US ASCII, German, French, Chinese, ...  However, it does this by
encoding everything in UTF-8, and it expects all configuration files
(including those read on Win32 machines) to be in UTF-8 format.
UTF-8 is typically the default on Linux machines, but not on all
Unix machines, nor on Windows, so you must take some care to ensure
that your locale is set properly before starting Bareos.

\index[general]{Windows!Configuration Files!UTF-8}
To ensure that Bareos configuration files can be correctly read including
foreign characters the {\bf LANG} environment variable
must end in {\bf .UTF-8}. A full example is {\bf en\_US.UTF-8}. The
exact syntax may vary a bit from OS to OS, and exactly how you define
it will also vary.  On most newer Win32 machines, you can use \command{notepad}
to edit the conf files, then choose output encoding UTF-8.

Bareos assumes that all filenames are in UTF-8 format on Linux and
Unix machines. On Win32 they are in Unicode (UTF-16), and will
be automatically converted to UTF-8 format.

\section{Resource Directive Format}
\index[general]{Resource Directive Format}
\index[general]{Format!Resource Directive}

Although, you won't need to know the details of all the directives a basic
knowledge of Bareos resource directives is essential. Each directive contained
within the resource (within the braces) is composed of a keyword followed by
an equal sign (=) followed by one or more values. The keywords must be one of
the known Bareos resource record keywords, and it may be composed of upper or
lower case characters and spaces.

Each resource definition MUST contain a Name directive, and may optionally
contain a Description directive. The Name directive is used to
uniquely identify the resource. The Description directive is (will be) used
during display of the Resource to provide easier human recognition. For
example:

\begin{bconfig}{Resource example}
Director {
  Name = "MyDir"
  Description = "Main Bareos Director"
  WorkingDirectory = "$HOME/bareos/bin/working"
}
\end{bconfig}
\hide{$}

Defines the Director resource with the name "MyDir" and a working directory
\directory{$HOME/bareos/bin/working}.
\hide{$}

\subsection{Quotes}
\label{sec:Quotes}
In general, if you want spaces in a name to the
right of the first equal sign (=), you must enclose that name within double
quotes. Otherwise quotes are not generally necessary because once defined,
quoted strings and unquoted strings are all equal.

However, if the configure directive defines a numbers, the number never has surrounding quotes.
This is even true if the number is specified together with its unit, like \parameter{365 days}.

Also do not prepend numbers by zeros (0), as these are not parsed in the expected manner (write 1 instead of 01).

\subsection{Comments}
\label{Comments}
\index[general]{Comments}

When reading the configuration file, blank lines are ignored and everything
after a hash sign (\#) until the end of the line is taken to be a comment. A
semicolon (;) is a logical end of line, and anything after the semicolon is
considered as the next statement. If a statement appears on a line by itself,
a semicolon is not necessary to terminate it, so generally in the examples in
this manual, you will not see many semicolons.


\subsection{Upper and Lower Case and Spaces}
\index[general]{Spaces!Upper/Lower Case}
\index[general]{Upper and Lower Case and Spaces}

Case (upper/lower) and spaces are totally ignored in the resource directive
keywords (the part before the equal sign).

Within the keyword (i.e. before the equal sign), spaces are not significant.
Thus the keywords: {\bf name}, {\bf Name}, and {\bf N a m e} are all
identical.

Spaces after the equal sign and before the first character of the value are
ignored.

In general, spaces within a value are significant (not ignored), and if the
value is a name, you must enclose the name in double quotes for the spaces to
be accepted. Names may contain up to 127 characters. Currently, a name may
contain any ASCII character. Within a quoted string, any character following a
backslash (\textbackslash{}) is taken as itself (handy for inserting
backslashes and double quotes (")).

Please note, however, that Bareos resource names as well as certain other
names (e.g. Volume names) must contain only letters (including ISO accented
letters), numbers, and a few special characters (space, underscore, ...).
All other characters and punctuation are invalid.

\subsection{Including other Configuration Files}
\label{Includes}
\index[general]{Including other Configuration Files}
\index[general]{Files!Including other Configuration}
\index[general]{Configuration!Including Files}

If you wish to break your configuration file into smaller pieces, you can do
so by including other files using the syntax \configdirective{@filename}
where \file{filename} is the full path and filename of another file.
The \configdirective{@filename}
specification can be given anywhere a primitive token would appear.

\begin{bconfig}{include configuration files}
@/etc/bareos/extra/clients.conf
\end{bconfig}

Since Bareos \sinceVersion{}{Including Configuration files by wildcard}{16.2.1} wildcards in pathes are supported:
\begin{bconfig}{include multiple configuration files}
@/etc/bareos/extra/*.conf
\end{bconfig}

% Before 
% this could be archived by
% If you wish include all files in a specific directory, you can use the following:
% \begin{bconfig}{include configuration files}
% # Include subfiles associated with configuration of clients.
% # They define the bulk of the Clients, Jobs, and FileSets.
% # Remember to "reload" the Director after adding a client file.
% #
% @|"sh -c 'for f in /etc/bareos/clientdefs/*.conf ; do echo @${f} ; done'"
% \end{bconfig}
% \hide{$}

By using \configdirective {@!command} it is also possible to include the output of a script as configuration:
\begin{bconfig}{use the output of a script as configuration}
@|"/etc/bareos/generate_configuration_to_stdout.sh"
\end{bconfig}
or if parameter should be used:
\begin{bconfig}{use the output of a script with parameter as configuration}
@|"sh -c '/etc/bareos/generate_client_configuration_to_stdout.sh clientname=client1.example.com'"
\end{bconfig}
The scripts are called at the start of the daemon. You should use them with care.


\subsection{Data Types}
\index[general]{Configuration!Data Types}
\index[general]{Data Type}
\label{DataTypes}

When parsing the resource directives, Bareos classifies the data according to
the types listed below. The first time you read this, it may appear a bit
overwhelming, but in reality, it is all pretty logical and straightforward.

\begin{description}

\item [acl]
    \index[general]{Data Type!acl}
    \label{DataTypeAcl}
This directive defines what is permitted to access.
It does this by a list of strings, separated by \parameter{,}.

The special keyword \parameter{*all*} allows unrestricted access.

Depending on the type of the ACL, the strings can by either resource names, paths or bconsole commands.


\item [auth-type]
    \index[general]{Data Type!auth-type}
    \label{DataTypeAuthType}
Specifies the authentication type that must be supplied when connecting to
a backup protocol that uses a specific authentication type.
Currently only used for \nameref{NDMPResource}.

The following values are allowed:
\begin{enumerate}
\item None - Use no password
\item Clear - Use clear text password
\item MD5 - Use MD5 hashing
\end{enumerate}


\item [path]
    \index[general]{Data Type!path}
    \label{DataTypeDirectory}
   A path is either a quoted or  non-quoted string. A path will be
passed to your  standard shell for expansion when it is scanned. Thus
constructs such as {\bf \$HOME} are interpreted to be  their correct values.
If can either reference to a file or a directory.


\item [integer]
    \index[general]{Data Type!integer}
    \label{DataTypeInteger}
   A 32 bit integer value. It may be positive or negative.

   Don't use quotes around the number, see \nameref{sec:Quotes}.


\item [long integer]
    \index[general]{Data Type!long integer}
    \label{DataTypeLongInteger}
   A 64 bit integer value. Typically these  are values such as bytes that can
exceed 4 billion and thus  require a 64 bit value.

   Don't use quotes around the number, see \nameref{sec:Quotes}.


\item [name]
    \index[general]{Data Type!name}
    \label{DataTypeName}
   A keyword or name consisting of alphanumeric characters, including the
hyphen, underscore, and dollar  characters. The first character of a {\bf
name} must be  a letter.  A name has a maximum length currently set to 127
bytes.  Typically keywords appear on the left side of an equal (i.e.  they are
Bareos keywords -- i.e. Resource names or  directive names). Keywords may not
be quoted.

\item [name-string]
    \index[general]{Data Type!name-string}
    \label{DataTypeNameString}
   A name-string is similar to a name,  except that the name may be quoted and
can thus contain  additional characters including spaces. Name strings  are
limited to 127 characters in length. Name strings  are typically used on the
right side of an equal (i.e.  they are values to be associated with a keyword).


\item [password]
    \index[general]{Data Type!password}
    \label{DataTypePassword}
   This is a Bareos password and it is stored internally in MD5 hashed format.


\item [path]
    \index[general]{Data Type!path}
    \label{DataTypePath}
    File name, including path.


\item [positive integer]
    \index[general]{Data Type!positive integer}
    \label{DataTypePositiveInteger}
   A 32 bit positive integer value.

   Don't use quotes around the number, see \nameref{sec:Quotes}.


\item [speed]
    \index[general]{Data Type!speed}
    \label{DataTypeSpeed}
    The speed parameter can be specified as k/s, kb/s, m/s or mb/s.

    Don't use quotes around the parameter, see \nameref{sec:Quotes}.


\item [string]
    \index[general]{Data Type!string}
    \label{DataTypeString}
   A quoted string containing virtually any  character including spaces, or a
non-quoted string. A  string may be of any length. Strings are typically
values  that correspond to filenames, directories, or system  command names. A
backslash (\textbackslash{}) turns the next character into  itself, so to
include a double quote in a string, you precede the  double quote with a
backslash. Likewise to include a backslash.


\item [string-list]
    \index[general]{Data Type!string list}
    \label{DataTypeStringList}
    Multiple strings, specified in multiple directives, or in a single directive, separated by \textbf{,}.




\item [net-address]
    \index[general]{Data Type!net-address}
    \label{DataTypeNetAddress}
Netaddress is either a domain name or an IP address specified as a
dotted quadruple in string or quoted string format.
This directive only permits a single address to be specified.
The \dtNetPort must be specific separate.
If multiple net-addresses are needed, check if it also possible to specify \dtNetAddresses (plural). 


\item [net-addresses]
    \index[general]{Data Type!net-addresses}
    \label{DataTypeNetAddresses}
Specify a set of net-addresses and net-ports.
Probably the simplest way to explain
this is to show an example:

\begin{bconfig}{net-addresses}
Addresses  = {
    ip = { addr = 1.2.3.4; port = 1205;}
    ipv4 = {
        addr = 1.2.3.4; port = http;}
    ipv6 = {
        addr = 1.2.3.4;
        port = 1205;
    }
    ip = {
        addr = 1.2.3.4
        port = 1205
    }
    ip = { addr = 1.2.3.4 }
    ip = { addr = 201:220:222::2 }
    ip = {
        addr = server.example.com
    }
}
\end{bconfig}

where ip, ip4, ip6, addr, and port are all keywords. Note, that  the address
can be specified as either a dotted quadruple, or  IPv6 colon notation, or as
a symbolic name (only in the ip specification).  Also, port can be specified
as a number or as the mnemonic value from  the \file{/etc/services} file.  If a port
is not specified, the default will be used. If an ip  section is specified,
the resolution can be made either by IPv4 or  IPv6. If ip4 is specified, then
only IPv4 resolutions will be permitted,  and likewise with ip6.


\item [net-port]
    \index[general]{Data Type!net-port}
    \label{DataTypeNetPort}
    Specify a network port (a positive  integer).

    Don't use quotes around the parameter, see \nameref{sec:Quotes}.


\item [resource]
    \index[general]{Data Type!resource}
    \label{DataTypeRes}
A resource defines a relation to the name of another resource.


\item [size]
    \index[general]{Data Type!size}
    \label{DataTypeSize}
    \label{Size1}
A size specified as bytes. Typically, this is  a floating point scientific
input format followed by an optional modifier. The  floating point input is
stored as a 64 bit integer value.  If a modifier is present, it must
immediately follow the  value with no intervening spaces. The following
modifiers are permitted:

\begin{description}
\item [k]
   1,024 (kilobytes)

\item [kb]
   1,000 (kilobytes)

\item [m]
   1,048,576 (megabytes)

\item [mb]
   1,000,000 (megabytes)

\item [g]
   1,073,741,824 (gigabytes)

\item [gb]
   1,000,000,000 (gigabytes)
\end{description}

    Don't use quotes around the parameter, see \nameref{sec:Quotes}.


\item [time]
    \index[general]{Data Type!time}
    \label{DataTypeTime}
    \label{Time}
A time or duration specified in seconds.  The time is stored internally as
a 64 bit integer value, but it is specified in two parts: a number part and
a modifier part.  The number can be an integer or a floating point number.
If it is entered in floating point notation, it will be rounded to the
nearest integer.  The modifier is mandatory and follows the number part,
either with or without intervening spaces.  The following modifiers are
permitted:

\begin{description}

\item [seconds]
   \index[dir]{seconds}
   seconds

\item [minutes]
   \index[dir]{minutes}
   minutes (60 seconds)

\item [hours]
   \index[dir]{hours}
   hours (3600 seconds)

\item [days]
   \index[dir]{days}
   days (3600*24 seconds)

\item [weeks]
   \index[dir]{weeks}
   weeks (3600*24*7 seconds)

\item [months]
   \index[dir]{months}
   months (3600*24*30 seconds)

\item [quarters]
   \index[dir]{quarters}
   quarters (3600*24*91 seconds)

\item [years]
   \index[dir]{years}
   years (3600*24*365 seconds)
\end{description}

Any abbreviation of these modifiers is also permitted (i.e.  {\bf seconds}
may be specified as {\bf sec} or {\bf s}).  A specification of {\bf m} will
be taken as months.

The specification of a time may have as many number/modifier parts as you
wish.  For example:

\footnotesize
\begin{verbatim}
1 week 2 days 3 hours 10 mins
1 month 2 days 30 sec
\end{verbatim}
\normalsize

are valid date specifications.

    Don't use quotes around the parameter, see \nameref{sec:Quotes}.


\item [audit-command-list]
    \index[general]{Data Type!audit command list}
    \label{DataTypeAuditCommandList}
    Specifies the commands that should be logged on execution (audited).
    E.g.
\begin{bconfig}{}
Audit Events = label
Audit Events = restore
\end{bconfig}
    Based on the type \dtStringList.



\item [\yesno]
    \index[general]{Data Type!\yesno}
    \index[general]{Data Type!boolean}
    \label{DataTypeYesNo}
   Either a \parameter{yes} or a \parameter{no} (or \parameter{true} or \parameter{false}).


\end{description}




\subsection{Variable Expansion}
    \label{VarsChapter}

Depending on the type of directive, Bareos will expand following variables:

\paragraph{Variable Expansion on Media Labels}

When labeling a new media, following Bareos internal variables can be used:

\begin{tabular}{p{2cm}p{7cm}}
\textbf{Internal Variable} & \textbf{Description} \\
\variable{Year} & Year \\
\variable{Month} & Month: 1-12 \\
\variable{Day} & Day: 1-31 \\
\variable{Hour} & Hour: 0-24 \\
\variable{Minute} & Minute: 0-59 \\
\variable{Second} & Second: 0-59 \\
\variable{WeekDay} & Day of the week: 0-6, using 0 for Sunday\\
\variable{Job} & Name of the Job \\
\variable{Dir} & Name of the Director \\
\variable{Level} & Job Level \\
\variable{Type} & Job Type \\
\variable{JobId} & JobId \\
\variable{JobName} & unique name of a job\\
\variable{Storage} & Name of the Storage Daemon\\
\variable{Client} &  Name of the Clients \\
\variable{NumVols} & Numbers of volumes in the pool\\
\variable{Pool} &  Name of the Pool  \\
\variable{Catalog} &  Name of the Catalog\\
\variable{MediaType} &  Type of the media
\end{tabular}

Additional, normal environment variables can be used, e.g.
\variable{HOME} oder \variable{HOSTNAME}.

\paragraph{Variable Expansion on RunScripts}

At the configuration of RunScripts following variables can be used:

\begin{tabular}{p{2cm}p{7cm}}
\textbf{Variable} & \textbf{Description} \\
\parameter{\%c} & Client's Name\\
\parameter{\%d} & Director's Name\\
\parameter{\%e} & Job Exit Code\\
\parameter{\%i} & JobId\\
\parameter{\%j} & Unique JobId\\
\parameter{\%l} & Job Level\\
\parameter{\%n} & Unadorned Job Name\\
\parameter{\%r} & Recipients\\
\parameter{\%s} & Since Time\\
\parameter{\%b} & Job Bytes \\
\parameter{\%B} & Job Bytes in human readable format \\
\parameter{\%f} & Job Files \\
\parameter{\%t} & Job Type (Backup, ...)\\
\parameter{\%v} & Read Volume Name (only on Director)\\
\parameter{\%V} & Write Volume Name (only on Director)
\end{tabular}



\paragraph{Variable Expansion in Autochanger Commands}

At the configuration of autochanger commands following variables can be used:


\begin{tabular}{p{2cm}p{7cm}}
\textbf{Variable} & \textbf{Description} \\
\parameter{\%a} & Archive Device Name\\
\parameter{\%c} & Changer Device Name\\
\parameter{\%d} & Changer Drive Index\\
\parameter{\%f} & Client's Name\\
\parameter{\%j} & Job Name\\
\parameter{\%o} & Command\\
\parameter{\%s} & Slot Base 0\\
\parameter{\%S} & Slot Base 1\\
\parameter{\%v} & Volume Name
\end{tabular}



\paragraph{Variable Expansion in Mount Commands}

At the configuration of mount commands following variables can be used:



\begin{tabular}{p{2cm}p{7cm}}
\textbf{Variable} & \textbf{Description} \\
\parameter{\%a} & Archive Device Name\\
\parameter{\%e} & Erase\\
\parameter{\%n} & Part Number\\
\parameter{\%m} & Mount Point\\
\parameter{\%v} & Last Part Name
\end{tabular}







\paragraph{Variable Expansion in Mail and Operator Commands}

At the configuration of mail and operator commands following variables can be used:

\begin{tabular}{p{2cm}p{7cm}}
\textbf{Variable} & \textbf{Description} \\
\parameter{\%c} & Client's Name\\
\parameter{\%d} & Director's Name\\
\parameter{\%e} & Job Exit Code\\
\parameter{\%i} & JobId\\
\parameter{\%j} & Unique Job Id\\
\parameter{\%l} & Job Level\\
\parameter{\%n} & Unadorned Job Name\\
\parameter{\%s} & Since Time\\
\parameter{\%t} & Job Type (Backup, ...)\\
\parameter{\%r} & Recipients\\
\parameter{\%v} & Read Volume Name\\
\parameter{\%V} & Write Volume Name\\
\parameter{\%b} & Job Bytes\\
\parameter{\%B} & Job Bytes in human readable format \\
\parameter{\%F} & Job Files
\end{tabular}



\section{Resource Types}
\index[general]{Types!Resource}
\index[general]{Resource Types}
\label{ResTypes}

The following table lists all current Bareos resource types. It shows what
resources must be defined for each service (daemon). The default configuration
files will already contain at least one example of each permitted resource.

\addcontentsline{lot}{table}{Resource Types}
\begin{longtable}{|l||c|c|c|c|}
 \hline
\multicolumn{1}{|c|| }{\bf Resource } &
\multicolumn{1}{c| }{ \ilink{Director}{DirectorConfChapter} } &
\multicolumn{1}{c| }{ \ilink{Client}{FiledConfChapter} } &
\multicolumn{1}{c| }{ \ilink{Storage}{StoredConfChapter} } &
\multicolumn{1}{c| }{ \ilink{Console}{ConsoleConfChapter}  } \\
 \hline
 \hline
{Autochanger} &                                 &                                 & \cmlink{StorageResourceAutochanger} &  \\
\hline
{Catalog }  & \cmlink{DirectorResourceCatalog}  &                                 &    &    \\
 \hline
{Client  }  & \cmlink{DirectorResourceClient}   & \cmlink{ClientResourceClient}   &    &    \\
 \hline
{Console }  & \cmlink{DirectorResourceConsole}  &                                 &                                  & \cmlink{ConsoleResourceConsole} \\
 \hline
{Device  }  &                                   &                                 & \cmlink{StorageResourceDevice}   &    \\
 \hline
{Director } & \cmlink{DirectorResourceDirector} & \cmlink{ClientResourceDirector} & \cmlink{StorageResourceDirector} & \cmlink{ConsoleResourceDirector} \\
 \hline
{FileSet }  & \cmlink{DirectorResourceFileSet}  &                                 &    &    \\
 \hline
{Job}       & \cmlink{DirectorResourceJob}      &                                 &    &    \\
 \hline
{JobDefs }  & \cmlink{DirectorResourceJobDefs}  &                                 &    &    \\
 \hline
{Message }  & \cmlink{ResourceMessages}         & \cmlink{ResourceMessages}       & \cmlink{ResourceMessages} &    \\
 \hline
{NDMP }     &                                   &                                 & \cmlink{StorageResourceNDMP} &    \\
 \hline
{Pool  }    & \cmlink{DirectorResourcePool}     &                                 &    &    \\
 \hline
{Profile}   & \cmlink{DirectorResourceProfile}  &                                 &    &    \\
 \hline
{Schedule } & \cmlink{DirectorResourceSchedule} &                                 &    &    \\
 \hline
{Storage }  & \cmlink{DirectorResourceStorage}  &                                 & \cmlink{StorageResourceStorage} & \\
\hline
\end{longtable}

\section{Names, Passwords and Authorization}
\label{Names}
\index[general]{Authorization!Names and Passwords}
\index[general]{Passwords}

In order for one daemon to contact another daemon, it must authorize itself
with a password. In most cases, the password corresponds to a particular name,
so both the name and the password must match to be authorized. Passwords are
plain text, any text.  They are not generated by any special process; just
use random text.

The default configuration files are automatically defined for correct
authorization with random passwords. If you add to or modify these files, you
will need to take care to keep them consistent.

Here is sort of a picture of what names/passwords in which files/Resources
must match up:

\begin{center}
\includegraphics[width=0.8\textwidth]{\idir Conf-Diagram}
\end{center}

In the left column, you will find the Director, Storage, and Client resources,
with their names and passwords -- these are all in \file{bareos-dir.conf}. In
the right column are where the corresponding values should be found in the
Console, Storage daemon (SD), and File daemon (FD) configuration files.

Please note that the address, \host{fw-sd}, that appears in the Storage
resource of the Director,
is passed to the File daemon in symbolic form. The File daemon then resolves it
to an IP address. For this reason, you must use either an IP address or a
resolvable fully qualified name. A name such as \host{localhost}, not being a fully
qualified name, will resolve in the File daemon to the \host{localhost} of the File
daemon, which is most likely not what is desired. The password used for the
File daemon to authorize with the Storage daemon is a temporary password
unique to each Job created by the daemons and is not specified in any .conf
file.
