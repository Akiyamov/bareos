\defDirective{Dir}{Director}{Backend Directory}{}{}{%
This directive specifies a directory from where the Bareos Director loads his dynamic backends.
}

\defDirective{Dir}{Director}{Name}{}{}{%
The director name used by the system  administrator.
}

\defDirective{Dir}{Director}{Absolute Job Timeout}{}{14.2.0}{%
}

\defDirective{Dir}{Director}{Auditing}{}{14.2.0}{%
This directive allows to en- or disable auditing of interaction with the Bareos Director.
If enabled, \ilink{audit messages}{MessageTypes} will be generated.
The \ilink{messages resource}{MessagesResource} configured in \linkResourceDirective{Dir}{Director}{Messages} 
defines, how these messages are handled.
}

\defDirective{Dir}{Director}{Audit Events}{}{14.2.0}{%
Specify which commands (see \nameref{sec:ConsoleCommands}) will be audited. If nothing is specified (and \linkResourceDirective{Dir}{Director}{Auditing} is enabled), all commands will be audited.
}

\defDirective{Dir}{Director}{Description}{}{}{%
The text field contains a  description of the Director that will be displayed
in the  graphical user interface. This directive is optional.
}

\defDirective{Dir}{Director}{Dir Address}{}{}{%
This directive is optional, but if it is specified, it will cause the
Director server (for the Console program) to bind to the specified address.
If this and the \linkResourceDirective{Dir}{Director}{Dir Addresses} directives are
not specified, the Director will bind to any available address (the
default).
}


\defDirective{Dir}{Director}{Dir Addresses}{}{}{%
Specify the ports and addresses on which the Director daemon will listen
for Bareos Console connections.

Please note that if you use the \linkResourceDirective{Dir}{Director}{Dir Addresses} directive, you must
not use either a \linkResourceDirective{Dir}{Director}{Dir Port} or a \linkResourceDirective{Dir}{Director}{Dir Address} directive in the same
resource.
}


\defDirective{Dir}{Director}{Dir Port}{}{}{%
Specify the port on which the  Director daemon will
listen for Bareos Console connections.  This same port number must be
specified in the Director resource  of the Console configuration file. The
default is 9101, so  normally this directive need not be specified.  This
directive should not be used if you specify \linkResourceDirective{Dir}{Director}{Dir Addresses} (N.B plural)
directive.
}

\defDirective{Dir}{Director}{Dir Source Address}{}{}{%
This record is optional, and if it is specified, it will cause the Director
server (when initiating connections to a storage or file daemon) to source
its connections from the specified address.  Only a single IP address may be
specified.  If this record is not specified, the Director server will source
its outgoing connections according to the system routing table (the default).
}


\defDirective{Dir}{Director}{FD Connect Timeout}{}{}{%
where \parameter{time} is the time that the Director should continue
attempting to contact the File daemon to start a job, and after which
the Director will cancel the job.  The default is 3 minutes.
}

\defDirective{Dir}{Director}{Heartbeat Interval}{}{}{%
This directive is optional and if specified will cause the Director to
set a keepalive interval (heartbeat) in seconds on each of the sockets
it opens for the Client resource.  This value will override any
specified at the Director level.  It is implemented only on systems
that provide the {\bf setsockopt} TCP\_KEEPIDLE function (Linux, ...).
The default value is zero, which means no change is made to the socket.
}

\defDirective{Dir}{Director}{Key Encryption Key}{}{}{%
This key is used to encrypt the Security Key that is exchanged between
the Director and the Storage Daemon for supporting Application Managed
Encryption (AME). For security reasons each Director should have a
different Key Encryption Key.
}

\defDirective{Dir}{Director}{Maximum Concurrent Jobs}{}{}{%
\label{DirMaxConJobs}%
\index[general]{Simultaneous Jobs}%
\index[general]{Concurrent Jobs}%
where {\textless}number{\textgreater}  is the maximum number of total Director Jobs that
should run  concurrently. The default is set to 1, but you may set it to a
larger number.

The Volume format becomes more complicated with
multiple simultaneous jobs, consequently, restores may take longer if
Bareos must sort through interleaved volume blocks from  multiple simultaneous
jobs. This can be avoided by having each simultaneous job write to
a different volume or  by using data spooling, which will first spool the data
to disk simultaneously, then write one spool file at a time to the volume
thus avoiding excessive interleaving of the different job blocks.

See also the section about \ilink{Concurrent Jobs}{ConcurrentJobs}.
}

\defDirective{Dir}{Director}{Maximum Console Connections}{}{}{%
where \parameter{number}  is the maximum number of Console Connections that
could run  concurrently. The default is set to 20, but you may set it to a
larger number.
}

\defDirective{Dir}{Director}{Messages}{}{}{%
The messages resource  specifies where to deliver Director messages that are
not associated  with a specific Job. Most messages are specific to a job and
will  be directed to the Messages resource specified by the job. However,
there are a messages that can occur when no job is running.
}


\defDirective{Dir}{Director}{NDMP Snooping}{}{13.2.0}{%
This directive enables the Snooping and pretty printing of NDMP protocol
information in debugging mode.
}

\defDirective{Dir}{Director}{NDMP Log Level}{}{13.2.0}{%
This directive sets the loglevel for the NDMP protocol library.
}


\defDirective{Dir}{Director}{Omit Defaults}{}{}{%
When showing the configuration, omit those parameter that have there default value assigned.
}

\defDirective{Dir}{Director}{Optimize For Size}{}{}{%
If set to \parameter{yes} this directive will use the optimizations
for memory size over speed. So it will try to use less memory which may lead
to a somewhat lower speed. Its currently mostly used for keeping all hardlinks
in memory.

If none of \linkResourceDirective{Dir}{Director}{Optimize For Size} and \linkResourceDirective{Dir}{Director}{Optimize For Speed} is enabled, \linkResourceDirective{Dir}{Director}{Optimize For Size} is enabled by default.
}

\defDirective{Dir}{Director}{Optimize For Speed}{}{}{%
If set to \parameter{yes} this directive will use the
optimizations for speed over the memory size. So it will try to use more memory
which lead to a somewhat higher speed. Its currently mostly used for keeping all
hardlinks in memory. Its relates to the \linkResourceDirective{Dir}{Director}{Optimize For Size}
option set either one to \parameter{yes} as they are mutually exclusive.
}


\defDirective{Dir}{Director}{Password}{}{}{%
Specifies the password that must be supplied for the default Bareos
Console to be authorized.
This password correspond to \linkResourceDirective{Console}{Director}{Password}
of the Console configuration file.

The password is plain text.
}


\defDirective{Dir}{Director}{Pid Directory}{}{}{%
This directive  is optional and specifies a directory in which the Director
may put its process Id file. The process Id file is used to  shutdown
Bareos and to prevent multiple copies of  Bareos from running simultaneously.
Standard shell expansion of the {\bf Directory}  is done when the
configuration file is read so that values such  as {\bf \$HOME} will be
properly expanded.

The PID directory specified must already exist and be
readable and writable by the Bareos daemon referencing it.

Typically on Linux systems, you will set this to:  \directory{/var/run}. If you are
not installing Bareos in the  system directories, you can use the {\bf Working
Directory} as  defined above.
}


\defDirective{Dir}{Director}{Plugin Directory}{}{}{%
%If a directory is specified, the Bareos Director tries to load 
%all \ilink{Director plugins}{dirPlugins} that are located in this directory.
%If \configdirective{Plugin Names} is also specified,
%only the specified plugins get loaded.
}



\defDirective{Dir}{Director}{Plugin Names}{}{}{%
If a \linkResourceDirective{Dir}{Director}{Plugin Directory} is specified
\configdirective{Plugin Names} defines, which \ilink{Director plugins}{dirPlugins} get loaded.

If \configdirective{Plugin Names} is not defined, all plugins get loaded,
otherwise the defined ones.
%
% \begin{bconfig}{Plugin Names}
% Plugin Names = python:otherplugin
% \end{bconfig}
}



\defDirective{Dir}{Director}{Query File}{}{}{%
This directive is required and specifies a directory and file in which
the Director can find the canned SQL statements for the {\bf Query}
command of the Console.
%
%Standard shell expansion of the {\bf Path} is
%done when the configuration file is read so that values such as {\bf
%\$HOME} will be properly expanded.
}


\defDirective{Dir}{Director}{Scripts Directory}{}{}{%
}


\defDirective{Dir}{Director}{SD Connect Timeout}{}{}{%
where \parameter{time} is the time that the Director should continue
attempting to contact the Storage daemon to start a job, and after which
the Director will cancel the job.  The default is 30 minutes.
}


\defDirective{Dir}{Director}{Statistics Collect Interval}{}{14.2.0}{%
Bareos offers the possibilty to collect statistic information from its connected devices.
To do so, \linkResourceDirective{Dir}{Storage}{Collect Statistics} must be enabled.
This interval defines, how often the Director connects to the attached Storage Daemons to collect the statistic information.
}

\defDirective{Dir}{Director}{Secure Erase Command}{}{}{%
When files are no longer needed, Bareos will delete (unlink) them.
With this directive, it will call the specified command to delete these files. See \nameref{sec:SecureEraseCommand} for details.
}

\defDirective{Dir}{Director}{Statistics Retention}{}{}{%
\label{PruneStatistics}%
The \configdirective{Statistics Retention} directive defines the length of time that
Bareos will keep statistics job records in the Catalog database after the
Job End time (in \texttt{JobHistory} table). When this time period expires,
and if user runs \texttt{prune stats} command, Bareos will prune (remove)
Job records that are older than the specified period.

Theses statistics records aren't use for restore purpose, but mainly for
capacity planning, billings, etc.
See chapter about \ilink{how to extract information from the catalog}{UseBareosCatalogToExtractInformationChapter}
 for additional information.

See the \ilink{Configuration chapter}{Time} of this manual for additional
details of time specification.
}


\defDirective{Dir}{Director}{Subscriptions}{}{12.4.4}{%
In case you want check that the number of active clients don't exceed a specific number,
you can define this number here and check with the \bcommand{status subscriptions}{} command.

However, this is only indended to give a hint. No active limiting is implemented.
}


\defDirective{Dir}{Director}{Sub Sys Directory}{}{}{%
}


\defDirective{Dir}{Director}{TLS Enable}{}{}{%
Bareos can be configured to encrypt all its network traffic.
See chapter \nameref{TlsDirectives} to see,
how the Bareos Director (and the other components) must be configured to use TLS.
}


\defDirective{Dir}{Director}{Ver Id}{}{}{%
where  \parameter{string} is an identifier which can be used for support purpose.
This string is displayed using the \bcommand{version}{} command.
}


\defDirective{Dir}{Director}{Working Directory}{}{}{%
This directive is optional and specifies a directory in which the Director
may put its status files. This directory should be used only  by Bareos but
may be shared by other Bareos daemons.
Standard shell expansion of the {\bf
directory}  is done when the configuration file is read so that values such
as {\bf \$HOME} will be properly expanded.

The working directory specified must already exist and be
readable and writable by the Bareos daemon referencing it.
}
