\chapter{Plugins}
\index[general]{Plugin}

The functionality of Bareos can be extended by plugins.
They do exists plugins for the different daemons (Director, Storage- and File-Daemon).

To use plugins, they must be enabled in the configuration (\configdirective{Plugin Directory} and optionally \configdirective{Plugin Names}).

If a \configdirective{Plugin Directory} is specified
\configdirective{Plugin Names} defines, which plugins get loaded.

If \configdirective{Plugin Names} is not defined.

\section{File Daemon Plugins}
\label{fdPlugins}

File Daemon plugins are configured by the \configdirective{Plugin} directive of a \ilink{File Set}{directive-fileset-plugin}.

\warning{Currently the plugin command is being stored as part of the backup. The restore command in your directive should be flexible enough if things might change in future, otherwise you could run into trouble.}

\subsection{bpipe Plugin}
\label{bpipe}
\index[general]{Plugin!bpipe}

The bpipe plugin is a generic pipe program, that simply transmits the data from a specified program to Bareos for backup, and
from Bareos to a specified program for restore. The purpose of the plugin is to provide an interface to any system program
for backup and restore. That allows you, for example, to do database backups without a local dump. By using different command 
lines to bpipe, you can backup any kind of data (ASCII or binary) depending on the program called.

On Linux, the Bareos bpipe plugin is part of the \package{bareos-filedaemon} package and is therefore installed on any system running the filedaemon.

The bpipe plugin is so simple and flexible, you may call it the 
"Swiss Army Knife" of the current existing plugins for Bareos.

The bpipe plugin is specified in the Include section of your Job's FileSet resource in your \file{bareos-dir.conf}.

\begin{bconfig}{bpipe fileset}
FileSet {
  Name = "MyFileSet"
  Include {
    Options {
      signature = MD5
      compression = gzip
    }
    Plugin = "bpipe:file=<filepath>:reader=<readprogram>:writer=<writeprogram>
  }
}
\end{bconfig}

The syntax and semantics of the Plugin directive require the first part of the string up to the colon to be the name of the plugin.
Everything after the first colon is ignored by the File daemon but is passed to the plugin. Thus the plugin writer may define the 
meaning of the rest of the string as he wishes. The full syntax of the plugin directive as interpreted by the bpipe plugin is:

\begin{bconfig}{bpipe directive}
Plugin = "<plugin>:file=<filepath>:reader=<readprogram>:writer=<writeprogram>"
\end{bconfig}

\begin{description}
\item[plugin] is the name of the plugin with the trailing -fd.so stripped off, so in this case, we would put bpipe in the field.

\item[filepath] specifies the namespace, which for bpipe is the pseudo path and filename under which the backup will be saved. This
pseudo path and filename will be seen by the user in the restore file tree. For example, if the value is \argument{/MySQL/mydump.sql},
the data
backed up by the plugin will be put under that \bquote{pseudo} path and filename.
You must be careful to choose a naming convention that is unique
to avoid a conflict with a path and filename that actually exists on your system.

\item[readprogram] for the bpipe plugin specifies the "reader" program that is called by the plugin during backup to read the data. bpipe
will call this program by doing a popen on it.

\item[writeprogram] for the bpipe plugin specifies the "writer" program that is called by the plugin during restore to write the data back 
to the filesystem.
\end{description}

Please note that the two items above describing the "reader" and "writer", these programs are "executed" by Bareos, which means 
there is no shell interpretation of any command line arguments you might use. If you want to use shell characters (redirection of input 
or output, ...), then we recommend that you put your command or commands in a shell script and execute the script. In addition if you
backup a file with reader program, when running the writer program during the restore, Bareos will not automatically create the path
to the file. Either the path must exist, or you must explicitly do so with your command or in a shell script.

See the examples about \nameref{backup-postgresql} and \nameref{backup-mysql}.


\subsection{PGSQL Plugin}

See chapter \nameref{backup-postgresql-plugin}.


\subsection{MSSQL Plugin}

See chapter \nameref{MSSQL}.

\subsection{LDAP Plugin}
\index[general]{Plugin!ldap}

This plugin is intended to backup (and restore) the contents of a LDAP server.
It uses normal LDAP operation for this.
The package \package{bareos-filedaemon-ldap-python-plugin} (\sinceVersion{Fd}{LDAP Plugin}{15.2.0}) contains an example configuration file, that must be adapted to your envirnoment.

\subsection{Cephfs Plugin}
\index[general]{Plugin!ceph!cephfs}
\index[general]{Ceph!Cephfs Plugin}

Opposite to the \ilink{Rados Backend}{SdBackendRados} that is used to store data on a CEPH Object Store,
this plugin is intended to backup a CEPH Object Store via the Cephfs interface to other media.
The package \package{bareos-filedaemon-ceph-plugin} (\sinceVersion{Fd}{Cephfs Plugin}{15.2.0}) contains an example configuration file, that must be adapted to your envirnoment.

\subsection{Rados Plugin}
\index[general]{Plugin!ceph!rados}
\index[general]{Ceph!Rados Plugin}

Opposite to the \ilink{Rados Backend}{SdBackendRados} that is used to store data on a CEPH Object Store,
this plugin is intended to backup a CEPH Object Store via the Rados interface to other media.
The package \package{bareos-filedaemon-ceph-plugin} (\sinceVersion{Fd}{CEPH Rados Plugin}{15.2.0}) contains an example configuration file, that must be adapted to your envirnoment.


\subsection{GlusterFS Plugin}
\index[general]{Plugin!glusterfs}
\index[general]{GlusterFS!Plugin}

Opposite to the \ilink{GFAPI Backend}{SdBackendGfapi} that is used to store data on a Gluster system,
this plugin is intended to backup data from a Gluster system to other media.
The package \package{bareos-filedaemon-glusterfs-plugin} (\sinceVersion{Fd}{GlusterFS Plugin}{15.2.0}) contains an example configuration file, that must be adapted to your envirnoment.


\subsection{python-fd Plugin}
\index[general]{Plugin!Python!File Daemon}

The \name{python-fd} plugin behaves similar to the \nameref{director-python-plugin}. Base plugins and an example get installed via the package bareos-filedaemon-python-plugin.
Configuration is done in the \nameref{DirectorResourceFileSet} on the director.

We basically distinguish between command-plugin and option-plugins. 

\subsubsection{Command Plugins}
Command plugins are used to replace or extend the FileSet definition in the File Section. If you have a command-plugin, 
you can use it like in this example:

\begin{bconfig}{bareos-dir.conf: Python FD command plugins}
FileSet {
  Name = "mysql"
  Include {
    Options {
      Signature = MD5 # calculate md5 checksum per file
    }
    File = "/etc"
    Plugin = "python:module_path=/usr/lib/bareos/plugins:module_name=bareos-fd-mysql"
  }
} 
\end{bconfig}

\index[general]{MySQL!Backup}
This example uses the \ilink{MySQL plugin}{backup-mysql-python} to backup MySQL dumps in addition to \file{/etc}.

\subsubsection{Option Plugins}
Option plugins are activated in the Options resource of a FileSet definition.

Example:

\begin{bconfig}{bareos-dir.conf: Python FD option plugins}
FileSet {
  Name = "option"
  Include {
    Options {
      Signature = MD5 # calculate md5 checksum per file
      Plugin = "python:module_path=/usr/lib/bareos/plugins:module_name=bareos-fd-file-interact"
    }
    File = "/etc"
    File = "/usr/lib/bareos/plugins"
  }
}
\end{bconfig}

This plugin bareos-fd-file-interact from \url{https://github.com/bareos/bareos-contrib/tree/master/fd-plugins/options-plugin-sample} has a method that is called before and after each file that goes into the backup,
it can be used as a template for whatever plugin wants to interact with files before or after backup.

\subsection{VMware Plugin}
\label{VMwarePlugin}
\index[general]{Plugin!VMware}

The \vmware\ Plugin can be
used for agentless backups of virtual machines running on
\vSphere.
It makes use of CBT (Changed Block Tracking) to do space efficient
full and incremental backups, see below for mandatory requirements.

\subsubsection{Status}

The current development status of the \vmware\
plugin is experimental.

The Plugin can do full, differential and incremental backup and restore of VM disks.

Current limitations amongst others are:
\begin{itemize}
\item it is not yet possible to exclude normal (dependent) VM disks from backups,
  but independent disks are excluded implicitly because they are not affected
  by snapshots which are required for CBT based backup.
\item the VM configuration is not yet backed up, so that it is not yet
  possible to recreate a completely deleted VM
\item the above implies that restore is still only possible to the same existing
  VM with exising virtual disks
\end{itemize}

\subsubsection{Requirements}
As the Plugin is based on the \vSphere\ Storage APIs for Data Protection,
which requires at least a \vSphere\ Essentials License.
It does not work with standalone unlicensed
VMware® ESXi\trademark.

\subsubsection{Installation}

First add an appropriate package repository from
\url{http://download.bareos.org/bareos/experimental/vmware-plugin/}.
Currently, packages exist for CentOS 7, RHEL 7, Debian 8, SLES 11 SP3 and SLES 12.
The procedure to add the repository is almost the same as described
in section \nameref{sec:InstallBareosPackages}. Note that the name of the
\path{.repo} for RPM based Linux distributions is \path{bareos:vmware.repo}
for this case.

Install the package \package{bareos-vmware-plugin} including its requirments
by using an appropriate package management tool
(eg. \command{yum}, \command{zypper}, \command{apt})

The \elink{FAQ}{http://www.bareos.org/en/faq.html} may have additional
useful information.

\subsubsection{Configuration}

First add a user account in vCenter that has full privileges by assigning
the account to an administrator role or by adding the account to a group
that is assigned to an administrator role. While any user account
with full privileges could be used, it is better practice to create a separate
user account, so that the actions by this account logged in vSphere are clearly
distinguishable. In the future a more detailed set of required role privilges
may be defined.

When using the vCenter appliance with embedded SSO, a user account usually has the
structure \command{<username>@vsphere.local}, it may be different when using
Active Directory as SSO in vCenter. For the examples here, we will use
\command{bakadm@vsphere.local} with the password \command{Bak.Adm-1234}.

For more details regarding users and permissions in vSphere see
\url{http://pubs.vmware.com/vsphere-55/topic/com.vmware.vsphere.security.doc/GUID-72BFF98C-C530-4C50-BF31-B5779D2A4BBB.html} and
\url{http://pubs.vmware.com/vsphere-55/topic/com.vmware.vsphere.security.doc/GUID-5372F580-5C23-4E9C-8A4E-EF1B4DD9033E.html}

Make sure to add or enable the following settings in \path|/etc/bareos/bareos-fd.conf|:

\begin{bconfig}{bareos-fd.conf: Plugin Settings}
...
FileDaemon {
...
  Plugin Directory = /usr/lib/bareos/plugins
  Plugin Names = python
...
}
\end{bconfig}

Note: Depending on Platform, the Plugin Directory may also be \path|/usr/lib64/bareos/plugins|

To define the backup of a VM in Bareos, a job definition and a fileset
resource must be added to the Bareos director confguration.
In vCenter, VMs are usually organized in datacenters and folders.
The following example shows how to configure the backup of the VM
named \textit{websrv1} in the datacenter \textit{mydc1}
folder \textit{webservers} on the vCenter server \command{vcenter.example.org}:

\begin{bconfig}{bareos-dir.conf: VMware Plugin Job and FileSet definition}
Job {
  Name = "vm-websrv1"
  JobDefs = "DefaultJob"
  FileSet = "vm-websrv1_fileset"
}

FileSet {
  Name = "vm-websrv1_fileset"

  Include {
    Options {
         signature = MD5
         Compression = GZIP
    }
    Plugin = "python:module_path=/usr/lib64/bareos/plugins/vmware_plugin:module_name=bareos-fd-vmware:dc=mydc1:folder=/webservers:vmname=websrv1:vcserver=vcenter.example.org:vcuser=bakadm@vsphere.local:vcpass=Bak.Adm-1234"
  }
}
\end{bconfig}

For VMs defined in the root-folder, \command{folder=/} must be specified
in the Plugin definition.

\subsubsection{Backup}

Before running the first backup, CBT (Changed Block Tracking) must be
enabled for the VMs to be backed up.

As of \url{http://kb.vmware.com/kb/2075984} manually enabling CBT is
currently not working properly. The API however works properly.
To enable CBT use the Script \command{vmware_cbt_tool.py}, it is packaged
in the bareos-vmware-plugin package:

\begin{commands}{usage of vmware\_cbt\_tool.py}
# <parameter>vmware_cbt_tool.py --help</parameter>
usage: vmware_cbt_tool.py [-h] -s HOST [-o PORT] -u USER [-p PASSWORD] -d
                          DATACENTER -f FOLDER -v VMNAME [--enablecbt]
                          [--disablecbt] [--resetcbt] [--info]

Process args for enabling/disabling/resetting CBT

optional arguments:
  -h, --help            show this help message and exit
  -s HOST, --host HOST  Remote host to connect to
  -o PORT, --port PORT  Port to connect on
  -u USER, --user USER  User name to use when connecting to host
  -p PASSWORD, --password PASSWORD
                        Password to use when connecting to host
  -d DATACENTER, --datacenter DATACENTER
                        DataCenter Name
  -f FOLDER, --folder FOLDER
                        Folder Name
  -v VMNAME, --vmname VMNAME
                        Names of the Virtual Machines
  --enablecbt           Enable CBT
  --disablecbt          Disable CBT
  --resetcbt            Reset CBT (disable, then enable)
  --info                Show information (CBT supported and enabled or
                        disabled)
\end{commands}

For the above configuration example, the command to enable CBT would be

\begin{commands}{Example using vmware\_cbt\_tool.py}
# <parameter>vmware_cbt_tool.py -s vcenter.example.org -u bakadm@vsphere.local -p Bak.Adm-1234 -d mydc1 -f /webservers -v websrv1 --enablecbt</parameter>
\end{commands}

Note: CBT does not work if the virtual hardware version is 6 or earlier.

After enabling CBT, Backup Jobs can be run or scheduled as usual,
for example in \command{bconsole}:

\bcommand{run}{job=vm-websrv1 level=Full}

\subsubsection{Restore}

For restore, the VM must be powered off and no snapshot must exist.
In \command{bconsole} use the restore menu 5, select the correct FileSet
and enter \bcommand{mark *}{}, then \bcommand{done}{}. After restore has finished,
the VM can be powered on.


\section{Storage Daemon Plugins}
\label{sdPlugins}

\subsection{autoxflate-sd}
\label{plugin-autoxflate-sd}
\index[general]{Plugin!autoxflate-sd}


This plugin is part of the \package{bareos-storage} package.

The autoxflate-sd plugin can inflate (decompress) and deflate (compress)
the data being written to or read from a device. It can also do both.

\begin{center}
\includegraphics[width=0.8\textwidth]{\idir autoxflate-functionblocks}
\end{center}

Therefore the autoxflate plugin inserts a inflate and a deflate function block
into the stream going to the device (called OUT) and coming from the device (called IN).

Each stream passes first the inflate function block, then the deflate function block.

The inflate blocks are controlled by the setting of the \linkResourceDirective{Sd}{Device}{Auto Inflate} directive.

The deflate blocks are controlled by the setting of the \linkResourceDirective{Sd}{Device}{Auto Deflate},
\linkResourceDirective{Sd}{Device}{Auto Deflate Algorithm} and \linkResourceDirective{Sd}{Device}{Auto Deflate Level} directives.

The inflate blocks, if enabled, will uncompress data if it is compressed using the
algorithm that was used during compression.

The deflate blocks, if enabled, will compress uncompressed data with the algorithm and level configured in the according directives.

The series connection of the inflate and deflate function blocks makes the plugin very flexible.

Szenarios where this plugin can be used are for example:
\begin{itemize}
   \item client computers with weak cpus can do backups without compression and let the sd do the compression when writing to disk
    \item compressed backups can be recompressed to a different compression format (e.g. gzip \textrightarrow\ lzo) using migration jobs
    \item client backups can be compressed with compression algorithms that the client itself does not support
\end{itemize}

Multi-core cpus will be utilized when using parallel jobs as the compression is done in each jobs' thread.

When the autoxflate plugin is configured, it will write some status information into the joblog.

\begin{bmessage}{used compression algorithm}
autodeflation: compressor on device FileStorage is FZ4H
\end{bmessage}

\begin{bmessage}{configured inflation and deflation blocks}
autoxflate-sd.c: FileStorage OUT:[SD->inflate=yes->deflate=yes->DEV] IN:[DEV->inflate=yes->deflate=yes->SD]
\end{bmessage}

\begin{bmessage}{overall deflation/inflation ratio}
autoxflate-sd.c: deflate ratio: 50.59%
\end{bmessage}

Additional \linkResourceDirective{Sd}{Storage}{Auto XFlate On Replication} can be configured at the Storage resource.


\subsection{scsicrypto-sd}
\index[general]{Plugin!scsicrypto-sd}

This plugin is part of the \package{bareos-storage-tape} package.

\subsubsection{General}

\paragraph{LTO Hardware Encryption}
\label{LTOHardwareEncryptionGeneral}

Modern tape-drives, for example LTO (from LTO4 onwards) support hardware encryption.
There are several ways of using encryption with these drives. The following three types of key management are available for encrypting drives. The transmission of the keys to the volumes is accomplished by either of the three:

\begin{itemize}
 \item A backup application that supports Application Managed Encryption (AME)
 \item A tape library that supports Library Managed Encryption (LME)
 \item A Key Management Appliance (KMA)
\end{itemize}

We added support for Application Managed Encryption (AME) scheme, where on labeling a crypto key is generated for a volume and when the volume is mounted, the crypto key is loaded. When finally the volume is unmounted, the key is cleared from the memory of the Tape Drive using the SCSI SPOUT command set.

If you have implemented Library Managed Encryption (LME) or a Key Management Appliance (KMA), there is no need to have support from Bareos on loading and clearing the encryption keys, as either the Library knows the per volume encryption keys itself, or it will ask the KMA for the encryption key when it needs it. For big installations you might consider using a KMA, but the Application Managed Encryption implemented in Bareos should also scale rather well and have a low overhead as the keys are only loaded and cleared when needed.

\paragraph{The scsicrypto-sd plugin}

The \command{scsicrypto-sd} hooks into the \pluginevent{unload}, \pluginevent{label read}, \pluginevent{label write} and \pluginevent{label verified} events for loading and clearing the key. It checks whether it it needs to clear the drive by either using an internal state (if it loaded a key before) or by checking the state of a special option that first issues an encrytion status query. If there is a connection to the director and the volume information is not available, it will ask the director for the data on the currently loaded volume. If no connection is available, a cache will be used which should contain the most recently mounted volumes. If an encryption key is available, it will be loaded into the drive's memory.

\paragraph{Changes in the director}

The director has been extended with additional code for handling hardware data encryption. The extra keyword \name{encrypt} on the label of a volume will force the director to generate a new semi-random passphrase for the volume, which will be stored in the database as part of the media information.

A passphrase is always stored in the database base64-encoded. When a so called \name{Key Encryption Key} is set in the config of the director, the passphrase is first wrapped using RFC3394 key wrapping and then base64-encoded. By using key wrapping, the keys in the database are safe against people sniffing the info, as the data is still encrypted using the Key Encryption Key (which in essence is just an extra passphrase of the same length as the volume passphrases used).

When the storage daemon needs to mount the volume, it will ask the director for the volume information and that protocol is extended with the exchange of the base64-wrapped encryption key (passphrase). The storage daemon provides an extra config option in which it records the Key Encryption Key of the particular director, and as such can unwrap the key sent into the original passphrase.

As can be seen from the above info we don't allow the user to enter a passphrase, but generate a semi-random passphrase using the openssl random functions (if available) and convert that into a readable ASCII stream of letters, numbers and most other characters, apart from the quotes and space etc. This will produce much stronger passphrases than when requesting the info from a user. As we store this information in the database, the user never has to enter these passphrases.

The volume label is written in unencrypted form to the volume, so we can always recognize a Bareos volume. When the key is loaded onto the drive, we set the decryption mode to mixed, so we can read both unencrypted and encrypted data from the volume. When no key or the wrong key has been loaded, the drive will give an IO error when trying to read the volume.
For disaster recovery you can store the Key Encryption Key and the content of the wrapped encryption keys somewhere safe and the \ilink{bscrypto}{bscrypto} tool together with the scsicrypto-sd plugin can be used to get access to your volumes, in case you ever lose your complete environment.

If you don't want to use the scsicrypto-sd plugin when doing DR and you are only reading one volume, you can also set the crypto key using the bscrypto tool. Because we use the mixed decryption mode, in which you can read both encrypted and unencrypted data from a volume, you can set the right encryption key before reading the volume label.

If you need to read more than one volume, you better use the scsicrypto-sd plugin with tools like bscan/bextract, as the plugin will then auto-load the correct encryption key when it loads the volume, similiarly to what the storage daemon does when performing backups and restores.

The volume label is unencrypted, so a volume can also be recognized by a non-encrypted installation, but it won't be able to read the actual data from it. Using an encrypted volume label doesn't add much security (there is no security-related info in the volume label anyhow) and it makes it harder to recognize either a labeled volume with encrypted data or an unlabeled new volume (both would return an IO-error on read of the label.)

\subsubsection{Configuration}

\paragraph{SCSI crypto setup}
%
The initial setup of SCSI crypto looks something like this:
%
\begin{itemize}
 \item Generate a Key Encryption Key e.g.
\begin{commands}{}
bscrypto -g -
\end{commands}
\end{itemize}

For details see \ilink{bscrypto}{bscrypto}.



\paragraph{Security Setup}

Some security levels need to be increased for the storage daemon to be able to use the low level SCSI interface for setting and getting the encryption status on a tape device.

The following additional security is needed for the following operating systems:

\subparagraph{Linux (SG\_IO ioctl interface):} The user running the storage daemon needs the following additional capabilities:
\index[sd]{Platform!Linux!Privileges}

\begin{itemize}
 \item \parameter{CAP_SYS_RAWIO} (see capabilities(7))
 \begin{itemize}
  \item On older kernels you might need \parameter{CAP_SYS_ADMIN}. Try \parameter{CAP_SYS_RAWIO} first and if that doesn't work try \parameter{CAP_SYS_ADMIN}
 \end{itemize}
 \item If you are running the storage daemon as another user than root (which has the \parameter{CAP_SYS_RAWIO} capability), you need to add it to the current set of capabilities.
 \item If you are using systemd, you could add this additional capability to the CapabilityBoundingSet parameter.
 \begin{itemize}
  \item For systemd add the following to the bareos-sd.service: \parameter{Capabilities=cap_sys_rawio+ep}
 \end{itemize}
\end{itemize}

You can also set up the extra capability on \command{bscrypto} and \command{bareos-sd} by running the following commands:

\begin{commands}{}
setcap cap_sys_rawio=ep bscrypto
setcap cap_sys_rawio=ep bareos-sd
\end{commands}

Check the setting with

\begin{commands}{}
getcap -v bscrypto
getcap -v bareos-sd
\end{commands}

\command{getcap} and \command{setcap} are part of libcap-progs.

If \command{bareos-sd} does not have the appropriate capabilities, all
other tape operations may still work correctly, but you will
get \bquote{Unable to perform SG\_IO ioctl} errors.



\subparagraph{Solaris (USCSI ioctl interface):} The user running the storage daemon needs the following additional privileges:
\index[sd]{Platform!Solaris!Privileges}
\begin{itemize}
 \item \parameter{PRIV_SYS_DEVICES} (see privileges(5))
\end{itemize}

If you are running the storage daemon as another user than root (which has the \parameter{PRIV_SYS_DEVICES} privilege), you need to add it to the current set of privileges.
This can be set up by setting this either as a project for the user, or as a set of extra privileges in the SMF definition starting the storage daemon. The SMF setup is the cleanest one.

For SMF make sure you have something like this in the instance block:

\begin{bconfig}{}
<method_context working_directory=":default"> <method_credential user="bareos" group="bareos" privileges="basic,sys_devices"/> </method_context>
\end{bconfig}

\paragraph{Changes in bareos-sd.conf}

\begin{itemize}
 \item Set the Key Encryption Key
   \begin{itemize}
   \item \linkResourceDirective{Sd}{Director}{Key Encryption Key} = \argument{passphrase}
   \end{itemize}
 \item Enable the loading of storage daemon plugins
   \begin{itemize}
   \item \linkResourceDirective{Sd}{Storage}{Plugin Directory} = \directory{path_to_sd_plugins}
   \end{itemize}
 \item Enable the SCSI encryption option
   \begin{itemize}
   \item \linkResourceDirective{Sd}{Device}{Drive Crypto Enabled} = yes
   \end{itemize}
 \item Enable this, if you want the plugin to probe the encryption status of the drive when it needs to clear a pending key
   \begin{itemize}
   \item \linkResourceDirective{Sd}{Device}{Query Crypto Status} = yes
   \end{itemize}
\end{itemize}

\paragraph{Changes in bareos-dir.conf}

\begin{itemize}
 \item Set the Key Encryption Key
   \begin{itemize}
   \item \linkResourceDirective{Dir}{Director}{Key Encryption Key} = \argument{passphrase}
   \end{itemize}
\end{itemize}

\subsubsection{Testing}

Restart the Storage Daemon and the Director.
After this you can label new volumes with the encrypt option, e.g.
\begin{bconfig}{}
label slots=1-5 barcodes encrypt
\end{bconfig}


\subsubsection{Disaster Recovery}

For Disaster Recovery (DR) you need the following information:

\begin{itemize}
 \item Actual bareos-sd.conf with config options enabled as described above, including, among others, a definition of a director with the Key Encryption Key used for creating the encryption keys of the volumes.
 \item The actual keys used for the encryption of the volumes.
\end{itemize}

This data needs to be availabe as a so called crypto cache file which is used by the plugin when no connection to the director can be made to do a lookup (most likely on DR).

Most of the times the needed information, e.g. the bootstrap info, is available on recently written volumes and most of the time the encryption cache will contain the most recent data, so a recent copy of the \file{bareos-sd.<portnr>.cryptoc} file in the working directory is enough most of the time. You can also save the info from database in a safe place and use bscrypto to populate this info 
(VolumeName \textrightarrow EncryptKey) into the crypto cache file used by \command{bextract} and \command{bscan}.
You can use \command{bscrypto} with the following flags to create a new or update an existing crypto cache file e.g.:

\begin{commands}{}
bscrypto -p /var/lib/bareos/bareos-sd.<portnr>.cryptoc
\end{commands}

\begin{itemize}
 \item A valid BSR file containing the location of the last safe of the database makes recovery much easier. Adding a post script to the database save job could collect the needed info and make sure its stored somewhere safe.
 \item Recover the database in the normal way e.g. for postgresql:
\begin{commands}{}
bextract -D <director_name> -c bareos-sd.conf -V <volname> \ /dev/nst0 /tmp -b bootstrap.bsr
/usr/lib64/bareos/create_bareos_database
/usr/lib64/bareos/grant_bareos_privileges
psql bareos < /tmp/var/lib/bareos/bareos.sql
\end{commands}
\end{itemize}

Or something similar (change paths to follow where you installed the software or where the package put it).

\textbf{Note:} As described at the beginning of this chapter, there are different types of key management, AME, LME and KMA. If the Library is set up for LME or KMA, it probably won't allow our AME setup and the scsi-crypto plugin will fail to set/clear the encryption key. To be able to use AME you need to \bquote{Modify Encryption Method} and set it to something like \bquote{Application Managed}.
If you decide to use LME or KMA you don't have to bother with the whole setup of AME which may for big libraries be easier, although the overhead of using AME even for very big libraries should be minimal.

\subsection{scsitapealert-sd}
\index[general]{Plugin!scsitapealert-sd}

This plugin is part of the \package{bareos-storage-tape} package.


\subsection{python-sd Plugin}
\index[general]{Plugin!Python!Storage Daemon}

The \name{python-sd} plugin behaves similar to the \nameref{director-python-plugin}.


\section{Director Plugins}
\label{dirPlugins}

\subsection{python-dir Plugin}
\label{director-python-plugin}
\index[general]{Plugin!Python!Director}

The \name{python-dir} plugin is intended to extend the functionality of the Bareos Director by Python code.
A working example is included.

\begin{itemize}
    \item install the \package{bareos-director-python-plugin} package
    \item change to the Bareos plugin directory (\directory{/usr/lib/bareos/plugins/} or \directory{/usr/lib64/bareos/plugins/})
    \item copy \file{bareos-dir.py.template} to \file{bareos-dir.py}
    \item activate the plugin in the Bareos Director configuration
    \item restart the Bareos Director
    \item change \file{bareos-dir.py} as required
    \item restart the Bareos Director
\end{itemize}


\subsubsection{Loading plugins}

Since \sinceVersion{dir}{multiple Python plugins}{14.4.0} multiple Python plugins can be loaded and plugin names can be arbitrary. Before this, the Python plugin always loads the file \file{bareos-dir.py}.

The director plugins are configured in the Job-Resource (or JobDefs resource). To load a Python plugin you need
\begin{itemize}
 \item [module\_path= ] pointing to your plugin directory (needs to be enabled in the Director resource, too
 \item [module\_name= ] Your plugin (without the suffix .py)
 \item [instance= ] default is '0', you can leave this, as long as you only have 1 Director Python plugin. If you have more than 1, start with instance=0 and increment the instance for each plugin.
 \item You can add plugin specific option key-value pairs, each pair separated by ':' key=value
\end{itemize}


Example:

\begin{bconfig}{bareos-dir.conf: Python Plugins}
Director {
  # ...
  # Plugin directory
  Plugin Directory = /usr/lib64/bareos/plugins
  # Load the python plugin
  Plugin Names = "python"
}

JobDefs {
  Name = "DefaultJob"
  Type = Backup
  # ...
  # Load the class based plugin with testoption=testparam
  Dir Plugin Options = "python:instance=0:module_path=/usr/lib64/bareos/plugins:module_name=bareos-dir-class-plugins:testoption=testparam
  # ...
}
\end{bconfig}

\subsubsection{Write your own Python Plugin}

Some plugin examples are available on \url{https://github.com/bareos/bareos-contrib}. 
The class-based approach lets you easily reuse stuff already
defined in the baseclass BareosDirPluginBaseclass, which ships with the \package{bareos-director-python-plugin} package.
The examples contain the plugin bareos-dir-nsca-sender, that submits the results and performance data of a backup job directly to Icinga\index[general]{Icinga} or Nagios\index[general]{Nagios|see{Icinga}} using the NSCA protocol.
