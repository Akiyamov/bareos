\chapter{Plugins}
\index[general]{Plugin}

\section{File Daemon Plugins}

File Daemon plugins are configured by the \configdirective{plugin} directive of a FileSet, see the \ilink{File Set}{directive-fileset-plugin} section.

\subsection{bpipe Plugin}
\label{bpipe}

The bpipe plugin is a generic pipe program, that simply transmits the data from a specified program to Bareos for backup, and
from Bareos to a specified program for restore. The purpose of the plugin is to provide an interface to any system program
for backup and restore. That allows you, for example, to do database backups without a local dump. By using different command 
lines to bpipe, you can backup any kind of data (ASCII or binary) depending on the program called.

On Linux, the Bareos bpipe plugin is part of the \package{bareos-filedaemon} package and is therefore installed on any system running the filedaemon.

The bpipe plugin is so simple and flexible, you may call it the 
"Swiss Army Knife" of the current existing plugins for Bareos.

The bpipe plugin is specified in the Include section of your Job's FileSet resource in your \file{bareos-dir.conf}.

\begin{bconfig}{bpipe fileset}
FileSet {
  Name = "MyFileSet"
  Include {
    Options {
      signature = MD5
      compression = gzip
    }
    Plugin = "bpipe:<path>:<backup-command>:<restore-command>"
  }
}
\end{bconfig}

The syntax and semantics of the Plugin directive require the first part of the string up to the colon to be the name of the plugin.
Everything after the first colon is ignored by the File daemon but is passed to the plugin. Thus the plugin writer may define the 
meaning of the rest of the string as he wishes. The full syntax of the plugin directive as interpreted by the bpipe plugin is:

\begin{bconfig}{bpipe directive}
Plugin = "<plugin>:<path>:<backup-command>:<restore-command>"
\end{bconfig}

\begin{description}
\item[plugin] is the name of the plugin with the trailing -fd.so stripped off, so in this case, we would put bpipe in the field.

\item[path] specifies the namespace, which for bpipe is the pseudo path and filename under which the backup will be saved. This
pseudo path and filename will be seen by the user in the restore file tree. For example, if the value is /MySQL/mydump.sql, the data
backed up by the plugin will be put under that "pseudo" path and filename. You must be careful to choose a naming convention that is unique
to avoid a conflict with a path and filename that actually exists on your system.

\item[backup-command] for the bpipe plugin specifies the "reader" program that is called by the plugin during backup to read the data. bpipe
will call this program by doing a popen on it.

\item[restore-command] for the bpipe plugin specifies the "writer" program that is called by the plugin during restore to write the data back 
to the filesystem.
\end{description}

Please note that the two items above describing the "reader" and "writer", these programs are "executed" by Bareos, which means 
there is no shell inerpretation of any command line arguments you might use. If you want to use shell characters (redirection of input 
or output, ...), then we recommend that you put your command or commands in a shell script and execute the script. In addition if you
backup a file with reader program, when running the writer program during the restore, Bareos will not automatically create the path
to the file. Either the path must exist, or you must explicitly do so with your command or in a shell script.

See the examples about \ilink{Backup PostgreSQL}{backup-postgresql} and \ilink{Backup MySQL}{backup-mysql}.

\subsection{MSSQL Plugin}

See chapter \ilink{Backup Of MSSQL Databases}{MSSQL}.

