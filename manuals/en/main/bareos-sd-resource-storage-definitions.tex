\defDirective{Sd}{Storage}{Absolute Job Timeout}{}{}{%
}

\defDirective{Sd}{Storage}{Allow Bandwidth Bursting}{}{}{%
}

\defDirective{Sd}{Storage}{Auto XFlate On Replication}{}{13.4.0}{%
This directive controls the \ilink{autoxflate-sd plugin}{plugin-autoxflate-sd}
plugin when replicating data inside one or
between two storage daemons (Migration/Copy Jobs). Normally the storage daemon will
use the autoinflate/autodeflate setting of the device when reading and writing
data to it which could mean that while reading it inflates the compressed data
and the while writing the other deflates it again. If you just want the data to
be exactly the same e.g. don't perform any on the fly uncompression and compression
while doing the replication of data you can set this option to no and it will
override any setting on the device for doing auto inflation/deflation when doing
data replication. This will not have any impact on any normal backup or restore jobs.
}

\defDirective{Sd}{Storage}{Client Connect Wait}{}{}{%
This directive defines an interval of time in seconds that
the Storage daemon will wait for a Client (the File daemon)
to connect.  Be aware that the
longer the Storage daemon waits for a Client, the more
resources will be tied up.
}

\defDirective{Sd}{Storage}{Collect Device Statistics}{}{}{%
}

\defDirective{Sd}{Storage}{Collect Job Statistics}{}{}{%
}

\defDirective{Sd}{Storage}{Compatible}{}{}{%
This directive enables the compatible mode of the storage daemon. In
this mode the storage daemon will try to write the storage data in a
compatible way with Bacula of which Bareos is a fork. This only works
for the data streams both share and not for any new datastreams which
are Bareos specific. Which may be read when used by a Bareos storage
daemon but might not be understood by any of the Bacula components
(dir/sd/fd).

The default setting of this directive was changed to no since Bareos \sinceVersion{sd}{Compatible = no}{15.2.0}.
}

\defDirective{Sd}{Storage}{Description}{}{}{%
}

\defDirective{Sd}{Storage}{Device Reserve By Media Type}{}{}{%
}

\defDirective{Sd}{Storage}{FD Connect Timeout}{}{}{%
}

\defDirective{Sd}{Storage}{Heartbeat Interval}{}{}{%
\index[general]{Broken pipe}%
This directive defines an interval of time in seconds.  When
the Storage daemon is waiting for the operator to mount a
tape, each time interval, it will send a heartbeat signal to
the File daemon.  The default interval is zero which disables
the heartbeat.  This feature is particularly useful if you
have a router that does not follow Internet
standards and times out an valid connection after a short
duration despite the fact that keepalive is set.  This usually
results in a broken pipe error message.
}

\defDirective{Sd}{Storage}{Maximum Bandwidth Per Job}{}{}{%
}

\defDirective{Sd}{Storage}{Maximum Concurrent Jobs}{}{}{%
This directive specifies the maximum number of Jobs that may run
concurrently. Each contact from the Director (e.g.  status request, job start
request) is considered as a Job, so if you want to be able to do a \bcommand{status}{}
request in the console at the same time as a Job is running, you
will need to set this value greater than 1.  To run simultaneous Jobs,
you will need to set a number of other directives in the Director's
configuration file.  Which ones you set depend on what you want, but you
will almost certainly need to set the \linkResourceDirective{Dir}{Storage}{Maximum Concurrent Jobs}.
Please refer to the \nameref{ConcurrentJobs} chapter.
}

\defDirective{Sd}{Storage}{Maximum Network Buffer Size}{}{}{%
}

\defDirective{Sd}{Storage}{Messages}{}{}{%
}

\defDirective{Sd}{Storage}{Name}{}{}{%
Specifies the Name of the Storage daemon.
}

\defDirective{Sd}{Storage}{NDMP Address}{}{}{%
This directive is optional, and if it is specified, it will cause the
Storage daemon server (for NDMP Tape Server connections) to bind
to the specified {\bf IP-Address}, which is either a domain name or an
IP address specified as a dotted quadruple.  If this directive is not
specified, the Storage daemon will bind to any available address (the
default).
}

\defDirective{Sd}{Storage}{NDMP Addresses}{}{}{%
Specify the ports and addresses on which the Storage daemon will listen
for NDMP Tape Server connections.  Normally, the default is sufficient and you
do not need to specify this directive.
}

\defDirective{Sd}{Storage}{NDMP Enable}{}{}{%
This directive enables the Native NDMP Tape Agent.
}

\defDirective{Sd}{Storage}{NDMP Log Level}{}{}{%
This directive sets the loglevel for the NDMP protocol library.
}

\defDirective{Sd}{Storage}{NDMP Port}{}{}{%
Specifies port number on which the Storage daemon listens for NDMP Tape Server
connections.
}

\defDirective{Sd}{Storage}{NDMP Snooping}{}{}{%
This directive enables the Snooping and pretty printing of NDMP protocol
information in debugging mode.
}

\defDirective{Sd}{Storage}{Pid Directory}{}{}{%
This directive specifies a directory in which the Storage Daemon may put its
process Id file files. The process Id file is used to  shutdown Bareos and to
prevent multiple copies of  Bareos from running simultaneously.
Standard shell expansion of the {\bf directory} is done when the
configuration file is read so that values such  as {\bf \$HOME} will be
properly expanded.
}

\defDirective{Sd}{Storage}{Plugin Directory}{}{}{%
This directive specifies a directory in which the Storage Daemon searches for
plugins with the name \file{<pluginname>-sd.so} which it will load at startup.
}

\defDirective{Sd}{Storage}{Plugin Names}{}{}{%
If a \linkResourceDirective{Sd}{Storage}{Plugin Directory} is specified
\configdirective{Plugin Names} defines, which \nameref{sdPlugins} get loaded.

If \configdirective{Plugin Names} is not defined, all plugins get loaded,
otherwise the defined ones.
}

\defDirective{Sd}{Storage}{Scripts Directory}{}{}{%
This directive is currently unused.
}

\defDirective{Sd}{Storage}{SD Address}{}{}{%
This directive is optional, and if it is specified, it will cause the
Storage daemon server (for Director and File daemon connections) to bind
to the specified IP-Address, which is either a domain name or an
IP address specified as a dotted quadruple.  
If this and the \linkResourceDirective{Sd}{Storage}{SD Addresses} directives are not
specified, the Storage daemon will bind to any available address (the
default).
}

\defDirective{Sd}{Storage}{SD Addresses}{}{}{%
Specify the ports and addresses on which the Storage daemon will listen for Director connections.
Using this directive, you can replace both the 
\linkResourceDirective{Sd}{Storage}{SD Port}
and
\linkResourceDirective{Sd}{Storage}{SD Address}
directives.
}

\defDirective{Sd}{Storage}{SD Connect Timeout}{}{}{%
}

\defDirective{Sd}{Storage}{SD Port}{}{}{%
Specifies port number on which the Storage daemon  listens for Director
connections.
}

\defDirective{Sd}{Storage}{SD Source Address}{}{}{%
}

\defDirective{Sd}{Storage}{Secure Erase Command}{}{}{%
When files are no longer needed, Bareos will delete (unlink) them.
With this directive, it will call the specified command to delete these files. See \nameref{sec:SecureEraseCommand} for details.
}

\defDirective{Sd}{Storage}{Statistics Collect Interval}{}{}{%
}

\defDirective{Sd}{Storage}{Sub Sys Directory}{}{}{%
}

\defDirective{Sd}{Storage}{TLS Allowed CN}{}{}{%
}

\defDirective{Sd}{Storage}{TLS Authenticate}{}{}{%
}

\defDirective{Sd}{Storage}{TLS CA Certificate Dir}{}{}{%
}

\defDirective{Sd}{Storage}{TLS CA Certificate File}{}{}{%
}

\defDirective{Sd}{Storage}{TLS Certificate}{}{}{%
}

\defDirective{Sd}{Storage}{TLS Certificate Revocation List}{}{}{%
}

\defDirective{Sd}{Storage}{TLS DH File}{}{}{%
}

\defDirective{Sd}{Storage}{TLS Enable}{}{}{%
Bareos can be configured to encrypt all its network traffic.
Chapter \nameref{TlsDirectives} explains
how the Bareos components must be configured to use TLS.
}

\defDirective{Sd}{Storage}{TLS Key}{}{}{%
}

\defDirective{Sd}{Storage}{TLS Require}{}{}{%
}

\defDirective{Sd}{Storage}{TLS Verify Peer}{}{}{%
}

\defDirective{Sd}{Storage}{Ver Id}{}{}{%
}

\defDirective{Sd}{Storage}{Working Directory}{}{}{%
This directive specifies a directory in which the Storage daemon may put
its status files. This directory should be used only  by {\bf Bareos},
but may be shared by other Bareos daemons provided the names given to each
daemon are unique.
}

