\defDirective{Dir}{Schedule}{Description}{}{}{%
}

\defDirective{Dir}{Schedule}{Enabled}{}{}{%
}

\defDirective{Dir}{Schedule}{Name}{}{}{%
The name of the schedule being defined.
}

\defDirective{Dir}{Schedule}{Run}{}{}{%
The Run directive defines when a Job is to be run, and what overrides if
any to apply.  You may specify multiple {\bf run} directives within a
{\bf Schedule} resource.  If you do, they will all be applied (i.e.
multiple schedules).  If you have two {\bf Run} directives that start at
the same time, two Jobs will start at the same time (well, within one
second of each other).

The {\bf Job-overrides} permit overriding the Level, the Storage, the
Messages, and the Pool specifications provided in the Job resource.  In
addition, the FullPool, the IncrementalPool, and the DifferentialPool
specifications permit overriding the Pool specification according to
what backup Job Level is in effect.

By the use of overrides, you may customize a particular Job.  For
example, you may specify a Messages override for your Incremental
backups that outputs messages to a log file, but for your weekly or
monthly Full backups, you may send the output by email by using a
different Messages override.

{\bf Job-overrides} are specified as: {\bf keyword=value} where the
keyword is Level, Storage, Messages, Pool, FullPool, DifferentialPool,
or IncrementalPool, and the {\bf value} is as defined on the respective
directive formats for the Job resource.  You may specify multiple {\bf
Job-overrides} on one {\bf Run} directive by separating them with one or
more spaces or by separating them with a trailing comma.  For example:

\begin{description}

\item [Level=Full]
\index[dir]{Level}
\index[dir]{Directive!Level}
is all files in the FileSet whether or not  they have changed.

\item [Level=Incremental]
\index[dir]{Level}
\index[dir]{Directive!Level}
is all files that have changed since  the last backup.

\item [Pool=Weekly]
\index[dir]{Pool}
\index[dir]{Directive!Pool}
specifies to use the Pool named {\bf Weekly}.

\item [Storage=DLT\_Drive]
\index[dir]{Storage}
\index[dir]{Directive!Storage}
specifies to use {\bf DLT\_Drive} for  the storage device.

\item [Messages=Verbose]
\index[dir]{Messages}
\index[dir]{Directive!Messages}
specifies to use the {\bf Verbose}  message resource for the Job.

\item [FullPool=Full]
\index[dir]{FullPool}
\index[dir]{Directive!FullPool}
specifies to use the Pool named {\bf Full}  if the job is a full backup, or
is upgraded from another type  to a full backup.

\item [DifferentialPool=Differential]
\index[dir]{DifferentialPool}
\index[dir]{Directive!DifferentialPool}
specifies to use the Pool named {\bf Differential} if the job is a
differential  backup.

\item [IncrementalPool=Incremental]
\index[dir]{IncrementalPool}
\index[dir]{Directive!IncrementalPool}
specifies to use the Pool  named {\bf Incremental} if the job is an
incremental  backup.

\item [Accurate=yes{\textbar}no]
\index[dir]{Accurate}
\index[dir]{Directive!Accurate}
tells Bareos to use or not the Accurate code for the specific job. It can
allow you to save memory and and CPU resources on the catalog server in some
cases.

\item [SpoolData=yes{\textbar}no]
\index[dir]{SpoolData}
\index[dir]{Directive!Spool Data}
tells Bareos to use or not to use spooling for the specific job.

\end{description}

{\bf Date-time-specification} determines when the  Job is to be run. The
specification is a repetition, and as  a default Bareos is set to run a job at
the beginning of the  hour of every hour of every day of every week of every
month  of every year. This is not normally what you want, so you  must specify
or limit when you want the job to run. Any  specification given is assumed to
be repetitive in nature and  will serve to override or limit the default
repetition. This  is done by specifying masks or times for the hour, day of the
month, day of the week, week of the month, week of the year,  and month when
you want the job to run. By specifying one or  more of the above, you can
define a schedule to repeat at  almost any frequency you want.

Basically, you must supply a {\bf month}, {\bf day}, {\bf hour}, and  {\bf
minute} the Job is to be run. Of these four items to be specified,  {\bf day}
is special in that you may either specify a day of the month  such as 1, 2,
... 31, or you may specify a day of the week such  as Monday, Tuesday, ...
Sunday. Finally, you may also specify a  week qualifier to restrict the
schedule to the first, second, third,  fourth, or fifth week of the month.

For example, if you specify only a day of the week, such as {\bf Tuesday}  the
Job will be run every hour of every Tuesday of every Month. That  is the {\bf
month} and {\bf hour} remain set to the defaults of  every month and all
hours.

Note, by default with no other specification, your job will run  at the
beginning of every hour. If you wish your job to run more than  once in any
given hour, you will need to specify multiple {\bf run}  specifications each
with a different minute.

The date/time to run the Job can be specified in the following way  in
pseudo-BNF:

\begin{longtable}{ l @{ ::= } p{0.5\textwidth} }
\bnfvar{week-keyword}    & 1st \pipe 2nd \pipe 3rd \pipe 4th \pipe 5th \pipe first \pipe
                     second \pipe third \pipe fourth \pipe fifth \pipe last \\
\bnfvar{wday-keyword}    & sun \pipe mon \pipe tue \pipe wed \pipe thu \pipe fri \pipe sat \pipe
                    sunday \pipe monday \pipe tuesday \pipe wednesday \pipe
                    thursday \pipe friday \pipe saturday \\
\bnfvar{week-of-year-keyword} & w00 \pipe w01 \pipe ... w52 \pipe w53 \\
\bnfvar{month-keyword}   & jan \pipe feb \pipe mar \pipe apr \pipe may \pipe jun \pipe jul \pipe
                    aug \pipe sep \pipe oct \pipe nov \pipe dec \pipe
                    january \pipe february \pipe ... \pipe december \\
\bnfvar{digit}           & 1 \pipe 2 \pipe 3 \pipe 4 \pipe 5 \pipe 6 \pipe 7 \pipe 8 \pipe 9 \pipe 0 \\
\bnfvar{number}          & \bnfvar{digit} \pipe \bnfvar{digit}\bnfvar{number} \\
\bnfvar{12hour}          & 0 \pipe 1 \pipe 2 \pipe ... 12 \\
\bnfvar{hour}            & 0 \pipe 1 \pipe 2 \pipe ... 23 \\
\bnfvar{minute}          & 0 \pipe 1 \pipe 2 \pipe ... 59 \\
\bnfvar{day}             & 1 \pipe 2 \pipe ... 31 \\
\bnfvar{time}            & \bnfvar{hour}:\bnfvar{minute} \pipe
                    \bnfvar{12hour}:\bnfvar{minute}am \pipe
                    \bnfvar{12hour}:\bnfvar{minute}pm \\
\bnfvar{time-spec}       & at \bnfvar{time} \pipe hourly \\
% ??? \bnfvar{date-keyword}    & on \pipe weekly \\
\bnfvar{day-range}       & \bnfvar{day}-\bnfvar{day} \\
\bnfvar{month-range}     & \bnfvar{month-keyword}-\bnfvar{month-keyword} \\
\bnfvar{wday-range}      & \bnfvar{wday-keyword}-\bnfvar{wday-keyword} \\
\bnfvar{range}           & \bnfvar{day-range} \pipe \bnfvar{month-range} \pipe
                          \bnfvar{wday-range} \\
\bnfvar{modulo}          & \bnfvar{day}/\bnfvar{day} \pipe \bnfvar{week-of-year-keyword}/\bnfvar{week-of-year-keyword} \\
\bnfvar{date}            & \bnfvar{date-keyword} \pipe \bnfvar{day} \pipe \bnfvar{range} \\
\bnfvar{date-spec}       & \bnfvar{date} \pipe \bnfvar{date-spec} \\
\bnfvar{day-spec}        & \bnfvar{day} \pipe \bnfvar{wday-keyword} \pipe
                    \bnfvar{day} \pipe \bnfvar{wday-range} \pipe
                    \bnfvar{week-keyword} \bnfvar{wday-keyword} \pipe
                    \bnfvar{week-keyword} \bnfvar{wday-range} \pipe
                    daily \\
\bnfvar{month-spec}      & \bnfvar{month-keyword} \pipe \bnfvar{month-range} \pipe monthly \\
\bnfvar{date-time-spec}  & \bnfvar{month-spec} \bnfvar{day-spec} \bnfvar{time-spec} \\
\end{longtable}
}
