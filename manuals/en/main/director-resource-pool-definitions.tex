\defDirective{Dir}{Pool}{Action On Purge}{}{}{%
This directive \configline{Action On Purge=Truncate} instructs Bareos to truncate the
volume when it is purged with the \configline{Purge Volume Action=Truncate}
command. It is useful to prevent disk based volumes from consuming too much
space.

\bconfigInput{config/DirPoolActionOnPurge1.conf}


You can schedule the truncate operation at the end of your CatalogBackup job
like in this example:

\bconfigInput{config/DirPoolActionOnPurge2.conf}
}

\defDirective{Dir}{Pool}{Auto Prune}{}{}{%
If AutoPrune is set to {\bf yes}, Bareos  will automatically
apply the Volume Retention period when new
Volume is needed and no appendable Volumes exist in the Pool.  Volume
pruning causes expired Jobs (older than the {\bf Volume Retention}
period) to be deleted from the Catalog and permits possible recycling of
the Volume.
}

\defDirective{Dir}{Pool}{Catalog}{}{}{%
This specifies the name of the catalog resource to be used for this Pool.
When a catalog is defined in a Pool it will override the definition in
the client (and the Catalog definition in a Job since
\sinceVersion{dir}{Job catalog overwriten by Pool catalog}{13.4.0}). e.g.
this catalog setting takes precedence over any other definition.
}

\defDirective{Dir}{Pool}{Catalog Files}{}{}{%
This directive defines whether or not you want the names of the files
that were saved to be put into the catalog.
If disabled, the Catalog database will be significantly smaller.
The disadvantage is that
you will not be able to produce a Catalog listing of the files backed up
for each Job (this is often called Browsing).  Also, without the File
entries in the catalog, you will not be able to use the Console \bcommand{restore}{} command
nor any other command that references File entries.
}

\defDirective{Dir}{Pool}{Cleaning Prefix}{}{}{%
This directive defines a prefix string, which if it matches the
beginning of a Volume name during labeling of a Volume, the Volume will
be defined with the VolStatus set to {\bf Cleaning} and thus Bareos will
never attempt to use this tape.  This is primarily for use with
autochangers that accept barcodes where the convention is that barcodes
beginning with {\bf CLN} are treated as cleaning tapes.

The default value for this directive is consequently set to {\bf CLN}, so
that in most cases the cleaning tapes are automatically recognized without
configuration.
If you use another prefix for your cleaning tapes, you can set this directive
accordingly.
}

\defDirective{Dir}{Pool}{Description}{}{}{%
}

\defDirective{Dir}{Pool}{File Retention}{}{}{%
The File Retention directive defines the length of time that  Bareos will
keep File records in the Catalog database after the End time of the
Job corresponding to the File records.

This directive takes precedence over Client directives of the same name. For
example, you can decide to increase Retention times for Archive or OffSite
Pool.

Note, this affects only records in the catalog database. It does not affect
your archive backups.

For more information see Client documentation about
\linkResourceDirective{Dir}{Client}{File Retention}
}

\defDirective{Dir}{Pool}{Job Retention}{}{}{%
The Job Retention directive defines the length of time that Bareos will keep
Job records in the Catalog database after the Job End time.  As with the
other retention periods, this affects only records in the catalog and not
data in your archive backup.

This directive takes precedence over Client directives of the same name.
For example, you can decide to increase Retention times for Archive or
OffSite Pool.

For more information see Client side documentation
\linkResourceDirective{Dir}{Client}{Job Retention}
}

\defDirective{Dir}{Pool}{Label Format}{}{}{%
This directive specifies the format of the labels contained in this
pool.  The format directive is used as a sort of template to create new
Volume names during automatic Volume labeling.

The format should be specified in double quotes (\path|"|), and consists of
letters, numbers and the special characters hyphen (\path|-|), underscore
(\path|_|), colon (\path|:|), and period (\path|.|), which are the legal
characters for a Volume name.

In addition, the format may contain a number of variable expansion
characters which will be expanded by a complex algorithm allowing you to
create Volume names of many different formats.  In all cases, the
expansion process must resolve to the set of characters noted above that
are legal Volume names.  Generally, these variable expansion characters
begin with a dollar sign (\path|$|)
or a left bracket (\path|[|).
For more details on
variable expansion, please see \nameref{sec:VariableExpansionVolumeLabels}.

If no variable expansion characters are found in the string, the Volume
name will be formed from the format string appended with the
a unique number that increases.  If you do not remove volumes from the
pool, this number should be the number of volumes plus one, but this
is not guaranteed. The unique number will be edited as four
digits with leading zeros.  For example, with a 
\configdirective{Label Format = "File-"},
the first volumes will be named \volume{File-0001}, \volume{File-0002}, ...

In almost all cases, you should enclose the format specification (part
after the equal sign) in double quotes (\path|"|).
}

\defDirective{Dir}{Pool}{Label Type}{ANSI{\textbar}IBM{\textbar}Bareos}{}{%
This directive is implemented in the
Director Pool resource and in the SD Device
resource (\linkResourceDirective{Sd}{Device}{Label Type}).
If it is specified in the SD Device resource, it will take
precedence over the value passed from the Director to the SD.
}

\defDirective{Dir}{Pool}{Maximum Block Size}{}{14.2.0}{%
The \configdirective{Maximum Block Size} can be defined here to define different block sizes per volume
or statically for all volumes at \linkResourceDirective{Sd}{Device}{Maximum Block Size}.
If not defined, its default is 63 KB.
Increasing this value could improve the throughput of writing to tapes.

\warning{However make sure to read the \ilink{Setting Block Sizes}{setblocksizes} chapter carefully before applying any changes.}
}

\defDirective{Dir}{Pool}{Maximum Volume Bytes}{}{}{%
This directive specifies the maximum number of bytes that can be written
to the Volume.  If you specify zero (the default), there is no limit
except the physical size of the Volume.  Otherwise, when the number of
bytes written to the Volume equals {\bf size} the Volume will be marked
{\bf Used}.  When the Volume is marked {\bf Used} it can no longer be
used for appending Jobs, much like the {\bf Full} status but it can be
recycled if recycling is enabled, and thus the Volume can be re-used
after recycling.  This value is checked and the {\bf Used} status set
while the job is writing to the particular volume.

This directive is particularly useful for restricting the size
of disk volumes, and will work correctly even in the case of
multiple simultaneous jobs writing to the volume.

The value defined by this directive in the bareos-dir.conf file is the
default value used when a Volume is created.  Once the volume is
created, changing the value in the bareos-dir.conf file will not change
what is stored for the Volume.  To change the value for an existing
Volume you must use the {\bf update} command in the Console.
}

\defDirective{Dir}{Pool}{Maximum Volume Files}{}{}{%
This directive specifies the maximum number of files that can be written
to the Volume.  If you specify zero (the default), there is no limit.
Otherwise, when the number of files written to the Volume equals {\bf
positive-integer} the Volume will be marked {\bf Used}.  When the Volume
is marked {\bf Used} it can no longer be used for appending Jobs, much
like the {\bf Full} status but it can be recycled if recycling is
enabled and thus used again.  This value is checked and the {\bf Used}
status is set only at the end of a job that writes to the particular
volume.

The value defined by this directive in the bareos-dir.conf file is the
default value used when a Volume is created.  Once the volume is
created, changing the value in the bareos-dir.conf file will not change
what is stored for the Volume.  To change the value for an existing
Volume you must use the {\bf update} command in the Console.
}

\defDirective{Dir}{Pool}{Maximum Volume Jobs}{}{}{%
This directive specifies the maximum number of Jobs that can be written
to the Volume.  If you specify zero (the default), there is no limit.
Otherwise, when the number of Jobs backed up to the Volume equals {\bf
positive-integer} the Volume will be marked {\bf Used}.  When the Volume
is marked {\bf Used} it can no longer be used for appending Jobs, much
like the {\bf Full} status but it can be recycled if recycling is
enabled, and thus used again.  By setting {\bf MaximumVolumeJobs} to
one, you get the same effect as setting {\bf UseVolumeOnce = yes}.

The value defined by this directive in the  bareos-dir.conf
file is the default value used when a Volume  is created. Once the volume is
created, changing the value  in the bareos-dir.conf file will not change what
is stored  for the Volume. To change the value for an existing Volume  you
must use the {\bf update} command in the Console.

If you are running multiple simultaneous jobs, this directive may not
work correctly because when a drive is reserved for a job, this
directive is not taken into account, so multiple jobs may try to
start writing to the Volume. At some point, when the Media record is
updated, multiple simultaneous jobs may fail since the Volume can no
longer be written.
}

\defDirective{Dir}{Pool}{Maximum Volumes}{}{}{%
This directive specifies the maximum number of volumes (tapes or files)
contained in the pool.  This directive is optional, if omitted or set to
zero, any number of volumes will be permitted.  In general, this
directive is useful for Autochangers where there is a fixed number of
Volumes, or for File storage where you wish to ensure that the backups
made to disk files do not become too numerous or consume too much space.
}

\defDirective{Dir}{Pool}{Migration High Bytes}{}{}{%
This directive specifies the number of bytes in the Pool which will
   trigger a migration if \linkResourceDirective{Dir}{Job}{Selection Type} = PoolOccupancy
   has been specified. The fact that the Pool
   usage goes above this level does not automatically trigger a migration
   job. However, if a migration job runs and has the PoolOccupancy selection
   type set, the Migration High Bytes will be applied.  Bareos does not
   currently restrict a pool to have only a single \linkResourceDirective{Dir}{Storage}{Media Type}, so you
   must keep in mind that if you mix Media Types in a Pool, the results
   may not be what you want, as the Pool count of all bytes will be
   for all Media Types combined.
}

\defDirective{Dir}{Pool}{Migration Low Bytes}{}{}{%
This directive specifies the number of bytes in the Pool which will
   stop a migration if \linkResourceDirective{Dir}{Job}{Selection Type} = PoolOccupancy
   has been specified and triggered by more than 
   \linkResourceDirective{Dir}{Pool}{Migration High Bytes}
   being in the pool. In other words, once a migration job
   is started with {\bf PoolOccupancy} migration selection and it
   determines that there are more than Migration High Bytes, the
   migration job will continue to run jobs until the number of
   bytes in the Pool drop to or below Migration Low Bytes.
}

\defDirective{Dir}{Pool}{Migration Time}{}{}{%
If \linkResourceDirective{Dir}{Job}{Selection Type} = PoolTime, 
the time specified here will be used. 
If the previous Backup Job or Jobs selected have been in the Pool longer than
the specified time, then they will be migrated.
}

\defDirective{Dir}{Pool}{Minimum Block Size}{}{}{%
The \configdirective{Minimum Block Size} can be defined here to define different block sizes per volume
or statically for all volumes at \linkResourceDirective{Sd}{Device}{Minimum Block Size}.
For details, see chapter \ilink{Setting Block Sizes}{setblocksizes}.
}

\defDirective{Dir}{Pool}{Name}{}{}{%
The name of the pool.
}

\defDirective{Dir}{Pool}{Next Pool}{}{}{%
This directive specifies the pool a Migration or Copy Job
and a Virtual Backup Job will write their data too.
This directive is required to define the Pool into which
   the data will be migrated. Without this directive, the migration job
   will terminate in error.
}

\defDirective{Dir}{Pool}{Pool Type}{}{}{%
This directive defines the pool type, which corresponds to the type of
Job being run.  It is required and may be one of the following:

\begin{description}
  \item [Backup]
  \item [*Archive]
  \item [*Cloned]
  \item [*Migration]
  \item [*Copy]
  \item [*Save]
\end{description}
Note, only Backup is currently implemented.
}

\defDirective{Dir}{Pool}{Purge Oldest Volume}{}{}{%
This directive instructs the Director to search for the oldest used
Volume in the Pool when another Volume is requested by the Storage
daemon and none are available.  The catalog is then {\bf purged}
irrespective of retention periods of all Files and Jobs written to this
Volume.  The Volume is then recycled and will be used as the next Volume
to be written.  This directive overrides any Job, File, or Volume
retention periods that you may have specified.

This directive can be useful if you have a fixed number of Volumes in
the Pool and you want to cycle through them and reusing the oldest one
when all Volumes are full, but you don't want to worry about setting
proper retention periods.  However, by using this option you risk losing
valuable data.

\warning{
Be aware that \configdirective{Purge Oldest Volume} disregards all retention
periods. If you have only a single Volume defined and you turn this
variable on, that Volume will always be immediately overwritten when it
fills!  So at a minimum, ensure that you have a decent number of Volumes
in your Pool before running any jobs.  If you want retention periods to
apply do not use this directive.\\
We \textbf{highly} recommend against using this directive, because it is
sure that some day, Bareos will purge a Volume that contains current
data.
}
}

\defDirective{Dir}{Pool}{Recycle}{}{}{%
This directive specifies whether or not Purged Volumes may be recycled.
If it is set to {\bf yes} (default) and Bareos needs a volume but finds
none that are appendable, it will search for and recycle (reuse) Purged
Volumes (i.e.  volumes with all the Jobs and Files expired and thus
deleted from the Catalog).  If the Volume is recycled, all previous data
written to that Volume will be overwritten. If Recycle is set to {\bf
no}, the Volume will not be recycled, and hence, the data will remain
valid.  If you want to reuse (re-write) the Volume, and the recycle flag
is no (0 in the catalog), you may manually set the recycle flag (update
command) for a Volume to be reused.

Please note that the value defined by this directive in the
bareos-dir.conf file is the default value used when a Volume is created.
Once the volume is created, changing the value in the bareos-dir.conf
file will not change what is stored for the Volume.  To change the value
for an existing Volume you must use the {\bf update} command in the
Console.

When all Job and File records have been pruned or purged from the
catalog for a particular Volume, if that Volume is marked as
Append, Full, Used, or Error, it will then be marked as Purged. Only
Volumes marked as Purged will be considered to be converted to the
Recycled state if the {\bf Recycle} directive is set to {\bf yes}.
}

\defDirective{Dir}{Pool}{Recycle Current Volume}{}{}{%
If Bareos needs a new Volume, this directive instructs Bareos to Prune
the volume respecting the Job and File retention periods.  If all Jobs
are pruned (i.e.  the volume is Purged), then the Volume is recycled and
will be used as the next Volume to be written.  This directive respects
any Job, File, or Volume retention periods that you may have specified,
and thus it is {\bf much} better to use it rather than the Purge Oldest
Volume directive.

This directive can be useful if you have: a fixed number of Volumes in
the Pool, you want to cycle through them, and you have specified
retention periods that prune Volumes before you have cycled through the
Volume in the Pool.

However, if you use this directive and have only one Volume in the Pool,
you will immediately recycle your Volume if you fill it and Bareos needs
another one.  Thus your backup will be totally invalid.  Please use this
directive with care.
}

\defDirective{Dir}{Pool}{Recycle Oldest Volume}{}{}{%
This directive instructs the Director to search for the oldest used
Volume in the Pool when another Volume is requested by the Storage
daemon and none are available.  The catalog is then {\bf pruned}
respecting the retention periods of all Files and Jobs written to this
Volume.  If all Jobs are pruned (i.e. the volume is Purged), then the
Volume is recycled and will be used as the next Volume to be written.
This directive respects any Job, File, or Volume retention periods that
you may have specified, and as such it is {\bf much} better to use this
directive than the Purge Oldest Volume.

This directive can be useful if you have a fixed number of Volumes in the
Pool and you want to cycle through them and you have specified the correct
retention periods.

However, if you use this directive and have only one
Volume in the Pool, you will immediately recycle your Volume if you fill
it and Bareos needs another one. Thus your backup will be totally invalid.
Please use this directive with care.
}

\defDirective{Dir}{Pool}{Recycle Pool}{}{}{%
This directive defines to which pool
the Volume will be placed (moved) when it is recycled. Without
this directive, a Volume will remain in the same pool when it is
recycled. With this directive, it can be moved automatically to any
existing pool during a recycle. This directive is probably most
useful when defined in the Scratch pool, so that volumes will
be recycled back into the Scratch pool. For more on the see the
\ilink{Scratch Pool}{TheScratchPool} section of this manual.

Although this directive is called RecyclePool, the Volume in
question is actually moved from its current pool to the one
you specify on this directive when Bareos prunes the Volume and
discovers that there are no records left in the catalog and hence
marks it as {\bf Purged}.
}

\defDirective{Dir}{Pool}{Scratch Pool}{}{}{%
This directive permits to specify a dedicate \textsl{Scratch} for the
current pool. This pool will replace the special pool named \textsl{Scrach}
for volume selection. For more information about \textsl{Scratch} see
\ilink{Scratch Pool}{TheScratchPool} section of this manual. This is useful
when using multiple storage sharing the same mediatype or when you want to
dedicate volumes to a particular set of pool.
}

\defDirective{Dir}{Pool}{Storage}{}{}{%
The Storage directive defines the name of the storage services where you
want to backup the FileSet data.  For additional details, see the
\nameref{DirectorResourceStorage} of this manual.
The Storage resource may also be specified in the Job resource,
but the value, if any, in the Pool resource overrides any value
in the Job. This Storage resource definition is not required by either
the Job resource or in the Pool, but it must be specified in
one or the other.  If not configuration error will result.
  We highly recommend
   that you define the Storage resource to be used in the Pool rather
   than elsewhere (job, schedule run, ...). Be aware that you theoretically can
give a list of storages here but only the first item from the list is actually
used for backup and restore jobs.
}

\defDirective{Dir}{Pool}{Use Catalog}{}{}{%
Store information into Catalog. In all pratical use cases, leave this value to its defaults.

}

\defDirective{Dir}{Pool}{Use Volume Once}{}{}{%
Use \linkResourceDirective{Dir}{Pool}{Maximum Volume Jobs} = 1 instead.
}

\defDirective{Dir}{Pool}{Volume Retention}{}{}{%
The Volume Retention directive defines the length of time that
Bareos will keep records associated with the Volume in
the Catalog database after the End time of each Job written to the
Volume.  When this time period expires, and if {\bf AutoPrune} is set to
{\bf yes} Bareos may prune (remove) Job records that are older than the
specified Volume Retention period if it is necessary to free up a
Volume.  Recycling will not occur until it is absolutely necessary to
free up a volume (i.e. no other writable volume exists).
All File records associated with pruned Jobs are also
pruned.  The time may be specified as seconds, minutes, hours, days,
weeks, months, quarters, or years.  The {\bf Volume Retention} is
applied independently of the {\bf Job Retention} and the {\bf File
Retention} periods defined in the Client resource.  This means that all
the retentions periods are applied in turn and that the shorter period
is the one that effectively takes precedence.  Note, that when the {\bf
Volume Retention} period has been reached, and it is necessary to obtain
a new volume, Bareos will prune both the Job and the File records.  This
pruning could also occur during a {\bf status dir} command because it
uses similar algorithms for finding the next available Volume.

It is important to know that when the Volume Retention period expires,
Bareos does not automatically recycle a Volume. It attempts to keep the
Volume data intact as long as possible before over writing the Volume.

By defining multiple Pools with different Volume Retention periods, you
may effectively have a set of tapes that is recycled weekly, another
Pool of tapes that is recycled monthly and so on.  However, one must
keep in mind that if your {\bf Volume Retention} period is too short, it
may prune the last valid Full backup, and hence until the next Full
backup is done, you will not have a complete backup of your system, and
in addition, the next Incremental or Differential backup will be
promoted to a Full backup.  As a consequence, the minimum {\bf Volume
Retention} period should be at twice the interval of your Full backups.
This means that if you do a Full backup once a month, the minimum Volume
retention period should be two months.

The default Volume retention period is 365 days, and either the default
or the value defined by this directive in the bareos-dir.conf file is
the default value used when a Volume is created.  Once the volume is
created, changing the value in the \file{bareos-dir.conf} file will not change
what is stored for the Volume.  To change the value for an existing
Volume you must use the {\bf update} command in the Console.
}

\defDirective{Dir}{Pool}{Volume Use Duration}{}{}{%
The Volume Use Duration directive defines the time period that the
Volume can be written beginning from the time of first data write to the
Volume.  If the time-period specified is zero (the default), the Volume
can be written indefinitely.  Otherwise, the next time a job
runs that wants to access this Volume, and the time period from the
first write to the volume (the first Job written) exceeds the
time-period-specification, the Volume will be marked {\bf Used}, which
means that no more Jobs can be appended to the Volume, but it may be
recycled if recycling is enabled.
% Using the command {\bf
% status dir} applies algorithms similar to running jobs, so
% during such a command, the Volume status may also be changed.
Once the Volume is
recycled, it will be available for use again.

You might use this directive, for example, if you have a Volume used for
Incremental backups, and Volumes used for Weekly Full backups.  Once the
Full backup is done, you will want to use a different Incremental
Volume.  This can be accomplished by setting the Volume Use Duration for
the Incremental Volume to six days.  I.e.  it will be used for the 6
days following a Full save, then a different Incremental volume will be
used.  Be careful about setting the duration to short periods such as 23
hours, or you might experience problems of Bareos waiting for a tape
over the weekend only to complete the backups Monday morning when an
operator mounts a new tape.

% The use duration is checked and the {\bf Used} status is set only at the
% end of a job that writes to the particular volume, which means that even
% though the use duration may have expired, the catalog entry will not be
% updated until the next job that uses this volume is run. This
% directive is not intended to be used to limit volume sizes
% and will not work correctly (i.e. will fail jobs) if the use
% duration expires while multiple simultaneous jobs are writing
% to the volume.

Please note that the value defined by this directive in the  bareos-dir.conf
file is the default value used when a Volume  is created. Once the volume is
created, changing the value  in the bareos-dir.conf file will not change what
is stored  for the Volume. To change the value for an existing Volume  you
must use the
\ilink{\bf update volume}{UpdateCommand} command in the Console.
}
