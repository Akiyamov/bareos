\defDirective{Dir}{Client}{Address}{}{}{%
Where the address is a host name, a fully qualified domain name, or a
network address in dotted quad notation for a Bareos File server daemon.
This directive is required.
}

\defDirective{Dir}{Client}{Auth Type}{}{}{%
Specifies the authentication type that must be supplied when connecting to
a backup protocol that uses a specific authentication type.
}

\defDirective{Dir}{Client}{Auto Prune}{}{}{%
If AutoPrune is set to  {\bf yes}, Bareos
will  automatically apply the File retention period and the Job  retention
period for the Client at the end of the Job.  If you leave the default {\bf AutoPrune = no},
pruning will not be done, and your Catalog will grow in size each time you
run a Job.  Pruning affects only information in the catalog and not data
stored in the backup archives (on Volumes), but if pruning deletes all data
referring to a certain volume, the volume is regarded as empty and will possibly
be overwritten before the volume retention has expired.
}

\defDirective{Dir}{Client}{Catalog}{}{}{%
This specifies the  name of the catalog resource to be used for this Client.
If none is specified the first defined catalog is used.
}

\defDirective{Dir}{Client}{Connection From Client To Director}{}{}{%
For details, see \nameref{sec:ClientInitiatedConnection}.
}

\defDirective{Dir}{Client}{Connection From Director To Client}{}{}{%
}

\defDirective{Dir}{Client}{Description}{}{}{%
}

\defDirective{Dir}{Client}{Enabled}{}{}{%
}

\defDirective{Dir}{Client}{FD Address}{}{}{%
}

\defDirective{Dir}{Client}{FD Password}{}{}{%
}

\defDirective{Dir}{Client}{FD Port}{}{}{%
Where the port is a port  number at which the Bareos File server daemon can
be contacted.  The default is 9102. For NDMP backups set this to 10000.
}

\defDirective{Dir}{Client}{File Retention}{}{}{%
The File Retention directive defines the length of time that  Bareos will
keep File records in the Catalog database after the End time of the
Job corresponding to the File records.
When this time period expires, and if
{\bf AutoPrune} is set to  {\bf yes} Bareos will prune (remove) File records
that  are older than the specified File Retention period. Note, this  affects
only records in the catalog database. It does not  affect your archive
backups.

File records  may actually be retained for a shorter period than you specify
on  this directive if you specify either a shorter \linkResourceDirective{Dir}{Client}{Job Retention}  or a
shorter \linkResourceDirective{Dir}{Pool}{Volume Retention} period. The shortest  retention period of the
three takes precedence.  The time may be expressed in seconds, minutes,
hours, days, weeks, months, quarters, or years. See the
\ilink{ Configuration chapter}{Time} of this  manual for
additional details of time specification.

The  default is 60 days.
}

\defDirective{Dir}{Client}{Hard Quota}{}{}{%
The amount of data determined by the Hard Quota directive sets the hard limit of backup space that cannot be exceeded. This is the maximum amount this client can back up before any backup job will be aborted.

If the Hard Quota is exceeded, the running job is terminated:

\bconfigInput{config/DirClientHardQuota1.conf}

}

\defDirective{Dir}{Client}{Heartbeat Interval}{}{}{%
This directive is optional and if specified will cause the Director to
set a keepalive interval (heartbeat) in seconds on each of the sockets
it opens for the Storage resource.  This value will override any
specified at the Director level.  It is implemented only on systems
(Linux, ...) that provide the {\bf setsockopt} TCP\_KEEPIDLE function.
The default value is zero, which means no change is made to the socket.
}

\defDirective{Dir}{Client}{Job Retention}{}{}{%
The Job Retention directive defines the length of time that  Bareos will keep
Job records in the Catalog database after the Job End time.  When
this time period expires, and if \linkResourceDirective{Dir}{Client}{Auto Prune} is set to {\bf yes}
Bareos will prune (remove) Job records that are older than the specified
File Retention period.  As with the other retention periods, this
affects only records in the catalog and not data in your archive backup.

If a Job record is selected for pruning, all associated File and JobMedia
records will also be pruned regardless of the File Retention period set.
As a consequence, you normally will set the File retention period to be
less than the Job retention period.  The Job retention period can actually
be less than the value you specify here if you set the \linkResourceDirective{Dir}{Pool}{Volume
Retention} directive to a smaller duration.  This is
because the Job retention period and the Volume retention period are
independently applied, so the smaller of the two takes precedence.

The Job retention period is specified as seconds,  minutes, hours, days,
weeks, months,  quarters, or years.  See the
\ilink{Configuration chapter}{Time} of this manual for
additional details of  time specification.

The default is 180 days.
}

\defDirective{Dir}{Client}{Maximum Bandwidth Per Job}{}{}{%
The speed parameter specifies the maximum allowed bandwidth that a job may use
when started for this Client. The speed parameter should be specified in
k/s, Kb/s, m/s or Mb/s.
}

\defDirective{Dir}{Client}{Maximum Concurrent Jobs}{}{}{%
This directive specifies the maximum number of Jobs with the current Client
that  can run concurrently. Note, this directive limits only Jobs  for Clients
with the same name as the resource in which it appears. Any  other
restrictions on the maximum concurrent jobs such as in  the Director, Job or
Storage resources will also apply in addition to  any limit specified here.
}

\defDirective{Dir}{Client}{Name}{}{}{%
The client name which will be used in the  Job resource directive or in the
console run command.
}

\defDirective{Dir}{Client}{NDMP Block Size}{}{}{%
This directive sets the default NDMP blocksize for this client.
}

\defDirective{Dir}{Client}{NDMP Log Level}{}{}{%
This directive sets the loglevel for the NDMP protocol library.
}

\defDirective{Dir}{Client}{Passive}{}{13.2.0}{%
The normal way of initializing the data channel (the channel where the backup data itself is transported)
is done by the file daemon (client) that connects to the storage daemon.

By using the client passive mode, the initialization of the datachannel is reversed, so that the storage daemon connects to the filedaemon.

See chapter \ilink{Passive Client}{PassiveClient}.
}

\defDirective{Dir}{Client}{Password}{}{}{%
This is the password to be  used when establishing a connection with the File
services, so  the Client configuration file on the machine to be backed up
must  have the same password defined for this Director.
If you have either \file{/dev/random} or {\bf bc} on your machine,
Bareos will generate a random  password during the configuration process,
otherwise it will  be left blank.

The password is plain text.  It is not generated through any special
process, but it is preferable for security reasons to make the text
random.
}

\defDirective{Dir}{Client}{Port}{}{}{%
}

\defDirective{Dir}{Client}{Protocol}{Native{\textbar}NDMP}{13.2.0}{%
The backup protocol to use to run the Job.

Currently the director understands the following protocols:
\begin{enumerate}
\item Native - The native Bareos protocol
\item NDMP - The NDMP protocol
\end{enumerate}
}

\defDirective{Dir}{Client}{Quota Include Failed Jobs}{}{}{%
When calculating the amount a client used take into consideration any failed Jobs.
}

\defDirective{Dir}{Client}{Soft Quota}{}{}{%
This is the amount after which there will be a warning issued that a client
is over his softquota. A client can keep doing backups until it hits the
hard quota or when the \linkResourceDirective{Dir}{Client}{Soft Quota Grace Period} is expired.
}

\defDirective{Dir}{Client}{Soft Quota Grace Period}{}{}{%
Time allowed for a client to be over its \linkResourceDirective{Dir}{Client}{Soft Quota} before it will be enforced.

When the amount of data backed up by the client outruns the value specified by the Soft Quota directive, the next start of a backup job will start the soft quota grace time period. This is written to the job log:

\bconfigInput{config/DirClientSoftQuotaGracePeriod1.conf}

In the Job Overview, the value of Grace Expiry Date: will then change from \parameter{Soft Quota was never exceeded} to the date when the grace time expires, e.g. \parameter{11-Dec-2012 04:09:05}.

During that period, it is possible to do backups even if the total amount of stored data exceeds the limit specified by soft quota.

If in this state, the job log will write:

\bconfigInput{config/DirClientSoftQuotaGracePeriod2.conf}

After the grace time expires, in the next backup job of the client, the value for Burst Quota will be set to the value that the client has stored at this point in time. Also, the job will be terminated. The following information in the job log shows what happened:

\bconfigInput{config/DirClientSoftQuotaGracePeriod3.conf}

At this point, it is not possible to do any backup of the client. To be able to do more backups, the amount of stored data for this client has to fall under the burst quota value.
}

\defDirective{Dir}{Client}{Strict Quotas}{}{}{%
The directive Strict Quotas determines whether, after the Grace Time Period is over,
to enforce the Burst Limit (Strict Quotas = {\bf No}) or
the Soft Limit (Strict Quotas = {\bf Yes}).

The Job Log shows either

\bconfigInput{config/DirClientStrictQuotas1.conf}

or

\bconfigInput{config/DirClientStrictQuotas2.conf}

}

\defDirective{Dir}{Client}{TLS Allowed CN}{}{}{%
}

\defDirective{Dir}{Client}{TLS Authenticate}{}{}{%
}

\defDirective{Dir}{Client}{TLS CA Certificate Dir}{}{}{%
}

\defDirective{Dir}{Client}{TLS CA Certificate File}{}{}{%
}

\defDirective{Dir}{Client}{TLS Certificate}{}{}{%
}

\defDirective{Dir}{Client}{TLS Certificate Revocation List}{}{}{%
}

\defDirective{Dir}{Client}{TLS Enable}{}{}{%
Bareos can be configured to encrypt all its network traffic.
See chapter \nameref{TlsDirectives} to see,
how the Bareos Director (and the other components) must be configured to use TLS.
}

\defDirective{Dir}{Client}{TLS Key}{}{}{%
}

\defDirective{Dir}{Client}{TLS Require}{}{}{%
}

\defDirective{Dir}{Client}{Username}{}{}{%
Specifies the username that must be supplied when authenticating.
Only used for the non Native protocols at the moment.

}
