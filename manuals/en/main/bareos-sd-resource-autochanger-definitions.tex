\defDirective{Sd}{Autochanger}{Changer Command}{}{}{%
This command specifies an external program and parameter to be called that will
automatically change volumes as required by Bareos.
This command is
invoked each time that Bareos wishes to manipulate the autochanger.

Most frequently,
you will specify the Bareos supplied \command{mtx-changer} script.

The
following substitutions are made in the command before it is sent to
the operating system for execution:

\begin{description}
\item[\%\%] \%
\item[\%a]  archive device name
\item[\%c]  changer device name
\item[\%d]  changer drive index base 0
\item[\%f]  Client's name
\item[\%j]  Job name
\item[\%o]  command  (loaded, load, or unload)
\item[\%s]  Slot base 0
\item[\%S]  Slot base 1
\item[\%v]  Volume name
\end{description}

A typical setting for this is \linkResourceDirectiveValue{Sd}{Autochanger}{Changer Command}{"/usr/lib/bareos/scripts/mtx-changer \%c \%o \%S \%a \%d"}.

Details of the three
commands currently used by Bareos  (loaded, load, unload) as well as the
output expected by  Bareos are given in the \ilink{Bareos Autochanger Interface}{autochanger-interface} section.

If it is specified here, it needs not to be specified in the Device
resource. If it is also specified in the Device resource, it will take
precedence over the one specified in the Autochanger resource.
}



\defDirective{Sd}{Autochanger}{Changer Device}{}{}{%
The specified device must be the generic SCSI device of the autochanger.

The changer device is additional to the the \linkResourceDirective{Sd}{Device}{Archive Device}.
This is because most autochangers are  controlled through a
different device than is used for reading and  writing the tapes.
For
example, on Linux, one normally uses the generic SCSI interface for
controlling the autochanger, but the standard SCSI interface for reading and
writing the  tapes.

On Linux, for the \linkResourceDirectiveValue{Sd}{Device}{Archive Device}{/dev/nst0},  you
would typically have \linkResourceDirectiveValue{Sd}{Device}{Changer Device}{/dev/sg0}.

On FreeBSD systems, the changer device will typically be on \file{/dev/pass0}
through \file{/dev/passN}.

On Solaris, the changer device will typically be some file under \file{/dev/rdsk}.

Please ensure that your Storage daemon has permission to access this
device.

It can be overwritten per device using the \linkResourceDirective{Sd}{Device}{Changer Device} directive.
}

\defDirective{Sd}{Autochanger}{Description}{}{}{%
}

\defDirective{Sd}{Autochanger}{Device}{}{}{%
Specifies the names of the Device resource or resources that correspond
to the autochanger drive.  If you have a multiple drive autochanger, you
must specify multiple Device names, each one referring to a separate
Device resource that contains a Drive Index specification that
corresponds to the drive number base zero.  You may specify multiple
device names on a single line separated by commas, and/or you may
specify multiple Device directives.
}

\defDirective{Sd}{Autochanger}{Name}{}{}{%
Specifies the Name of the Autochanger.  This name is used in the
Director's Storage definition to refer to the autochanger.
}

