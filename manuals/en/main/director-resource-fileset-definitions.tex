\defDirective{Dir}{FileSet}{Description}{}{}{%
Information only.
}

\defDirective{Dir}{FileSet}{Enable VSS}{}{}{%
\index[dir]{Windows!Enable VSS}%
If this directive is set to \parameter{yes} the File daemon will be notified
that the user wants to use a Volume Shadow Copy Service (VSS) backup
for this job. This directive is effective only on the Windows File Daemon.
It permits a consistent copy
of open files to be made for cooperating writer applications, and for
applications that are not VSS away, Bareos can at least copy open files.
The Volume Shadow Copy will only be done on Windows drives where the
drive (e.g. C:, D:, ...) is explicitly mentioned in a \configdirective{File}
directive.
For more information, please see the
\ilink{Windows}{VSS} chapter of this manual.
}

\defDirective{Dir}{FileSet}{Exclude}{}{}{%
Describe the files, that should get excluded from a backup, see section about the \nameref{fileset-exclude}.
}

\defDirective{Dir}{FileSet}{Ignore File Set Changes}{}{}{%
Normally, if you modify the FileSet Include or Exclude lists,
the next backup will be forced to a Full so that Bareos can
guarantee that any additions or deletions are properly saved.

We strongly recommend against setting this directive to yes,
since doing so may cause you to have an incomplete set of backups.

If this directive is set to {\bf yes}, any changes you make to the
FileSet Include or Exclude lists, will not force a Full during
subsequent backups.

The default is {\bf no}, in which case, if you change the Include or
Exclude, Bareos will force a Full backup to ensure that everything is
properly backed up.
}

\defDirective{Dir}{FileSet}{Include}{}{}{%
Describe the files, that should get included to a backup, see section about the \nameref{fileset-include}.
}

\defDirective{Dir}{FileSet}{Name}{}{}{%
The name of the FileSet resource.
}
