%%
%%

\chapter{Installing Bareos Webui}
\label{sec:webui}
\label{sec:install-webui}
\index[general]{Webui}
\index[general]{Webui!Install}
\index[general]{bareos-webui}

This chapter addresses the installation process of the Bareos Webui.

Since \sinceVersion{dir}{bareos-webui}{15.2.0} bareos-webui is part of the Bareos project and available for a number of platforms.


\section{System Requirements}

\begin{itemize}
\item A working Bareos environment, Bareos $>$= 15.2
\item A Bareos platform, where bareos-webui packages are provided.
\item An Apache 2.x Webserver with mod-rewrite, mod-php5 and mod-setenv
\item PHP $>$= 5.3.3
\item Zend Framework 2.2.x or later
  \textbf{Note:} Unfortunately, not all distributions offer a Zend Framework 2 package.
  The following list shows where to get the Zend Framework 2 package.
  \begin{itemize}
  \item RHEL, CentOS
    \begin{itemize}
    \item \url{https://fedoraproject.org/wiki/EPEL}
    \item \url{https://apps.fedoraproject.org/packages/php-ZendFramework2}
    \end{itemize}

  \item Fedora
    \begin{itemize}
    \item \url{https://apps.fedoraproject.org/packages/php-ZendFramework2}
    \end{itemize}

  \item SUSE, Debian, Ubuntu
    \begin{itemize}
    \item \url{http://download.bareos.org/bareos}
    \end{itemize}
  \end{itemize}
\end{itemize}

\section{Installation}

\subsection{Adding the Bareos Repository}

If not already done, add the Bareos repository that is matching your Linux distribution. Please have a look at the chapter \nameref{sec:InstallBareosPackages} for more information on how to achieve this.

\subsection{Install the bareos-webui package}

After adding the repository simply install the bareos-webui package via your package manager.

\begin{itemize}
 \item RHEL, CentOS and Fedora
\begin{commands}{}
yum install bareos-webui
\end{commands}
 or
\begin{commands}{}
dnf install bareos-webui
\end{commands}
\end{itemize}

\begin{itemize}
 \item SUSE Linux Enterprise Server (SLES), openSUSE
\begin{commands}{}
zypper install bareos-webui
\end{commands}
\end{itemize}

\begin{itemize}
 \item Debian, Ubuntu
\begin{commands}{}
apt-get install bareos-webui
\end{commands}
\end{itemize}


\subsection{Configuration of restricted consoles and profile resources}

You can have multiple consoles with different names and passwords, sort of like multiple users, each with different privileges. As a default, these consoles can do absolutely nothing – no commands whatsoever. You give them privileges or rather access to commands and resources by specifying access control lists (ACLs) in the director’s console resource. The ACLs are specified by a directive followed by a list of access names.

It is required to add at least one restricted named console in your director configuration (\file{bareos-dir.conf}) for bareos-webui. The restricted named consoles, configured in your \file{bareos-dir.conf}, are used for authentication and access control. The name and password directives of the restricted consoles are the credentials you have to provide during authentication to the webui as username and password.

% For full access and functionality relating the director connection, the following commands are currently needed by the webui and have to be made available via the \linkResourceDirective{Dir}{Profile}{Command ACL} in the profile the restricted consoles use.
% 
% \begin{itemize}
%  \item status
%  \item messages
%  \item show
%  \item version
%  \item run
%  \item rerun
%  \item cancel
%  \item use
%  \item restore
%  \item list, llist
%  \item .api
%  \item .bvfs\_update
%  \item .bvfs\_lsdirs
%  \item .bvfs\_lsfiles
%  \item .bvfs\_versions
%  \item .bvfs\_restore
%  \item .jobs
%  \item .clients
%  \item .filesets
% \end{itemize}

The bareos-webui package provides a default console and profile configuration under \directory{/etc/bareos/bareos-dir.d/},
which have to be included at the bottom of your \file{/etc/bareos/bareos-dir.conf} and edited to your needs.

\begin{commands}{add webui-consoles and webui-profiles to the Bareos Director configuration}
echo "@/etc/bareos/bareos-dir.d/webui-consoles.conf" >> /etc/bareos/bareos-dir.conf
echo "@/etc/bareos/bareos-dir.d/webui-profiles.conf" >> /etc/bareos/bareos-dir.conf
\end{commands}

\begin{bconfig}{webui-consoles.conf}
#
# Restricted console used by bareos-webui
#
Console {
  Name = user1
  Password = "CHANGEME"
  Profile = webui
}
\end{bconfig}

For more details about the console resource configuration, please have a look at the chapter \nameref{DirectorResourceConsole}.

\begin{bconfig}{webui-profiles.conf}
#
# bareos-webui default profile resource
#
Profile {
  Name = webui
  CommandACL = status, messages, show, version, run, rerun, cancel, .api, .bvfs_*, list, llist, use, restore, .jobs, .filesets, .clients
  Job ACL = *all*
  Schedule ACL = *all*
  Catalog ACL = *all*
  Pool ACL = *all*
  Storage ACL = *all*
  Client ACL = *all*
  FileSet ACL = *all*
  Where ACL = *all*
}
\end{bconfig}

For more details about profile resource configuration in bareos, please have a look at the chapter \nameref{DirectorResourceProfile}.

\warning{Do not forget to reload your new director configuration.}

\subsection{Configure your Apache Webserver}

\index[general]{Apache!bareos-webui}

If you have installed from package, a default configuration is provided, please see \file{/etc/apache2/conf.d/bareos-webui.conf}, \file{/etc/httpd/conf.d/bareos-webui.conf} or \file{/etc/apache2/available-conf/bareos-webui.conf}.

The required Apache modules, \argument{setenv}, \argument{rewrite} and \argument{php} are enabled via package postinstall script. You simply need to restart your apache webserver manually.

\subsection{Configure your /etc/bareos-webui/directors.ini}

Configure your directors in \file{/etc/bareos-webui/directors.ini} to match your settings, which you have chosen in the previous steps.

The configuration file \file{/etc/bareos-webui/directors.ini} should look similar to this:

\begin{bconfig}{Bareos-webui directors.ini}
;
; Bareos WebUI Configuration
; File: /etc/bareos-webui/directors.ini
;

;
; Section localhost-dir
;
[localhost-dir]

; Enable or disable section. Possible values are "yes" or "no", the default is "yes".
enabled = "yes"

; Fill in the IP-Address or FQDN of you director.
diraddress = "localhost"

; Default value is 9101
dirport = 9101

; Note: TLS has not been tested and documented, yet.
;tls_verify_peer = false
;server_can_do_tls = false
;server_requires_tls = false
;client_can_do_tls = false
;client_requires_tls = false
;ca_file = ""
;cert_file = ""
;cert_file_passphrase = ""
;allowed_cns = ""

;
; Section remote-dir
;
[remote-dir]
enabled = "no"
diraddress = "hostname"
dirport = 9101
; Note: TLS has not been tested and documented, yet.
;tls_verify_peer = false
;server_can_do_tls = false
;server_requires_tls = false
;client_can_do_tls = false
;client_requires_tls = false
;ca_file = ""
;cert_file = ""
;cert_file_passphrase = ""
;allowed_cns = ""
\end{bconfig}

You can add as many directors as you want.

Now you are able to login by calling http://HOSTNAME/bareos-webui in your browser of choice. Your login credentials are defined in your Bareos Director Console configuration.

\section{Additional information}

\subsection{SELinux}
\index[general]{SELinux!bareos-webui}

To install bareos-webui on a system with SELinux enabled, the following additional steps have to be performed.
\begin{itemize}
 \item Allow HTTPD scripts and modules to connect to the network
\begin{commands}{}
setsebool -P httpd_can_network_connect on
\end{commands}
\end{itemize}

\subsection{NGINX}
\index[general]{nginx!bareos-webui}

If you prefer to use bareos-webui on Nginx with php5-fpm instead of Apache,
a basic working configuration could look like this:

\begin{bconfig}{bareos-webui on nginx}
server {

        listen       9100;
        server_name  bareos;
        root         /var/www/bareos-webui/public;

        location / {
                index index.php;
                try_files $uri $uri/ /index.php?$query_string;
        }

        location ~ .php$ {

                include snippets/fastcgi-php.conf;

                # With php5-cgi alone pass the PHP
                # scripts to FastCGI server
                # listening on 127.0.0.1:9000

                # fastcgi_pass 127.0.0.1:9000;

                # With php5-fpm:

                fastcgi_pass unix:/var/run/php5-fpm.sock;

        }

}
\end{bconfig}


\subsection{Installation from source}

For information about installing from source, please refer to \url{https://github.com/bareos/bareos-webui/blob/master/doc/INSTALL.md}.

% In this example we assume a running Apache2 Webserver with PHP5 and mod\_rewrite is already installed and configured properly.
%
% \subsection*{Step 1 - Get the source}
%
% \begin{commands}{}
% cd /srv/www/htdocs
%
% git clone https://github.com/bareos/bareos-webui.git
% \end{commands}
%
% or
%
% \begin{commands}{}
% cd /srv/www/htdocs
%
% wget https://github.com/bareos/bareos-webui/archive/Release/15.2.1.tar.gz
%
% tar xf 15.2.1.tar.gz
%
% \end{commands}
%
% \subsection*{Step 2 - Get Zend Framework 2}
%
% If not installed in other ways we get Zend Framework 2 via Composer. Read more about Composer here: \url{https://getcomposer.org/}
%
% \begin{commands}{}
% cd bareos-webui
%
% ./composer.phar install
% \end{commands}
%
% \subsection*{Step 3 - Configure Apache2 Webserver}
%
% You can for example copy the default configuration file and edit it to your needs.
%
% \begin{commands}{}
% cd /etc/apache2/conf.d
%
% cp /srv/www/htdocs/bareos-webui/install/apache/bareos-webui.conf .
%
% vim bareos-webui.conf
% \end{commands}
%
% In this example it looks like the following.
%
% \begin{bconfig}{Webui Apache configuration file}
% #
% # Bareos WebUI Apache configuration file
% #
%
% # Environment Variable for Application Debugging
% # Set to "development" to turn on debugging mode or
% # "production" to turn off debugging mode.
% <IfModule env_module>
%         SetEnv "APPLICATION_ENV" "production"
% </IfModule>
%
% Alias /bareos-webui  /srv/www/htdocs/bareos-webui/public
%
% <Directory /srv/www/htdocs/bareos-webui/public>
%
%         Options FollowSymLinks
%         AllowOverride None
%
%         # Following module checks are only done to support
%         # Apache 2.2,
%         # Apache 2.4 with mod_access_compat and
%         # Apache 2.4 without mod_access_compat
%         # in the same configuration file.
%         # Feel free to adapt it to your needs.
%
%         # Apache 2.2
%         <IfModule !mod_authz_core.c>
%                 Order deny,allow
%                 Allow from all
%         </IfModule>
%
%         # Apache 2.4
%         <IfModule mod_authz_core.c>
%                 <IfModule mod_access_compat.c>
%                     Order deny,allow
%                 </IfModule>
%                 Require all granted
%         </IfModule>
%
%         <IfModule mod_rewrite.c>
%                 RewriteEngine on
%                 RewriteBase /bareos-webui
%                 RewriteCond %{REQUEST_FILENAME} -s [OR]
%                 RewriteCond %{REQUEST_FILENAME} -l [OR]
%                 RewriteCond %{REQUEST_FILENAME} -d
%                 RewriteRule ^.*$ - [NC,L]
%                 RewriteRule ^.*$ index.php [NC,L]
%         </IfModule>
%
%         <IfModule mod_php5.c>
%                 php_flag magic_quotes_gpc off
%                 php_flag register_globals off
%         </IfModule>
%
% </Directory>
% \end{bconfig}
%
% Restart your Apache Webserver.
%
% \subsection*{Step 4 - Configure the directors}
%
% Create a required configuration file \file{/etc/bareos-webui/directors.ini} and edit it to your needs.
%
% \textbf{Note:} The location and name of the directors.ini is mandatory.
%
% \begin{commands}{}
% mkdir /etc/bareos-webui
%
% cd /etc/bareos-webui
%
% cp /srv/www/htdocs/bareos-webui/install/directors.ini .
%
% vim directors.ini
% \end{commands}
%
% A basic directors.ini should look similar to this:
%
% \begin{bconfig}{Webui directors.ini}
% [director121-dir]
% enabled = "yes"
% diraddress = "10.10.121.100"
% dirport = 9101
%
% [director122-dir]
% enabled = "yes"
% diraddress = "10.10.122.100"
% dirport = 9101
%
% [director123-dir]
% enabled = "no"
% diraddress = "10.10.123.100"
% dirport = 9101
% \end{bconfig}
%
% \subsection*{Step 5 - Create restricted named consoles}
%
% Configure restricted named consoles and reload your director configuration. Those consoles are like user accounts and used by the webui. Therefore you need need to create for example two files with the following content on the host your director runs on and you want to be able to connect to via the webui.
%
% \begin{bconfig}{}
% #
% # Preparations:
% #
% # include this configuration file in bareos-dir.conf by
% # @/etc/bareos/bareos-dir.d/webui-consoles.conf
% #
%
% #
% # Restricted consoles used by bareos-webui
% #
% Console {
%   Name = user1
%   Password = "user1"
%   Profile = profile1
% }
% Console {
%   Name = user2
%   Password = "user3"
%   Profile = profile2
% }
% Console {
%   Name = user3
%   Password = "user3"
%   Profile = profile3
% }
% \end{bconfig}
%
% \begin{bconfig}{}
% #
% # Preparations:
% #
% # include this configuration file in bareos-dir.conf by
% # @/etc/bareos/bareos-dir.d/webui-profiles.conf
% #
%
% #
% # bareos-webui profile resources
% #
% Profile {
%   Name = profile1
%   CommandACL = status, messages, show, version, run, rerun, cancel, .api, .bvfs_*, list, llist, use, restore, .jobs, .filesets, .clients
%   Job ACL = *all*
%   Schedule ACL = *all*
%   Catalog ACL = *all*
%   Pool ACL = *all*
%   Storage ACL = *all*
%   Client ACL = *all*
%   FileSet ACL = *all*
%   Where ACL = *all*
% }
%
% Profile {
%   Name = profile2
%   CommandACL = status, messages, show, version, run, rerun, cancel, .api, .bvfs_*, list, llist, use, restore, .jobs, .filesets, .clients
%   Job ACL = *all*
%   Schedule ACL = *all*
%   Catalog ACL = *all*
%   Pool ACL = *all*
%   Storage ACL = *all*
%   Client ACL = *all*
%   FileSet ACL = *all*
%   Where ACL = *all*
% }
%
% Profile {
%   Name = profile3
%   CommandACL = status, messages, show, version, run, rerun, cancel, .api, .bvfs_*, list, llist, use, restore, .jobs, .filesets, .clients
%   Job ACL = *all*
%   Schedule ACL = *all*
%   Catalog ACL = *all*
%   Pool ACL = *all*
%   Storage ACL = *all*
%   Client ACL = *all*
%   FileSet ACL = *all*
%   Where ACL = *all*
% }
% \end{bconfig}
%
% After including both files and reloading the director you are done and able to use the webui.
%
% Login via your favorite browser by calling: \url{http:///bareos-webui/}.

