%%
%%

\chapter{Disaster Recovery Using Bareos}
\label{RescueChapter}
\index[general]{Disaster!Recovery}
\index[general]{Recovery!Disaster Recovery}

\section{General}

When disaster strikes, you must have a plan, and you must have prepared in
advance, otherwise the work of recovering your system and your files will be
considerably greater.  For example, if you have not previously saved the
partitioning information for your hard disk, how can you properly rebuild
it if the disk must be replaced?

Unfortunately, many of the steps one must take before and immediately after
a disaster are very much dependent on the operating system in use.  As a
consequence, this chapter will discuss in detail disaster recovery only for selected operating systems.

\subsection{Important Considerations}

Here are a few important considerations concerning disaster recovery that
you should take into account before a disaster strikes.

\begin{itemize}
\item If the building which houses your computers burns down or is otherwise
   destroyed, do you have off-site backup data?
\item Disaster recovery is much easier if you have several machines. If  you
   have a single machine, how will you handle unforeseen events  if your only
   machine is down?
\item Do you want to protect your whole system and use Bareos to  recover
   everything? Or do you want to try to restore your system from  the original
   installation disks, apply any other updates and only restore the user files?
\end{itemize}

\section{Steps to Take Before Disaster Strikes}
\label{sec:before-disaster}
\index[general]{Disaster!Before}

\begin{itemize}
\item Create a rescue or CDROM for your systems. Generally,
   they are offered by each distribution, and there are many good
   rescue disks on the Web
\item Ensure that you always have a valid bootstrap file for your backup and
   that it is saved to an alternate machine.  This will permit you to
   easily do a full restore of your system.
\item If possible copy your catalog nightly to an alternate machine.  If you
   have a valid bootstrap file, this is not necessary, but  can be very useful if
   you do not want to reload everything.
\item Ensure that you always have a valid \ilink{bootstrap}{BootstrapChapter} file for your  catalog
   backup that is saved to an alternate machine. This will  permit you to restore
   your catalog more easily if needed.
\item Try out how to use the Rescue CDROM before you are forced to use it
    in an emergency situation.
\item Make a copy of your Bareos .conf files, particularly your
   bareos-dir.conf, and your bareos-sd.conf files, because if your server
   goes down, these files will be needed to get it back up and running,
   and they can be difficult to rebuild from memory.
\end{itemize}


\section{Bare Metal Recovery of Bareos Clients}
\label{sec:BareMetalRestoreClient}

A so called "Bare Metal" recovery is one where you start with an empty hard
disk and you restore your machine.

Generally, following components are required for a Bare Metal Recovery:
\begin{itemize}
\item A rescue CDROM containing a copy of your OS and including the Bareos File daemon, including all libraries
\item The Bareos client configuration files
\item Useful: a copy of your hard disk information
\item A full Bareos backup of your system
\end{itemize}


\subsection{Linux}
\label{sec:rear}
\index[general]{Disaster!Recovery!Linux}

From the Relax-and-Recover web site
(\elink{http://relax-and-recover.org}{http://relax-and-recover.org}):
\begin{quote}
 Relax-and-Recover is a setup-and-forget Linux bare metal disaster recovery
solution.
It is easy to set up and requires no maintenance so there is no excuse for not
using it.
\end{quote}
Relax-and-Recover (ReaR) is quite easy to use with Bareos.

\subsubsection{Installation}
Bareos is a supported backend for ReaR $>=$ 1.15.
To use the \parameter{BAREOS_CLIENT} option, ReaR $>=$ 1.17 is required.
If ReaR $>=$ 1.17 is not part of your distribution, 
check the \elink{download section on the
ReaR website}{http://relax-and-recover.org/download/}.

\subsubsection{Configuration}
Assuming you have a working Bareos configuration on the system you want to
protect with ReaR and Bareos references this system by the name \parameter{bareosclient-fd}, 
the only configuration for ReaR is:

\begin{config}{}
BACKUP=BAREOS
BAREOS_CLIENT=bareosclient-fd
\end{config}

You also need to specify in your ReaR configuration file
(\file{/etc/rear/local.conf})
where
you want to store your recovery images. 
Please refer to the \elink{ReaR documentation}{http://relax-and-recover.org/documentation/} for details.

For example, if you want to create an ISO
image and store it to an NFS server
with the IP Address 192.168.10.1, you can use the following configuration:

\begin{config}{Full Rear configuration in /etc/rear/local.conf}
# This is default:
#OUTPUT=ISO
# Where to write the iso image
# You can use NFS, if you want to write your iso image to a nfs server
# If you leave this blank, it will
# be written to: /var/lib/rear/output/
OUTPUT_URL=nfs://192.168.10.1/rear
BACKUP=BAREOS
BAREOS_CLIENT=bareosclient-fd
\end{config}

\subsubsection{Backup}
If you have installed and configured ReaR on your system, type
\begin{commands}{Create Rescue Image}
<command>rear</command><parameter> -v mkrescue</parameter>
\end{commands}
to create the rescue image. If you used the configuration example above, you
will get a bootable ISO image which can be burned onto a CD.

\warning{This will not create a Bareos backup on your system! You will have to do that by
other means, e.g. by a regular Bareos backup schedule.
Also \command{rear mkbackup} will not create a backup. 
In this configuration it will only create the rescue ISO 
(same as the \command{rear mkrescue} command).}

\subsubsection{Recovery}

In case, you want to recover your system,
boot it using the generated ReaR recovery ISO.
After booting log in as user \user{root} and type
\begin{commands}{Restore your system using Rear and Bareos}
<command>rear</command><parameter> recover</parameter>
\end{commands}
ReaR will now use the most recent backup from Bareos to restore your system.
When the restore job has finished, ReaR will start a new shell which you can use
to verify if the system has been restored correctly. The restored system can be
found under the \path|/mnt/local| directory.
When you are done< with the verification, type 'exit' to leave the shell, getting
back to the recovery process. Finally, you will be asked to confirm that
everything is correct. Type 'yes' to continue. After that, ReaR will restore
your bootloader. Recovery is complete.

\hide{
\subsection{FreeBSD}
\label{FreeBSD1}
\index[general]{Disaster!Recovery!FreeBSD}
\index[general]{FreeBSD!Disaster Recovery}

The same basic techniques described above also apply to FreeBSD. Although we
don't yet have a fully automated procedure, Alex Torres Molina has provided us
with the following instructions with a few additions from Jesse Guardiani and
Dan Langille:

\begin{enumerate}
\item Boot with the FreeBSD installation disk
\item Go to Custom, Partition and create your slices and go to Label and
   create the partitions that you want. Apply changes.
\item Go to Fixit to start an emergency console.
\item Create devs ad0 .. .. if they don't exist under /mnt2/dev (in my  situation)
   with MAKEDEV. The device or devices you  create depend on what hard drives you
   have. ad0 is your  first ATA drive. da0 would by your first SCSI drive.  Under
OS version 5 and greater, your device files are  most likely automatically
created for you.
\item mkdir /mnt/disk
   this is the root of the new disk
\item mount /mnt2/dev/ad0s1a /mnt/disk
   mount /mnt2/dev/ad0s1c /mnt/disk/var
   mount /mnt2/dev/ad0s1d /mnt/disk/usr
.....
The same hard drive issues as above apply here too.  Note, under OS version 5
or higher, your disk devices may  be in /dev not /mnt2/dev.
\item Network configuration (ifconfig xl0 ip/mask + route add default
   ip-gateway)
\item mkdir /mnt/disk/tmp
\item cd /mnt/disk/tmp
\item Copy bareos-fd and bareos-fd.conf to this path
\item If you need to, use sftp to copy files, after which you must do this:
   ln -s /mnt2/usr/bin /usr/bin
\item chmod u+x bareos-fd
\item Modify bareos-fd.conf to fit this machine
\item Copy /bin/sh to /mnt/disk, necessary for chroot
\item Don't forget to put your bareos-dir's IP address and domain  name in
   /mnt/disk/etc/hosts if it's not on a public net.  Otherwise the FD on the
   machine you are restoring to  won't be able to contact the SD and DIR on the
remote machine.
\item mkdir -p /mnt/disk/var/db/bareos
\item chroot /mnt/disk /tmp/bareos-fd -c /tmp/bareos-fd.conf
   to start bareos-fd
\item Now you can go to bareos-dir and restore the job with the entire
   contents of the broken server.
\item You must create /proc
\end{enumerate}
}

\hide{
\subsection{Solaris}
\label{solaris}
\index[general]{Solaris!Disaster Recovery}
\index[general]{Disaster!Recovery!Solaris}

The same basic techniques described above apply to Solaris:

\begin{itemize}
\item the same restrictions as those given for Linux apply
\item you will need to create a Rescue disk
   \end{itemize}

However, during the recovery phase, the boot and disk preparation procedures
are different:

\begin{itemize}
\item there is no need to create an emergency boot disk  since it is an
   integrated part of the Solaris boot.
\item you must partition and format your hard disk by hand  following manual
   procedures as described in W. Curtis Preston's  book "Unix Backup \&
   Recovery"
\end{itemize}

Once the disk is partitioned, formatted and mounted, you can continue with
bringing up the network and reloading Bareos.

\subsubsection{Preparing Solaris Before a Disaster}

As mentioned above, before a disaster strikes, you should prepare the
information needed in the case of problems. To do so, in the {\bf
rescue/solaris} subdirectory enter:

\footnotesize
\begin{verbatim}
su
./getdiskinfo
./make_rescue_disk
\end{verbatim}
\normalsize

The {\bf getdiskinfo} script will, as in the case of Linux described above,
create a subdirectory {\bf diskinfo} containing the output from several system
utilities. In addition, it will contain the output from the {\bf SysAudit}
program as described in Curtis Preston's book. This file {\bf
diskinfo/sysaudit.bsi} will contain the disk partitioning information that
will allow you to manually follow the procedures in the "Unix Backup \&
Recovery" book to repartition and format your hard disk. In addition, the
{\bf getdiskinfo} script will create a {\bf start\_network} script.

Once you have your disks repartitioned and formatted, do the following:

\begin{itemize}
\item Start Your Network with the {\bf start\_network} script
\item Restore the Bareos File daemon as documented above
\item Perform a Bareos restore of all your files using the same  commands as
   described above for Linux
\item Re-install your boot loader using the instructions outlined  in the
   "Unix Backup \& Recovery" book  using installboot
\end{itemize}
}

\hide{
\subsection{Windows}
\label{Win3233}
\index[general]{Disaster!Recovery!Windows}
\index[general]{Windows!Disaster Recovery}

Due to open system files, and registry problems, Bareos cannot save and
restore a complete Win2K/XP/NT environment.

A suggestion by Damian Coutts using Microsoft's NTBackup utility in
conjunction with Bareos should permit a Full bare metal restore of Win2K/XP
(and possibly NT systems). His suggestion is to do an NTBackup of the critical
system state prior to running a Bareos backup with the following command:

\footnotesize
\begin{verbatim}
ntbackup backup systemstate /F c:\systemstate.bkf
\end{verbatim}
\normalsize

The {\bf backup} is the command, the {\bf systemstate} says to backup only the
system state and not all the user files, and the {\bf /F
c:\textbackslash{}systemstate.bkf} specifies where to write the state file.
this file must then be saved and restored by Bareos. This command
can be put in a Client Run Before Job directive so that it is automatically
run during each backup, and thus saved to a Bareos Volume.

To restore the system state, you first reload a base operating system, then
you would use Bareos to restore all the users files and to recover the {\bf
c:\textbackslash{}systemstate.bkf} file, and finally, run {\bf NTBackup} and
{\bf catalogue} the system statefile, and then select it for restore. The
documentation says you can't run a command line restore of the systemstate.

This procedure has been confirmed to work by Ludovic Strappazon -- many
thanks!

A new tool is provided in the form of a bareos plugin for the BartPE rescue
CD. BartPE is a self-contained WindowsXP boot CD which you can make using the
PeBuilder tools available at
\elink{http://www.nu2.nu/pebuilder/}{http://www.nu2.nu/pebuilder/} and a valid
Windows XP SP1 CDROM. The plugin is provided as a zip archive. Unzip the file
and copy the bareos directory into the plugin directory of your BartPE
installation. Edit the configuration files to suit your installation and build
your CD according to the instructions at Bart's site. This will permit you to
boot from the cd, configure and start networking, start the bareos file client
and access your director with the console program. The programs menu on the
booted CD contains entries to install the file client service, start the file
client service, and start the WX-Console. You can also open a command line
window and CD Programs\textbackslash{}Bareos and run the command line console
bconsole.
}

\section{Restoring a Bareos Server}
\index[general]{Restore!Bareos Server}
\label{sec:RestoreServer}

Above, we considered how to recover a client machine where a valid Bareos
server was running on another machine. However, what happens if your server
goes down and you no longer have a running Director, Catalog, or Storage
daemon? There are several solutions:

\begin{enumerate}
\item Bring up static versions of your Director, Catalog, and Storage  daemon
   on the damaged machine.

\item Move your server to another machine.

\item Use a Hot Spare Server on another Machine.
\end{enumerate}

The first option, is very difficult because it requires you to have created a
static version of the Director and the Storage daemon as well as the Catalog.
If the Catalog uses MySQL or PostgreSQL, this may or may not be possible. In
addition, to loading all these programs on a bare system (quite possible), you
will need to make sure you have a valid driver for your tape drive.

The second suggestion is probably a much simpler solution, and one I have done
myself. To do so, you might want to consider the following steps:

\begin{itemize}
\item Install the same database server as on the original system.
\item Install Bareos and initialize the Bareos database.
\item Ideally, you will have a copy of all the Bareos conf files that
   were being used on your server. If not, you will at a minimum need
   create a bareos-dir.conf that has the same Client resource that
   was used to backup your system.
\item If you have a valid saved Bootstrap file as created for your damaged
   machine with WriteBootstrap, use it to restore the files to the damaged
   machine, where you have loaded a static Bareos File daemon using the
   Rescue disk).  This is done by using the restore command and at
   the yes/mod/no prompt, selecting {\bf mod} then specifying the path to
   the bootstrap file.
\item If you have the Bootstrap file, you should now be back up and  running,
   if you do not have a Bootstrap file, continue with the  suggestions below.
\item Using {\bf bscan} scan the last set of backup tapes into your  MySQL,
   PostgreSQL or SQLite database.
\item Start Bareos, and using the Console {\bf restore} command,  restore the
   last valid copy of the Bareos database and the Bareos configuration
   files.
\item Move the database to the correct location.
\item Start the database, and restart Bareos. Then use the Console {\bf
   restore} command, restore all the files  on the damaged machine, where you
   have loaded a Bareos File  daemon using the Rescue disk.
\end{itemize}

For additional details of restoring your database, please see the
\nameref{sec:RestoreCatalog} chapter.


