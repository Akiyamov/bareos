\defDirective{Fd}{Client}{Absolute Job Timeout}{}{}{%
}

\defDirective{Fd}{Client}{Allow Bandwidth Bursting}{}{}{%
This option sets if there is Bandwidth Limiting in effect if the limiting
code can use bursting e.g. when from a certain timeslice not all bandwidth
is used this bandwidth can be used in a next timeslice. Which mean you will
get a smoother limiting which will be much closer to the actual limit set
but it also means that sometimes the bandwidth will be above the setting here.
}

\defDirective{Fd}{Client}{Allowed Job Command}{}{}{%
This directive filters what type of jobs the filedaemon should allow.
Until now we had the -b (backup only) and -r (restore only) flags which could
be specified at the startup of the filedaemon.

Allowed Job Command can be defined globally for all directors by
adding it to the global filedaemon resource or for a specific director when
added to the director resource.

You specify all commands you want to be executed by the filedaemon. When you
don't specify the option it will be empty which means all commands are allowed.

The following example shows how to use this functionality:
\bconfigInput{config/FdClientAllowedJobCommand1.conf}


All commands that are allowed are specified each on a new line with the
Allowed Job Command keyword.

The following job commands are recognized:

\begin{description}
    \item[backup] allow backups to be made
    \item[restore] allow restores to be done
    \item[verify] allow verify jobs to be done
    \item[estimate] allow estimate cmds to be executed
    \item[runscript] allow runscripts to run
\end{description}

Only the important commands the filedaemon can perform are filtered, as
some commands are part of the above protocols and by disallowing
the action the other commands are not invoked at all.

If runscripts are not needed it would we recommend as security measure to disable
running those or only allow the commands that you really want to be used.

Runscripts are particularly a problem as they allow the filedaemon to run
arbitrary commands. You may also look into the Allowed Script Dir keyword to
limit the impact of the runscript command.
}

\defDirective{Fd}{Client}{Allowed Script Dir}{}{}{%
This directive limits the impact of the runscript command of the filedaemon.

It can be specified either for all directors  by adding it to the global filedaemon resource
or for a specific director when added to the director resource.

All directories in which the scripts or commands are located
that you allow to be run by the runscript command of the filedaemon. Any
program not in one of these paths (or subpaths) cannot be used. The
implementation checks if the full path of the script starts with one of the
specified paths.

The following example shows how to use this functionality:

\bconfigInput{config/FdClientAllowedScriptDir1.conf}
}

\defDirective{Fd}{Client}{Always Use LMDB}{}{}{%
}

\defDirective{Fd}{Client}{Compatible}{}{}{%
This directive enables the compatible mode of the file daemon. In
this mode the file daemon will try to be as compatible to a native
Bacula file daemon as possible. Enabling this option means that
certain new options available in Bareos cannot be used as they would
lead to data not being able to be restored by a Native Bareos file daemon.

The default value for this directive was changed from yes to no since Bareos Version 15.

When you want to use bareos-only features, the value of compatible must be no.
}

\defDirective{Fd}{Client}{Description}{}{}{%
}

\defDirective{Fd}{Client}{FD Address}{}{}{%
This record is optional,  and if it is specified, it will cause the File
daemon server (for  Director connections) to bind to the specified {\bf
IP-Address},  which is either a domain name or an IP address specified as a
dotted quadruple.
}

\defDirective{Fd}{Client}{FD Addresses}{}{}{%
Specify the ports and addresses on which the File daemon listens for
Director connections.  Probably the simplest way to explain is to show
an example:

\bconfigInput{config/FdClientFDAddresses1.conf}

where ip, ip4, ip6, addr, and port are all keywords. Note, that  the address
can be specified as either a dotted quadruple, or  IPv6 colon notation, or as
a symbolic name (only in the ip specification).  Also, the port can be specified
as a number or as the mnemonic value from  the /etc/services file.  If a port
is not specified, the default will be used. If an ip  section is specified,
the resolution can be made either by IPv4 or  IPv6. If ip4 is specified, then
only IPv4 resolutions will be permitted,  and likewise with ip6.
}

\defDirective{Fd}{Client}{FD Port}{}{}{%
This specifies the port number  on which the Client listens for Director
connections. It must agree  with the FDPort specified in the Client resource
of the Director's  configuration file.
}

\defDirective{Fd}{Client}{FD Source Address}{}{}{%
This record is optional,  and if it is specified, it will cause the File
daemon server (for  Storage connections) to bind to the specified {\bf
IP-Address},  which is either a domain name or an IP address specified as a
dotted quadruple. If this record is not specified, the kernel will choose
the best address according to the routing table (the default).
}

\defDirective{Fd}{Client}{Heartbeat Interval}{}{}{%
This record defines an interval of time in seconds.  For each heartbeat that the
File daemon receives from the Storage daemon, it will forward it to the
Director.  In addition, if no heartbeat has been received from the
Storage daemon and thus forwarded the File daemon will send a heartbeat
signal to the Director and to the Storage daemon to keep the channels
active.  Setting the interval to 0 (zero) disables the heartbeat.
This feature is particularly useful if you have a router
that does not follow Internet standards and times out a valid
connection after a short duration despite the fact that keepalive is
set. This usually results in a broken pipe error message.

% If you continue getting broken pipe error messages despite using the
% Heartbeat Interval, and you are using Windows, you should consider
% upgrading your ethernet driver.  This is a known problem with NVidia
% NForce 3 drivers (4.4.2 17/05/2004), or try the following workaround
% suggested by Thomas Simmons for Win32 machines:
%
% Browse to:
% Start {\textgreater} Control Panel {\textgreater} Network Connections
%
% Right click the connection for the nvidia adapter and select properties.
% Under the General tab, click "Configure...". Under the Advanced tab set
% "Checksum Offload" to disabled and click OK to save the change.

% Lack of communications, or communications that get interrupted can
% also be caused by Linux firewalls where you have a rule that throttles
% connections or traffic.
}

\defDirective{Fd}{Client}{LMDB Threshold}{}{}{%
}

\defDirective{Fd}{Client}{Maximum Bandwidth Per Job}{}{}{%
The speed parameter specifies the maximum allowed bandwidth that a job may
use. The speed parameter should be specified in k/s, kb/s, m/s or mb/s.
}

\defDirective{Fd}{Client}{Maximum Concurrent Jobs}{}{}{%
where {\textless}number{\textgreater} is the maximum number of Jobs that should run
concurrently. Each contact from the Director (e.g. status request, job start
request) is considered as a Job, so if you want to be able to do a {\bf
status} request in the console at the same time as a Job is running, you
will need to set this value greater than 1.
}

\defDirective{Fd}{Client}{Maximum Network Buffer Size}{}{}{%
where {\textless}bytes{\textgreater} specifies the initial network buffer  size to use with
the File daemon. This size will be adjusted down  if it is too large until it
is accepted by the OS. Please use  care in setting this value since if it is
too large, it will  be trimmed by 512 bytes until the OS is happy, which may
require  a large number of system calls. The default value is 65,536 bytes.

Note, on certain Windows machines, there are reports that the
transfer rates are very slow and this seems to be related to
the default 65,536 size. On systems where the transfer rates
seem abnormally slow compared to other systems, you might try
setting the Maximum Network Buffer Size to 32,768 in both the
File daemon and in the Storage daemon.
}

\defDirective{Fd}{Client}{Messages}{}{}{%
}

\defDirective{Fd}{Client}{Name}{}{}{%
The client name that must be used  by the Director when connecting. Generally,
it is a good idea  to use a name related to the machine so that error messages
can be easily identified if you have multiple Clients.  This directive is
required.
}

\defDirective{Fd}{Client}{Pid Directory}{}{}{%
This directive specifies a directory in which the File Daemon
may put its process Id file files. The process Id file is used to  shutdown
Bareos and to prevent multiple copies of  Bareos from running simultaneously.

The Bareos file daemon uses a platform specific default value,
that is defined at compile time.
Typically on Linux systems, it is set to \path|/var/lib/bareos/| or \path|/var/run/|.
}

\defDirective{Fd}{Client}{Pki Cipher}{}{}{%
See the \ilink{Data Encryption}{DataEncryption} chapter of this manual.
}

\defDirective{Fd}{Client}{Pki Encryption}{}{}{%
See the \ilink{Data Encryption}{DataEncryption} chapter of this manual.
}

\defDirective{Fd}{Client}{Pki Key Pair}{}{}{%
See the \ilink{Data Encryption}{DataEncryption} chapter of this manual.
}

\defDirective{Fd}{Client}{Pki Master Key}{}{}{%
See the \ilink{Data Encryption}{DataEncryption} chapter of this manual.
}

\defDirective{Fd}{Client}{Pki Signatures}{}{}{%
See the \ilink{Data Encryption}{DataEncryption} chapter of this manual.
}

\defDirective{Fd}{Client}{Pki Signer}{}{}{%
See the \ilink{Data Encryption}{DataEncryption} chapter of this manual.
}

\defDirective{Fd}{Client}{Plugin Directory}{}{}{%
This directive specifies a directory in which the File Daemon searches for
plugins with the name {\bf {\textless}pluginname{\textgreater}-fd.so} which it will load at startup.
Typically on Linux systems, it is set to \path|/usr/lib/bareos/plugins/| or \path|/usr/lib64/bareos/plugins/|.
}

\defDirective{Fd}{Client}{Plugin Names}{}{}{%
}

\defDirective{Fd}{Client}{Scripts Directory}{}{}{%
}

\defDirective{Fd}{Client}{SD Connect Timeout}{}{}{%
This  record defines an interval of time that  the File daemon will try to
connect to the  Storage daemon. If no connection
is made in the specified time interval, the File daemon  cancels the Job.
}

\defDirective{Fd}{Client}{Sub Sys Directory}{}{}{%
}

\defDirective{Fd}{Client}{TLS Authenticate}{}{}{%
}

\defDirective{Fd}{Client}{TLS CA Certificate Dir}{}{}{%
}

\defDirective{Fd}{Client}{TLS CA Certificate File}{}{}{%
}

\defDirective{Fd}{Client}{TLS Certificate}{}{}{%
}

\defDirective{Fd}{Client}{TLS Certificate Revocation List}{}{}{%
}

\defDirective{Fd}{Client}{TLS Enable}{}{}{%
}

\defDirective{Fd}{Client}{TLS Key}{}{}{%
}

\defDirective{Fd}{Client}{TLS Require}{}{}{%
}

\defDirective{Fd}{Client}{TLS Verify Peer}{}{}{%
}

\defDirective{Fd}{Client}{Ver Id}{}{}{%
}

\defDirective{Fd}{Client}{Working Directory}{}{}{%
This directive is optional and specifies a directory in which the File
daemon  may put its status files.
%\TODO{This directory should be used only by {\bf
%Bareos}, but may be shared by other Bareos daemons provided the daemon
%names on the {\bf Name} definition are unique for each daemon.}

% The bareos file daemon uses a platform specific default value, that is defined at compile time.
% For Linux systems, the default is \path|/var/lib/bareos/|.
% For Windows systems it is \%TEMP\%.

On Win32 systems, in some circumstances you may need to specify a drive
letter in the specified working directory path.  Also, please be sure
that this directory is writable by the SYSTEM user otherwise restores
may fail (the bootstrap file that is transferred to the File daemon from
the Director is temporarily put in this directory before being passed
to the Storage daemon).
}
