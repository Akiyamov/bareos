
\chapter{Bareos Programs}
    \label{sec:Utilities}

This document describes the utility programs written to aid Bareos users and
developers in dealing with Volumes external to Bareos and to perform other useful tasks.

\section{Parameter}

\subsection{Specifying the Configuration File}
\index[general]{Specifying the Configuration File}

Each of the utilities that deal with Volumes require a valid
Storage daemon configuration file \path|bareos-sd.conf| (actually, the only part of the
configuration file that these programs need is the {\bf Device} resource
definitions). This permits the programs to find the configuration parameters
for your archive device (generally a tape drive).
Using the {\bf -c} option a custom storage daemon configuration file can be selected.


\subsection{Specifying a Device Name For a Tape}
\index[general]{Tape!Device Name}
\index[general]{Specifying a Device Name For a Tape}

Each of these programs require a {\bf device-name} where the Volume can be
found. In the case of a tape, this is the physical device name such as {\bf
/dev/nst0} or {\bf /dev/rmt/0ubn} depending on your system. For the program to
work, it must find the identical name in the Device resource of the
configuration file. See below for specifying Volume names.

Please note that if you have Bareos running and you want to use
one of these programs, you will either need to stop the Storage daemon, or
{\bf unmount} any tape drive you want to use, otherwise the drive
will {\bf busy} because Bareos is using it.


\subsection{Specifying a Device Name For a File}
\index[general]{File!Device Name}
\index[general]{Specifying a Device Name For a File}

If you are attempting to read or write an archive file rather than a tape, the
{\bf device-name} should be the full path to the archive location including
the filename. The filename (last part of the specification) will be stripped
and used as the Volume name, and the path (first part before the filename)
must have the same entry in the configuration file. So, the path is equivalent
to the archive device name, and the filename is equivalent to the volume name.
The default file storage path is \fileStoragePath.

\subsection{Specifying Volumes}
\index[general]{Volumes!Specifying}
\index[general]{Specifying Volumes}
\index[general]{Bootstrap}

Often you must specify the Volume name to the programs below.
The best method to do so is to specify a
{\bf bootstrap} file on the command line with the {\bf -b} option. As part of
the bootstrap file, you will then specify the Volume name or Volume names if
more than one volume is needed. For example, suppose you want to read tapes
{\bf tape1} and {\bf tape2}. First construct a {\bf bootstrap} file named say,
{\bf list.bsr} which contains:

\footnotesize
\begin{verbatim}
Volume=test1|test2
\end{verbatim}
\normalsize

where each Volume is separated by a vertical bar. Then simply use:

\footnotesize
\begin{verbatim}
bls -b list.bsr /dev/nst0
\end{verbatim}
\normalsize

In the case of Bareos Volumes that are on files, you may simply append volumes
as follows:

\footnotesize
\begin{verbatim}
bls /tmp/test1\|test2
\end{verbatim}
\normalsize

where the backslash (\textbackslash{}) was necessary as a shell escape to
permit entering the vertical bar (|).

And finally, if you feel that specifying a Volume name is a bit complicated
with a bootstrap file, you can use the {\bf -V} option (on all programs except
{\bf bcopy}) to specify one or more Volume names separated by the vertical bar
(|). For example,

\footnotesize
\begin{verbatim}
bls -V Vol001 /dev/nst0
\end{verbatim}
\normalsize

You may also specify an asterisk (*) to indicate that the program should
accept any volume. For example:

\footnotesize
\begin{verbatim}
bls -V* /dev/nst0
\end{verbatim}
\normalsize


\section{Bareos Daemons}

\subsection{Daemon Command Line Options}
\label{daemon-command-line-options}
\index[general]{Daemon!Command Line Options}
\index[general]{Command Line Options!Daemon}

Each of the three daemons (Director, File, Storage) accepts a small set of
options on the command line. In general, each of the daemons as well as the
Console program accepts the following options:

\begin{description}

\item [-c {\textless}file{\textgreater}]
   Define the file to use as a  configuration file. The default is the daemon
   name followed  by {\bf .conf} i.e. {\bf bareos-dir.conf} for the Director,
   {\bf bareos-fd.conf} for the File daemon, and {\bf bareos-sd}  for the Storage
   daemon.

\item [-d nnn]
   Set the debug level to {\bf nnn}. 
   Generally anything between 50 and 200 is reasonable. 
   The higher the number, the more output is produced. The output is
   written to standard output.
   The debug level can also be set during runtime, see section \ilink{bconsole: setdebug}{bcommandSetdebug}.

\item [-f]
   Run the daemon in the foreground. This option is  needed to run the daemon
   under the debugger.

\item [-g {\textless}group{\textgreater}]
   Run the daemon under this group.  This must be a group name, not a GID.

\item [-s]
   Do not trap signals. This option is needed to run  the daemon under the
   debugger.

\item [-t]
   Read the configuration file and print any error messages,  then immediately
   exit. Useful for syntax testing of  new configuration files.

\item [-u {\textless}user{\textgreater}]
   Run the daemon as this user.  This must be a user name, not a UID.

\item [-v]
   Be more verbose or more complete in printing error  and informational
   messages. Recommended.

\item [-?]
   Print the version and list of options.

\end{description}


\subsection{bareos-dir}
\label{command-bareos-dir}
\index[general]{Command!bareos-dir}
\index[dir]{Command Line Options}

Bareos Director.
\TODO{Bareos Director describe command line arguments}

\subsection{bareos-sd}
\label{command-bareos-sd}
\index[general]{Command!bareos-sd}
\index[sd]{Command Line Options}

Baroes Storage Daemon
\TODO{Bareos Storage Daemon describe command line arguments}


\subsection{bareos-fd}
\label{command-bareos-fd}
\index[general]{Command!bareos-fd}
\index[fd]{Command Line Options}

Baroes File Daemon
\TODO{Bareos File Daemon describe command line arguments}


\section{Interactive Programs}

\subsection{bconsole}

There is an own chapter on \command{bconsole}.
Please refer to chapter \nameref{sec:bconsole}.

\subsection{bareos-webui}

See \nameref{sec:webui}.

\subsection{bat}
\index[general]{Command!bat}
\label{bat}



\section{Volume Utility Commands}
    \index[general]{Volume Utility Tools}
    \index[general]{Tools!Volume Utility}

\warning{If you using Bareos with non-default block sizes defined in the pools, it might be neccessary to specifiy the \configdirective{maximum block level} also in the storage device resource, see \ilink{Direct access to Volumes with non-default blocksizes}{direct-access-to-volumes-with-non-default-blocksizes}.}

\subsection{bls}
\label{bls}
\index[general]{bls}
\index[general]{Command!bls}

{\bf bls} can be used to do an {\bf ls} type listing of a {\bf Bareos} tape or
file. It is called:

\footnotesize
\begin{verbatim}
Usage: bls [options] <device-name>
       -b <file>       specify a bootstrap file
       -c <file>       specify a Storage configuration file
       -D <director>   specify a director name specified in the Storage
                       configuration file for the Key Encryption Key selection
       -d <nn>         set debug level to <nn>
       -dt             print timestamp in debug output
       -e <file>       exclude list
       -i <file>       include list
       -j              list jobs
       -k              list blocks
    (no j or k option) list saved files
       -L              dump label
       -p              proceed inspite of errors
       -v              be verbose
       -V              specify Volume names (separated by |)
       -?              print this message
\end{verbatim}
\normalsize

For example, to list the contents of a tape:

\footnotesize
\begin{verbatim}
bls -V Volume-name /dev/nst0
\end{verbatim}
\normalsize

Or to list the contents of a file:

\footnotesize
\begin{verbatim}
bls /var/lib/bareos/storage/testvol
or
bls -V testvol /var/lib/bareos/storage
\end{verbatim}
\normalsize

Note that, in the case of a file, the Volume name becomes the filename, so in
the above example, you will replace the {\bf Volume-name} with the name of the volume
(file) you wrote.

Normally if no options are specified, {\bf bls} will produce the equivalent
output to the {\bf ls -l} command for each file on the tape. Using other
options listed above, it is possible to display only the Job records, only the
tape blocks, etc. For example:

\footnotesize
\begin{verbatim}
bls: butil.c:282-0 Using device: "/var/lib/bareos/storage" for reading.
12-Sep 18:30 bls JobId 0: Ready to read from volume "testvol" on device "FileStorage" (/var/lib/bareos/storage).
bls JobId 1: -rwxr-xr-x   1 root     root            4614 2013-01-22 22:24:11  /usr/sbin/service
bls JobId 1: -rwxr-xr-x   1 root     root           13992 2013-01-22 22:24:12  /usr/sbin/rtcwake
bls JobId 1: -rwxr-xr-x   1 root     root            6243 2013-02-06 11:01:29  /usr/sbin/update-fonts-scale
bls JobId 1: -rwxr-xr-x   1 root     root           43240 2013-01-22 22:24:10  /usr/sbin/grpck
bls JobId 1: -rwxr-xr-x   1 root     root           16894 2013-01-22 22:24:11  /usr/sbin/update-rc.d
bls JobId 1: -rwxr-xr-x   1 root     root            9480 2013-01-22 22:47:43  /usr/sbin/gss_clnt_send_err
...
bls JobId 456: -rw-r-----   1 root     bareos          1008 2013-05-23 13:17:45  /etc/bareos/bareos-fd.conf
bls JobId 456: -rw-r-----   1 bareos   bareos          6026 2013-04-22 12:00:33  /etc/bareos/bareos-sd.conf.dpkg-dist
bls JobId 456: drwxr-xr-x   2 root     root            4096 2013-07-04 17:40:21  /etc/bareos/
12-Sep 18:30 bls JobId 0: End of Volume at file 0 on device "FileStorage" (/var/lib/bareos/storage), Volume "testvol"
12-Sep 18:30 bls JobId 0: End of all volumes.
2972 files found.
\end{verbatim}
\normalsize


\subsubsection{Show Label Information}
\index[general]{bls!Label}

Using the \parameter{-L} the label information of a Volume is shown:

\begin{commandOut}{bls, Show Volume Label}{}{bls -L /var/lib/bareos/storage/testvol}
bls: butil.c:282-0 Using device: "/var/lib/bareos/storage" for reading.
12-Sep 18:41 bls JobId 0: Ready to read from volume "testvol" on device "FileStorage" (/var/lib/bareos/storage).

Volume Label:
Id                : Bareos 0.9 mortal
VerNo             : 10
VolName           : File002
PrevVolName       :
VolFile           : 0
LabelType         : VOL_LABEL
LabelSize         : 147
PoolName          : Default
MediaType         : File
PoolType          : Backup
HostName          : debian6
Date label written: 06-Mar-2013 17:21
\end{commandOut}


\subsubsection{Listing Jobs}
\index[general]{Listing Jobs with bls}
\index[general]{bls!Listing Jobs}

If you are listing a Volume to determine what Jobs to restore, normally the
{\bf -j} option provides you with most of what you will need as long as you
don't have multiple clients. For example:

\begin{commandOut}{bls, Listing Jobs}{}{bls /var/lib/bareos/storage/testvol -j}
bls: butil.c:282-0 Using device: "/var/lib/bareos/storage" for reading.
12-Sep 18:33 bls JobId 0: Ready to read from volume "testvol" on device "FileStorage" (/var/lib/bareos/storage).
Volume Record: File:blk=0:193 SessId=1 SessTime=1362582744 JobId=0 DataLen=158
Begin Job Session Record: File:blk=0:64705 SessId=1 SessTime=1362582744 JobId=1
   Job=BackupClient1.2013-03-06_17.22.48_05 Date=06-Mar-2013 17:22:51 Level=F Type=B
End Job Session Record: File:blk=0:6499290 SessId=1 SessTime=1362582744 JobId=1
   Date=06-Mar-2013 17:22:52 Level=F Type=B Files=162 Bytes=6,489,071 Errors=0 Status=T
Begin Job Session Record: File:blk=0:6563802 SessId=2 SessTime=1362582744 JobId=2
   Job=BackupClient1.2013-03-06_23.05.00_02 Date=06-Mar-2013 23:05:02 Level=I Type=B
End Job Session Record: File:blk=0:18832687 SessId=2 SessTime=1362582744 JobId=2
   Date=06-Mar-2013 23:05:02 Level=I Type=B Files=3 Bytes=12,323,791 Errors=0 Status=T
...
Begin Job Session Record: File:blk=0:319219736 SessId=299 SessTime=1369307832 JobId=454
   Job=BackupClient1.2013-09-11_23.05.00_25 Date=11-Sep-2013 23:05:03 Level=I Type=B
End Job Session Record: File:blk=0:319219736 SessId=299 SessTime=1369307832 JobId=454
   Date=11-Sep-2013 23:05:03 Level=I Type=B Files=0 Bytes=0 Errors=0 Status=T
Begin Job Session Record: File:blk=0:319284248 SessId=301 SessTime=1369307832 JobId=456
   Job=BackupCatalog.2013-09-11_23.10.00_28 Date=11-Sep-2013 23:10:03 Level=F Type=B
End Job Session Record: File:blk=0:320694269 SessId=301 SessTime=1369307832 JobId=456
   Date=11-Sep-2013 23:10:03 Level=F Type=B Files=12 Bytes=1,472,681 Errors=0 Status=T
12-Sep 18:32 bls JobId 0: End of Volume at file 0 on device "FileStorage" (/var/lib/bareos/storage), Volume "testvol"
12-Sep 18:32 bls JobId 0: End of all volumes.
\end{commandOut}

Adding the {\bf -v} option will display virtually all information that is
available for each record.

\subsubsection{Listing Blocks}
\index[general]{Listing Blocks with bls}
\index[general]{bls!Listing Blocks}

Normally, except for debugging purposes, you will not need to list Bareos
blocks (the "primitive" unit of Bareos data on the Volume). However, you can
do so with:

\footnotesize
\begin{verbatim}
bls -k /tmp/File002
bls: butil.c:148 Using device: /tmp
Block: 1 size=64512
Block: 2 size=64512
...
Block: 65 size=64512
Block: 66 size=19195
bls: Got EOF on device /tmp
End of File on device
\end{verbatim}
\normalsize

By adding the {\bf -v} option, you can get more information, which can be
useful in knowing what sessions were written to the volume:

\footnotesize
\begin{verbatim}
bls -k -v /tmp/File002
Date label written: 2002-10-19 at 21:16
Block: 1 blen=64512 First rec FI=VOL_LABEL SessId=1 SessTim=1035062102 Strm=0 rlen=147
Block: 2 blen=64512 First rec FI=6 SessId=1 SessTim=1035062102 Strm=DATA rlen=4087
Block: 3 blen=64512 First rec FI=12 SessId=1 SessTim=1035062102 Strm=DATA rlen=5902
Block: 4 blen=64512 First rec FI=19 SessId=1 SessTim=1035062102 Strm=DATA rlen=28382
...
Block: 65 blen=64512 First rec FI=83 SessId=1 SessTim=1035062102 Strm=DATA rlen=1873
Block: 66 blen=19195 First rec FI=83 SessId=1 SessTim=1035062102 Strm=DATA rlen=2973
bls: Got EOF on device /tmp
End of File on device
\end{verbatim}
\normalsize

Armed with the SessionId and the SessionTime, you can extract just about
anything.

If you want to know even more, add a second {\bf -v} to the command line to
get a dump of every record in every block.

\footnotesize
\begin{verbatim}
bls -k -v -v /tmp/File002
bls: block.c:79 Dump block  80f8ad0: size=64512 BlkNum=1
               Hdrcksum=b1bdfd6d cksum=b1bdfd6d
bls: block.c:92    Rec: VId=1 VT=1035062102 FI=VOL_LABEL Strm=0 len=147 p=80f8b40
bls: block.c:92    Rec: VId=1 VT=1035062102 FI=SOS_LABEL Strm=-7 len=122 p=80f8be7
bls: block.c:92    Rec: VId=1 VT=1035062102 FI=1 Strm=UATTR len=86 p=80f8c75
bls: block.c:92    Rec: VId=1 VT=1035062102 FI=2 Strm=UATTR len=90 p=80f8cdf
bls: block.c:92    Rec: VId=1 VT=1035062102 FI=3 Strm=UATTR len=92 p=80f8d4d
bls: block.c:92    Rec: VId=1 VT=1035062102 FI=3 Strm=DATA len=54 p=80f8dbd
bls: block.c:92    Rec: VId=1 VT=1035062102 FI=3 Strm=MD5 len=16 p=80f8e07
bls: block.c:92    Rec: VId=1 VT=1035062102 FI=4 Strm=UATTR len=98 p=80f8e2b
bls: block.c:92    Rec: VId=1 VT=1035062102 FI=4 Strm=DATA len=16 p=80f8ea1
bls: block.c:92    Rec: VId=1 VT=1035062102 FI=4 Strm=MD5 len=16 p=80f8ec5
bls: block.c:92    Rec: VId=1 VT=1035062102 FI=5 Strm=UATTR len=96 p=80f8ee9
bls: block.c:92    Rec: VId=1 VT=1035062102 FI=5 Strm=DATA len=1783 p=80f8f5d
bls: block.c:92    Rec: VId=1 VT=1035062102 FI=5 Strm=MD5 len=16 p=80f9668
bls: block.c:92    Rec: VId=1 VT=1035062102 FI=6 Strm=UATTR len=95 p=80f968c
bls: block.c:92    Rec: VId=1 VT=1035062102 FI=6 Strm=DATA len=32768 p=80f96ff
bls: block.c:92    Rec: VId=1 VT=1035062102 FI=6 Strm=DATA len=32768 p=8101713
bls: block.c:79 Dump block  80f8ad0: size=64512 BlkNum=2
               Hdrcksum=9acc1e7f cksum=9acc1e7f
bls: block.c:92    Rec: VId=1 VT=1035062102 FI=6 Strm=contDATA len=4087 p=80f8b40
bls: block.c:92    Rec: VId=1 VT=1035062102 FI=6 Strm=DATA len=31970 p=80f9b4b
bls: block.c:92    Rec: VId=1 VT=1035062102 FI=6 Strm=MD5 len=16 p=8101841
...
\end{verbatim}
\normalsize

\subsection{bextract}
\label{bextract}
\index[general]{bextract}
\index[general]{Command!bextract}
\index[general]{Disaster!Recovery!bextract}


If you find yourself using \command{bextract}, you probably have done
something wrong. For example, if you are trying to recover a file
but are having problems, please see the \nameref{sec:RestoreCatalog} chapter.

Normally, you will restore files by running a {\bf Restore} Job from the {\bf
Console} program. However, {\bf bextract} can be used to extract a single file
or a list of files from a Bareos tape or file. In fact, {\bf bextract} can be
a useful tool to restore files to an empty system assuming you are able to
boot, you have statically linked {\bf bextract} and you have an appropriate
{\bf bootstrap} file.

Please note that some of the current limitations of bextract are:

\begin{enumerate}
\item It cannot restore access control lists (ACL) that have been
      backed up along with the file data.
\item It cannot restore encrypted files.
\item The command line length is relatively limited,
      which means that you cannot enter a huge number of volumes.  If you need to
      enter more volumes than the command line supports, please use a bootstrap
      file (see below).
\item Extracting files from a Windows backup on a Linux system
      will only extract the plain files, not the addtional Windows file information.
      If you have to extract files from a Windows backup,
      you should use the Windows version of bextract.
\end{enumerate}


It is called:

\footnotesize
\begin{verbatim}
Usage: bextract <options> <bareos-archive-device-name> <directory-to-store-files>
       -b <file>       specify a bootstrap file
       -c <file>       specify a Storage configuration file
       -D <director>   specify a director name specified in the Storage
                       configuration file for the Key Encryption Key selection
       -d <nn>         set debug level to <nn>
       -dt             print timestamp in debug output
       -e <file>       exclude list
       -i <file>       include list
       -p              proceed inspite of I/O errors
       -v              verbose
       -V <volumes>    specify Volume names (separated by |)
       -?              print this message
\end{verbatim}
\normalsize

where {\bf device-name} is the Archive Device (raw device name or full
filename) of the device to be read, and {\bf directory-to-store-files} is a
path prefix to prepend to all the files restored.

NOTE: On Windows systems, if you specify a prefix of say d:/tmp, any file that
would have been restored to {\bf c:/My Documents} will be restored to {\bf
d:/tmp/My Documents}. That is, the original drive specification will be
stripped. If no prefix is specified, the file will be restored to the original
drive.

\subsubsection{Extracting with Include or Exclude Lists}

Using the {\bf -e} option, you can specify a file containing a list of files
to be excluded. Wildcards can be used in the exclusion list. This option will
normally be used in conjunction with the {\bf -i} option (see below). Both the
{\bf -e} and the {\bf -i} options may be specified at the same time as the
{\bf -b} option. The bootstrap filters will be applied first, then the include
list, then the exclude list.

Likewise, and probably more importantly, with the {\bf -i} option, you can
specify a file that contains a list (one file per line) of files and
directories to include to be restored. The list must contain the full filename
with the path. If you specify a path name only, all files and subdirectories
of that path will be restored. If you specify a line containing only the
filename (e.g. {\bf my-file.txt}) it probably will not be extracted because
you have not specified the full path.

For example, if the file {\bf include-list} contains:

\footnotesize
\begin{verbatim}
/etc/bareos
/usr/sbin
\end{verbatim}
\normalsize

Then the command:

\footnotesize
\begin{verbatim}
bextract -i include-list -V Volume /dev/nst0 /tmp
\end{verbatim}
\normalsize

will restore from the Bareos archive {\bf /dev/nst0} all files and directories
in the backup from {\bf /etc/bareos} and from {\bf /usr/sbin}. The
restored files will be placed in a file of the original name under the
directory {\bf /tmp} (i.e. /tmp/etc/bareos/... and
/tmp/usr/sbin/...).

\subsubsection{Extracting With a Bootstrap File}

The {\bf -b} option is used to specify a {\bf bootstrap} file containing the
information needed to restore precisely the files you want. Specifying a {\bf
bootstrap} file is optional but recommended because it gives you the most
control over which files will be restored. For more details on the {\bf
bootstrap} file, please see
\ilink{Restoring Files with the Bootstrap File}{BootstrapChapter}
chapter of this document. Note, you may also use a bootstrap file produced by
the {\bf restore} command. For example:

\footnotesize
\begin{verbatim}
bextract -b bootstrap-file /dev/nst0 /tmp
\end{verbatim}
\normalsize

The bootstrap file allows detailed specification of what files you want
restored (extracted). You may specify a bootstrap file and include and/or
exclude files at the same time. The bootstrap conditions will first be
applied, and then each file record seen will be compared to the include and
exclude lists.

\subsubsection{Extracting From Multiple Volumes}

If you wish to extract files that span several Volumes, you can specify the
Volume names in the bootstrap file or you may specify the Volume names on the
command line by separating them with a vertical bar. See the section above
under the {\bf bls} program entitled {\bf Listing Multiple Volumes} for more
information. The same techniques apply equally well to the {\bf bextract}
program or read the \ilink{Bootstrap}{BootstrapChapter}
chapter of this document.

\subsubsection{Extracting Under Windows}
\index[general]{Windows!bextract}

\warning{If you use \command{bextract} under Windows, the ordering of the parameters is essential.}

To use \command{bextract}, the Bareos Storage Daemon must be installed.
As bextract works on tapes or disk volumes, these must be configured in the Storage Daemon configuration file,
normally found at \file{C:\ProgrammData\Bareos\bareos-sd.conf}.
However, it is not required to start the Bareos Storage Daemon.
Normally, if the Storage Daemon would be able to run, \command{bextract} would not be required.

After installing, \command{bextract} can be called via command line:

\begin{commands}{Call of bextract}
C:\Program Files\Bareos .\bextract.exe -c "C:\ProgrammData\Bareos\bareos-sd.conf" -V <Volume> <YourStorage> <YourDestination>
\end{commands}

If you want to use exclude or include files you need to write them like you do on Linux. That means each path begins with
a "/" and not with "yourdrive:/". You need to specify the parameter {\bf -e exclude.list} as first parameter.
For example:

\begin{config}{Example exclude.list}
/Program Files/Bareos/bareos-dir.exe
/ProgramData/
\end{config}

\begin{commands}{Call bextract with exclude list}
C:\Program Files\Bareos .\bextract.exe -e exclude.list -c "C:\ProgrammData\Bareos\bareos-sd.conf" -V <Volume> <YourStorage> <YourDestination>
\end{commands}


\subsection{bscan}
\label{bscan}
\index[general]{bscan}
\index[general]{Command!bscan}

If you find yourself using this program, you have probably done something
wrong. For example, the best way to recover a lost or damaged Bareos
database is to reload the database by using the bootstrap file that
was written when you saved it (default Bareos-dir.conf file).

The {\bf bscan} program can be used to re-create a database (catalog)
records from the backup information written to one or more Volumes.  This
is normally needed only if one or more Volumes have been pruned or purged
from your catalog so that the records on the Volume are no longer in the
catalog, or for Volumes that you have archived.  Note, if you scan in
Volumes that were previously purged, you will be able to do restores from
those Volumes.  However, unless you modify the Job and File retention times
for the Jobs that were added by scanning, the next time you run any backup Job
with the same name, the records will be pruned again.  Since it takes a
long time to scan Volumes this can be very frustrating.

With some care, {\bf bscan} can also be used to synchronize your existing
catalog with a Volume.  Although we have never seen a case of bscan
damaging a catalog, since bscan modifies your catalog, we recommend that
you do a simple ASCII backup of your database before running {\bf bscan}
just to be sure.  See \ilink{Compacting Your Database}{CompactingMySQL} for
the details of making a copy of your database.

{\bf bscan} can also be useful in a disaster recovery situation, after the
loss of a hard disk, if you do not have a valid {\bf bootstrap} file for
reloading your system, or if a Volume has been recycled but not overwritten,
you can use {\bf bscan} to re-create your database, which can then be used to
{\bf restore} your system or a file to its previous state.

It is called:

\footnotesize
\begin{verbatim}

Usage: bscan [options] <Bareos-archive>
       -B <driver name>  specify the database driver name (default NULL) <postgresql|mysql|sqlite>
       -b bootstrap      specify a bootstrap file
       -c <file>         specify configuration file
       -d <nn>           set debug level to nn
       -dt               print timestamp in debug output
       -m                update media info in database
       -D <director>     specify a director name specified in the Storage
                         configuration file for the Key Encryption Key selection
       -n <name>         specify the database name (default Bareos)
       -u <user>         specify database user name (default Bareos)
       -P <password>     specify database password (default none)
       -h <host>         specify database host (default NULL)
       -t <port>         specify database port (default 0)
       -p                proceed inspite of I/O errors
       -r                list records
       -s                synchronize or store in database
       -S                show scan progress periodically
       -v                verbose
       -V <Volumes>      specify Volume names (separated by |)
       -w <dir>          specify working directory (default from conf file)
       -?                print this message
\end{verbatim}
\normalsize

As Bareos supports loading its database backend dynamically you need to specify
the right database driver to use using the {\bf -B} option.

If you are using MySQL or PostgreSQL, there is no need to supply a working
directory since in that case, bscan knows where the databases are. However, if
you have provided security on your database, you may need to supply either the
database name ({\bf -b} option), the user name ({\bf -u} option), and/or the
password ({\bf -p}) options.

NOTE: before {\bf bscan} can work, it needs at least a bare bones valid
database.  If your database exists but some records are missing because
they were pruned, then you are all set. If your database was lost or
destroyed, then you must first ensure that you have the SQL program running
(MySQL or PostgreSQL), then you must create the Bareos database (normally
named bareos), and you must create the Bareos tables.
This is explained in
\nameref{sec:CreateDatabase} chapter of the manual. Finally, before
scanning into an empty database, you must start and stop the Director with
the appropriate Bareos-dir.conf file so that it can create the Client and
Storage records which are not stored on the Volumes.  Without these
records, scanning is unable to connect the Job records to the proper
client.

Forgetting for the moment the extra complications of a full rebuild of
your catalog, let's suppose that you did a backup to Volumes "Vol001"
and "Vol002", then sometime later all records of one or both those
Volumes were pruned or purged from the
database. By using {\bf bscan} you can recreate the catalog entries for
those Volumes and then use the {\bf restore} command in the Console to restore
whatever you want. A command something like:

\footnotesize
\begin{verbatim}
bscan -v -V Vol001\|Vol002 /dev/nst0
\end{verbatim}
\normalsize

will give you an idea of what is going to happen without changing
your catalog. Of course, you may need to change the path to the Storage
daemon's conf file, the Volume name, and your tape (or disk) device name. This
command must read the entire tape, so if it has a lot of data, it may take a
long time, and thus you might want to immediately use the command listed
below. Note, if you are writing to a disk file, replace the device name with
the path to the directory that contains the Volumes. This must correspond to
the Archive Device in the conf file.

Then to actually write or store the records in the catalog, add the {\bf -s}
option as follows:

\footnotesize
\begin{verbatim}
bscan -s -m -v -V Vol001\|Vol002 /dev/nst0
\end{verbatim}
\normalsize

When writing to the database, if bscan finds existing records, it will
generally either update them if something is wrong or leave them alone. Thus
if the Volumes you are scanning are all or partially in the catalog already, no
harm will be done to that existing data. Any missing data will simply be
added.

If you have multiple tapes, you should scan them with:

\footnotesize
\begin{verbatim}
bscan -s -m -v -V Vol001\|Vol002\|Vol003 /dev/nst0
\end{verbatim}
\normalsize

Since there is a limit on the command line length (511 bytes) accepted
by {\bf bscan}, if you have too many Volumes, you will need to manually
create a bootstrap file.  See the \ilink{Bootstrap}{BootstrapChapter}
chapter of this manual for more details, in particular the section
entitled \ilink{Bootstrap for bscan}{bscanBootstrap}. Basically, the
.bsr file for the above example might look like:

\footnotesize
\begin{verbatim}
Volume=Vol001
Volume=Vol002
Volume=Vol003
\end{verbatim}
\normalsize

Note: {\bf bscan} does not support supplying Volume names on the
command line and at the same time in a bootstrap file.  Please
use only one or the other.

You should, always try to specify the tapes in the order they are written.
If you do not, any Jobs that span a volume may not be fully or properly
restored. However, bscan can handle scanning tapes that are not sequential.  Any
incomplete records at the end of the tape will simply be ignored in that
case.  If you are simply repairing an existing catalog, this may be OK, but
if you are creating a new catalog from scratch, it will leave your database
in an incorrect state.  If you do not specify all necessary Volumes on a
single bscan command, bscan will not be able to correctly restore the
records that span two volumes.  In other words, it is much better to
specify two or three volumes on a single bscan command (or in a .bsr file)
rather than run bscan two or three times, each with a single volume.

Note, the restoration process using bscan is not identical to the original
creation of the catalog data. This is because certain data such as Client
records and other non-essential data such
as volume reads, volume mounts, etc is not stored on the Volume, and thus is
not restored by bscan. The results of bscanning are, however, perfectly valid,
and will permit restoration of any or all the files in the catalog using the
normal Bareos console commands.  If you are starting with an empty catalog
and expecting bscan to reconstruct it, you may be a bit disappointed, but
at a minimum, you must ensure that your Bareos-dir.conf file is the same
as what it previously was -- that is, it must contain all the appropriate
Client resources so that they will be recreated in your new database {\bf
before} running bscan. Normally when the Director starts, it will recreate
any missing Client records in the catalog.  Another problem you will have
is that even if the Volumes (Media records) are recreated in the database,
they will not have their autochanger status and slots properly set. As a
result, you will need to repair that by using the {\bf update slots}
command.  There may be other considerations as well.  Rather than
bscanning, you should always attempt to recover you previous catalog
backup.


\subsubsection{Using bscan to Compare a Volume to an existing Catalog}
\index[general]{Catalog!Using bscan to Compare a Volume to an existing}

If you wish to compare the contents of a Volume to an existing catalog without
changing the catalog, you can safely do so if and only if you do {\bf not}
specify either the {\bf -m} or the {\bf -s} options.
However, the comparison routines are not as good or as thorough
as they should be, so we don't particularly recommend this mode other than for
testing.

\subsubsection{Using bscan to Recreate a Catalog from a Volume}
\index[general]{Catalog!Recreate Using bscan}
\index[general]{bscan!Recreate Catalog}

This is the mode for which {\bf bscan} is most useful. You can either {\bf
bscan} into a freshly created catalog, or directly into your existing catalog
(after having made an ASCII copy as described above). Normally, you should
start with a freshly created catalog that contains no data.

Starting with a single Volume named {\bf TestVolume1}, you run a command such
as:

\footnotesize
\begin{verbatim}
bscan -V TestVolume1 -v -s -m /dev/nst0
\end{verbatim}
\normalsize

If there is more than one volume, simply append it to the first one separating
it with a vertical bar. You may need to precede the vertical bar with a
forward slash escape the shell -- e.g. {\bf
TestVolume1\textbackslash{}|TestVolume2}. The {\bf -v} option was added for
verbose output (this can be omitted if desired). The {\bf -s} option that
tells {\bf bscan} to store information in the database. The physical device
name {\bf /dev/nst0} is specified after all the options.

{\bf} For example, after having done a full backup of a directory, then two
incrementals, I reinitialized the SQLite database as described above, and
using the bootstrap.bsr file noted above, I entered the following command:

\footnotesize
\begin{verbatim}
bscan -b bootstrap.bsr -v -s /dev/nst0
\end{verbatim}
\normalsize

which produced the following output:

\footnotesize
\begin{verbatim}
bscan: bscan.c:182 Using Database: Bareos, User: bacula
bscan: bscan.c:673 Created Pool record for Pool: Default
bscan: bscan.c:271 Pool type "Backup" is OK.
bscan: bscan.c:632 Created Media record for Volume: TestVolume1
bscan: bscan.c:298 Media type "DDS-4" is OK.
bscan: bscan.c:307 VOL_LABEL: OK for Volume: TestVolume1
bscan: bscan.c:693 Created Client record for Client: Rufus
bscan: bscan.c:769 Created new JobId=1 record for original JobId=2
bscan: bscan.c:717 Created FileSet record "Users Files"
bscan: bscan.c:819 Updated Job termination record for new JobId=1
bscan: bscan.c:905 Created JobMedia record JobId 1, MediaId 1
bscan: Got EOF on device /dev/nst0
bscan: bscan.c:693 Created Client record for Client: Rufus
bscan: bscan.c:769 Created new JobId=2 record for original JobId=3
bscan: bscan.c:708 Fileset "Users Files" already exists.
bscan: bscan.c:819 Updated Job termination record for new JobId=2
bscan: bscan.c:905 Created JobMedia record JobId 2, MediaId 1
bscan: Got EOF on device /dev/nst0
bscan: bscan.c:693 Created Client record for Client: Rufus
bscan: bscan.c:769 Created new JobId=3 record for original JobId=4
bscan: bscan.c:708 Fileset "Users Files" already exists.
bscan: bscan.c:819 Updated Job termination record for new JobId=3
bscan: bscan.c:905 Created JobMedia record JobId 3, MediaId 1
bscan: Got EOF on device /dev/nst0
bscan: bscan.c:652 Updated Media record at end of Volume: TestVolume1
bscan: bscan.c:428 End of Volume. VolFiles=3 VolBlocks=57 VolBytes=10,027,437
\end{verbatim}
\normalsize

The key points to note are that {\bf bscan} prints a line when each major
record is created. Due to the volume of output, it does not print a line for
each file record unless you supply the {\bf -v} option twice or more on the
command line.

In the case of a Job record, the new JobId will not normally be the same as
the original Jobid. For example, for the first JobId above, the new JobId is
1, but the original JobId is 2. This is nothing to be concerned about as it is
the normal nature of databases. {\bf bscan} will keep everything straight.

Although {\bf bscan} claims that it created a Client record for Client: Rufus
three times, it was actually only created the first time. This is normal.

You will also notice that it read an end of file after each Job (Got EOF on
device ...). Finally the last line gives the total statistics for the bscan.

If you had added a second {\bf -v} option to the command line, Bareos would
have been even more verbose, dumping virtually all the details of each Job
record it encountered.

Now if you start Bareos and enter a {\bf list jobs} command to the console
program, you will get:

\footnotesize
\begin{verbatim}
+-------+----------+------------------+------+-----+----------+----------+---------+
| JobId | Name     | StartTime        | Type | Lvl | JobFiles | JobBytes | JobStat |
+-------+----------+------------------+------+-----+----------+----------+---------+
| 1     | usersave | 2002-10-07 14:59 | B    | F   | 84       | 4180207  | T       |
| 2     | usersave | 2002-10-07 15:00 | B    | I   | 15       | 2170314  | T       |
| 3     | usersave | 2002-10-07 15:01 | B    | I   | 33       | 3662184  | T       |
+-------+----------+------------------+------+-----+----------+----------+---------+
\end{verbatim}
\normalsize

which corresponds virtually identically with what the database contained
before it was re-initialized and restored with bscan. All the Jobs and Files
found on the tape are restored including most of the Media record. The Volume
(Media) records restored will be marked as {\bf Full} so that they cannot be
rewritten without operator intervention.

It should be noted that {\bf bscan} cannot restore a database to the exact
condition it was in previously because a lot of the less important information
contained in the database is not saved to the tape. Nevertheless, the
reconstruction is sufficiently complete, that you can run {\bf restore}
against it and get valid results.

An interesting aspect of restoring a catalog backup using {\bf bscan} is
that the backup was made while Bareos was running and writing to a tape. At
the point the backup of the catalog is made, the tape Bareos is writing to
will have say 10 files on it, but after the catalog backup is made, there
will be 11 files on the tape Bareos is writing.  This there is a difference
between what is contained in the backed up catalog and what is actually on
the tape.  If after restoring a catalog, you attempt to write on the same
tape that was used to backup the catalog, Bareos will detect the difference
in the number of files registered in the catalog compared to what is on the
tape, and will mark the tape in error.

There are two solutions to this problem. The first is possibly the simplest
and is to mark the volume as Used before doing any backups.  The second is
to manually correct the number of files listed in the Media record of the
catalog.  This procedure is documented elsewhere in the manual and involves
using the {\bf update volume} command in {\bf bconsole}.

\subsubsection{Using bscan to Correct the Volume File Count}
\index[general]{bscan!Correct Volume File Count}
\index[general]{Volume!File Count}

If the Storage daemon crashes during a backup Job, the catalog will not be
properly updated for the Volume being used at the time of the crash. This
means that the Storage daemon will have written say 20 files on the tape, but
the catalog record for the Volume indicates only 19 files.

Bareos refuses to write on a tape that contains a different number of files
from what is in the catalog. To correct this situation, you may run a {\bf
bscan} with the {\bf -m} option (but {\bf without} the {\bf -s} option) to
update only the final Media record for the Volumes read.

\subsubsection{After bscan}
\index[general]{bscan!after}

If you use {\bf bscan} to enter the contents of the Volume into an existing
catalog, you should be aware that the records you entered may be immediately
pruned during the next job, particularly if the Volume is very old or had been
previously purged. To avoid this, after running {\bf bscan}, you can manually
set the volume status (VolStatus) to {\bf Read-Only} by using the {\bf update}
command in the catalog. This will allow you to restore from the volume without
having it immediately purged. When you have restored and backed up the data,
you can reset the VolStatus to {\bf Used} and the Volume will be purged from
the catalog.

\subsection{bcopy}
\label{bcopy}
\index[general]{bcopy}
\index[general]{Command!bcopy}

The {\bf bcopy} program can be used to copy one {\bf Bareos} archive file to
another. For example, you may copy a tape to a file, a file to a tape, a file
to a file, or a tape to a tape. For tape to tape, you will need two tape
drives. In the
process of making the copy, no record of the information written to the new
Volume is stored in the catalog. This means that the new Volume, though it
contains valid backup data, cannot be accessed directly from existing catalog
entries. If you wish to be able to use the Volume with the Console restore
command, for example, you must first bscan the new Volume into the catalog.

\footnotesize
\begin{verbatim}
Usage: bcopy [-d debug_level] <input-archive> <output-archive>
       -b bootstrap    specify a bootstrap file
       -c <file>       specify configuration file
       -D <director>   specify a director name specified in the Storage
                       configuration file for the Key Encryption Key selection
       -dnn            set debug level to nn
       -dt             print timestamp in debug output
       -i              specify input Volume names (separated by |)
       -o              specify output Volume names (separated by |)
       -p              proceed inspite of I/O errors
       -v              verbose
       -w dir          specify working directory (default /tmp)
       -?              print this message
\end{verbatim}
\normalsize

By using a {\bf bootstrap} file, you can copy parts of a Bareos archive file
to another archive.

One of the objectives of this program is to be able to recover as much data as
possible from a damaged tape. However, the current version does not yet have
this feature.

As this is a new program, any feedback on its use would be appreciated. In
addition, I only have a single tape drive, so I have never been able to test
this program with two tape drives.

\subsection{btape}
\label{btape}
\index[general]{btape}
\index[general]{Command!btape}

This program permits a number of elementary tape operations via a tty command
interface. It works only with tapes and not with other kinds of Bareos
storage media (DVD, File, ...).
The {\bf test} command, described below,
can be very useful for testing older tape drive compatibility problems.
Aside from initial testing of tape drive compatibility with {\bf Bareos},
{\bf btape} will be mostly used by developers writing new tape drivers.

{\bf btape} can be dangerous to use with existing {\bf Bareos} tapes because
it will relabel a tape or write on the tape if so requested regardless that
the tape may contain valuable data, so please be careful and use it only on
blank tapes.

To work properly, {\bf btape} needs to read the Storage daemon's configuration
file. As a default, it will look for {\bf Bareos-sd.conf} in the current
directory. If your configuration file is elsewhere, please use the {\bf -c}
option to specify where.

The physical device name must be specified on the command line, and this
same device name must be present in the Storage daemon's configuration file
read by {\bf btape}

\footnotesize
\begin{verbatim}
Usage: btape <options> <device_name>
       -b <file>     specify bootstrap file
       -c <file>     set configuration file to file
       -D <director> specify a director name specified in the Storage
                     configuration file for the Key Encryption Key selection
       -d <nn>       set debug level to nn
       -dt           print timestamp in debug output
       -p            proceed inspite of I/O errors
       -s            turn off signals
       -v            be verbose
       -?            print this message.
\end{verbatim}
\normalsize

\subsubsection{Using btape to Verify your Tape Drive}
\index[general]{Drive!Verify using btape}

An important reason for this program is to ensure that a Storage daemon
configuration file is defined so that Bareos will correctly read and write
tapes.

It is highly recommended that you run the {\bf test} command before running
your first Bareos job to ensure that the parameters you have defined for your
storage device (tape drive) will permit {\bf Bareos} to function properly. You
only need to mount a blank tape, enter the command, and the output should be
reasonably self explanatory. Please see the
\ilink{Tape Testing}{TapeTestingChapter} Chapter of this manual for
the details.

\subsubsection{btape Commands}

The full list of commands are:

\footnotesize
\begin{verbatim}
  Command    Description
  =======    ===========
  autochanger test autochanger
  bsf        backspace file
  bsr        backspace record
  cap        list device capabilities
  clear      clear tape errors
  eod        go to end of Bareos data for append
  eom        go to the physical end of medium
  fill       fill tape, write onto second volume
  unfill     read filled tape
  fsf        forward space a file
  fsr        forward space a record
  help       print this command
  label      write a Bareos label to the tape
  load       load a tape
  quit       quit btape
  rawfill    use write() to fill tape
  readlabel  read and print the Bareos tape label
  rectest    test record handling functions
  rewind     rewind the tape
  scan       read() tape block by block to EOT and report
  scanblocks Bareos read block by block to EOT and report
  speed      report drive speed
  status     print tape status
  test       General test Bareos tape functions
  weof       write an EOF on the tape
  wr         write a single Bareos block
  rr         read a single record
  qfill      quick fill command
\end{verbatim}
\normalsize

The most useful commands are:

\begin{itemize}
\item test -- test writing records and EOF marks and  reading them back.
\item fill -- completely fill a volume with records, then  write a few records
   on a second volume, and finally,  both volumes will be read back.
   This command writes blocks containing random data, so your drive will
   not be able to compress the data, and thus it is a good test of
   the real physical capacity of your tapes.
\item readlabel -- read and dump the label on a Bareos tape.
\item cap -- list the device capabilities as defined in the  configuration
   file and as perceived by the Storage daemon.
\end{itemize}

The {\bf readlabel} command can be used to display the details of a Bareos
tape label. This can be useful if the physical tape label was lost or damaged.

In the event that you want to relabel a Bareos volume, you can simply use the
{\bf label} command which will write over any existing label. However, please
note for labeling tapes, we recommend that you use the {\bf label} command in
the {\bf Console} program since it will never overwrite a valid Bareos tape.

\paragraph{Testing your Tape Drive}
\label{sec:btapespeed}

To determine the best configuration of your tape drive, you can run the new
\texttt{speed} command available in the \texttt{btape} program.

This command can have the following arguments:
\begin{itemize}
\item[\texttt{file\_size=n}] Specify the Maximum File Size for this test
  (between 1 and 5GB). This counter is in GB.
\item[\texttt{nb\_file=n}] Specify the number of file to be written. The amount
  of data should be greater than your memory ($file\_size*nb\_file$).
\item[\texttt{skip\_zero}] This flag permits to skip tests with constant
  data.
\item[\texttt{skip\_random}] This flag permits to skip tests with random
  data.
\item[\texttt{skip\_raw}] This flag permits to skip tests with raw access.
\item[\texttt{skip\_block}] This flag permits to skip tests with Bareos block
  access.
\end{itemize}

\begin{verbatim}
*speed file_size=3 skip_raw
btape.c:1078 Test with zero data and Bareos block structure.
btape.c:956 Begin writing 3 files of 3.221 GB with blocks of 129024 bytes.
++++++++++++++++++++++++++++++++++++++++++
btape.c:604 Wrote 1 EOF to "Drive-0" (/dev/nst0)
btape.c:406 Volume bytes=3.221 GB. Write rate = 44.128 MB/s
...
btape.c:383 Total Volume bytes=9.664 GB. Total Write rate = 43.531 MB/s

btape.c:1090 Test with random data, should give the minimum throughput.
btape.c:956 Begin writing 3 files of 3.221 GB with blocks of 129024 bytes.
+++++++++++++++++++++++++++++++++++++++++++
btape.c:604 Wrote 1 EOF to "Drive-0" (/dev/nst0)
btape.c:406 Volume bytes=3.221 GB. Write rate = 7.271 MB/s
+++++++++++++++++++++++++++++++++++++++++++
...
btape.c:383 Total Volume bytes=9.664 GB. Total Write rate = 7.365 MB/s
\end{verbatim}

When using compression, the random test will give your the minimum throughput
of your drive . The test using constant string will give you the maximum speed
of your hardware chain. (cpu, memory, scsi card, cable, drive, tape).

You can change the block size in the Storage Daemon configuration file.




\section{Other Programs}

The following programs are general utility programs and in general do not need
a configuration file nor a device name.


\subsection{bsmtp}
\label{bsmtp}
\index[general]{bsmtp}
\index[general]{Command!bsmtp}

\command{bsmtp} is a simple mail transport program that permits more flexibility
than the standard mail programs typically found on Unix systems. It can even
be used on Windows machines.

It is called:
\begin{commands}{bsmtp}
Usage: bsmtp [-f from] [-h mailhost] [-s subject] [-c copy] [recipient ...t
       -4          forces bsmtp to use IPv4 addresses only.
       -6          forces bsmtp to use IPv6 addresses only.
       -8          set charset to UTF-8
       -a          use any ip protocol for address resolution
       -c          set the Cc: field
       -d <nn>     set debug level to <nn>
       -dt         print a timestamp in debug output
       -f          set the From: field
       -h          use mailhost:port as the SMTP server
       -s          set the Subject: field
       -r          set the Reply-To: field
       -l          set the maximum number of lines to send (default: unlimited)
       -?          print this message.
\end{commands}

If the {\bf -f} option is not specified, \command{bsmtp} will use your userid. If
the option {\bf -h} is not specified \command{bsmtp} will use the value in the environment
variable {\bf bsmtpSERVER} or if there is none {\bf localhost}. By default
port 25 is used.

If a line count limit is set with the {\bf -l} option, \command{bsmtp} will
not send an email with a body text exceeding that number of lines. This
is especially useful for large restore job reports where the list of
files restored might produce very long mails your mail-server would
refuse or crash. However, be aware that you will probably suppress the
job report and any error messages unless you check the log file written
by the Director (see the messages resource in this manual for details).


{\bf recipients} is a space separated list of email recipients.

The body of the email message is read from standard input.

An example of the use of \command{bsmtp} would be to put the following statement
in the \ilink{Messages resource}{MessagesResource} of your \file{bareos-dir.conf} file.

\begin{bconfig}{bsmtp in Message resource}
Mail Command     = "bsmtp -h mail.example.com -f \"\(Bareos\) %r\" -s \"Bareos: %t %e of %c %l\" %r"
Operator Command = "bsmtp -h mail.example.com -f \"\(Bareos\) %r\" -s \"Bareos: Intervention needed for %j\" %r"
\end{bconfig}

You have to replace {\bf mail.example.com} with the fully
qualified name of your SMTP (email) server, which normally listens on port
25. For more details on the substitution characters (e.g. \%r) used in the
above line, please see the documentation of the
\ilink{MailCommand in the Messages Resource}{mailcommand}
chapter of this manual.

It is HIGHLY recommended that you test one or two cases by hand to make sure
that the {\bf mailhost} that you specified is correct and that it will accept
your email requests. Since {\bf bsmtp} always uses a TCP connection rather
than writing in the spool file, you may find that your {\bf from} address is
being rejected because it does not contain a valid domain, or because your
message is caught in your spam filtering rules. Generally, you should specify
a fully qualified domain name in the {\bf from} field, and depending on
whether your bsmtp gateway is Exim or Sendmail, you may need to modify the
syntax of the from part of the message. Please test.

When running \command{bsmtp} by hand, you will need to terminate the message by
entering a ctrl-d in column 1 of the last line.
% TODO: is "column" the correct terminology for this?

If you are getting incorrect dates (e.g. 1970) and you are
running with a non-English language setting, you might try adding
a \command{LANG=C} immediately before the \command{bsmtp} call.

In general, \command{bsmtp} attempts to cleanup email addresses that you
specify in the from, copy, mailhost, and recipient fields, by adding
the necessary {\textless} and {\textgreater} characters around the address part.  However,
if you include a {\bf display-name} (see RFC 5332), some SMTP servers
such as Exchange may not accept the message if the {\bf display-name} is
also included in {\textless} and {\textgreater}.  As mentioned above, you must test, and
if you run into this situation, you may manually add the {\textless} and {\textgreater}
to the Bareos \linkResourceDirective{Dir}{Messages}{Mail Command} or \linkResourceDirective{Dir}{Messages}{Operator Command} and when
\command{bsmtp} is formatting an address if it already contains a {\textless} or
{\textgreater} character, it will leave the address unchanged.

\subsection{bareos-dbcheck}
    \label{bareos-dbcheck}
    \label{dbcheck}
\index[general]{bareos-dbcheck}
\index[general]{Command!bareos-dbcheck}
\index[general]{Catalog!database check}


{\bf bareos-dbcheck} is a simple program that will search for logical
inconsistencies in the Bareos tables in your database, and optionally fix them.
It is a database maintenance routine, in the sense that it can
detect and remove unused rows, but it is not a database repair
routine. To repair a database, see the tools furnished by the
database vendor.  Normally bareos-dbcheck should never need to be run,
but if Bareos has crashed or you have a lot of Clients, Pools, or
Jobs that you have removed, it could be useful.

It is called:

\footnotesize
\begin{verbatim}
Usage: dbcheck [-c config ] [-B] [-C catalog name] [-d debug level] [-D driver name] <working-directory> <bareos-database> <user> <password> [<dbhost>] [<dbport>]
       -b                batch mode
       -C                catalog name in the director conf file
       -c                Director conf filename
       -B                print catalog configuration and exit
       -d <nn>           set debug level to <nn>
       -dt               print a timestamp in debug output
       -D <driver name>  specify the database driver name (default NULL) <postgresql|mysql|sqlite>
       -f                fix inconsistencies
       -v                verbose
       -?                print this message
\end{verbatim}
\normalsize

As Bareos supports loading its database backend dynamically you need to specify
the right database driver to use using the {\bf -D} option.

If the \textbf{-B} option is specified, bareos-dbcheck will print out catalog
information in a simple text based format. This is useful to backup it in a
secure way.

\begin{verbatim}
 $ bareos-dbcheck -B
 catalog=MyCatalog
 db_type=SQLite
 db_name=regress
 db_driver=
 db_user=regress
 db_password=
 db_address=
 db_port=0
 db_socket=
\end{verbatim} %$

If the {\bf -c} option is given with the Director's conf file, there is no
need to enter any of the command line arguments, in particular the working
directory as dbcheck will read them from the file.

If the {\bf -f} option is specified, {\bf bareos-dbcheck} will repair ({\bf fix}) the
inconsistencies it finds. Otherwise, it will report only.

If the {\bf -b} option is specified, {\bf bareos-dbcheck} will run in batch mode, and
it will proceed to examine and fix (if -f is set) all programmed inconsistency
checks. If the {\bf -b} option is not specified, {\bf bareos-dbcheck} will enter
interactive mode and prompt with the following:

\footnotesize
\begin{verbatim}
Hello, this is the database check/correct program.
Please select the function you want to perform.
     1) Toggle modify database flag
     2) Toggle verbose flag
     3) Repair bad Filename records
     4) Repair bad Path records
     5) Eliminate duplicate Filename records
     6) Eliminate duplicate Path records
     7) Eliminate orphaned Jobmedia records
     8) Eliminate orphaned File records
     9) Eliminate orphaned Path records
    10) Eliminate orphaned Filename records
    11) Eliminate orphaned FileSet records
    12) Eliminate orphaned Client records
    13) Eliminate orphaned Job records
    14) Eliminate all Admin records
    15) Eliminate all Restore records
    16) All (3-15)
    17) Quit
Select function number:
\end{verbatim}
\normalsize

By entering 1 or 2, you can toggle the modify database flag (-f option) and
the verbose flag (-v). It can be helpful and reassuring to turn off the modify
database flag, then select one or more of the consistency checks (items 3
through 9) to see what will be done, then toggle the modify flag on and re-run
the check.

The inconsistencies examined are the following:

\begin{itemize}
\item Duplicate filename records. This can happen if you accidentally run  two
   copies of Bareos at the same time, and they are both adding  filenames
   simultaneously. It is a rare occurrence, but will create  an inconsistent
   database. If this is the case, you will receive  error messages during Jobs
   warning of duplicate database records.  If you are not getting these error
   messages, there is no reason  to run this check.
\item Repair bad Filename records. This checks and corrects filenames  that
   have a trailing slash. They should not.
\item Repair bad Path records. This checks and corrects path names  that do
   not have a trailing slash. They should.
\item Duplicate path records. This can happen if you accidentally run  two
   copies of Bareos at the same time, and they are both adding  filenames
   simultaneously. It is a rare occurrence, but will create  an inconsistent
   database. See the item above for why this occurs and  how you know it is
   happening.
\item Orphaned JobMedia records. This happens when a Job record is deleted
   (perhaps by a user issued SQL statement), but the corresponding  JobMedia
   record (one for each Volume used in the Job) was not deleted.  Normally, this
   should not happen, and even if it does, these records  generally do not take
   much space in your database. However, by running  this check, you can
   eliminate any such orphans.
\item Orphaned File records. This happens when a Job record is deleted
   (perhaps by a user issued SQL statement), but the corresponding  File record
   (one for each Volume used in the Job) was not deleted.  Note, searching for
   these records can be {\bf very} time consuming (i.e.  it may take hours) for a
   large database. Normally this should not  happen as Bareos takes care to
   prevent it. Just the same, this  check can remove any orphaned File records.
   It is recommended that  you run this once a year since orphaned File records
   can take a  large amount of space in your database. You might
   want to ensure that you have indexes on JobId, FilenameId, and
   PathId for the File table in your catalog before running this
   command.
\item Orphaned Path records. This condition happens any time a directory is
   deleted from your system and all associated Job records have been purged.
   During standard purging (or pruning) of Job records, Bareos does  not check
   for orphaned Path records. As a consequence, over a period  of time, old
   unused Path records will tend to accumulate and use  space in your database.
   This check will eliminate them. It is recommended that you run this
   check at least once a year.
\item Orphaned Filename records. This condition happens any time a file is
   deleted from your system and all associated Job records have been purged.
   This can happen quite frequently as there are quite a large number  of files
   that are created and then deleted. In addition, if you  do a system update or
   delete an entire directory, there can be  a very large number of Filename
   records that remain in the catalog  but are no longer used.

   During standard purging (or pruning) of Job records, Bareos does  not check
   for orphaned Filename records. As a consequence, over a period  of time, old
   unused Filename records will accumulate and use  space in your database. This
   check will eliminate them. It is strongly  recommended that you run this check
   at least once a year, and for  large database (more than 200 Megabytes), it is
   probably better to  run this once every 6 months.
\item Orphaned Client records. These records can remain in the database  long
   after you have removed a client.
\item Orphaned Job records. If no client is defined for a job or you  do not
   run a job for a long time, you can accumulate old job  records. This option
   allow you to remove jobs that are not  attached to any client (and thus
   useless).
\item All Admin records. This command will remove all Admin records,
   regardless of their age.
\item All Restore records. This command will remove all Restore records,
   regardless of their age.
\end{itemize}


If you are using MySQL, \command{bareos-dbcheck} will ask you if you want to create temporary
indexes to speed up orphaned Path and Filename elimination.

%\index[general]{bvfs}
If you are using bvfs (e.g. used by \ilink{bareos-webui}{sec:webui}),
don't eliminate orphaned path, else you will
have to rebuild \variable{brestore_pathvisibility} and
\variable{brestore_pathhierarchy} indexes.

Normally
you should never need to run \command{bareos-dbcheck} in spite of the
recommendations given above, which are given so that users don't
waste their time running bareos-dbcheck too often.

\subsection{bregex}
\label{bregex}
\index[general]{bregex}
\index[general]{Command!bregex}

{\bf bregex} is a simple program that will allow you to test
regular expressions against a file of data. This can be useful
because the regex libraries on most systems differ, and in
addition, regex expressions can be complicated.

{\bf bregex} is found in the src/tools directory and it is
normally installed with your system binaries. To run it, use:

\begin{verbatim}
Usage: bregex [-d debug_level] -f <data-file>
       -f          specify file of data to be matched
       -l          suppress line numbers
       -n          print lines that do not match
       -?          print this message.
\end{verbatim}

The {\textless}data-file{\textgreater} is a filename that contains lines
of data to be matched (or not) against one or more patterns.
When the program is run, it will prompt you for a regular
expression pattern, then apply it one line at a time against
the data in the file. Each line that matches will be printed
preceded by its line number.  You will then be prompted again
for another pattern.

Enter an empty line for a pattern to terminate the program. You
can print only lines that do not match by using the -n option,
and you can suppress printing of line numbers with the -l option.

This program can be useful for testing regex expressions to be
applied against a list of filenames.

\subsection{bwild}
\label{bwild}
\index[general]{bwild}
\index[general]{Command!bwild}

{\bf bwild} is a simple program that will allow you to test
wild-card expressions against a file of data.

{\bf bwild} is found in the src/tools directory and it is
normally installed with your system binaries. To run it, use:

\begin{verbatim}
Usage: bwild [-d debug_level] -f <data-file>
       -f          specify file of data to be matched
       -l          suppress line numbers
       -n          print lines that do not match
       -?          print this message.
\end{verbatim}

The {\textless}data-file{\textgreater} is a filename that contains lines
of data to be matched (or not) against one or more patterns.
When the program is run, it will prompt you for a wild-card
pattern, then apply it one line at a time against
the data in the file. Each line that matches will be printed
preceded by its line number.  You will then be prompted again
for another pattern.

Enter an empty line for a pattern to terminate the program. You
can print only lines that do not match by using the -n option,
and you can suppress printing of line numbers with the -l option.

This program can be useful for testing wild expressions to be
applied against a list of filenames.

% \section{testfind}
% \label{testfind}
% \index[general]{Testfind}
% \index[general]{Command!testfind}
%
% {\bf testfind} permits listing of files using the same search engine that is
% used for the {\bf Include} resource in Job resources. Note, much of the
% functionality of this program (listing of files to be included) is present in
% the
% \ilink{estimate command}{estimate} in the Console program.
%
% The original use of testfind was to ensure that Bareos's file search engine
% was correct and to print some statistics on file name and path length.
% However, you may find it useful to see what Bareos would do with a given {\bf
% Include} resource. The {\bf testfind} program can be found in the {\bf
% {\textless}Bareos-source{\textgreater}/src/tools} directory of the source distribution.
% Though it is built with the make process, it is not normally "installed".
%
% It is called:
%
% \footnotesize
% \begin{verbatim}
% Usage: testfind [-d debug_level] [-] [pattern1 ...]
%        -a          print extended attributes (Win32 debug)
%        -dnn        set debug level to nn
%        -           read pattern(s) from stdin
%        -?          print this message.
% Patterns are used for file inclusion -- normally directories.
% Debug level>= 1 prints each file found.
% Debug level>= 10 prints path/file for catalog.
% Errors are always printed.
% Files/paths truncated is a number with len> 255.
% Truncation is only in the catalog.
% \end{verbatim}
% \normalsize
%
% Where a pattern is any filename specification that is valid within an {\bf
% Include} resource definition. If none is specified, {\bf /} (the root
% directory) is assumed. For example:
%
% \footnotesize
% \begin{verbatim}
% testfind /bin
% \end{verbatim}
% \normalsize
%
% Would print the following:
%
% \footnotesize
% \begin{verbatim}
% Dir: /bin
% Reg: /bin/bash
% Lnk: /bin/bash2 -> bash
% Lnk: /bin/sh -> bash
% Reg: /bin/cpio
% Reg: /bin/ed
% Lnk: /bin/red -> ed
% Reg: /bin/chgrp
% ...
% Reg: /bin/ipcalc
% Reg: /bin/usleep
% Reg: /bin/aumix-minimal
% Reg: /bin/mt
% Lnka: /bin/gawk-3.1.0 -> /bin/gawk
% Reg: /bin/pgawk
% Total files    : 85
% Max file length: 13
% Max path length: 5
% Files truncated: 0
% Paths truncated: 0
% \end{verbatim}
% \normalsize
%
% Even though {\bf testfind} uses the same search engine as {\bf Bareos}, each
% directory to be listed, must be entered as a separate command line entry or
% entered one line at a time to standard input if the {\bf -} option was
% specified.
%
% Specifying a debug level of one (i.e. {\bf -d1}) on the command line will
% cause {\bf testfind} to print the raw filenames without showing the Bareos
% internal file type, or the link (if any). Debug levels of 10 or greater cause
% the filename and the path to be separated using the same algorithm that is
% used when putting filenames into the Catalog database.


\subsection{bpluginfo}
\label{bpluginfo}
\index[general]{bpluginfo}
\index[general]{Command!bpluginfo}

The main purpose of bpluginfo is to display different information about Bareos plugin. You can
use it to check a plugin name, author, license and short description. You can use -f option  to
display API implemented by the plugin. Some plugins may require additional '-a' option for val-
idating a Bareos Daemons API. In most cases it is not required.
