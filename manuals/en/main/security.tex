%%
%%

\chapter{Bareos Security Issues}
\label{SecurityChapter}
\index[general]{Security}

\begin{itemize}
\item Security means being able to restore your files, so read the
   \ilink{Critical Items Chapter}{Critical} of this manual.
\item The clients ({\bf bareos-fd}) must run as root to be able to access  all
   the system files.
\item It is not necessary to run the Director as root.
\item It is not necessary to run the Storage daemon as root, but you  must
   ensure that it can open the tape drives, which are often restricted to root
   access by default. In addition, if you do not run the Storage daemon as root,
   it will not be able to automatically set your tape drive parameters on most
   OSes since these functions, unfortunately require root access.
\item You should restrict access to the Bareos configuration files,  so that
   the passwords are not world-readable. The {\bf Bareos}  daemons are password
   protected using CRAM-MD5 (i.e. the password is not  sent across the network).
   This will ensure that not everyone  can access the daemons. It is a reasonably
   good protection, but  can be cracked by experts.
\item If you are using the recommended ports 9101, 9102, and 9103, you  will
   probably want to protect these ports from external access  using a firewall
   and/or using tcp wrappers ({\bf etc/hosts.allow}).
\item By default, all data that is sent across the network is unencrypted.
   However, Bareos does support TLS (transport layer security) and can
   encrypt transmitted data.  Please read the
   \ilink{TLS (SSL) Communications Encryption}{CommEncryption}
   section of this manual.
\item You should ensure that the Bareos working directories are  readable and
   writable only by the Bareos daemons.
\item The default Bareos \command{grant_bareos_privileges} script  grants all
   permissions to use the MySQL (and PostgreSQL) database without a  password. If you want
   security, please tighten this up!
\item Don't forget that Bareos is a network program, so anyone anywhere  on
   the network with the console program and the Director's password  can access
   Bareos and the backed up data.
\item You can restrict what IP addresses Bareos will bind to by using the
   appropriate {\bf DirAddress}, {\bf FDAddress}, or {\bf SDAddress}  records in
   the respective daemon configuration files.
\end{itemize}



\label{wrappers}
\section{Configuring and Testing TCP Wrappers}
\index[general]{Configuring and Testing TCP Wrappers}
\index[general]{TCP Wrappers}
\index[general]{Wrappers!TCP}
\index[general]{libwrappers}

The TCP wrapper functionality is available on different plattforms.
Be default, it is activated on Bareos for Linux.
With this enabled, you may control who may access your
daemons.  This control is done by modifying the file: {\bf
/etc/hosts.allow}.  The program name that {\bf Bareos} uses when
applying these access restrictions is the name you specify in the
daemon configuration file (see below for examples).
You must not use the {\bf twist} option in your {\bf
/etc/hosts.allow} or it will terminate the Bareos daemon when a
connection is refused.

\hide{
Dan Langille has provided the following information on configuring and
testing TCP wrappers with Bareos.

If you read hosts\_options(5), you will see an option called twist. This
option replaces the current process by an instance of the specified shell
command. Typically, something like this is used:

\footnotesize
\begin{verbatim}
ALL : ALL \
 : severity auth.info \
 : twist /bin/echo "You are not welcome to use %d from %h."
\end{verbatim}
\normalsize

The libwrap code tries to avoid {\bf twist} if it runs in a resident process,
but that test will not protect the first hosts\_access() call. This will
result in the process (e.g. bareos-fd, bareos-sd, bareos-dir) being terminated
if the first connection to their port results in the twist option being
invoked. The potential, and I stress potential, exists for an attacker to
prevent the daemons from running. This situation is eliminated if your
/etc/hosts.allow file contains an appropriate rule set. The following example
is sufficient:

\footnotesize
\begin{verbatim}
undef-fd : localhost : allow
undef-sd : localhost : allow
undef-dir : localhost : allow
undef-fd : ALL : deny
undef-sd : ALL : deny
undef-dir : ALL : deny
\end{verbatim}
\normalsize

You must adjust the names to be the same as the Name directives found
in each of the daemon configuration files. They are, in general, not the
same as the binary daemon names. It is not possible to use the
daemon names because multiple daemons may be running on the same machine
but with different configurations.

In these examples, the Director is undef-dir, the
Storage Daemon is undef-sd, and the File Daemon is undef-fd. Adjust to suit
your situation. The above example rules assume that the SD, FD, and DIR all
reside on the same box. If you have a remote FD client, then the following
rule set on the remote client will suffice:

\footnotesize
\begin{verbatim}
undef-fd : director.example.org : allow
undef-fd : ALL : deny
\end{verbatim}
\normalsize

where director.example.org is the host which will be contacting the client
(ie. the box on which the Bareos Director daemon runs). The use of "ALL :
deny" ensures that the twist option (if present) is not invoked. To properly
test your configuration, start the daemon(s), then attempt to connect from an
IP address which should be able to connect. You should see something like
this:

\footnotesize
\begin{verbatim}
$ telnet undef 9103
Trying 192.168.0.56...
Connected to undef.example.org.
Escape character is '^]'.
Connection closed by foreign host.
$
\end{verbatim}
\normalsize

This is the correct response. If you see this:

\footnotesize
\begin{verbatim}
$ telnet undef 9103
Trying 192.168.0.56...
Connected to undef.example.org.
Escape character is '^]'.
You are not welcome to use undef-sd from xeon.example.org.
Connection closed by foreign host.
$
\end{verbatim}
\normalsize

then twist has been invoked and your configuration is not correct and you need
to add the deny statement. It is important to note that your testing must
include restarting the daemons after each connection attempt. You can also
tcpdchk(8) and tcpdmatch(8) to validate your /etc/hosts.allow rules. Here is a
simple test using tcpdmatch:

\footnotesize
\begin{verbatim}
$ tcpdmatch undef-dir xeon.example.org
warning: undef-dir: no such process name in /etc/inetd.conf
client: hostname xeon.example.org
client: address 192.168.0.18
server: process undef-dir
matched: /etc/hosts.allow line 40
option: allow
access: granted
\end{verbatim}
\normalsize

If you are running Bareos as a standalone daemon, the warning above can be
safely ignored. Here is an example which indicates that your rules are missing
a deny statement and the twist option has been invoked.

\footnotesize
\begin{verbatim}
$ tcpdmatch undef-dir 10.0.0.1
warning: undef-dir: no such process name in /etc/inetd.conf
client: address 10.0.0.1
server: process undef-dir
matched: /etc/hosts.allow line 91
option: severity auth.info
option: twist /bin/echo "You are not welcome to use
  undef-dir from 10.0.0.1."
access: delegated
\end{verbatim}
\normalsize
}

\hide{
\section{Running as non-root}
\index[general]{Running as non-root}

Security advice from Dan Langille:
% TODO: don't use specific name

% TODO: don't be too specific on operating system

% TODO: maybe remove personalization?

It is a good idea to run daemons with the lowest possible privileges.  In
other words, if you can, don't run applications as root which do  not have to
be root.  The Storage Daemon and the Director Daemon do not need to be root.
The  File Daemon needs to be root in order to access all files on your system.
In order to run as non-root, you need to create a user and a group.  Choosing
{\tt bareos} as both the user name and the group name sounds like a good idea
to me.

The FreeBSD port creates this user and group for you.
Here is what those entries looked like on my FreeBSD laptop:

\footnotesize
\begin{verbatim}
bareos:*:1002:1002::0:0:Bareos Daemon:/var/db/bareos:/sbin/nologin
\end{verbatim}
\normalsize

I used vipw to create this entry. I selected a User ID and Group ID  of 1002
as they were unused on my system.

I also created a group in /etc/group:

\footnotesize
\begin{verbatim}
bareos:*:1002:
\end{verbatim}
\normalsize

The bareos user (as opposed to the Bareos daemon) will have a home  directory
of {\tt /var/db/bareos} which is the  default location for the Bareos
database.

Now that you have both a bareos user and a bareos group, you can  secure the
bareos home directory by issuing this command:

\footnotesize
\begin{verbatim}
chown -R bareos:bareos /var/db/bareos/
\end{verbatim}
\normalsize

This ensures that only the bareos user can access this directory.  It also
means that if we run the Director and the Storage daemon  as bareos, those
daemons also have restricted access. This would not be  the case if they were
running as root.

It is important to note that the storage daemon actually needs  to be in the
operator group for normal access to tape drives etc (at  least on a FreeBSD
system, that's how things are set up by default)  Such devices are normally
chown root:operator. It is easier and less  error prone  to make Bareos a
member of that group than it is to play around  with system permissions.

Starting the Bareos daemons

To start the bareos daemons on a FreeBSD system, issue the following command:

\footnotesize
\begin{verbatim}
/usr/local/etc/rc.d/bareos-dir start
/usr/local/etc/rc.d/bareos-sd  start
/usr/local/etc/rc.d/bareos-fd  start
\end{verbatim}
\normalsize

To confirm they are all running:

\footnotesize
\begin{verbatim}
$ ps auwx | grep bareos
root   63418 0.0 0.3 1856 1036 ?? Ss 4:09PM 0:00.00
    /usr/local/sbin/bareos-fd -v -c /usr/local/etc/bareos-fd.conf
bareos 63416 0.0 0.3 2040 1172 ?? Ss 4:09PM 0:00.01
    /usr/local/sbin/bareos-sd -v -c /usr/local/etc/bareos-sd.conf
bareos 63422 0.0 0.4 2360 1440 ?? Ss 4:09PM 0:00.00
    /usr/local/sbin/bareos-dir -v -c /usr/local/etc/bareos-dir.conf
\end{verbatim}
\normalsize
}

\section{Secure Erase Command}
\label{sec:SecureEraseCommand}

From \url{https://en.wikipedia.org/w/index.php?title=Data_erasure&oldid=675388437}:
\begin{quote}
    Strict industry standards and government regulations are in place that
    force organizations to mitigate the risk of unauthorized exposure of
    confidential corporate and government data. Regulations in the United
    States include HIPAA (Health Insurance Portability and Accountability
    Act); FACTA (The Fair and Accurate Credit Transactions Act of 2003);
    GLB (Gramm-Leach Bliley); Sarbanes-Oxley Act (SOx); and Payment Card
    Industry Data Security Standards (PCI DSS) and the Data Protection Act
    in the United Kingdom. Failure to comply can result in fines and damage
    to company reputation, as well as civil and criminal liability.
\end{quote}
    
Bareos supports the secure erase of files that usually are simply deleted.
Bareos uses an external command to do the secure erase itself.

This makes it easy to choose a tool that meets the secure erase requirements.


To configure this functionality, a new configuration directive with the name 
\configdirective{Secure Erase Command} has been introduced.

This directive is optional and can be configured in:

\begin{itemize}
  \item \linkResourceDirective{Dir}{Director}{Secure Erase Command}
  \item \linkResourceDirective{Sd}{Storage}{Secure Erase Command}
  \item \linkResourceDirective{Fd}{Client}{Secure Erase Command}
\end{itemize}

This directive configures the secure erase command globally for the daemon it was configured in.

If set, the secure erase command is used to delete files instead of the normal delete routine.

If files are securely erased during a job, the secure delete command output will be shown in the job log.

\begin{logging}{bareos.log}
08-Sep 12:58 win-fd JobId 10: secure_erase: executing C:/cygwin64/bin/shred.exe "C:/temp/bareos-restores/C/Program Files/Bareos/Plugins/bareos_fd_consts.py"
08-Sep 12:58 win-fd JobId 10: secure_erase: executing C:/cygwin64/bin/shred.exe "C:/temp/bareos-restores/C/Program Files/Bareos/Plugins/bareos_sd_consts.py"
08-Sep 12:58 win-fd JobId 10: secure_erase: executing C:/cygwin64/bin/shred.exe "C:/temp/bareos-restores/C/Program Files/Bareos/Plugins/bpipe-fd.dll"
\end{logging}

The current status of the secure erase command is also shown in the output of status director, status client and status storage.

If the secure erase command is configured, the current value is printed.

Example:
\begin{bconsole}{}
* <input>status dir</input>
backup1.example.com-dir Version: 15.3.0 (24 August 2015) x86_64-suse-linux-gnu suse openSUSE 13.2 (Harlequin) (x86_64)
Daemon started 08-Sep-15 12:50. Jobs: run=0, running=0 mode=0 db=sqlite3
 Heap: heap=290,816 smbytes=89,166 max_bytes=89,166 bufs=334 max_bufs=335
 secure erase command='/usr/bin/wipe -V'
\end{bconsole}

Example for Secure Erase Command Settings:

\begin{description}
  \item[Linux:] \configdirective{Secure Erase Command = "/usr/bin/wipe -V"}
  \item[Windows:] \configdirective{Secure Erase Command = "C:/cygwin64/bin/shred.exe"}
\end{description}


Our tests with the \command{sdelete} command was not successful, as \command{sdelete} seems to stay active in the background.
