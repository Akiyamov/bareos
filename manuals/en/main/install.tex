If you are like me, you want to get Bareos running immediately to get a feel
for it, then later you want to go back and read about all the details. This
chapter attempts to accomplish just that: get you going quickly without all
the details.

Bareos comes prepackaged for a number of Linux distributions.
So the easiert way to get to a running Bareos installation, 
is to use a platform where prepacked Bareos packages are available.
Additional information can be found in the chapter \ilink{Operating Systems}{SupportedOSes}.

If you are a platform, for which Bareos is not available in a prepackaged format,
please refer to chapter \vref{Building}.

If Bareos is available as a package, 
only 4 steps are required to get to a running Bareos System:
\begin{itemize}
    \item Subscribe to the repository according to your platform and version
    \item Install the Bareos package
    \item Prepare database backend
    \item Start the daemons
\end{itemize}

This will start a very basic Bareos installation which will regularly backup a directory to disk.
In order to fit it to your needs, you'll have to adapt the configuration and might want to backup other clients.

\section{Subscribe to a Software Repository and Install Software}

You'll find Bareos binary package repositories at \url{http://download.bareos.org/bareos}.
Add the repository to your system. 
You will find the public key in the repodata subdirectory 
(for Debian the distribution directory itself). 
See your distribution's documentation for details.
 
You will then have to install the package bareos. 

This will install all client- and server components and the postgresql backend. 
We recommend to use postgresql, therefore this is the default. 
You can also use mysql or sqlite3 
(where sqlite3 is intended for testing purposes only, by no means recommended for productive use).

If you want to use mysql rather than postgresql, you need two packages: bareos and bareos-database-mysql.

Here are some working examples:

\subsection{Install on RedHat based Linux Distributions}
\subsubsection{RHEL 6}

For adding the repository:
\begin{verbatim}
wget -O /etc/yum.repos.d/bareos.repo http://download.bareos.org/bareos/release/12.4/RHEL_6/bareos.repo
\end{verbatim}

For using PostgreSQL backend:
\begin{verbatim}
yum install bareos bareos-database-postgresql
\end{verbatim}

For using MySQL backend:
\begin{verbatim}
yum install bareos bareos-database-mysql
\end{verbatim}

\subsubsection{CentOS 6}

Just replace RHEL by CentOS in the repository URL

\subsubsection{RHEL 5}

yum in RHEL 5 has slightly different behaviour as far as dependency resolving is concerned: it sometimes install a dependent package after the one that has the dependency defined. To make sure that it works, install the desired Bareos database backend package first in a separate step:

First add the repository using:
\begin{verbatim}
wget -O /etc/yum.repos.d/bareos.repo http://download.bareos.org/bareos/release/12.4/RHEL_5/bareos.repo
\end{verbatim}

After that you can install with PostgreSQL backend:
\begin{verbatim}
yum install bareos-database-postgresql
yum install bareos
\end{verbatim}

Or install with MySQL backend:
\begin{verbatim}
yum install bareos-database-mysql
yum install bareos
\end{verbatim}

\subsubsection{CentOS 5}

Again, just replace RHEL with CentOS in the repository URL.

\subsubsection{Fedora 18}

For adding the repository:
\begin{verbatim}
wget -O /etc/yum.repos.d/bareos.repo http://download.bareos.org/bareos/release/12.4/Fedora_18/bareos.repo
\end{verbatim}

After that you can install with PostgreSQL backend:
\begin{verbatim}
yum install bareos bareos-database-postgresql
\end{verbatim}

Or install with MySQL backend:
\begin{verbatim}
yum install bareos bareos-database-mysql
\end{verbatim}

\subsection{Install on SUSE based Linux Distributions}

\subsubsection{SLES11 SP2}

\begin{verbatim}
zypper addrepo --refresh http://download.bareos.org/bareos/release/12.4/SLE_11_SP2/bareos.repo
zypper install bareos bareos-database-mysql
\end{verbatim}
In this example using mysql as database backend.

\subsubsection{openSUSE 12.3}

\begin{verbatim}
zypper addrepo --refresh http://download.bareos.org/bareos/release/12.4/openSUSE_12.3/bareos.repo
zypper install bareos bareos-database-mysql
\end{verbatim}


\subsection{Install on Debian based Linux Distributions}


\subsubsection{Debian 6}

\begin{verbatim}
URL=http://download.bareos.org/bareos/release/12.4/Debian_6.0/
printf "deb $URL /\n" > /etc/apt/sources.list.d/bareos.list

# add package key
wget -q $URL/Release.key -O- | apt-key add -

apt-get update
apt-get install bareos bareos-database-postgresql
\end{verbatim}

\subsubsection{Ubuntu 12.04}

\begin{verbatim}
URL=http://download.bareos.org/bareos/release/12.4/xUbuntu_12.04/
printf "deb $URL /\n" > /etc/apt/sources.list.d/bareos.list

# add package key
wget -q $URL/Release.key -O- | apt-key add -

apt-get update
apt-get install bareos bareos-database-postgresql
\end{verbatim}

\section{Prepare Bareos database}
    \label{CreateDatabase}

We assume that you have already your database installed and basically running.
Currently the database backend Postgres and MySQL are recommended. The Sqlite database backend is only intended for testing purposes.

The easiest way to set up a database is using an system account that have passwordless local access to the database. 
Often this is the user root for MySQL and the user postgres for PostgreSQL.

\subsection{PostgreSQL}
If your are using PostgreSql and your PostgreSql admin is is postgres (default), use following commands:

\begin{verbatim}
su postgres -c /usr/lib/bareos/scripts/create_bareos_database
su postgres -c /usr/lib/bareos/scripts/make_bareos_tables
su postgres -c /usr/lib/bareos/scripts/grant_bareos_privileges
\end{verbatim}


\subsection{MySQL}
Make sure, that \user{root} has direct access to the local MySQL server. 
Check if the command \command{mysql} connects to the database without defining the the password. 
This is the default on RedHat and SUSE distributions. 
On other systems (Debian, Ubuntu),
create the file \file{/root/.my.cnf} with your authentication informations:
\begin{verbatim}
[client]
user=root
password=YourPasswordForAccessingMysqlAsRoot
\end{verbatim}

Setup the Bareos database tables by following commands:
\begin{verbatim}
/usr/lib/bareos/scripts/create_bareos_database
/usr/lib/bareos/scripts/make_bareos_tables
/usr/lib/bareos/scripts/grant_bareos_privileges
\end{verbatim}

After this, the file \file{/root/.my.cnf} is no longer required for Bareos.



\section{Start the daemons}
\begin{verbatim}
service bareos-dir start
service bareos-sd start
service bareos-fd start
\end{verbatim}

You will eventually have to allow access to the ports 9101-9103, used by Bareos.

Now you should be able to access the director using the bconsole.
