If you are like me, you want to get Bareos running immediately to get a feel
for it, then later you want to go back and read about all the details. This
chapter attempts to accomplish just that: get you going quickly without all
the details.

Bareos comes prepackaged for a number of Linux distributions.
So the easiest way to get to a running Bareos installation, 
is to use a platform where prepacked Bareos packages are available.
Additional information can be found in the chapter \ilink{Operating Systems}{SupportedOSes}.

%\TODO{If you are using a platform, for which Bareos is not available in a prepackaged format,
%please refer to the \ilink{Building chapter}{compile}.}

If Bareos is available as a package, 
only 5 steps are required to get to a running Bareos System:
\begin{enumerate}
    \item \nameref{sec:AddSoftwareRepository}
    \item \nameref{sec:ChooseDatabaseBackend}
    \item \nameref{sec:InstallBareosPackages}
    \item \nameref{sec:CreateDatabase}
    \item \nameref{sec:StartDaemons}
\end{enumerate}

This will start a very basic Bareos installation which will regularly backup a directory to disk.
In order to fit it to your needs, you'll have to adapt the configuration and might want to backup other clients.

\section{Decide about the Bareos release to use}
    \label{sec:AddSoftwareRepository}

\begin{itemize}
   \item \url{http://download.bareos.org/bareos/release/latest/}
\end{itemize}

You'll find Bareos binary package repositories at \url{http://download.bareos.org/}.
The latest stable released version is available at \url{http://download.bareos.org/bareos/release/latest/}.

The public key to verify the repository is also in repository directory
(\file{Release.key} for Debian based distributions, \file{repodata/repomd.xml.key} for RPM based distributions).

Section \nameref{sec:InstallBareosPackages} describes how to add the software repository to your system.


\section{Decide about the Database Backend}
    \label{sec:ChooseDatabaseBackend}

Next you have to decide, what database backend you want to use.
Bareos supports following database backends:
\begin{itemize}
    \item PostgreSQL by package \package{bareos-database-postgresql}
    \item MySQL by package \package{bareos-database-mysql}
    \item Sqlite by package \package{bareos-database-sqlite3} \\
        \warning{The Sqlite backend is only intended for testing, not for productive use.}
\end{itemize}

The PostgreSQL backend is the default.
However, the MySQL backend is also supported,
while the Sqlite backend is intended for testing purposes only.

The Bareos database packages have there dependencies only to the database client packages, 
therefore the database itself must be installed manually.


\section{Install the Bareos Software Packages}
    \label{sec:InstallBareosPackages}

You will have to install the package \package{bareos} 
and the database backend package (\package{bareos-database-*}) you want to use.
The corresponding database should already be installed and running, see \nameref{sec:ChooseDatabaseBackend}.

If you do not explicitly choose a database backend, your operating system installer will choose one for you.
The default should be PostgreSQL, but depending on your operating system and the already installed packages, 
this may differ.

The package \package{bareos} is only a meta package, that contains dependencies to the main components of Bareos, see \nameref{sec:BareosPackages}. 
If you want to setup a distributed environment (like one Director, separate database server, multiple Storage daemons)
you have to choose the corresponding Bareos packages to install on each hosts instead of just installing the \package{bareos} package.


\subsection{Install on RedHat based Linux Distributions}

\subsubsection{RHEL$\ge$7, CentOS$\ge$7, Fedora}
\index[general]{Platform!RHEL}
\index[general]{Platform!CentOS}
\index[general]{Platform!Fedora}

Bareos \sinceVersion{dir}{requires!jansson}{15.2.0} requires the \ilink{Jansson library}{jansson} package.
On RHEL 7 it is available through the RHEL Server Optional channel. On CentOS 7 and Fedora is it included on the main repository.

\begin{commands}{Bareos installation on RHEL $\ge$ 7 / CentOS $\ge$ 7 / Fedora}
#
# define parameter
#

DIST=RHEL_7
# or
# DIST=Fedora_22
# DIST=CentOS_7

DATABASE=postgresql
# or
# DATABASE=mysql

# add the Bareos repository
URL=http://download.bareos.org/bareos/release/latest/$DIST
wget -O /etc/yum.repos.d/bareos.repo $URL/bareos.repo

# install Bareos packages
yum install bareos bareos-database-$DATABASE
\end{commands}
\hide{$}

\subsubsection{RHEL 6, CentOS 6}
\index[general]{Platform!RHEL!6}
\index[general]{Platform!CentOS!6}

Bareos \sinceVersion{dir}{requires!jansson}{15.2.0} requires the \ilink{Jansson library}{jansson} package.
This package is available on \elink{EPEL}{https://fedoraproject.org/wiki/EPEL} 6. Make sure, it is available on your system.

\begin{commands}{Bareos installation on RHEL $\ge$ 6 / CentOS $\ge$ 6}
#
# define parameter
#

DIST=RHEL_6
# DIST=CentOS_6

DATABASE=postgresql
# or
# DATABASE=mysql

# add the Bareos repository
URL=http://download.bareos.org/bareos/release/latest/$DIST
wget -O /etc/yum.repos.d/bareos.repo $URL/bareos.repo

# install Bareos packages
yum install bareos bareos-database-$DATABASE
\end{commands}
\hide{$}


\subsubsection{RHEL 5, CentOS 5}
\index[general]{Platform!RHEL!5}
\index[general]{Platform!CentOS!5}

yum in RHEL 5/CentOS 5 has slightly different behaviour as far as dependency resolving is concerned: it sometimes install a dependent package after the one that has the dependency defined. To make sure that it works, install the desired Bareos database backend package first in a separate step:

\begin{commands}{Bareos installation on RHEL 5 / CentOS 5}
#
# define parameter
#

DIST=RHEL_5
# or
# DIST=CentOS_5

DATABASE=postgresql
# or
# DATABASE=mysql

# add the Bareos repository
URL=http://download.bareos.org/bareos/release/latest/$DIST
wget -O /etc/yum.repos.d/bareos.repo $URL/bareos.repo

# install Bareos packages
yum install bareos-database-$DATABASE
yum install bareos
\end{commands}
\hide{$}

\subsection{Install on SUSE based Linux Distributions}

\subsubsection{SUSE Linux Enterprise Server (SLES), openSUSE}
\index[general]{Platform!SLES}
\index[general]{Platform!openSUSE}

\begin{commands}{Bareos installation on SLES / openSUSE}
#
# define parameter
#

DIST=SLE_12
# or
# DIST=SLE_11_SP4
# DIST=SLE_11_SP3
# DIST=openSUSE_Leap_42.1
# DIST=openSUSE_13.2
# DIST=openSUSE_13.1

DATABASE=postgresql
# or
# DATABASE=mysql

# add the Bareos repository
URL=http://download.bareos.org/bareos/release/latest/$DIST
zypper addrepo --refresh $URL/bareos.repo

# install Bareos packages
zypper install bareos bareos-database-$DATABASE
\end{commands}
\hide{$}



\subsection{Install on Debian based Linux Distributions}

\subsubsection{Debian / Ubuntu}
\index[general]{Platform!Debian}
\index[general]{Platform!Ubuntu}

Bareos \sinceVersion{dir}{requires!jansson}{15.2.0} requires the \ilink{Jansson library}{jansson} package.
On Ubuntu is it available in Ubuntu Universe. In Debian, is it included in the main repository.

\begin{commands}{Bareos installation on Debian / Ubuntu}
#
# define parameter
#

DIST=Debian_8.0
# or
# DIST=Debian_7.0
# DIST=xUbuntu_14.04
# DIST=xUbuntu_12.04

DATABASE=postgresql
# or
# DATABASE=mysql

URL=http://download.bareos.org/bareos/release/latest/$DIST/

# add the Bareos repository
printf "deb $URL /\n" > /etc/apt/sources.list.d/bareos.list

# add package key
wget -q $URL/Release.key -O- | apt-key add -

# install Bareos packages
apt-get update
apt-get install bareos bareos-database-$DATABASE
\end{commands}

If you prefer using the versions of Bareos directly integrated into the distributions, 
please note that there are some differences, see \nameref{sec:DebianOrgLimitations}.

\section{Prepare Bareos database}
    \label{sec:CreateDatabase}

We assume that you have already your database installed and basically running.
Currently the database backends PostgreSQL and MySQL are recommended. The Sqlite database backend is only intended for testing purposes.

The easiest way to set up a database is using an system account that have passwordless local access to the database. 
Often this is the user \user{root} for MySQL and the user \user{postgres} for PostgreSQL.

For details, see chapter \nameref{CatMaintenanceChapter}.

\subsection{Debian based Linux Distributions}

Since Bareos \sinceVersion{dir}{dbconfig-common (Debian)}{14.2.0} the Debian (and Ubuntu) based packages support the \package{dbconfig-common} mechanism to create and update the Bareos database.

Follow the instructions during install to configure it according to your needs.

\begin{center}
\includegraphics[width=0.45\textwidth]{\idir dbconfig-1-enable}
\includegraphics[width=0.45\textwidth]{\idir dbconfig-2-select-database-type}
\end{center}

If you decide not to use \package{dbconfig-common} (selecting \parameter{<No>} on the initial dialog), 
follow the instructions for \nameref{sec:CreateDatabaseOtherDistributions}.

The selectable database backends depend on the \package{bareos-database-*} packages installed.

For details see \nameref{sec:dbconfig}.

\subsection{Other Platforms}
    \label{sec:CreateDatabaseOtherDistributions}

\subsubsection{PostgreSQL}
If your are using PostgreSQL and your PostgreSQL administration user is \user{postgres} (default), use following commands:

\begin{commands}{Setup Bareos catalog with PostgreSQL}
su postgres -c /usr/lib/bareos/scripts/create_bareos_database
su postgres -c /usr/lib/bareos/scripts/make_bareos_tables
su postgres -c /usr/lib/bareos/scripts/grant_bareos_privileges
\end{commands}


\subsubsection{MySQL/MariaDB}
Make sure, that \user{root} has direct access to the local MySQL server. 
Check if the command \command{mysql} connects to the database without defining the password.
This is the default on RedHat and SUSE distributions. 
On other systems (Debian, Ubuntu),
create the file \file{~/.my.cnf} with your authentication informations:

\begin{config}{MySQL credentials file .my.cnf}
[client]
host=localhost
user=root
password=<input>YourPasswordForAccessingMysqlAsRoot</input>
\end{config}

It is recommended, to secure the Bareos database connection with a password.
See \ilink{Catalog Maintenance -- MySQL}{catalog-maintenance-mysql} about how to archieve this.
For testing, using a password-less MySQL connection is probable okay.
Setup the Bareos database tables by following commands:
\begin{commands}{Setup Bareos catalog with MySQL}
/usr/lib/bareos/scripts/create_bareos_database
/usr/lib/bareos/scripts/make_bareos_tables
/usr/lib/bareos/scripts/grant_bareos_privileges
\end{commands}

As some Bareos updates require a database schema update,
therefore the file \file{/root/.my.cnf} might also be useful in the future.


\section{Start the daemons}
    \label{sec:StartDaemons}

\begin{commands}{Start the Bareos Daemons}
service bareos-dir start
service bareos-sd start
service bareos-fd start
\end{commands}

You will eventually have to allow access to the ports 9101-9103, used by Bareos.

Now you should be able to access the director using the bconsole.

When you want to use the bareos-webui, please refer to the chapter \nameref{sec:install-webui}.

\chapter{Installing Bareos Webui}
\label{sec:webui}
\label{sec:install-webui}
\index[general]{Webui}
\index[general]{Webui!Install}

This chapter addresses the installation process of the \bareosWebui.

Since \sinceVersion{dir}{bareos-webui}{15.2.0} \bareosWebui is part of the Bareos project and available for a number of platforms.

\begin{center}
  \includegraphics[width=0.8\textwidth]{\idir bareos-webui-jobs}
\end{center}

\section{Features}

\begin{itemize}
\item Intuitive web interface
\item Multilinugual
\item Can access multiple directors and catalogs
\item Individual accounts and ACL support via Bareos restricted named consoles
%\item Display of the most recent job status in a dashboard widget
%\item Display of current director messages in a dashboard widget
\item Tape Autochanger management, with the possibility to label, import/export media and update your autochanger slot status
\item Temporarly enable or disable jobs, clients and schedules and also see their current state
\item Show
    \begin{itemize}
    \item Detailed information about Jobs, Clients, Filesets, Pools, Volumes, Storages, Schedules, Logs and Director messages
    \item Filedaemon, Storage- and Director updates
    \item Client, Director, Storage and Scheduler status
    \end{itemize}
%\item Browse through your pools and media/volumes to keep track of retentions, volume status and usage
\item Backup Jobs
    \begin{itemize}
    \item Start, cancel, rerun and restore from.
    \item Show the file list of backup jobs
    \end{itemize}
\item Restore files by browsing through a filetree of your backup jobs.
    \begin{itemize}
    \item Merge your backup jobs history and filesets of a client or use a single backup job for restore.
    \item Restore files to a different client instead of the origin
    \end{itemize}
\item bconsole interface (limited to non-interactive commands)
\end{itemize}

\section{System Requirements}

\begin{itemize}
\item A platform, for which the \package{bareos-webui} package is available, see \nameref{sec:BareosPackages}.
\item A working Bareos environment.
\item \bareosDir version $>=$ \bareosWebui version.
\item The \bareosWebui can be installed on any host. It does not have to be installed on the same as the \bareosDir.
\item The default installation uses an Apache webserver with mod-rewrite, mod-php and mod-setenv.
\item PHP $>$= 5.3.23
\item On SUSE Linux Enterprise 12 you need the additional SUSE Linux Enterprise Module for Web Scripting 12.
\end{itemize}

\subsection{Version $<$ 16.2}

\bareosWebui \sinceVersion{}{bareos-webui incorporates Zend Framework 2}{16.2.4} incorporates the required Zend Framework 2 components, no extra Zend Framework installation is required.
For older versions of \package{bareos-webui}, you must install Zend Framework separately.
Unfortunately, not all distributions offer Zend Framework 2 packages.
The following list shows where to get the Zend Framework 2 package:

\begin{itemize}
  \item RHEL, CentOS
    \begin{itemize}
    \item \url{https://fedoraproject.org/wiki/EPEL}
    \item \url{https://apps.fedoraproject.org/packages/php-ZendFramework2}
    \end{itemize}

  \item Fedora
    \begin{itemize}
    \item \url{https://apps.fedoraproject.org/packages/php-ZendFramework2}
    \end{itemize}

  \item SUSE, Debian, Ubuntu
    \begin{itemize}
    \item \url{http://download.bareos.org/bareos}
    \end{itemize}
\end{itemize}

Also be aware, that older versions of \bareosDir do not support the \nameref{sec:SubdirectoryConfigurationScheme}
and therefore Bareos configuration resource files must be included manually.

\section{Installation}

\subsection{Adding the Bareos Repository}

If not already done, add the Bareos repository that is matching your Linux distribution. Please have a look at the chapter \nameref{sec:InstallBareosPackages} for more information on how to achieve this.

\subsection{Install the bareos-webui package}

After adding the repository simply install the bareos-webui package via your package manager.

\begin{itemize}
 \item RHEL, CentOS and Fedora
\begin{commands}{}
yum install bareos-webui
\end{commands}
 or
\begin{commands}{}
dnf install bareos-webui
\end{commands}
\end{itemize}

\begin{itemize}
 \item SUSE Linux Enterprise Server (SLES), openSUSE
\begin{commands}{}
zypper install bareos-webui
\end{commands}
\end{itemize}

\begin{itemize}
 \item Debian, Ubuntu
\begin{commands}{}
apt-get install bareos-webui
\end{commands}
\end{itemize}

\subsection{Minimal Configuration}

This assumes, \bareosDir and \bareosWebui are installed on the same host.

\begin{enumerate}

\item If you are using SELinux, allow HTTPD scripts and modules make network connections:
\begin{commands}{}
setsebool -P httpd_can_network_connect on
\end{commands}
For details, see \nameref{sec:webui-selinux}.

\item Restart Apache (to load configuration provided by bareos-webui, see \nameref{sec:webui-apache})

\item \label{item:webui-create-user}
Use \command{bconsole} to create a user with name \name{admin} and password \name{secret} and permissions defined in \resourcename{Dir}{Profile}{webui-admin}:
\begin{bconsole}{add a named console}
*<input>configure add console name=admin password=secret profile=webui-admin</input>
\end{bconsole}
Of course, you can choose other names and passwords.
For details, see \nameref{sec:webui-console}.

\item Login to http://HOSTNAME/bareos-webui with username and password as created in \ref{item:webui-create-user}.

\end{enumerate}


\subsection{Configuration Details}


\subsubsection{Create a restricted consoles}
    \label{sec:webui-console}

There is not need, that \bareosWebui itself provide a user management.
Instead it uses so named \resourcetype{Dir}{Console} defined in the \bareosDir.
You can have multiple consoles with different names and passwords, sort of like multiple users, each with different privileges.

At least one \resourcetype{Dir}{Console} is required to use the \bareosWebui.

To allow a user with name \name{admin} and password \name{secret} to access the \bareosDir
with permissions defined in the \resourcename{Dir}{Profile}{webui-admin} (see \nameref{sec:webui-profile}),
either
\begin{itemize}
\item create a file \file{/etc/bareos/bareos-dir.d/console/admin.conf} with following content:
\begin{bareosConfigResource}{bareos-dir}{console}{admin}
Console {
  Name = "admin"
  Password = "secret"
  Profile = "webui-admin"
}
\end{bareosConfigResource}

To enable this, reload or restart your \bareosDir.
\item or use the \command{bconsole}:
\begin{bconsole}{add console}
*<input>configure add console name=admin password=secret profile=webui-admin</input>
\end{bconsole}
\end{itemize}


For details, please read \nameref{DirectorResourceConsole}.



\subsubsection{Configuration of profile resources}
    \label{sec:webui-profile}

The package \package{bareos-webui} comes with a predefined profile for \bareosWebui: \resourcename{Dir}{Profile}{webui-admin}.

If your \bareosWebui is installed on another system than the \bareosDir, you have to copy the profile to the \bareosDir.

This is the default profile, giving access to all Bareos resources and allowing all commands used by the \bareosWebui:

\begin{bareosConfigResource}{bareos-dir}{profile}{webui-admin}
Profile {
  Name = webui-admin
  CommandACL = !.bvfs_clear_cache, !.exit, !.sql, !configure, !create, !delete, !purge, !sqlquery, !umount, !unmount, *all*
  Job ACL = *all*
  Schedule ACL = *all*
  Catalog ACL = *all*
  Pool ACL = *all*
  Storage ACL = *all*
  Client ACL = *all*
  FileSet ACL = *all*
  Where ACL = *all*
  Plugin Options ACL = *all*
}
\end{bareosConfigResource}

The \resourcetype{Dir}{Profile} itself does not give any access to the \bareosDir,
but can be used by \resourcetype{Dir}{Console}, which do give access to the \bareosDir, see \nameref{sec:webui-console}.

For details, please read \nameref{DirectorResourceProfile}.

\subsubsection{SELinux}
\label{sec:webui-selinux}
\index[general]{SELinux!bareos-webui}

To use \bareosDir on a system with SELinux enabled,
permission must be given to HTTPD to make network connections:
\begin{commands}{}
setsebool -P httpd_can_network_connect on
\end{commands}


\subsubsection{Configure your Apache Webserver}
\index[general]{Apache!bareos-webui}
\label{sec:webui-apache}

The package \package{bareos-webui} provides a default configuration for Apache.
Depending on your distribution, it is installed at \file{/etc/apache2/conf.d/bareos-webui.conf}, \file{/etc/httpd/conf.d/bareos-webui.conf} or \file{/etc/apache2/available-conf/bareos-webui.conf}.

The required Apache modules, \argument{setenv}, \argument{rewrite} and \argument{php} are enabled via package postinstall script.
However, after installing the \package{bareos-webui} package, you need to restart your Apache webserver manually.

\subsubsection{Configure your /etc/bareos-webui/directors.ini}
\index[general]{Configuration!WebUI}
\label{sec:webui-configuration-files}

Configure your directors in \file{/etc/bareos-webui/directors.ini} to match your settings.

The configuration file \file{/etc/bareos-webui/directors.ini} should look similar to this:

\begin{bconfig}{/etc/bareos-webui/directors.ini}
;
; Bareos WebUI Configuration File
;
; File: /etc/bareos-webui/directors.ini
;

;------------------------------------------------------------------------------
; Section localhost-dir
;------------------------------------------------------------------------------
[localhost-dir]

; Enable or disable section. Possible values are "yes" or "no", the default is "yes".
enabled = "yes"

; Fill in the IP-Address or FQDN of you director.
diraddress = "localhost"

; Default value is 9101
dirport	= 9101

; Set catalog to explicit value if you have multiple catalogs
;catalog = "MyCatalog"

; TLS verify peer
; Possible values: true or false
tls_verify_peer = false

; Server can do TLS
; Possible values: true or false
server_can_do_tls = false

; Server requires TLS
; Possible values: true or false
server_requires_tls = false

; Client can do TLS
; Possible values: true or false
client_can_do_tls = false

; Client requires TLS
; Possible value: true or false
client_requires_tls = false

; Path to the certificate authority file
; E.g. ca_file = "/etc/bareos-webui/tls/BareosCA.crt"
;ca_file = ""

; Path to the cert file which needs to contain the client certificate and the key in PEM encoding
; E.g. ca_file = "/etc/bareos-webui/tls/restricted-named-console.pem"
;cert_file = ""

; Passphrase needed to unlock the above cert file if set
;cert_file_passphrase = ""

; Allowed common names
; E.g. allowed_cns = "host1.example.com"
;allowed_cns = ""

;------------------------------------------------------------------------------
; Section another-host-dir
;------------------------------------------------------------------------------
[another-host-dir]
enabled = "no"
diraddress = "192.168.120.1"
dirport = 9101
;catalog = "MyCatalog"
;tls_verify_peer = false
;server_can_do_tls = false
;server_requires_tls = false
;client_can_do_tls = false
;client_requires_tls = false
;ca_file = ""
;cert_file = ""
;cert_file_passphrase = ""
;allowed_cns = ""

\end{bconfig}

You can add as many directors as you want, also the same host with a different name and different catalog, if you have multiple catalogs.

\subsubsection{Configure your /etc/bareos-webui/configuration.ini}

Since \sinceVersion{}{/etc/bareos-webui/configuration.ini}{16.2.2}
you are able to configure some parameters of the \bareosWebui to your needs.

\begin{bconfig}{/etc/bareos-webui/configuration.ini}
;
; Bareos WebUI Configuration File
;
; File: /etc/bareos-webui/configuration.ini
;

;------------------------------------------------------------------------------
; SESSION SETTINGS
;------------------------------------------------------------------------------
;
[session]
; Default: 3600 seconds
timeout=3600

;------------------------------------------------------------------------------
; DASHBOARD SETTINGS
;------------------------------------------------------------------------------
[dashboard]
; Autorefresh Interval
; Default: 60000 milliseconds
autorefresh_interval=60000

;------------------------------------------------------------------------------
; TABLE SETTINGS
;------------------------------------------------------------------------------
[tables]
; Possible values for pagination
; Default: 10,25,50,100
pagination_values=10,25,50,100

; Default number of rows per page
; for possible values see pagination_values
; Default: 25
pagination_default_value=25

; State saving - restore table state on page reload.
; Default: false
save_previous_state=false

;------------------------------------------------------------------------------
; VARIOUS SETTINGS
;------------------------------------------------------------------------------
[autochanger]
; Pooltype for label to use as filter.
; Default: none
labelpooltype=scratch

\end{bconfig}

\section{Upgrade from 15.2 to 16.2}

\subsection{Console/Profile changes}

The \bareosWebui Director profile shipped with Bareos 15.2 (\resourcename{Dir}{Profile}{webui} in the file \file{/etc/bareos/bareos-dir.d/webui-profiles.conf}) is not sufficient to use the \bareosWebui 16.2.
This has several reasons:
\begin{enumerate}
  \item The handling of \dt{Acl}s is more strict in Bareos 16.2 than it has been in Bareos 15.2.
    Substring matching is no longer enabled, therefore you need to change \bcommand{.bvfs_*} to \bcommand{.bvfs_.*}
    in your \linkResourceDirective{Dir}{Profile}{Command ACL} to have a proper regular expression.
    Otherwise the restore module won't work any longer, especially the file browser.

  \item The \bareosWebui 16.2 uses following additional commands:
\begin{itemize}
\item .help
\item .schedule
\item .pools
\item import
\item export
\item update
\item release
\item enable
\item disable
\end{itemize}

\end{enumerate}

If you used an unmodified \file{/etc/bareos/bareos-dir.d/webui-profiles.conf} file,
the easiest way is to overwrite it with the new profile file \file{/etc/bareos/bareos-dir.d/profile/webui-admin.conf}.
The new \resourcename{Dir}{Profile}{webui-admin} allows all commands, except of the dangerous ones, see \nameref{sec:webui-profile}.

\subsection{directors.ini}

Since \sinceVersion{general}{Webui offers limited support for multiple catalogs}{16.2.0} it is possible to work with different catalogs. Therefore the catalog parameter has been introduced. If you don't set a catalog explicitly the default \resourcename{Dir}{Catalog}{MyCatalog} will be used. Please see \nameref{sec:webui-configuration-files} for more details.

\subsection{configuration.ini}

Since 16.2 the \bareosWebui introduced an additional configuration file besides the directors.ini file named configuration.ini where you are able to adjust some parameters of the webui to your needs. Please see \nameref{sec:webui-configuration-files} for more details.


\section{Additional information}

\subsection{NGINX}
\index[general]{nginx!bareos-webui}

If you prefer to use \bareosWebui on Nginx with php5-fpm instead of Apache,
a basic working configuration could look like this:

\begin{bconfig}{bareos-webui on nginx}
server {

        listen       9100;
        server_name  bareos;
        root         /var/www/bareos-webui/public;

        location / {
                index index.php;
                try_files $uri $uri/ /index.php?$query_string;
        }

        location ~ .php$ {

                include snippets/fastcgi-php.conf;

                # php5-cgi alone:
                # pass the PHP
                # scripts to FastCGI server
                # listening on 127.0.0.1:9000
                #fastcgi_pass 127.0.0.1:9000;

                # php5-fpm:
                fastcgi_pass unix:/var/run/php5-fpm.sock;

                # APPLICATION_ENV:  set to 'development' or 'production'
                #fastcgi_param APPLICATION_ENV development;
                fastcgi_param APPLICATION_ENV production;

        }

}
\end{bconfig}

This will make the \bareosWebui accessible at http://bareos:9100/ (assuming your DNS resolve the hostname \host{bareos} to the NGINX server).


\chapter{Updating Bareos}
\label{bareos-update}

In most cases, a Bareos update is simply done by a package update of the distribution.
Please remind, that Bareos Director and Bareos Storage Daemon must always have the same version.
The version of the File Daemon may differ, see chapter about \ilink{backward compatibility}{backward-compability}.

\section{Updating the database scheme}

Sometimes improvements in Bareos make it neccessary to update the database scheme.

\warning{If the Bareos catalog database does not have the current schema, the Bareos Director refuses to start.}

Detailed information can then be found in the log file \logfileUnix.

Take a look into the \ilink{Release Notes}{releasenotes} to see which Bareos updates do require a database scheme update.


\subsection{Debian based Linux Distributions}

Since Bareos \sinceVersion{dir}{dbconfig-common (Debian)}{14.2.0} the Debian (and Ubuntu) based packages support the \package{dbconfig-common} mechanism to create and update the Bareos database.
If this is properly configured, the database schema will be automatically adapted by the Bareos packages.

\warning{When using the PostgreSQL backend and updating to Bareos $<$ 14.2.3, it is necessary to manually grant database permissions, normally by using}
\begin{commands}{}
<command> </command><parameter>su - postgres -c /usr/lib/bareos/scripts/grant_bareos_privileges</parameter>
\end{commands}
For details see \nameref{sec:dbconfig}.

If you disabled the usage of \package{dbconfig-common}, 
follow the instructions for \nameref{sec:UpdateDatabaseOtherDistributions}.

\subsection{Other Platforms}
    \label{sec:UpdateDatabaseOtherDistributions}

This has to be done as database administrator.
On most platforms Bareos knows only about the credentials to access the Bareos database,
but not about the database administrator to modify the database schema.

The task of updating the database schema is done by the script
\command{/usr/lib/bareos/scripts/update_bareos_tables}.

However, this script requires administration access to the database.
Depending on your distribution and your database, this requires different preparations.
More details can be found in chapter \ilink{Catalog Maintenance}{CatMaintenanceChapter}.

\warning{If you're updating to Bareos $<=$ 13.2.3 and have configured the Bareos database during install using Bareos environment variables (\variable{db_name}, \variable{db_user} or \variable{db_password}, see \nameref{CatMaintenanceChapter}), make sure to have these variables defined in the same way when calling the update and grant scripts. Newer versions of Bareos read these variables from the Director configuration file \configFileDirUnix. However, make sure that the user running the database scripts has read access to this file (or set the environment variables). The \user{postgres} user normally does not have the required permissions.}

\subsubsection{PostgreSQL}
If your are using PostgreSQL and your PostgreSQL administrator is \user{postgres} (default), use following commands:

\begin{commands}{Update PostgreSQL database schema}
su postgres -c /usr/lib/bareos/scripts/update_bareos_tables
su postgres -c /usr/lib/bareos/scripts/grant_bareos_privileges
\end{commands}

The \command{grant_bareos_privileges} command is required, if new databases tables are introduced. It does not hurt to run it multiple times.

After this, restart the Bareos Director and verify it starts without problems.

\subsubsection{MySQL/MariaDB}
Make sure, that \user{root} has direct access to the local MySQL server.
Check if the command \command{mysql} without parameter connects to the database.
If not, you may be required to adapt your local MySQL config file \file{~/.my.cnf}.
It should look similar to this:

\begin{config}{MySQL credentials file .my.cnf}
[client]
host=localhost
user=root
password=<input>YourPasswordForAccessingMysqlAsRoot</input>
\end{config}

If you are able to connect via the \command{mysql} to the database, run the following script from the Unix prompt:
\begin{commands}{Update MySQL database schema}
/usr/lib/bareos/scripts/update_bareos_tables
\end{commands}

Currently on MySQL is it not neccessary to run \command{grant_bareos_privileges}, because access to the database is already given using wildcards.

After this, restart the Bareos Director and verify it starts without problems.
