\section{Passive Clients}
\label{PassiveClient}

The normal way of initializing the data channel (the channel where the backup data itself is transported)
is done by the file daemon (client) that connects to the storage daemon.

In many setups, this can cause problems, as this means that:
\begin{itemize}
    \item The client must be able to resolve the name of the storage daemon (Often not true, you have to do tricks with the hosts file)
    \item The client must be allowed to create a new connection.
    \item The client must be able to connect to the storage daemon over the network (often difficult over NAT or Firewall)
\end{itemize}

By using Passive Client, the initialization of the datachannel is reversed, so that the storage daemon connects to the filedaemon.
This solves almost every problem created by Firewalls, NAT-gateways and resolving issues, as

\begin{itemize}
    \item The storage daemon initiates the connection, and thus can pass thru the same or similar firewallrules that the director already has to access the fileadaemon.
    \item The client never initiates any connection, thus can be completely firewalled.
    \item The client never needs any name resolution and is totally independent from any resolving issues.
\end{itemize}

\subsection{Usage}

To use this new feature, just configure  \textbf{passive=yes} in the client definition of the director daemon:
\begin{bconfig}{Enable passive mode in bareos-dir.conf}
Client {
   Name = client1-fd
   Password = "secretpassword"
   <input>Passive = yes</input>
   [...]
}
\end{bconfig}

Also, you need to set \configdirective{compatible=no} in the \file{bareos-fd.conf} configuration file:
\begin{bconfig}{Disable compatible mode for the Bareos filedaemon in bareos-fd.conf} 
Director {
  Name = bareos-dir
  Password = "secretpassword"
}

FileDaemon {
   Name = client1-fd
   [...]
   <input>Compatible = no</input>
}
\end{bconfig}
