%%
%%

\chapter{The Windows Version of Bareos}
\label{Win32Chapter}
\index[general]{Windows}

At the current time only the File daemon or Client program has
been thouroughly tested on Windows and is suitable for a
production environment. As a consequence, when we
speak of the Windows version of Bareos below, we are referring to
the File daemon (client) only.

The Windows version of the Bareos File daemon has been tested on WinXP,
Windows Server 2003, Windows Server 2008, Vista, Windows 7 and Windows 8
systems.  The Windows version of Bareos is a native Win32 port, but there
are very few source code changes to the Unix code, which means that the
Windows version is for the most part running code that has long proved
stable on Unix systems.

Once installed Bareos normally runs as a system service. This means that it is
immediately started by the operating system when the system is booted, and
runs in the background even if there is no user logged into the system.

\section{Windows Installation}
\label{Windows:Installation}
\label{Windows:Configuration:Files}
\index[general]{Installation!Windows}
\index[general]{Windows!File Daemon!Installation}

Normally, you will install the Windows version of Bareos from the binaries.
The winbareos binary packages are provided under \path|http://download.bareos.org/bareos/release/12.4/windows|.
Additionally, there are OPSI packages available under \path|http://download.bareos.org/bareos/release/12.4/windows/opsi|.

This install is standard Windows .exe that runs an install wizard using the
NSIS Free Software installer, so if you have already installed Windows
software, it should be very familiar to you.
Providing you do not already
have Bareos installed, the installer installs the binaries and dlls in
\path|C:\Program Files\Bareos| and the configuration files
in Windows XP \path|c:\Documents and Settings\All Users\Application Data\Bareos|.
and from Windows Vista + in \path|c:\ProgramData\Bareos|.

In addition, the {\bf Start\-{\textgreater}All Programs\-{\textgreater}Bareos} menu item
will be created during the installation, and on that menu, you
will find items for editing the configuration files, displaying
the document, and starting \command{BAT} or \command{bconsole}.


Finally, proceed with the installation.

\pagebreak

\subsection{Graphical Installation}

Here are the important steps.

\begin{itemize}
\itemsep0pt
\item You must be logged in as an Administrator to the local machine
to do a correct installation, if not, please do so before continuing.
\item For a standard installation you may only select the "Tray-Monitor"
and the "Open Firewall for Client" as additional optional components.

\begin{figure}[htbp]
\centering
\includegraphics[scale=0.8]{\idir win-install-1.eps}
\caption{Optional component selection}
\label{fig:win32clientoptionalcomponents}
\end{figure}

\item You need to fill in the name of your bareos director in the client configuration
dialogue and the FQDN or ip address of your client.

\begin{figure}[htbp]
\centering
\includegraphics[scale=0.8]{\idir win-install-2.eps}
\caption{Client configuration}
\label{fig:win32clientconfiguration}
\end{figure}

\pagebreak

\item Add the client ressource to your Bareos Director Configuration and a
job ressource for the client as it is also described in the default bareos-dir.conf

\begin{figure}[htbp]
\centering
\includegraphics[scale=0.8]{\idir win-install-3.eps}
\caption{Client ressource for the Director Configuration}
\label{fig:win32clientresource}
\end{figure}

\end{itemize}

\subsection{Commandline Installation}

The windows installer is significantly enhanced since Version 12.4.4. All inputs
that are given during interactive install can now directly be configured on the
commandline, so that an automatic silent install is possible.

\large Commandline Switches
\normalsize

\begin{description}
\item[/?] shows the list of available parameters.
\item[/S] sets the installer to silent. The Installation is done without user interaction.
This switch is also available for the uninstaller.
\item[/CLIENTNAME] sets the name of the client ressource
\item[/CLIENTPASSWORD] sets the password to access the client
\item[/DIRECTORNAME] sets the name of the director to access the client and of the director to
accessed by bconsole and bat
\item[/CLIENTADDRESS] network address of the client
\item[/CLIENTMONITORPASSWORD] sets the password for monitor access
\item[/DIRECTORADDRESS] sets network address of the director for bconsole or bat access
\item[/DIRECTORPASSWORD] set the password to access the director
\item[/D=\path|C:\specify\installation\directory|] (Important: It has to be the last option!)
\item[/SILENTKEEPCONFIG] keep configuration files on silent uninstall and use exinsting config files during silent install
\end{description}

By setting the Installation Parameters via commandline and using the silent installer,
you can install the bareos client without having to do any configuration after the
installation e.g. as follows:

\footnotesize
\begin{verbatim}
c:\winbareos.exe /S /CLIENTNAME=hostname-fd /CLIENTPASSWORD="verysecretpassword" /DIRECTORNAME=bareos-dir
\end{verbatim}
\normalsize

\section{Dealing with Windows Problems}
\label{problems}
\index[general]{Problem!Windows}
\index[general]{Windows!Dealing with Problems}

If you are not using the portable option, and you have VSS
(Volume Shadow Copy) enabled in the Director, and you experience
problems with Bareos not being able to open files, it is most
likely that you are running an antivirus program that blocks
Bareos from doing certain operations. In this case, disable the
antivirus program and try another backup.  If it succeeds, either
get a different (better) antivirus program or use something like
RunClientJobBefore/After to turn off the antivirus program while
the backup is running.

If turning off anti-virus software does not resolve your VSS
problems, you might have to turn on VSS debugging.  The following
link describes how to do this:
\elink{http://support.microsoft.com/kb/887013/en-us}{http://support.microsoft.com/kb/887013/en-us}.

In Microsoft Windows Small Business Server 2003 the VSS Writer for Exchange
is turned off by default. To turn it on, please see the following link:
\elink{http://support.microsoft.com/default.aspx?scid=kb;EN-US;Q838183}{
http://support.microsoft.com/default.aspx?scid=kb;EN-US;Q838183}


The most likely source of problems is authentication when the Director
attempts to connect to the File daemon that you installed. This can occur if
the names and the passwords defined in the File daemon's configuration file
{\bf bareos-fd.conf} file on
the Windows machine do not match with the names and the passwords in the
Director's configuration file {\bf bareos-dir.conf} located on your Unix/Linux
server.

More specifically, the password found in the {\bf Client} resource in the
Director's configuration file must be the same as the password in the {\bf
Director} resource of the File daemon's configuration file. In addition, the
name of the {\bf Director} resource in the File daemon's configuration file
must be the same as the name in the {\bf Director} resource of the Director's
configuration file.

It is a bit hard to explain in words, but if you understand that a Director
normally has multiple Clients and a Client (or File daemon) may permit access
by multiple Directors, you can see that the names and the passwords on both
sides must match for proper authentication.

Running Unix like programs on Windows machines is a bit frustrating because
the Windows command line shell (DOS Window) is rather primitive. As a
consequence, it is not generally possible to see the debug information and
certain error messages that Bareos prints. With a bit of work, however, it is
possible. When everything else fails and you want to {\bf see} what is going
on, try the following:

\footnotesize
\begin{verbatim}
   Start a DOS shell Window.
   c:\Program Files\bareos\bareos-fd -t >out
   type out
\end{verbatim}
\normalsize

The precise path to bareos-fd depends on where it is installed.
The {\bf -t} option will cause Bareos to read the configuration file, print
any error messages and then exit. the {\bf {\textgreater}} redirects the output to the
file named {\bf out}, which you can list with the {\bf type} command.

If something is going wrong later, or you want to run {\bf Bareos} with a
debug option, you might try starting it as:

\footnotesize
\begin{verbatim}
   c:\Program Files\bareos\bin\bareos-fd -d 100 >out
\end{verbatim}
\normalsize

In this case, Bareos will run until you explicitly stop it, which will give
you a chance to connect to it from your Unix/Linux server.

In addition, you should look in the System Applications log on the Control
Panel to find any Windows errors that Bareos got during the startup process.

Finally, due to the above problems, when you turn on debugging, and specify
\bcommand{trace=on} on a setdebug command in the Console, Bareos will write the debug
information to the file \file{bareos.trace} in the directory from which Bareos
is executing.

You may wish to start the daemon with debug mode on rather than doing it
using bconsole. To do so, edit the following registry key:

\footnotesize
\begin{verbatim}
HKEY_LOCAL_MACHINE\HARDWARE\SYSTEM\CurrentControlSet\Services\Bareos-dir
\end{verbatim}
\normalsize

using regedit, then add -dnn after the /service option, where nn represents
the debug level you want.

\section{Windows Compatibility Considerations}
\label{Compatibility}
\index[general]{Windows!Compatibility Considerations}


If you are not using the VSS (Volume Shadow Copy) option described in the
next section of this chapter, and if any applications are running during
the backup and they have files opened exclusively, Bareos will not be able
to backup those files, so be sure you close your applications (or tell your
users to close their applications) before the backup.  Fortunately, most
Microsoft applications do not open files exclusively so that they can be
backed up.  However, you will need to experiment.  In any case, if Bareos
cannot open the file, it will print an error message, so you will always
know which files were not backed up.
If Volume Shadow Copy Service is enabled, Bareos is able to backing up any
file.

During backup, Bareos doesn't know about the system registry, so you will
either need to write it out to an ASCII file using \command{regedit /e} or use a
program specifically designed to make a copy or backup the registry.



\section{Volume Shadow Copy Service}
\index[general]{Windows!Volume Shadow Copy Service}
\index[general]{Windows!VSS}
\label{VSS}

VSS is available on Windows XP and newer. From the perspective of
a backup-solution for Windows, this is an extremely important step. VSS
allows Bareos to backup open files and even to interact with applications like
RDBMS to produce consistent file copies. VSS aware applications are called
VSS Writers, they register with the OS so that when Bareos wants to do a
Snapshot, the OS will notify the register Writer programs, which may then
create a consistent state in their application, which will be backed up.
Examples for these writers are "MSDE" (Microsoft database
engine), "Event Log Writer", "Registry Writer" plus 3rd
party-writers.  If you have a non-vss aware application a shadow copy is still generated
and the open files can be backed up, but there is no guarantee
that the file is consistent.

Bareos produces a message from each of the registered writer programs
when it is doing a VSS backup so you know which ones are correctly backed
up.

Technically Bareos creates a shadow copy as soon as the backup process
starts. It does then backup all files from the shadow copy and destroys the
shadow copy after the backup process. Please have in mind, that VSS
creates a snapshot and thus backs up the system at the state it had
when starting the backup. It will disregard file changes which occur during
the backup process.

VSS can be turned on by placing an

\index[dir]{Enable VSS}
\index[general]{Enable VSS}
\begin{verbatim}
Enable VSS = yes
\end{verbatim}

in your FileSet resource.

The VSS aware File daemon has the letters VSS on the signon line that
it produces when contacted by the console. For example:
\begin{verbatim}
Tibs-fd Version: 1.37.32 (22 July 2005) VSS Windows XP MVS NT 5.1.2600
\end{verbatim}
the VSS is shown in the line above. This only means that the File daemon
is capable of doing VSS not that VSS is turned on for a particular backup.
There are two ways of telling if VSS is actually turned on during a backup.
The first is to look at the status output for a job, e.g.:
\footnotesize
\begin{verbatim}
Running Jobs:
JobId 1 Job NightlySave.2005-07-23_13.25.45 is running.
    VSS Backup Job started: 23-Jul-05 13:25
    Files=70,113 Bytes=3,987,180,650 Bytes/sec=3,244,247
    Files Examined=75,021
    Processing file: c:/Documents and Settings/user/My Documents/My Pictures/Misc1/Sans titre - 39.pdd
    SDReadSeqNo=5 fd=352
\end{verbatim}
\normalsize
Here, you see under Running Jobs that JobId 1 is "VSS Backup Job started ..."
This means that VSS is enabled for that job.  If VSS is not enabled, it will
simply show "Backup Job started ..." without the letters VSS.

The second way to know that the job was backed up with VSS is to look at the
Job Report, which will look something like the following:
\footnotesize
\begin{verbatim}
23-Jul 13:25 rufus-dir: Start Backup JobId 1, Job=NightlySave.2005-07-23_13.25.45
23-Jul 13:26 rufus-sd: Wrote label to prelabeled Volume "TestVolume001" on device "DDS-4" (/dev/nst0)
23-Jul 13:26 rufus-sd: Spooling data ...
23-Jul 13:26 Tibs: Generate VSS snapshots. Driver="VSS WinXP", Drive(s)="C"
23-Jul 13:26 Tibs: VSS Writer: "MSDEWriter", State: 1 (VSS_WS_STABLE)
23-Jul 13:26 Tibs: VSS Writer: "Microsoft Writer (Bootable State)", State: 1 (VSS_WS_STABLE)
23-Jul 13:26 Tibs: VSS Writer: "WMI Writer", State: 1 (VSS_WS_STABLE)
23-Jul 13:26 Tibs: VSS Writer: "Microsoft Writer (Service State)", State: 1 (VSS_WS_STABLE)
\end{verbatim}
\normalsize
In the above Job Report listing, you see that the VSS snapshot was generated for drive C (if
other drives are backed up, they will be listed on the {\bf Drive(s)="C"}.  You also see the
reports from each of the writer program.  Here they all report VSS\_WS\_STABLE, which means
that you will get a consistent snapshot of the data handled by that writer.

\section{VSS Problems}
\index[general]{Windows!Problem!VSS}
\index[fd]{Windows!Problem!VSS}
\index[general]{Problem!Windows!VSS}


If you are experiencing problems such as VSS hanging on MSDE, first try
running {\bf vssadmin} to check for problems, then try running {\bf
ntbackup} which also uses VSS to see if it has similar problems. If so, you
know that the problem is in your Windows machine and not with Bareos.

The FD hang problems were reported with {\bf MSDEwriter} when:
\begin{itemize}
\item a local firewall locked local access to the MSDE TCP port (MSDEwriter
seems to use TCP/IP and not Named Pipes).
\item msdtcs was installed to run under "localsystem": try running msdtcs
under  networking account (instead of local system) (com+ seems to work
better with this configuration).
\end{itemize}


\section{Windows Firewalls}
\index[general]{Firewall!Windows}
\index[general]{Windows!Firewall}

The Windows buildin Firewall is enabled since Windows version WinXP SP2.
The Bareos installer opens the required network ports for Bareos.
However, if you are using another Firewall, you might need to manually open the Bareos network port 9102/TCP.

\section{Windows Port Usage}
\index[general]{Windows!Port Usage}

If you want to see if the File daemon has properly opened the port and is
listening, you can enter the following command in a shell window:

\footnotesize
\begin{verbatim}
   netstat -an | findstr 910[123]
\end{verbatim}
\normalsize


\section{Windows Restore Problems}
\index[general]{Problem!Windows Restore}
\index[general]{Windows!Restore Problem}

Please see the
\ilink{Restore Chapter}{Windows} of this manual for problems
that you might encounter doing a restore.

\section{Windows Backup Problems}
\index[general]{Problem!Windows Backup}
\index[general]{Windows!Backup Problems}

If during a Backup, you get the message:
{\bf ERR=Access is denied} and you are using the portable option,
you should try both adding both the non-portable (backup API) and
the Volume Shadow Copy options to your Director's conf file.

In the Options resource:
\footnotesize
\begin{verbatim}
portable = no
\end{verbatim}
\normalsize

In the FileSet resource:
\footnotesize
\begin{verbatim}
enablevss = yes
\end{verbatim}
\normalsize

In general, specifying these two options should allow you to backup
any file on a Windows system.  However, in some cases, if users
have allowed to have full control of their folders, even system programs
such a Bareos can be locked out.  In this case, you must identify
which folders or files are creating the problem and do the following:

\begin{enumerate}
\item Grant ownership of the file/folder to the Administrators group,
with the option to replace the owner on all child objects.
\item Grant full control permissions to the Administrators group,
and change the user's group to only have Modify permission to
the file/folder and all child objects.
\end{enumerate}

Thanks to Georger Araujo for the above information.

\section{Windows Ownership and Permissions Problems}
\index[general]{Problem!Windows Ownership and Permissions}
\index[general]{Windows!Ownership and Permissions Problems}

If you restore files backed up from Windows to an alternate directory,
Bareos may need to create some higher level directories that were not saved
(or restored). In this case, the File daemon will create them under the SYSTEM
account because that is the account that Bareos runs under as a service and with full access permission.
However, there may be cases where you have problems accessing those files even
if you run as administrator. In principle, Microsoft supplies you with the way
to cease the ownership of those files and thus change the permissions.
However, a much better solution to working with and changing Win32 permissions
is the program {\bf SetACL}, which can be found at
\elink{http://setacl.sourceforge.net/}{http://setacl.sourceforge.net/}.

If you have not installed Bareos while running as Administrator
and if Bareos is not running as a Process with the userid (User Name) SYSTEM,
then it is very unlikely that it will have sufficient permission to
access all your files.

Some users have experienced problems restoring files that participate in
the Active Directory. They also report that changing the userid under which
Bareos (bareos-fd.exe) runs, from SYSTEM to a Domain Admin userid, resolves
the problem.


%hopefully not needed anymore
\hide{
\section{Manually resetting the Permissions}
\index[general]{Windows!Manually resetting the Permissions}

The following solution was provided by Dan Langille {\textless}dan at langille in
the dot org domain{\textgreater}. The steps are performed using Windows 2000 Server but
they should apply to most Win32 platforms. The procedure outlines how to deal
with a problem which arises when a restore creates a top-level new directory.
In this example, "top-level" means something like {\bf
c:\textbackslash{}src}, not {\bf c:\textbackslash{}tmp\textbackslash{}src}
where {\bf c:\textbackslash{}tmp} already exists. If a restore job specifies /
as the {\bf Where:} value, this problem will arise.

The problem appears as a directory which cannot be browsed with Windows
Explorer. The symptoms include the following message when you try to click on
that directory:

\includegraphics{\idir access-is-denied.eps}

If you encounter this message, the following steps will change the permissions
to allow full access.

\begin{enumerate}
\item right click on the top level directory (in this example, {\bf c:/src})
   and  select {\bf Properties}.
\item click on the Security tab.
\item If the following message appears, you can ignore it, and click on {\bf
   OK}.

\includegraphics{\idir view-only.eps}

You should see something like this:

\includegraphics{\idir properties-security.eps}
\item click on Advanced
\item click on the Owner tab
\item Change the owner to something other than the current owner (which is
   {\bf SYSTEM} in this example as shown below).

\includegraphics{\idir properties-security-advanced-owner.eps}
\item ensure the "Replace owner on subcontainers and objects" box is
   checked
\item click on OK
\item When the message "You do not have permission to read the contents of
   directory c:\textbackslash{}src\textbackslash{}basis. Do you wish to replace
   the directory permissions with permissions granting you Full Control?", click
on Yes.

\includegraphics{\idir confirm.eps}
\item Click on OK to close the Properties tab
   \end{enumerate}

With the above procedure, you should now have full control over your restored
directory.

In addition to the above methods of changing permissions, there is a Microsoft
program named {\bf cacls} that can perform similar functions.
}


\section{Fixing the Windows Boot Record}
\index[general]{Windows!Fixing the Boot Record}

An effective way to restore a Windows backup is to install Windows on a different
hard drive and restore the backup.  Then run the
recovery CD and run

\begin{verbatim}
diskpart
   select disk 0
   select part 1
   active
   exit

bootrec /rebuldbcd
bootrec /fixboot
bootrec /fixmbr
\end{verbatim}


\section{Windows Specific File daemon Command Line}
\index[general]{Windows!File Daemon!Command Line Options}
\index[fd]{Windows!Command Line Options}

These options are not normally seen or used by the user, and are documented
here only for information purposes. At the current time, to change the default
options, you must either manually run {\bf Bareos} or you must manually edit
the system registry and modify the appropriate entries.

In order to avoid option clashes between the options necessary for {\bf
Bareos} to run on Windows and the standard Bareos options, all Windows
specific options are signaled with a forward slash character (/), while as
usual, the standard Bareos options are signaled with a minus (-), or a minus
minus (\verb:--:). All the standard Bareos options can be used on the Windows
version. In addition, the following Windows only options are implemented:

\begin{description}

\item [/service ]
   Start Bareos as a service

\item [/run ]
   Run the Bareos application

\item [/install ]
   Install Bareos as a service in the system registry

\item [/remove ]
   Uninstall Bareos from the system registry

\item [/about ]
   Show the Bareos about dialogue box

\item [/status ]
   Show the Bareos status dialogue box

\item [/events ]
   Show the Bareos events dialogue box (not  yet implemented)

\item [/kill ]
   Stop any running {\bf Bareos}

\item [/help ]
   Show the Bareos help dialogue box
\end{description}

It is important to note that under normal circumstances the user should never
need to use these options as they are normally handled by the system
automatically once Bareos is installed. However, you may note these options in
some of the .bat files that have been created for your use.
