
\chapter{Director Configuration}
\label{DirectorChapter}
\label{DirectorConfChapter}
\index[general]{Director!Configuration}
\index[general]{Configuration!Director}

Of all the configuration files needed to run {Bareos}, the Director's is
the most complicated and the one that you will need to modify the most often
as you add clients or modify the FileSets.

For a general discussion of configuration files and resources including the
recognized data types see \nameref{ConfigureChapter}.

%\section{Director Resource Types}
\index[general]{Types!Director Resource}
\index[general]{Director!Resource Types}
\index[dir]{Resource Types}

Everything revolves around a job and is tied to a job in one
way or another.

The \bareosDir knows about following resource types:

\begin{itemize}
\item
   \nameref{DirectorResourceDirector} -- to  define the Director's
   name and its access password used for authenticating the Console program.
   Only a single  Director resource definition may appear in the Director's
   configuration file.
\item
   \nameref{DirectorResourceJob} -- to define the backup/restore Jobs
   and to tie together the Client, FileSet and Schedule resources to  be used
   for each Job. Normally, you will Jobs of different names corresponding
   to each client (i.e. one Job per client, but a different one with a different name
   for each client).
\item
   \nameref{DirectorResourceJobDefs} -- optional resource for
   providing defaults for Job resources.
\item
   \nameref{DirectorResourceSchedule} -- to define when a Job has to
   run. You may have any number of Schedules, but each job will reference only
   one.
\item
   \nameref{DirectorResourceFileSet} -- to define the set of files
   to be backed up for each Client. You may have any number of
   FileSets but each Job will reference only one.
\item
   \nameref{DirectorResourceClient} -- to define what Client is to be
   backed up. You will generally have multiple Client definitions. Each
   Job will reference only a single client.
\item
   \nameref{DirectorResourceStorage} -- to define on what physical
   device the Volumes should be mounted. You may have one or
   more Storage definitions.
\item
   \nameref{DirectorResourcePool} -- to define the pool of Volumes
   that can be used for a particular Job. Most people use a
   single default Pool.  However, if you have a large number
   of clients or volumes, you may want to have multiple Pools.
   Pools allow you to restrict a Job (or a Client) to use
   only a particular set of Volumes.
\item
   \nameref{DirectorResourceCatalog} -- to define in what database to
   keep the list of files and the Volume names where they are backed up.
   Most people only use a single catalog.
   It is possible, however not adviced and not supported to use multiple catalogs,
   see \nameref{MultipleCatalogs}.
\item
   \nameref{DirectorResourceMessages} -- to define where error and
   information messages are to be sent or logged. You may define
   multiple different message resources and hence direct particular
   classes of messages to different users or locations (files, ...).
\end{itemize}



\section{Director Resource}
\label{DirectorResourceDirector}
\index[general]{Director Resource}
\index[general]{Resource!Director}

The Director resource defines the attributes of the Directors running on the
network. Only a single Director
resource is allowed.


The following is an example of a valid Director resource definition:

\begin{bconfig}{Director Resource example}
Director {
  Name = bareos-dir
  Password = secretpassword
  QueryFile = "/etc/bareos/query.sql"
  Maximum Concurrent Jobs = 10
  Messages = Daemon
}
\end{bconfig}

\input{autogenerated/bareos-dir-resource-director-table.tex}
\defDirective{Dir}{Director}{Backend Directory}{}{}{%
This directive specifies a directory from where the Bareos Director loads his dynamic backends.
}

\defDirective{Dir}{Director}{Name}{}{}{%
The director name used by the system  administrator.
}

\defDirective{Dir}{Director}{Absolute Job Timeout}{}{14.2.0}{%
}

\defDirective{Dir}{Director}{Auditing}{}{14.2.0}{%
This directive allows to en- or disable auditing of interaction with the Bareos Director.
If enabled, \ilink{audit messages}{MessageTypes} will be generated.
The \ilink{messages resource}{MessagesResource} configured in \linkResourceDirective{Dir}{Director}{Messages} 
defines, how these messages are handled.
}

\defDirective{Dir}{Director}{Audit Events}{}{14.2.0}{%
Specify which commands (see \nameref{sec:ConsoleCommands}) will be audited. If nothing is specified (and \linkResourceDirective{Dir}{Director}{Auditing} is enabled), all commands will be audited.
}

\defDirective{Dir}{Director}{Description}{}{}{%
The text field contains a  description of the Director that will be displayed
in the  graphical user interface. This directive is optional.
}

\defDirective{Dir}{Director}{Dir Address}{}{}{%
This directive is optional, but if it is specified, it will cause the
Director server (for the Console program) to bind to the specified address.
If this and the \linkResourceDirective{Dir}{Director}{Dir Addresses} directives are
not specified, the Director will bind to any available address (the
default).
}


\defDirective{Dir}{Director}{Dir Addresses}{}{}{%
Specify the ports and addresses on which the Director daemon will listen
for Bareos Console connections.

Please note that if you use the \linkResourceDirective{Dir}{Director}{Dir Addresses} directive, you must
not use either a \linkResourceDirective{Dir}{Director}{Dir Port} or a \linkResourceDirective{Dir}{Director}{Dir Address} directive in the same
resource.
}


\defDirective{Dir}{Director}{Dir Port}{}{}{%
Specify the port on which the  Director daemon will
listen for Bareos Console connections.  This same port number must be
specified in the Director resource  of the Console configuration file. The
default is 9101, so  normally this directive need not be specified.  This
directive should not be used if you specify \linkResourceDirective{Dir}{Director}{Dir Addresses} (N.B plural)
directive.
}

\defDirective{Dir}{Director}{Dir Source Address}{}{}{%
This record is optional, and if it is specified, it will cause the Director
server (when initiating connections to a storage or file daemon) to source
its connections from the specified address.  Only a single IP address may be
specified.  If this record is not specified, the Director server will source
its outgoing connections according to the system routing table (the default).
}


\defDirective{Dir}{Director}{FD Connect Timeout}{}{}{%
where \parameter{time} is the time that the Director should continue
attempting to contact the File daemon to start a job, and after which
the Director will cancel the job.  The default is 3 minutes.
}

\defDirective{Dir}{Director}{Heartbeat Interval}{}{}{%
This directive is optional and if specified will cause the Director to
set a keepalive interval (heartbeat) in seconds on each of the sockets
it opens for the Client resource.  This value will override any
specified at the Director level.  It is implemented only on systems
that provide the {\bf setsockopt} TCP\_KEEPIDLE function (Linux, ...).
The default value is zero, which means no change is made to the socket.
}

\defDirective{Dir}{Director}{Key Encryption Key}{}{}{%
This key is used to encrypt the Security Key that is exchanged between
the Director and the Storage Daemon for supporting Application Managed
Encryption (AME). For security reasons each Director should have a
different Key Encryption Key.
}

\defDirective{Dir}{Director}{Maximum Concurrent Jobs}{}{}{%
\label{DirMaxConJobs}%
\index[general]{Simultaneous Jobs}%
\index[general]{Concurrent Jobs}%
where {\textless}number{\textgreater}  is the maximum number of total Director Jobs that
should run  concurrently. The default is set to 1, but you may set it to a
larger number.

The Volume format becomes more complicated with
multiple simultaneous jobs, consequently, restores may take longer if
Bareos must sort through interleaved volume blocks from  multiple simultaneous
jobs. This can be avoided by having each simultaneous job write to
a different volume or  by using data spooling, which will first spool the data
to disk simultaneously, then write one spool file at a time to the volume
thus avoiding excessive interleaving of the different job blocks.

See also the section about \ilink{Concurrent Jobs}{ConcurrentJobs}.
}

\defDirective{Dir}{Director}{Maximum Console Connections}{}{}{%
where \parameter{number}  is the maximum number of Console Connections that
could run  concurrently. The default is set to 20, but you may set it to a
larger number.
}

\defDirective{Dir}{Director}{Messages}{}{}{%
The messages resource  specifies where to deliver Director messages that are
not associated  with a specific Job. Most messages are specific to a job and
will  be directed to the Messages resource specified by the job. However,
there are a messages that can occur when no job is running.
}


\defDirective{Dir}{Director}{NDMP Snooping}{}{13.2.0}{%
This directive enables the Snooping and pretty printing of NDMP protocol
information in debugging mode.
}

\defDirective{Dir}{Director}{NDMP Log Level}{}{13.2.0}{%
This directive sets the loglevel for the NDMP protocol library.
}


\defDirective{Dir}{Director}{Omit Defaults}{}{}{%
When showing the configuration, omit those parameter that have there default value assigned.
}

\defDirective{Dir}{Director}{Optimize For Size}{}{}{%
If set to \parameter{yes} this directive will use the optimizations
for memory size over speed. So it will try to use less memory which may lead
to a somewhat lower speed. Its currently mostly used for keeping all hardlinks
in memory.

If none of \linkResourceDirective{Dir}{Director}{Optimize For Size} and \linkResourceDirective{Dir}{Director}{Optimize For Speed} is enabled, \linkResourceDirective{Dir}{Director}{Optimize For Size} is enabled by default.
}

\defDirective{Dir}{Director}{Optimize For Speed}{}{}{%
If set to \parameter{yes} this directive will use the
optimizations for speed over the memory size. So it will try to use more memory
which lead to a somewhat higher speed. Its currently mostly used for keeping all
hardlinks in memory. Its relates to the \linkResourceDirective{Dir}{Director}{Optimize For Size}
option set either one to \parameter{yes} as they are mutually exclusive.
}


\defDirective{Dir}{Director}{Password}{}{}{%
Specifies the password that must be supplied for the default Bareos
Console to be authorized.  The same password must appear in the {\bf
Director} resource of the Console configuration file.  For added
security, the password is never passed across the network but instead a
challenge response hash code created with the password.  
Bareos tries to generate a random password during the configuration
process, otherwise it will be left blank and you must manually supply
it.

The password is plain text.
}


\defDirective{Dir}{Director}{Pid Directory}{}{}{%
This directive  is optional and specifies a directory in which the Director
may put its process Id file. The process Id file is used to  shutdown
Bareos and to prevent multiple copies of  Bareos from running simultaneously.
Standard shell expansion of the {\bf Directory}  is done when the
configuration file is read so that values such  as {\bf \$HOME} will be
properly expanded.

The PID directory specified must already exist and be
readable and writable by the Bareos daemon referencing it.

Typically on Linux systems, you will set this to:  \directory{/var/run}. If you are
not installing Bareos in the  system directories, you can use the {\bf Working
Directory} as  defined above.
}


\defDirective{Dir}{Director}{Plugin Directory}{}{}{%
%If a directory is specified, the Bareos Director tries to load 
%all \ilink{Director plugins}{dirPlugins} that are located in this directory.
%If \configdirective{Plugin Names} is also specified,
%only the specified plugins get loaded.
}



\defDirective{Dir}{Director}{Plugin Names}{}{}{%
If a \linkResourceDirective{Dir}{Director}{Plugin Directory} is specified
\configdirective{Plugin Names} defines, which \ilink{Director plugins}{dirPlugins} get loaded.

If \configdirective{Plugin Names} is not defined, all plugins get loaded,
otherwise the defined ones.
%
% \begin{bconfig}{Plugin Names}
% Plugin Names = python:otherplugin
% \end{bconfig}
}



\defDirective{Dir}{Director}{Query File}{}{}{%
This directive is required and specifies a directory and file in which
the Director can find the canned SQL statements for the {\bf Query}
command of the Console.
%
%Standard shell expansion of the {\bf Path} is
%done when the configuration file is read so that values such as {\bf
%\$HOME} will be properly expanded.
}


\defDirective{Dir}{Director}{Scripts Directory}{}{}{%
}


\defDirective{Dir}{Director}{SD Connect Timeout}{}{}{%
where \parameter{time} is the time that the Director should continue
attempting to contact the Storage daemon to start a job, and after which
the Director will cancel the job.  The default is 30 minutes.
}


\defDirective{Dir}{Director}{Statistics Collect Interval}{}{14.2.0}{%
Bareos offers the possibilty to collect statistic information from its connected devices.
To do so, \linkResourceDirective{Dir}{Storage}{Collect Statistics} must be enabled.
This interval defines, how often the Director connects to the attached Storage Daemons to collect the statistic information.
}

\defDirective{Dir}{Director}{Secure Erase Command}{}{}{%
When files are no longer needed, Bareos will delete (unlink) them.
With this directive, it will call the specified command to delete these files. See \nameref{sec:SecureEraseCommand} for details.
}

\defDirective{Dir}{Director}{Statistics Retention}{}{}{%
\label{PruneStatistics}%
The \configdirective{Statistics Retention} directive defines the length of time that
Bareos will keep statistics job records in the Catalog database after the
Job End time (in \texttt{JobHistory} table). When this time period expires,
and if user runs \texttt{prune stats} command, Bareos will prune (remove)
Job records that are older than the specified period.

Theses statistics records aren't use for restore purpose, but mainly for
capacity planning, billings, etc.
See chapter about \ilink{how to extract information from the catalog}{UseBareosCatalogToExtractInformationChapter}
 for additional information.

See the \ilink{Configuration chapter}{Time} of this manual for additional
details of time specification.
}


\defDirective{Dir}{Director}{Subscriptions}{}{12.4.4}{%
In case you want check that the number of active clients don't exceed a specific number,
you can define this number here and check with the \bcommand{status subscriptions}{} command.

However, this is only indended to give a hint. No active limiting is implemented.
}


\defDirective{Dir}{Director}{Sub Sys Directory}{}{}{%
}


\defDirective{Dir}{Director}{TLS Enable}{}{}{%
Bareos can be configured to encrypt all its network traffic.
See chapter \nameref{TlsDirectives} to see,
how the Bareos Director (and the other components) must be configured to use TLS.
}


\defDirective{Dir}{Director}{Ver Id}{}{}{%
where  \parameter{string} is an identifier which can be used for support purpose.
This string is displayed using the \bcommand{version}{} command.
}


\defDirective{Dir}{Director}{Working Directory}{}{}{%
This directive is optional and specifies a directory in which the Director
may put its status files. This directory should be used only  by Bareos but
may be shared by other Bareos daemons.
Standard shell expansion of the {\bf
directory}  is done when the configuration file is read so that values such
as {\bf \$HOME} will be properly expanded.

The working directory specified must already exist and be
readable and writable by the Bareos daemon referencing it.
}

\input{autogenerated/bareos-dir-resource-director-description.tex}



\section{Job Resource}
\label{DirectorResourceJob}
\label{JobResource}
\index[general]{Resource!Job}
\index[general]{Job!Resource}

The Job resource defines a Job (Backup, Restore, ...) that Bareos must
perform. Each Job resource definition contains the name of a Client and
a FileSet to backup, the Schedule for the Job, where the data
are to be stored, and what media Pool can be used. In effect, each Job
resource must specify What, Where, How, and When or FileSet, Storage,
Backup/Restore/Level, and Schedule respectively. Note, the FileSet must
be specified for a restore job for historical reasons, but it is no longer used.

Only a single type ({\bf Backup}, {\bf Restore}, ...) can be specified for any
job. If you want to backup multiple FileSets on the same Client or multiple
Clients, you must define a Job for each one.

Note, you define only a single Job to do the Full, Differential, and
Incremental backups since the different backup levels are tied together by
a unique Job name.  Normally, you will have only one Job per Client, but
if a client has a really huge number of files (more than several million),
you might want to split it into to Jobs each with a different FileSet
covering only part of the total files.

Multiple Storage daemons are not currently supported for Jobs, so if
you do want to use multiple storage daemons, you will need to create
a different Job and ensure that for each Job that the combination of
Client and FileSet are unique.  The Client and FileSet are what Bareos
uses to restore a client, so if there are multiple Jobs with the same
Client and FileSet or multiple Storage daemons that are used, the
restore will not work.  This problem can be resolved by defining multiple
FileSet definitions (the names must be different, but the contents of
the FileSets may be the same).

\input{autogenerated/bareos-dir-resource-job-table.tex}
\defDirective{Dir}{Job}{Accurate}{}{}{%
In accurate mode, the File daemon knowns exactly which files were present
after the last backup. So it is able to handle deleted or renamed files.

When restoring a FileSet for a specified date (including "most
recent"), Bareos is able to restore exactly the files and
directories that existed at the time of the last backup prior to
that date including ensuring that deleted files are actually deleted,
and renamed directories are restored properly.

In this mode, the File daemon must keep data concerning all files in
memory.  So If you do not have sufficient memory, the backup may
either be terribly slow or fail.

%%   $$ memory = \sum_{i=1}^{n}(strlen(path_i + file_i) + sizeof(CurFile))$$

For 500.000 files (a typical desktop linux system), it will require
approximately 64 Megabytes of RAM on your File daemon to hold the
required information.
}

\defDirective{Dir}{Job}{Add Prefix}{}{}{%
This directive applies only to a Restore job and specifies a prefix to the
directory name of all files being restored.  This will use \ilink{File
Relocation}{filerelocation} feature.
}

\defDirective{Dir}{Job}{Add Suffix}{}{}{%
This directive applies only to a Restore job and specifies a suffix to all
files being restored.  This will use \ilink{File Relocation}{filerelocation}
feature.

Using \texttt{Add Suffix=.old}, \texttt{/etc/passwd} will be restored to
\texttt{/etc/passwsd.old}
}

\defDirective{Dir}{Job}{Allow Duplicate Jobs}{}{}{%
\begin{figure}[htbp]
  \centering
  \includegraphics[width=0.5\textwidth]{\idir duplicate-real}
  \caption{Allow Duplicate Jobs usage}
  \label{fig:allowduplicatejobs}
\end{figure}
A duplicate job in the sense we use it here means a second or subsequent job
with the same name starts.  This happens most frequently when the first job
runs longer than expected because no tapes are available.

If this directive is enabled duplicate jobs will be run.  If
the directive is set to {\bf no} then only one job of a given name
may run at one time, and the action that Bareos takes to ensure only
one job runs is determined by the other directives (see below).

If {\bf Allow Duplicate Jobs} is set to {\bf no} and two jobs
are present and none of the three directives given below permit
cancelling a job, then the current job (the second one started)
will be cancelled.
}

\defDirective{Dir}{Job}{Allow Higher Duplicates}{}{}{%
}

\defDirective{Dir}{Job}{Allow Mixed Priority}{}{}{%
When
set to {\bf yes} (default {\bf no}), this job may run even if lower
priority jobs are already running.  This means a high priority job
will not have to wait for other jobs to finish before starting.
The scheduler will only mix priorities when all running jobs have
this set to true.

Note that only higher priority jobs will start early.  Suppose the
director will allow two concurrent jobs, and that two jobs with
priority 10 are running, with two more in the queue.  If a job with
priority 5 is added to the queue, it will be run as soon as one of
the running jobs finishes.  However, new priority 10 jobs will not
be run until the priority 5 job has finished.
}


\defDirective{Dir}{Job}{Backup Format}{}{}{%
The backup format used for protocols which support multiple formats.
By default, it uses the Bareos \parameter{Native} Backup format.
Other protocols,
like NDMP supports different backup formats for instance:
\begin{itemize}
\item Dump
\item Tar
\item SMTape
\end{itemize}
}

\defDirective{Dir}{Job}{Base}{}{}{%
The Base directive permits to specify the list of jobs that will be used during
Full backup as base. This directive is optional. See the \ilink{Base Job
chapter}{basejobs} for more information.
}

\defDirective{Dir}{Job}{Bootstrap}{}{}{%
The Bootstrap directive specifies a bootstrap file that, if provided,
will be used during {\bf Restore} Jobs and is ignored in other Job
types.  The {\bf bootstrap} file contains the list of tapes to be used
in a restore Job as well as which files are to be restored.
Specification of this directive is optional, and if specified, it is
used only for a restore job.  In addition, when running a Restore job
from the console, this value can be changed.

If you use the {\bf Restore} command in the Console program, to start a
restore job, the {\bf bootstrap} file will be created automatically from
the files you select to be restored.

For additional details of the {\bf bootstrap} file, please see
\ilink{Restoring Files with the Bootstrap File}{BootstrapChapter} chapter
of this manual.
}

\defDirective{Dir}{Job}{Cancel Lower Level Duplicates}{}{}{%
If \linkResourceDirective{Dir}{Job}{Allow Duplicate Jobs} is set to \parameter{no} and this
directive is set to \parameter{yes}, Bareos will choose between duplicated
jobs the one with the highest level.  For example, it will cancel a
previous Incremental to run a Full backup.  It works only for Backup
jobs.
If the levels of the duplicated
jobs are the same, nothing is done and the directives
\linkResourceDirective{Dir}{Job}{Cancel Queued Duplicates} and \linkResourceDirective{Dir}{Job}{Cancel Running Duplicates}
will be examined.
}

\defDirective{Dir}{Job}{Cancel Queued Duplicates}{}{}{%
If \linkResourceDirective{Dir}{Job}{Allow Duplicate Jobs} is set to \parameter{no} and
if this directive is set to \parameter{yes} any job that is
already queued to run but not yet running will be canceled.
}

\defDirective{Dir}{Job}{Cancel Running Duplicates}{}{}{%
If \linkResourceDirective{Dir}{Job}{Allow Duplicate Jobs} is set to \parameter{no} and
if this directive is set to \parameter{yes} any job that is already running
will be canceled.
}

\defDirective{Dir}{Job}{Catalog}{}{13.4.0}{%
This specifies the name of the catalog resource to be used for this Job.
When a catalog is defined in a Job it will override the definition in
the client.
}

\defDirective{Dir}{Job}{Client}{}{}{%
The Client directive  specifies the Client (File daemon) that will be used in
the  current Job. Only a single Client may be specified in any one Job.  The
Client runs on the machine to be backed up,  and sends the requested files to
the Storage daemon for backup,  or receives them when restoring. For
additional details, see the
\nameref{DirectorResourceClient} of this chapter.
This directive is required
For versions before 13.3.0, this directive is required for all Jobtypes.
For \sinceVersion{dir}{Director Job Resource isn't required for Copy or Migrate jobs}{13.3.0}
it is required for all Jobtypes but Copy or Migrate jobs.
}

\defDirective{Dir}{Job}{Client Run After Job}{}{}{%
The specified {\bf command} is run on the client machine as soon
as data spooling is complete in order to allow restarting applications
on the client as soon as possible. .

Note, please see the notes above in {\bf RunScript}
concerning Windows clients.
}

\defDirective{Dir}{Job}{Client Run Before Job}{}{}{%
This directive is the same as {\bf Run Before Job} except that the
program is run on the client machine.  The same restrictions apply to
Unix systems as noted above for the {\bf RunScript}.
}

\defDirective{Dir}{Job}{Description}{}{}{
}

\defDirective{Dir}{Job}{Differential Backup Pool}{}{}{%
The {\bf Differential Backup Pool} specifies a Pool to be used for
Differential backups.  It will override any Pool specification during a
Differential backup.  This directive is optional.
}

\defDirective{Dir}{Job}{Differential Max Runtime}{}{}{%
The time specifies the maximum allowed time that a Differential backup job may
run, counted from when the job starts ({\bf not} necessarily the same as when
the job was scheduled).
}

\defDirective{Dir}{Job}{Differential Max Wait Time}{}{}{%
This directive has been deprecated in favor of
\linkResourceDirective{Dir}{Job}{Differential Max Runtime}.
}

\defDirective{Dir}{Job}{Dir Plugin Options}{}{}{%
These settings are plugin specific, see \nameref{dirPlugins}.
}

\defDirective{Dir}{Job}{Enabled}{}{}{%
This directive allows you to enable or disable automatic execution
  via the scheduler of a Job.
}

\defDirective{Dir}{Job}{FD Plugin Options}{}{}{%
These settings are plugin specific, see \nameref{fdPlugins}.
}

\defDirective{Dir}{Job}{File Set}{}{}{%
The FileSet directive specifies the FileSet that will be used in the
current Job.  The FileSet specifies which directories (or files) are to
be backed up, and what options to use (e.g.  compression, ...).  Only a
single FileSet resource may be specified in any one Job.  For additional
details, see the \ilink{FileSet Resource section}{FileSetResource} of
this chapter.
This directive is required (For versions before 13.3.0 for all Jobtypes
and for versions after that for all Jobtypes but Copy and Migrate).
}

\defDirective{Dir}{Job}{Full Backup Pool}{}{}{%
The {\bf Full Backup Pool} specifies a Pool to be used for Full backups.
It will override any Pool specification during a Full backup.  This
directive is optional.
}

\defDirective{Dir}{Job}{Full Max Runtime}{}{}{%
The time specifies the maximum allowed time that a Full backup job may
run, counted from when the job starts ({\bf not} necessarily the same as when
the job was scheduled).
}

\defDirective{Dir}{Job}{Full Max Wait Time}{}{}{%
This directive has been deprecated in favor of
\linkResourceDirective{Dir}{Job}{Full Max Runtime}.
}

\defDirective{Dir}{Job}{Incremental Backup Pool}{}{}{%
The {\bf Incremental Backup Pool} specifies a Pool to be used for
Incremental backups.  It will override any Pool specification during an
Incremental backup.  This directive is optional.
}

\defDirective{Dir}{Job}{Incremental Max Runtime}{}{}{%
The time specifies the maximum allowed time that an Incremental backup job may
run, counted from when the job starts, ({\bf not} necessarily the same as when
the job was scheduled).
}

\defDirective{Dir}{Job}{Incremental Max Wait Time}{}{}{%
This directive has been deprecated in favor of
\linkResourceDirective{Dir}{Job}{Incremental Max Runtime}
}

\defDirective{Dir}{Job}{Job Defs}{}{}{
If a Job Defs resource name is specified, all the values contained in the
named JobDefs resource will be used as the defaults for the current Job.
Any value that you explicitly define in the current Job resource, will
override any defaults specified in the JobDefs resource.  The use of
this directive permits writing much more compact Job resources where the
bulk of the directives are defined in one or more JobDefs.  This is
particularly useful if you have many similar Jobs but with minor
variations such as different Clients.
}

\defDirective{Dir}{Job}{Job To Verify}{}{}{
}

\defDirective{Dir}{Job}{Level}{}{}{%
The Level directive specifies the default Job level to be run.  Each
different Job Type (Backup, Restore, Verify, ...) has a different set of Levels
that can be specified.  The Level is normally overridden by a different
value that is specified in the {\bf Schedule} resource.  This directive
is not required, but must be specified either by a {\bf Level} directive
or as an override specified in the {\bf Schedule} resource.


\begin{description}
    \item [Backup] \hfill \\
        For a {\bf Backup} Job, the Level may be one of the  following:

\begin{description}

\item [Full] \hfill \\
\index[dir]{Full}
When the Level is set to Full all files in the FileSet whether or not
they have changed will be backed up.

\item [Incremental] \hfill \\
\index[dir]{Incremental}
When the Level is set to Incremental all files specified in the FileSet
that have changed since the last successful backup of the the same Job
using the same FileSet and Client, will be backed up.  If the Director
cannot find a previous valid Full backup then the job will be upgraded
into a Full backup.  When the Director looks for a valid backup record
in the catalog database, it looks for a previous Job with:

\begin{itemize}
\item The same Job name.
\item The same Client name.
\item The same FileSet (any change to the definition of  the FileSet such as
adding or deleting a file in the  Include or Exclude sections constitutes a
different FileSet.
\item The Job was a Full, Differential, or Incremental backup.
\item The Job terminated normally (i.e. did not fail or was not  canceled).
\item The Job started no longer ago than {\bf Max Full Interval}.
\end{itemize}

If all the above conditions do not hold, the Director will upgrade  the
Incremental to a Full save. Otherwise, the Incremental  backup will be
performed as requested.

The File daemon (Client) decides which files to backup for an
Incremental backup by comparing start time of the prior Job (Full,
Differential, or Incremental) against the time each file was last
"modified" (st\_mtime) and the time its attributes were last
"changed"(st\_ctime).  If the file was modified or its attributes
changed on or after this start time, it will then be backed up.

Some virus scanning software may change st\_ctime while
doing the scan.  For example, if the virus scanning program attempts to
reset the access time (st\_atime), which Bareos does not use, it will
cause st\_ctime to change and hence Bareos will backup the file during
an Incremental or Differential backup.  In the case of Sophos virus
scanning, you can prevent it from resetting the access time (st\_atime)
and hence changing st\_ctime by using the \parameter{--no-reset-atime}
option.  For other software, please see their manual.

When Bareos does an Incremental backup, all modified files that are
still on the system are backed up.  However, any file that has been
deleted since the last Full backup remains in the Bareos catalog,
which means that if between a Full save and the time you do a
restore, some files are deleted, those deleted files will also be
restored.  The deleted files will no longer appear in the catalog
after doing another Full save.

In addition, if you move a directory rather than copy it, the files in
it do not have their modification time (st\_mtime) or their attribute
change time (st\_ctime) changed.  As a consequence, those files will
probably not be backed up by an Incremental or Differential backup which
depend solely on these time stamps.  If you move a directory, and wish
it to be properly backed up, it is generally preferable to copy it, then
delete the original.

However, to manage deleted files or directories changes in the
catalog during an Incremental backup you can use \nameref{accuratemode}.
This is quite memory consuming process.

\item [Differential] \hfill \\
\index[dir]{Differential}
When the Level is set to Differential
all files specified in the FileSet that have changed since the last
successful Full backup of the same Job will be backed up.
If the Director cannot find a
valid previous Full backup for the same Job, FileSet, and Client,
backup, then the Differential job will be upgraded into a Full backup.
When the Director looks for a valid Full backup record in the catalog
database, it looks for a previous Job with:

\begin{itemize}
\item The same Job name.
\item The same Client name.
\item The same FileSet (any change to the definition of  the FileSet such as
adding or deleting a file in the  Include or Exclude sections constitutes a
different FileSet.
\item The Job was a FULL backup.
\item The Job terminated normally (i.e. did not fail or was not  canceled).
\item The Job started no longer ago than {\bf Max Full Interval}.
\end{itemize}

If all the above conditions do not hold, the Director will  upgrade the
Differential to a Full save. Otherwise, the  Differential backup will be
performed as requested.

The File daemon (Client) decides which files to backup for a
differential backup by comparing the start time of the prior Full backup
Job against the time each file was last "modified" (st\_mtime) and the
time its attributes were last "changed" (st\_ctime).  If the file was
modified or its attributes were changed on or after this start time, it
will then be backed up.  The start time used is displayed after the {\bf
Since} on the Job report.  In rare cases, using the start time of the
prior backup may cause some files to be backed up twice, but it ensures
that no change is missed.

When Bareos does a Differential backup, all modified files that are
still on the system are backed up.  However, any file that has been
deleted since the last Full backup remains in the Bareos catalog, which
means that if between a Full save and the time you do a restore, some
files are deleted, those deleted files will also be restored.  The
deleted files will no longer appear in the catalog after doing another
Full save.  However, to remove deleted files from the catalog during a
Differential backup is quite a time consuming process and not currently
implemented in Bareos.  It is, however, a planned future feature.

As noted above, if you move a directory rather than copy it, the
files in it do not have their modification time (st\_mtime) or
their attribute change time (st\_ctime) changed.  As a
consequence, those files will probably not be backed up by an
Incremental or Differential backup which depend solely on these
time stamps.  If you move a directory, and wish it to be
properly backed up, it is generally preferable to copy it, then
delete the original. Alternatively, you can move the directory, then
use the {\bf touch} program to update the timestamps.

%% TODO: merge this with incremental
However, to manage deleted files or directories changes in the
catalog during an Differential backup you can use \ilink{accurate mode}{accuratemode}.
This is quite memory consuming process. See  for more details.

Every once and a while, someone asks why we need Differential
backups as long as Incremental backups pickup all changed files.
There are possibly many answers to this question, but the one
that is the most important for me is that a Differential backup
effectively merges
all the Incremental and Differential backups since the last Full backup
into a single Differential backup.  This has two effects: 1.  It gives
some redundancy since the old backups could be used if the merged backup
cannot be read.  2.  More importantly, it reduces the number of Volumes
that are needed to do a restore effectively eliminating the need to read
all the volumes on which the preceding Incremental and Differential
backups since the last Full are done.

\item [VirtualFull] \hfill \\
\index[dir]{VirtualFull Backup}
When the Level is set to VirtualFull, a new Full backup is generated from the last existing Full backup and the matching the Differential- and Incremental-Backups. This means, a Full backup will get available, without transfering all the data from the client to backup server again. The new Full backup will be stored in the pool defined in \linkResourceDirective{Dir}{Pool}{Next Pool}. The process of generating a VirtualFull backup is similar to the process described in  \nameref{MigrationChapter}.



\end{description}

    \item [Restore] \hfill \\
        For a {\bf Restore} Job, no level needs to be specified.

    \item [Verify] \hfill \\
        For a {\bf Verify} Job, the Level may be one of the  following:

\begin{description}

\item [InitCatalog] \hfill \\
\index[dir]{InitCatalog}
does a scan of the specified {\bf FileSet} and stores the file
attributes in the Catalog database.  Since no file data is saved, you
might ask why you would want to do this.  It turns out to be a very
simple and easy way to have a {\bf Tripwire} like feature using {\bf
Bareos}.  In other words, it allows you to save the state of a set of
files defined by the {\bf FileSet} and later check to see if those files
have been modified or deleted and if any new files have been added.
This can be used to detect system intrusion.  Typically you would
specify a {\bf FileSet} that contains the set of system files that
should not change (e.g.  /sbin, /boot, /lib, /bin, ...).  Normally, you
run the {\bf InitCatalog} level verify one time when your system is
first setup, and then once again after each modification (upgrade) to
your system.  Thereafter, when your want to check the state of your
system files, you use a {\bf Verify} {\bf level = Catalog}.  This
compares the results of your {\bf InitCatalog} with the current state of
the files.

\item [Catalog] \hfill \\
\index[dir]{Catalog}
Compares the current state of the files against the state previously
saved during an {\bf InitCatalog}.  Any discrepancies are reported.  The
items reported are determined by the {\bf verify} options specified on
the {\bf Include} directive in the specified {\bf FileSet} (see the {\bf
FileSet} resource below for more details).  Typically this command will
be run once a day (or night) to check for any changes to your system
files.

\warning{If you run two Verify Catalog jobs on the same client at
the same time, the results will certainly be incorrect.  This is because
Verify Catalog modifies the Catalog database while running in order to
track new files.}

\item [VolumeToCatalog] \hfill \\
\index[dir]{VolumeToCatalog}
This level causes Bareos to read the file attribute data written to the
Volume from the last backup Job for the job specified on the {\bf VerifyJob}
directive.  The file attribute data are compared to the
values saved in the Catalog database and any differences are reported.
This is similar to the {\bf DiskToCatalog} level except that instead of
comparing the disk file attributes to the catalog database, the
attribute data written to the Volume is read and compared to the catalog
database.  Although the attribute data including the signatures (MD5 or
SHA1) are compared, the actual file data is not compared (it is not in
the catalog).

VolumeToCatalog jobs need a client to extract the metadata, but this
client does not have to be the original client. We suggest to use the
client on the backup server itself for maximum performance.

\warning{If you run two Verify VolumeToCatalog jobs on the same
client at the same time, the results will certainly be incorrect.  This
is because the Verify VolumeToCatalog modifies the Catalog database
while running.}

\item [DiskToCatalog] \hfill \\
\index[dir]{DiskToCatalog}
This level causes Bareos to read the files as they currently are on
disk, and to compare the current file attributes with the attributes
saved in the catalog from the last backup for the job specified on the
{\bf VerifyJob} directive.  This level differs from the {\bf VolumeToCatalog}
level described above by the fact that it doesn't compare against a
previous Verify job but against a previous backup.  When you run this
level, you must supply the verify options on your Include statements.
Those options determine what attribute fields are compared.

This command can be very useful if you have disk problems because it
will compare the current state of your disk against the last successful
backup, which may be several jobs.

Note, the current implementation does not identify files that
have been deleted.
\end{description}

\end{description}
}

\defDirective{Dir}{Job}{Max Diff Interval}{}{}{%
The time specifies the maximum allowed age (counting from start time) of
the most recent successful Differential backup that is required in order to run
Incremental backup jobs. If the most recent Differential backup
is older than this interval, Incremental backups will be
upgraded to Differential backups automatically. If this directive is not present,
or specified as 0, then the age of the previous Differential backup is not
considered.
}

\defDirective{Dir}{Job}{Max Full Interval}{}{}{%
The time specifies the maximum allowed age (counting from start time) of
the most recent successful Full backup that is required in order to run
Incremental or Differential backup jobs. If the most recent Full backup
is older than this interval, Incremental and Differential backups will be
upgraded to Full backups automatically. If this directive is not present,
or specified as 0, then the age of the previous Full backup is not
considered.
}

\defDirective{Dir}{Job}{Max Run Time}{}{}{%
The time specifies the maximum allowed time that a job may run, counted
from when the job starts, ({\bf not} necessarily the same as when the
job was scheduled).

By default, the the watchdog thread will kill any Job that has run more
than 6 days.  The maximum watchdog timeout is independent of MaxRunTime
and cannot be changed.
}

\defDirective{Dir}{Job}{Max Start Delay}{}{}{%
The time specifies the maximum delay between the scheduled time and the
actual start time for the Job.  For example, a job can be scheduled to
run at 1:00am, but because other jobs are running, it may wait to run.
If the delay is set to 3600 (one hour) and the job has not begun to run
by 2:00am, the job will be canceled.  This can be useful, for example,
to prevent jobs from running during day time hours.  The default is 0
which indicates no limit.
}

\defDirective{Dir}{Job}{Max Virtual Full Interval}{}{14.4.0}{%
The time specifies the maximum allowed age (counting from start time) of
the most recent successful Virtual Full backup that is required in order to run
Incremental or Differential backup jobs. If the most recent Virtual Full backup
is older than this interval, Incremental and Differential backups will be
upgraded to Virtual Full backups automatically. If this directive is not present,
or specified as 0, then the age of the previous Virtual Full backup is not
considered.
}

\defDirective{Dir}{Job}{Max Wait Time}{}{}{%
The time specifies the maximum allowed time that a job may block waiting
for a resource (such as waiting for a tape to be mounted, or waiting for
the storage or file daemons to perform their duties), counted from the
when the job starts, ({\bf not} necessarily the same as when the job was
scheduled).

\begin{figure}[htbp]
  \centering
  \includegraphics[width=13cm]{\idir different_time}
  \caption{Job time control directives}
  \label{fig:differenttime}
\end{figure}
}

\defDirective{Dir}{Job}{Maximum Bandwidth}{}{}{%
The speed parameter specifies the maximum allowed bandwidth that a job may
use.
}

\defDirective{Dir}{Job}{Maximum Concurrent Jobs}{}{}{%
where {\textless}number{\textgreater} is the maximum number of Jobs from the current
Job resource that can run concurrently.  Note, this directive limits
only Jobs with the same name as the resource in which it appears.  Any
other restrictions on the maximum concurrent jobs such as in the
Director, Client, or Storage resources will also apply in addition to
the limit specified here.  The default is set to 1, but you may set it
to a larger number.  We strongly recommend that you read the WARNING
documented under \ilink{ Maximum Concurrent Jobs}{DirMaxConJobs} in the
Director's resource.
}

\defDirective{Dir}{Job}{Maxrun Sched Time}{}{}{%
The time specifies the maximum allowed time that a job may run, counted from
when the job was scheduled. This can be useful to prevent jobs from running
during working hours. We can see it like \texttt{Max Start Delay + Max Run
Time}.
}

\defDirective{Dir}{Job}{Messages}{}{}{%
The Messages directive defines what Messages resource should be used for
this job, and thus how and where the various messages are to be
delivered.  For example, you can direct some messages to a log file, and
others can be sent by email.  For additional details, see the
\ilink{Messages Resource}{MessagesChapter} Chapter of this manual.  This
directive is required.
}

\defDirective{Dir}{Job}{Name}{}{}{%
The Job name. This name can be specified  on the {\bf Run} command in the
console program to start a job. If the  name contains spaces, it must be
specified between quotes. It is  generally a good idea to give your job the
same name as the Client  that it will backup. This permits easy
identification of jobs.

When the job actually runs, the unique Job Name will consist  of the name you
specify here followed by the date and time the  job was scheduled for
execution. This directive is required.

}

\defDirective{Dir}{Job}{Next Pool}{}{}{%
A Next Pool override used for Migration/Copy and Virtual Backup Jobs.
}

\defDirective{Dir}{Job}{Plugin Options}{}{}{
}

\defDirective{Dir}{Job}{Pool}{}{}{%
The Pool directive defines the pool of Volumes where your data can be
backed up.  Many Bareos installations will use only the {\bf Default}
pool.  However, if you want to specify a different set of Volumes for
different Clients or different Jobs, you will probably want to use
Pools.  For additional details, see the \nameref{DirectorResourcePool}
of this chapter.  This directive is required.

In case of a Copy or Migration job,
   this setting determines what Pool will be examined
   for finding JobIds to migrate.  The exception to this is when
   \linkResourceDirective{Dir}{Job}{Selection Type} = SQLQuery, 
   and although a Pool directive must still be
   specified, no Pool is used, unless you specifically include it in the
   SQL query.  Note, in any case, the Pool resource defined by the Pool
   directive must contain a \linkResourceDirective{Dir}{Pool}{Next Pool} = ... directive to define the
   Pool to which the data will be migrated.
}

\defDirective{Dir}{Job}{Prefer Mounted Volumes}{}{}{%
If the Prefer Mounted Volumes directive is set to {\bf yes},
the Storage daemon is requested to select either an Autochanger or
a drive with a valid Volume already mounted in preference to a drive
that is not ready.  This means that all jobs will attempt to append
to the same Volume (providing the Volume is appropriate -- right Pool,
... for that job), unless you are using multiple pools.
If no drive with a suitable Volume is available, it
will select the first available drive.  Note, any Volume that has
been requested to be mounted, will be considered valid as a mounted
volume by another job.  This if multiple jobs start at the same time
and they all prefer mounted volumes, the first job will request the
mount, and the other jobs will use the same volume.

If the directive is set to {\bf no}, the Storage daemon will prefer
finding an unused drive, otherwise, each job started will append to the
same Volume (assuming the Pool is the same for all jobs).  Setting
Prefer Mounted Volumes to no can be useful for those sites
with multiple drive autochangers that prefer to maximize backup
throughput at the expense of using additional drives and Volumes.
This means that the job will prefer to use an unused drive rather
than use a drive that is already in use.

Despite the above, we recommend against setting this directive to
{\bf no} since
it tends to add a lot of swapping of Volumes between the different
drives and can easily lead to deadlock situations in the Storage
daemon. We will accept bug reports against it, but we cannot guarantee
that we will be able to fix the problem in a reasonable time.

A better alternative for using multiple drives is to use multiple
pools so that Bareos will be forced to mount Volumes from those Pools
on different drives.
}

\defDirective{Dir}{Job}{Prefix Links}{}{}{%
If a {\bf Where} path prefix is specified for a recovery job, apply it
to absolute links as well.  The default is {\bf No}.  When set to {\bf
Yes} then while restoring files to an alternate directory, any absolute
soft links will also be modified to point to the new alternate
directory.  Normally this is what is desired -- i.e.  everything is self
consistent.  However, if you wish to later move the files to their
original locations, all files linked with absolute names will be broken.
}

\defDirective{Dir}{Job}{Priority}{}{}{%
This directive permits you to control the order in which your jobs will
be run by specifying a positive non-zero number. The higher the number,
the lower the job priority. Assuming you are not running concurrent jobs,
all queued jobs of priority 1 will run before queued jobs of priority 2
and so on, regardless of the original scheduling order.

The priority only affects waiting jobs that are queued to run, not jobs
that are already running.  If one or more jobs of priority 2 are already
running, and a new job is scheduled with priority 1, the currently
running priority 2 jobs must complete before the priority 1 job is
run, unless Allow Mixed Priority is set.

If you want to run concurrent jobs you should
keep these points in mind:

\begin{itemize}
\item See \nameref{ConcurrentJobs} on how to setup
concurrent jobs.

\item Bareos concurrently runs jobs of only one priority at a time.  It
will not simultaneously run a priority 1 and a priority 2 job.

\item If Bareos is running a priority 2 job and a new priority 1 job is
scheduled, it will wait until the running priority 2 job terminates even
if the Maximum Concurrent Jobs settings would otherwise allow two jobs
to run simultaneously.

\item Suppose that bareos is running a priority 2 job and a new priority 1
job is scheduled and queued waiting for the running priority 2 job to
terminate.  If you then start a second priority 2 job, the waiting
priority 1 job will prevent the new priority 2 job from running
concurrently with the running priority 2 job.  That is: as long as there
is a higher priority job waiting to run, no new lower priority jobs will
start even if the Maximum Concurrent Jobs settings would normally allow
them to run.  This ensures that higher priority jobs will be run as soon
as possible.
\end{itemize}

If you have several jobs of different priority, it may not best to start
them at exactly the same time, because Bareos must examine them one at a
time.  If by Bareos starts a lower priority job first, then it will run
before your high priority jobs.  If you experience this problem, you may
avoid it by starting any higher priority jobs a few seconds before lower
priority ones.  This insures that Bareos will examine the jobs in the
correct order, and that your priority scheme will be respected.
}

\defDirective{Dir}{Job}{Protocol}{}{}{%
The backup protocol to use to run the Job. If not set it will default
to {\bf Native} currently the director understand the following protocols:
\begin{enumerate}
\item Native - The native Bareos protocol
\item NDMP - The NDMP protocol
\end{enumerate}
}

\defDirective{Dir}{Job}{Prune Files}{}{}{%
Normally, pruning of Files from the Catalog is specified on a Client by
Client basis in the Client resource with the {\bf AutoPrune} directive.
If this directive is specified (not normally) and the value is {\bf
yes}, it will override the value specified in the Client resource.
}

\defDirective{Dir}{Job}{Prune Jobs}{}{}{%
Normally, pruning of Jobs from the Catalog is specified on a Client by
Client basis in the Client resource with the {\bf AutoPrune} directive.
If this directive is specified (not normally) and the value is {\bf
yes}, it will override the value specified in the Client resource.
}

\defDirective{Dir}{Job}{Prune Volumes}{}{}{%
Normally, pruning of Volumes from the Catalog is specified on a Pool by
Pool basis in the Pool resource with the {\bf AutoPrune} directive.
Note, this is different from File and Job pruning which is done on a
Client by Client basis.  If this directive is specified (not normally)
and the value is {\bf yes}, it will override the value specified in the
Pool resource.
}

\defDirective{Dir}{Job}{Purge Migration Job}{}{}{%
  This directive may be added to the Migration Job definition in the Director
  configuration file to purge the job migrated at the end of a migration.
}

\defDirective{Dir}{Job}{Regex Where}{}{}{%
This directive applies only to a Restore job and specifies a regex filename
manipulation of all files being restored.  This will use \ilink{File
Relocation}{filerelocation} feature.

For more informations about how use this option, see
\nameref{regexwhere}.
}

\defDirective{Dir}{Job}{Replace}{}{}{%
This directive applies only to a Restore job and specifies what happens
when Bareos wants to restore a file or directory that already exists.
You have the following options for {\bf replace-option}:

\begin{description}

\item [always]
\index[dir]{always}
when the file to be restored already exists, it is deleted and then
replaced by the copy that was backed up.  This is the default value.

\item [ifnewer]
\index[dir]{ifnewer}
if the backed up file (on tape) is newer than the existing file, the
existing file is deleted and replaced by the back up.

\item [ifolder]
\index[dir]{ifolder}
if the backed up file (on tape) is older than the existing file, the
existing file is deleted and replaced by the back up.

\item [never]
\index[dir]{never}
if the backed up file already exists, Bareos skips  restoring this file.
\end{description}
}

\defDirective{Dir}{Job}{Rerun Failed Levels}{}{}{%
If this directive is set to {\bf yes} (default no), and Bareos detects that
a previous job at a higher level (i.e.  Full or Differential) has failed,
the current job level will be upgraded to the higher level.  This is
particularly useful for Laptops where they may often be unreachable, and if
a prior Full save has failed, you wish the very next backup to be a Full
save rather than whatever level it is started as.

There are several points that must be taken into account when using this
directive: first, a failed job is defined as one that has not terminated
normally, which includes any running job of the same name (you need to
ensure that two jobs of the same name do not run simultaneously);
secondly, the {\bf Ignore FileSet Changes} directive is not considered
when checking for failed levels, which means that any FileSet change will
trigger a rerun.
}

\defDirective{Dir}{Job}{Reschedule Interval}{}{}{%
If you have specified {\bf Reschedule On Error = yes} and the job
terminates in error, it will be rescheduled after the interval of time
specified by {\bf time-specification}.  See \ilink{the time
specification formats}{Time} in the Configure chapter for details of
time specifications.  If no interval is specified, the job will not be
rescheduled on error.
}

\defDirective{Dir}{Job}{Reschedule On Error}{}{}{%
If this directive is enabled, and the job terminates in error, the job
will be rescheduled as determined by the {\bf Reschedule Interval} and
{\bf Reschedule Times} directives.  If you cancel the job, it will not
be rescheduled.  The default is {\bf no} (i.e.  the job will not be
rescheduled).

This specification can be useful for portables, laptops, or other
machines that are not always connected to the network or switched on.
}

\defDirective{Dir}{Job}{Reschedule Times}{}{}{%
This directive specifies the maximum number of times to reschedule the
job.  If it is set to zero (the default) the job will be rescheduled an
indefinite number of times.
}

\defDirective{Dir}{Job}{Run}{}{}{%
\index[dir]{Clone a Job}%
The Run directive (not to be confused with the Run option in a
Schedule) allows you to start other jobs or to clone jobs. By using the
cloning keywords (see below), you can backup
the same data (or almost the same data) to two or more drives
at the same time. The {\bf job-name} is normally the same name
as the current Job resource (thus creating a clone). However, it
may be any Job name, so one job may start other related jobs.

The part after the equal sign must be enclosed in double quotes,
and can contain any string or set of options (overrides) that you
can specify when entering the Run command from the console. For
example {\bf storage=DDS-4 ...}.  In addition, there are two special
keywords that permit you to clone the current job. They are {\bf level=\%l}
and {\bf since=\%s}. The \%l in the level keyword permits
entering the actual level of the current job and the \%s in the since
keyword permits putting the same time for comparison as used on the
current job.  Note, in the case of the since keyword, the \%s must be
enclosed in double quotes, and thus they must be preceded by a backslash
since they are already inside quotes. For example:

% \begin{bconfig}{}^^J
%    run = "Nightly-backup level=\%l since=\\"\%s\\" storage=DDS-4"^^J
% \end{bconfig}
\bconfigInput{config/DirJobRun1.conf}

A cloned job will not start additional clones, so it is not
possible to recurse.

Please note that all cloned jobs, as specified in the Run directives are
submitted for running before the original job is run (while it is being
initialized). This means that any clone job will actually start before
the original job, and may even block the original job from starting
until the original job finishes unless you allow multiple simultaneous
jobs.  Even if you set a lower priority on the clone job, if no other
jobs are running, it will start before the original job.

If you are trying to prioritize jobs by using the clone feature (Run
directive), you will find it much easier to do using a \linkResourceDirective{Dir}{Job}{Run Script}
resource, or a \linkResourceDirective{Dir}{Job}{Run Before Job} directive.
}

\defDirective{Dir}{Job}{Run After Failed Job}{}{}{
The specified command is run as an external program after the current
job terminates with any error status.  This directive is not required.  The
command string must be a valid program name or name of a shell script. If
the exit code of the program run is non-zero, Bareos will print a
warning message. Before submitting the specified command to the
operating system, Bareos performs character substitution as described above
for the {\bf RunScript} directive. Note, if you wish that your script
will run regardless of the exit status of the Job, you can use this:

% \begin{bconfig}{Run Script that runs after all jobs (successful and failed)}^^J
% Run Script \{^^J
% \ Command = "echo test"^^J
% \ Runs When = After^^J
% \ Runs On Failure = yes^^J
% \ Runs On Client  = no^^J
% \ Runs On Success = yes    \# default, you can drop this line^^J
% \}^^J
% \end{bconfig} 
\bconfigInput{config/DirJobRunAfterFailedJob1.conf}
}

\defDirective{Dir}{Job}{Run After Job}{}{}{%
The specified {\bf command} is run as an external program if the current
job terminates normally (without error or without being canceled).  This
directive is not required.  If the exit code of the program run is
non-zero, Bareos will print a warning message.  Before submitting the
specified command to the operating system, Bareos performs character
substitution as described above for the {\bf RunScript} directive.

%An example of the use of this directive is given in the
%\ilink{Tips Chapter}{JobNotification} of this manual.

See the \linkResourceDirective{Dir}{Job}{Run After Failed Job} if you
want to run a script after the job has terminated with any
non-normal status.
}

\defDirective{Dir}{Job}{Run Before Job}{}{}{%
The specified command is run as an external program prior to running the
current Job.  This directive is not required, but if it is defined, and if the
exit code of the program run is non-zero, the current Bareos job will be
canceled.

% \begin{bconfig}{}^^J
% Run Before Job = "echo test"^^J
% \end{bconfig}
\bconfigInput{config/DirJobRunBeforeJob1.conf}

   it's equivalent to :

% \begin{bconfig}{}^^J
% RunScript \{^^J
% \  Command = "echo test"^^J
% \  Runs On Client = No^^J
% \  Runs When = Before^^J
% \}^^J
% \end{bconfig}
\bconfigInput{config/DirJobRunBeforeJob2.conf}
%
% Lutz Kittler has pointed out that using the RunBeforeJob directive can be a
% simple way to modify your schedules during a holiday.  For example, suppose
% that you normally do Full backups on Fridays, but Thursday and Friday are
% holidays.  To avoid having to change tapes between Thursday and Friday when
% no one is in the office, you can create a RunBeforeJob that returns a
% non-zero status on Thursday and zero on all other days.  That way, the
% Thursday job will not run, and on Friday the tape you inserted on Wednesday
% before leaving will be used.
}

\defDirective{Dir}{Job}{Run Script}{}{}{
The RunScript directive behaves like a resource in that it
requires opening and closing braces around a number of directives
that make up the body of the runscript.

The specified {\bf Command} (see below for details) is run as an external
program prior or after the current Job.  This is optional.  By default, the
program is executed on the Client side like in \texttt{ClientRunXXXJob}.

\textbf{Console} options are special commands that are sent to the director instead
of the OS. At this time, console command ouputs are redirected to log with
the jobid 0.

You can use following console command : \texttt{delete}, \texttt{disable},
\texttt{enable}, \texttt{estimate}, \texttt{list}, \texttt{llist},
\texttt{memory}, \texttt{prune}, \texttt{purge}, \texttt{reload},
\texttt{status}, \texttt{setdebug}, \texttt{show}, \texttt{time},
\texttt{trace}, \texttt{update}, \texttt{version}, \texttt{.client},
\texttt{.jobs}, \texttt{.pool}, \texttt{.storage}.  See console chapter for
more information. You need to specify needed information on command line, nothing
will be prompted. Example:

% \begin{bconfig}{Run Script Console commands}^^J
%    Console = "prune files client=\%c"^^J
%    Console = "update stats age=3"^^J
% \end{bconfig}
\bconfigInput{config/DirJobRunScript1.conf}

You can specify more than one Command/Console option per RunScript.

You can use following options may be specified in the body
of the runscript:\\

\begin{tabular}{|c|c|c|l}
\hline
Options         & Value  & Default & Information   \\
\hline
\hline
Runs On Success & Yes{\textbar}No & {\bf Yes} & Run command if JobStatus is successful\\
\hline
Runs On Failure & Yes{\textbar}No & {\bf No} & Run command if JobStatus isn't successful\\
\hline
Runs On Client  & Yes{\textbar}No & {\bf Yes} & Run command on client\\
\hline
Runs When       & Before{\textbar}After{\textbar}Always{\textbar}\textsl{AfterVSS} & {\bf Never} & When run commands\\
\hline
Fail Job On Error & Yes/No & {\bf Yes} & Fail job if script returns
                                          something different from 0 \\
\hline
Command          &       &          & Path to your script\\
\hline
Console          &       &          & Console command\\
\hline
\end{tabular}
   \\

Any output sent by the command to standard output will be included in the
Bareos job report.  The command string must be a valid program name or name
of a shell script.

In addition, the command string is parsed then fed to the OS,
which means that the path will be searched to execute your specified
command, but there is no shell interpretation. As a consequence, if you
invoke complicated commands or want any shell features such as redirection
or piping, you must call a shell script and do it inside that script.
Alternatively, it is possible to use \command{sh -c '...'} in the command
definition to force shell interpretation, see example below.

Before submitting the specified command to the operating system, Bareos
performs character substitution of the following characters:

\label{character substitution}
\footnotesize
\begin{longtable}{ l l }
    \%\% & \% \\
    \%b & Job Bytes \\
    \%c & Client's name \\
    \%d & Daemon's name (Such as host-dir or host-fd) \\
    \%D & Director's name (Also valid on file daemon) \\
    \%e & Job Exit Status \\
    \%f & Job FileSet (Only on director side) \\
    \%F & Job Files \\
    \%h & Client address \\
    \%i & Job Id \\
    \%j & Unique Job Id \\
    \%l & Job Level \\
    \%n & Job name \\
    \%p & Pool name (Only on director side) \\
    \%P & Daemon PID \\
    \%s & Since time \\
    \%t & Job type (Backup, ...) \\
    \%v & Read Volume name(s) (Only on director side) \\
    \%V & Write Volume name(s) (Only on director side) \\
    \%w & Storage name (Only on director side) \\
    \%x & Spooling enabled? ("yes" or "no") \\
\end{longtable}
\normalsize

Some character substitutions are not available in all situations. The Job Exit
Status code \%e edits the following values:

\index[dir]{Exit Status}
\begin{itemize}
\item OK
\item Error
\item Fatal Error
\item Canceled
\item Differences
\item Unknown term code
\end{itemize}

   Thus if you edit it on a command line, you will need to enclose
   it within some sort of quotes.


You can use these following shortcuts:\\

\begin{tabular}{|l|c|c|c|c|c}
\hline
Keyword & RunsOnSuccess & RunsOnFailure  & FailJobOnError & Runs On Client & RunsWhen  \\
\hline
\hline
\linkResourceDirective{Dir}{Job}{Run Before Job}         &        &       & Yes     & No     & Before \\
\hline
\linkResourceDirective{Dir}{Job}{Run After Job}          &  Yes   &   No  &         & No     & After  \\
\hline
\linkResourceDirective{Dir}{Job}{Run After Failed Job}   &  No    &  Yes  &         & No     & After  \\
\hline
\linkResourceDirective{Dir}{Job}{Client Run Before Job}  &        &       & Yes     & Yes    & Before \\
\hline
\linkResourceDirective{Dir}{Job}{Client Run After Job}   &  Yes   &   No  &         & Yes    & After  \\
\hline
\end{tabular}

Examples:
% \begin{bconfig}{Run Script Examples}^^J
% RunScript \{^^J
% \ \   RunsWhen = Before^^J
% \ \   FailJobOnError = No^^J
% \ \   Command = "/etc/init.d/apache stop"^^J
% \}^^J
% ^^J
% RunScript \{^^J
% \ \   RunsWhen = After^^J
% \ \   RunsOnFailure = yes^^J
% \ \   Command = "/etc/init.d/apache start"^^J
% \}^^J
% \end{bconfig}
\bconfigInput{config/DirJobRunScript2.conf}

{\bf Notes about ClientRunBeforeJob}

For compatibility reasons, with this shortcut, the command is executed
directly when the client receive it. And if the command is in error, other
remote runscripts will be discarded. To be sure that all commands will be
sent and executed, you have to use RunScript syntax.

{\bf Special Windows Considerations}
\index[general]{Windows!Run Script}

You can run scripts just after snapshots initializations with
\textsl{AfterVSS} keyword.

In addition, for a Windows client, please take
note that you must ensure a correct path to your script.  The script or
program can be a .com, .exe or a .bat file.  If you just put the program
name in then Bareos will search using the same rules that cmd.exe uses
(current directory, Bareos bin directory, and PATH).  It will even try the
different extensions in the same order as cmd.exe.
The command can be anything that cmd.exe or command.com will recognize
as an executable file.

However, if you have slashes in the program name then Bareos figures you
are fully specifying the name, so you must also explicitly add the three
character extension.

The command is run in a Win32 environment, so Unix like commands will not
work unless you have installed and properly configured Cygwin in addition
to and separately from Bareos.

The System \%Path\% will be searched for the command.  (under the
environment variable dialog you have have both System Environment and
User Environment, we believe that only the System environment will be
available to bareos-fd, if it is running as a service.)

System environment variables can be referenced with \%var\% and
used as either part of the command name or arguments.

So if you have a script in the Bareos\\bin directory then the following lines
should work fine:

% \footnotesize
%\begin{bconfig}{Windows systemstate Run Script}^^J
%         Client Run Before Job = "systemstate"^^J
% or^^J
%         Client Run Before Job = "systemstate.bat"^^J
% or^^J
%         ClientRunBeforeJob = "\\"C:/Program Files/Bareos/systemstate.bat\\""^^J
%\end{bconfig}
% \normalsize

% \begin{bconfig}{Windows systemstate Run Script}^^J
%         Client Run Before Job = "systemstate"^^J
% or^^J
%         Client Run Before Job = "systemstate.bat"^^J
% or^^J
%         Client Run Before Job = "\\"C:/Program Files/Bareos/systemstate.bat\\""^^J
% \end{bconfig}
\bconfigInput{config/DirJobRunScript3.conf}

% \begin{itemize}
%     \item \path|Client Run Before Job = "systemstate"|
% or
%     \item \path|Client Run Before Job = "systemstate.bat"|
% or
%     \item \path|ClientRunBeforeJob = "\\"C:/Program Files/Bareos/systemstate.bat\\""|
% \end{itemize}

The outer set of quotes is removed when the configuration file is parsed.
You need to escape the inner quotes so that they are there when the code
that parses the command line for execution runs so it can tell what the
program name is.

% \footnotesize
% \begin{verbatim}
% ClientRunBeforeJob = "\"C:/Program Files/Software
%      Vendor/Executable\" /arg1 /arg2 \"foo bar\""
% \end{verbatim}
% \normalsize

The special characters \configCharsToQuote
will need to be quoted,
if they are part of a filename or argument.

If someone is logged in, a blank "command" window running the commands
will be present during the execution of the command.

Some Suggestions from Phil Stracchino for running on Win32 machines with
the native Win32 File daemon:

\begin{enumerate}
\item You might want the ClientRunBeforeJob directive to specify a .bat
      file which runs the actual client-side commands, rather than trying
      to run (for example) regedit /e directly.
\item The batch file should explicitly 'exit 0' on successful completion.
\item The path to the batch file should be specified in Unix form:

    \configline{Client Run Before Job = "c:/bareos/bin/systemstate.bat"}

    rather than DOS/Windows form:

    INCORRECT: \configline{Client Run Before Job = "c:\bareos\bin\systemstate.bat"}
\end{enumerate}

For Win32, please note that there are certain limitations:

\configline{Client Run Before Job = "C:/Program Files/Bareos/bin/pre-exec.bat"}

Lines like the above do not work because there are limitations of
cmd.exe that is used to execute the command.
Bareos prefixes the string you supply with \command{cmd.exe /c}.  To test that
your command works you should type \command{cmd /c "C:/Program Files/test.exe"} at a
cmd prompt and see what happens.  Once the command is correct insert a
backslash (\textbackslash{}) before each double quote ("), and
then put quotes around the whole thing when putting it in
the director's .conf file.  You either need to have only one set of quotes
or else use the short name and don't put quotes around the command path.

Below is the output from cmd's help as it relates to the command line
passed to the /c option.

If /C or /K is specified, then the remainder of the command line after
the switch is processed as a command line, where the following logic is
used to process quote (") characters:

\begin{enumerate}
\item
If all of the following conditions are met, then quote characters
on the command line are preserved:
\begin{itemize}
\item no /S switch.
\item exactly two quote characters.
\item no special characters between the two quote characters,
where special is one of: \configCharsToQuote
\item there are one or more whitespace characters between the
the two quote characters.
\item the string between the two quote characters is the name
of an executable file.
\end{itemize}

\item  Otherwise, old behavior is to see if the first character is
a quote character and if so, strip the leading character and
remove the last quote character on the command line, preserving
any text after the last quote character.
\end{enumerate}

% The following example of the use of the Client Run Before Job directive was
% submitted by a user:
%
% You could write a shell script to back up a DB2 database to a FIFO. The shell
% script is:
%
% \footnotesize
% \begin{verbatim}
%  #!/bin/sh
%  # ===== backupdb.sh
%  DIR=/u01/mercuryd
%
%  mkfifo $DIR/dbpipe
%  db2 BACKUP DATABASE mercuryd TO $DIR/dbpipe WITHOUT PROMPTING &
%  sleep 1
% \end{verbatim}
% \normalsize
%
%The following line in the Job resource in the bareos-dir.conf file:
% \footnotesize
% \begin{verbatim}
%  Client Run Before Job = "su - mercuryd -c \"/u01/mercuryd/backupdb.sh '%t'
% '%l'\""
% \end{verbatim}
% \normalsize
%
% When the job is run, you will get messages from the output of the script
% stating that the backup has started. Even though the command being run is
% backgrounded with \&, the job will block until the "db2 BACKUP DATABASE"
% command, thus the backup stalls.
%
% To remedy this situation, the "db2 BACKUP DATABASE" line should be changed to
% the following:
%
% % \footnotesize
% \begin{verbatim}
%  db2 BACKUP DATABASE mercuryd TO $DIR/dbpipe WITHOUT PROMPTING > $DIR/backup.log
% 2>&1 < /dev/null &
% \end{verbatim}
% \normalsize
%
% It is important to redirect the input and outputs of a backgrounded command to
% /dev/null to prevent the script from blocking.
}

\defDirective{Dir}{Job}{Save File History}{}{14.2.0}{%
\index[dir]{NDMP!File History}%
Allow disabling storing the file history, as this causes problems problems with some implementations of NDMP (out-of-order metadata).
}

\defDirective{Dir}{Job}{Schedule}{}{}{%
The Schedule directive defines what schedule is to be used for the Job.
The schedule in turn determines when the Job will be automatically
started and what Job level (i.e.  Full, Incremental, ...) is to be run.
This directive is optional, and if left out, the Job can only be started
manually using the Console program.  Although you may specify only a
single Schedule resource for any one job, the Schedule resource may
contain multiple {\bf Run} directives, which allow you to run the Job at
many different times, and each {\bf run} directive permits overriding
the default Job Level Pool, Storage, and Messages resources.  This gives
considerable flexibility in what can be done with a single Job.  For
additional details, see \nameref{DirectorResourceSchedule}.
}

\defDirective{Dir}{Job}{SD Plugin Options}{}{}{%
These settings are plugin specific, see \nameref{sdPlugins}.
}

\defDirective{Dir}{Job}{Selection Pattern}{}{}{%
Selection Patterns is only used for Copy and Migration jobs, see \nameref{MigrationChapter}.
The interpretation of its value depends on the selected \linkResourceDirective{Dir}{Job}{Selection Type}.

  For the OldestVolume and SmallestVolume, this
  Selection pattern is not used (ignored).

  For the Client, Volume, and Job
  keywords, this pattern must be a valid regular expression that will filter
  the appropriate item names found in the Pool.

  For the SQLQuery keyword, this pattern must be a valid \command{SELECT} SQL statement
  that returns JobIds.
}

\defDirective{Dir}{Job}{Selection Type}{}{}{%
Selection Type is only used for Copy and Migration jobs, see \nameref{MigrationChapter}.
It determines how a migration job
  will go about selecting what JobIds to migrate. In most cases, it is
  used in conjunction with a \linkResourceDirective{Dir}{Job}{Selection Pattern}
  to give you fine
  control over exactly what JobIds are selected.
  The possible values are:
  \begin{description}
  \item [SmallestVolume] This selection keyword selects the volume with the
        fewest bytes from the Pool to be migrated.  The Pool to be migrated
        is the Pool defined in the Migration Job resource.  The migration
        control job will then start and run one migration backup job for
        each of the Jobs found on this Volume.  The Selection Pattern, if
        specified, is not used.

  \item [OldestVolume] This selection keyword selects the volume with the
        oldest last write time in the Pool to be migrated.  The Pool to be
        migrated is the Pool defined in the Migration Job resource.  The
        migration control job will then start and run one migration backup
        job for each of the Jobs found on this Volume.  The Selection
        Pattern, if specified, is not used.

  \item [Client] The Client selection type, first selects all the Clients
        that have been backed up in the Pool specified by the Migration
        Job resource, then it applies the \linkResourceDirective{Dir}{Job}{Selection Pattern}
        as a regular expression to the list of Client names, giving
        a filtered Client name list.  All jobs that were backed up for those
        filtered (regexed) Clients will be migrated.
        The migration control job will then start and run one migration
        backup job for each of the JobIds found for those filtered Clients.

  \item [Volume] The Volume selection type, first selects all the Volumes
        that have been backed up in the Pool specified by the Migration
        Job resource, then it applies the \linkResourceDirective{Dir}{Job}{Selection Pattern}
        as a regular expression to the list of Volume names, giving
        a filtered Volume list.  All JobIds that were backed up for those
        filtered (regexed) Volumes will be migrated.
        The migration control job will then start and run one migration
        backup job for each of the JobIds found on those filtered Volumes.

  \item [Job] The Job selection type, first selects all the Jobs (as
        defined on the \linkResourceDirective{Dir}{Job}{Name} directive in a Job resource)
        that have been backed up in the Pool specified by the Migration
        Job resource, then it applies the \linkResourceDirective{Dir}{Job}{Selection Pattern}
        as a regular expression to the list of Job names, giving
        a filtered Job name list.  All JobIds that were run for those
        filtered (regexed) Job names will be migrated.  Note, for a given
        Job named, they can be many jobs (JobIds) that ran.
        The migration control job will then start and run one migration
        backup job for each of the Jobs found.

  \item [SQLQuery] The SQLQuery selection type, used the 
        \linkResourceDirective{Dir}{Job}{Selection Pattern}
        as an SQL query to obtain the JobIds to be migrated.
        The Selection Pattern must be a valid SELECT SQL statement for your
        SQL engine, and it must return the JobId as the first field
        of the SELECT.

  \item [PoolOccupancy] This selection type will cause the Migration job
        to compute the total size of the specified pool for all Media Types
        combined. If it exceeds the \linkResourceDirective{Dir}{Pool}{Migration High Bytes} 
        defined in
        the Pool, the Migration job will migrate all JobIds beginning with
        the oldest Volume in the pool (determined by Last Write time) until
        the Pool bytes drop below the \linkResourceDirective{Dir}{Pool}{Migration Low Bytes} 
        defined in the
        Pool. This calculation should be consider rather approximative because
        it is made once by the Migration job before migration is begun, and
        thus does not take into account additional data written into the Pool
        during the migration.  In addition, the calculation of the total Pool
        byte size is based on the Volume bytes saved in the Volume (Media) database
        entries. The bytes calculate for Migration is based on the value stored
        in the Job records of the Jobs to be migrated. These do not include the
        Storage daemon overhead as is in the total Pool size. As a consequence,
        normally, the migration will migrate more bytes than strictly necessary.

  \item [PoolTime] The PoolTime selection type will cause the Migration job to
        look at the time each JobId has been in the Pool since the job ended.
        All Jobs in the Pool longer than the time specified on 
        \linkResourceDirective{Dir}{Pool}{Migration Time}
        directive in the Pool resource will be migrated.

  \item [PoolUncopiedJobs] This selection which copies all jobs from a pool
        to an other pool which were not copied before is available only for copy Jobs.

  \end{description}

}

\defDirective{Dir}{Job}{Spool Attributes}{}{}{%
The default is set to {\bf no}, which means that the File attributes are
sent by the Storage daemon to the Director as they are stored on tape.
However, if you want to avoid the possibility that database updates will
slow down writing to the tape, you may want to set the value to {\bf
yes}, in which case the Storage daemon will buffer the File attributes
and Storage coordinates to a temporary file in the Working Directory,
then when writing the Job data to the tape is completed, the attributes
and storage coordinates will be sent to the Director.

NOTE: When \linkResourceDirective{Dir}{Job}{Spool Data} is set to yes, Spool Attributes is also
automatically set to yes.
}

\defDirective{Dir}{Job}{Spool Data}{}{}{%
If this directive is set  to {\bf yes} (default no), the Storage daemon will
be requested  to spool the data for this Job to disk rather than write it
directly to the Volume (normally a tape).

Thus the data is written in large blocks to the Volume rather than small
blocks.  This directive is particularly useful when running multiple
simultaneous backups to tape.  Once all the data arrives or the spool
files' maximum sizes are reached, the data will be despooled and written
to tape.

Spooling data prevents interleaving data from several job and reduces or
eliminates tape drive stop and start commonly known as "shoe-shine".

We don't recommend using this option if you are writing to a disk file
using this option will probably just slow down the backup jobs.

NOTE: When this directive is set to yes, \linkResourceDirective{Dir}{Job}{Spool Attributes} is also
automatically set to yes.
}

\defDirective{Dir}{Job}{Spool Size}{}{}{%
where the bytes specify the maximum spool size for this job.
The default is take from Device Maximum Spool Size limit.
}

\defDirective{Dir}{Job}{Storage}{}{}{%
The Storage directive defines the name of the storage services where you
want to backup the FileSet data.  For additional details, see the
\nameref{DirectorResourceStorage} of this manual.
The Storage resource may also be specified in the Job's Pool resource,
in which case the value in the Pool resource overrides any value
in the Job. This Storage resource definition is not required by either
the Job resource or in the Pool, but it must be specified in
one or the other, if not an error will result.
}

\defDirective{Dir}{Job}{Strip Prefix}{}{}{
This directive applies only to a Restore job and specifies a prefix to remove
from the directory name of all files being restored.  This will use the
\ilink{File Relocation}{filerelocation} feature.

Using \texttt{Strip Prefix=/etc}, \texttt{/etc/passwd} will be restored to
\texttt{/passwd}

Under Windows, if you want to restore \texttt{c:/files} to \texttt{d:/files},
you can use:

% \begin{bconfig}{}^^J
%  Strip Prefix = c:^^J
%  Add Prefix = d:^^J
% \end{bconfig}
% 
\bconfigInput{config/DirJobStripPrefix1.conf}
}

\defDirective{Dir}{Job}{Type}{}{}{%
The {\bf Type} directive specifies  the Job type, which may be one of the
following: {\bf Backup},  {\bf Restore}, {\bf Verify}, or {\bf Admin}. This
directive  is required. Within a particular Job Type, there are also Levels
as discussed in the next item.

\begin{description}

\item [Backup] \hfill \\
\index[dir]{Backup}
Run a backup Job. Normally you will  have at least one Backup job for each
client you want  to save. Normally, unless you turn off cataloging,  most all
the important statistics and data concerning  files backed up will be placed
in the catalog.

\item [Restore] \hfill \\
\index[dir]{Restore}
Run a restore Job.  Normally, you will specify only one Restore job
which acts as a sort of prototype that you will modify using the console
program in order to perform restores.  Although certain basic
information from a Restore job is saved in the catalog, it is very
minimal compared to the information stored for a Backup job -- for
example, no File database entries are generated since no Files are
saved.

{\bf Restore} jobs cannot be
automatically started by the scheduler as is the case for Backup, Verify
and Admin jobs. To restore files, you must use the {\bf restore} command
in the console.


\item [Verify] \hfill \\
\index[dir]{Verify}
Run a verify Job. In general, {\bf verify}  jobs permit you to compare the
contents of the catalog  to the file system, or to what was backed up. In
addition,  to verifying that a tape that was written can be read,  you can
also use {\bf verify} as a sort of tripwire  intrusion detection.

\item [Admin] \hfill \\
\index[dir]{Admin}
Run an admin Job. An {\bf Admin} job can  be used to periodically run catalog
pruning, if you  do not want to do it at the end of each {\bf Backup}  Job.
Although an Admin job is recorded in the  catalog, very little data is saved.

\item [Migrate]
   defines the job that is run as being a
   Migration Job.  A Migration Job is a sort of control job and does not have
   any Files associated with it, and in that sense they are more or less like
   an Admin job.  Migration jobs simply check to see if there is anything to
   Migrate then possibly start and control new Backup jobs to migrate the data
   from the specified Pool to another Pool.  Note, any original JobId that
   is migrated will be marked as having been migrated, and the original
   JobId can nolonger be used for restores; all restores will be done from
   the new migrated Job.

\item [Copy]
   defines the job that is run as being a
   Copy Job.  A Copy Job is a sort of control job and does not have
   any Files associated with it, and in that sense they are more or less like
   an Admin job.  Copy jobs simply check to see if there is anything to
   Copy then possibly start and control new Backup jobs to copy the data
   from the specified Pool to another Pool.  Note that when a copy is
   made, the original JobIds are left unchanged. The new copies can not
   be used for restoration unless you specifically choose them by JobId.
   If you subsequently delete a JobId that has a copy, the copy will be
   automatically upgraded to a Backup rather than a Copy, and it will
   subsequently be used for restoration.
\end{description}
}

\defDirective{Dir}{Job}{Verify Job}{}{}{%
If you run a verify job without this directive, the last job run will be
compared with the catalog, which means that you must immediately follow
a backup by a verify command.  If you specify a {\bf Verify Job} Bareos
will find the last job with that name that ran.  This permits you to run
all your backups, then run Verify jobs on those that you wish to be
verified (most often a {\bf VolumeToCatalog}) so that the tape just
written is re-read.
}

\defDirective{Dir}{Job}{Where}{}{}{%
This directive applies only to a Restore job and specifies a prefix to
the directory name of all files being restored.  This permits files to
be restored in a different location from which they were saved.  If {\bf
Where} is not specified or is set to backslash ({\bf /}), the files will
be restored to their original location.  By default, we have set {\bf
Where} in the example configuration files to be {\bf
/tmp/bareos-restores}.  This is to prevent accidental overwriting of
your files.
}

\defDirective{Dir}{Job}{Write Bootstrap}{}{}{%
The {\bf writebootstrap} directive specifies a file name where Bareos
will write a {\bf bootstrap} file for each Backup job run.  This
directive applies only to Backup Jobs.  If the Backup job is a Full
save, Bareos will erase any current contents of the specified file
before writing the bootstrap records.  If the Job is an Incremental
or Differential
save, Bareos will append the current bootstrap record to the end of the
file.

Using this feature, permits you to constantly have a bootstrap file that
can recover the current state of your system.  Normally, the file
specified should be a mounted drive on another machine, so that if your
hard disk is lost, you will immediately have a bootstrap record
available.  Alternatively, you should copy the bootstrap file to another
machine after it is updated. Note, it is a good idea to write a separate
bootstrap file for each Job backed up including the job that backs up
your catalog database.

If the {\bf bootstrap-file-specification} begins with a vertical bar
(\textbar), Bareos will use the specification as the name of a program to which
it will pipe the bootstrap record.  It could for example be a shell
script that emails you the bootstrap record.

Before opening the file or executing the
specified command, Bareos performs
\ilink{character substitution}{character substitution} like in RunScript
directive. To automatically manage your bootstrap files, you can use
this in your {\bf JobDefs} resources:
% \begin{bconfig}{Write Bootstrap Example}^^J
% Job Defs \{^^J
% \ \ Write Bootstrap = "\%c_\%n.bsr"^^J
% \ \ ...^^J
% \}^^J
% \end{bconfig}
\bconfigInput{config/DirJobWriteBootstrap1.conf}

For more details on using this file, please see chapter \nameref{BootstrapChapter}.
}

\defDirective{Dir}{Job}{Write Part After Job}{}{}{
}

\defDirective{Dir}{Job}{Write Verify List}{}{}{
}

\input{autogenerated/bareos-dir-resource-job-description.tex}


The following is an example of a valid Job resource definition:

\begin{bconfig}{Job Resource Example}
Job {
  Name = "Minou"
  Type = Backup
  Level = Incremental                 # default
  Client = Minou
  FileSet="Minou Full Set"
  Storage = DLTDrive
  Pool = Default
  Schedule = "MinouWeeklyCycle"
  Messages = Standard
}
\end{bconfig}

\section{JobDefs Resource}
\label{DirectorResourceJobDefs}
\index[general]{Job!JobDefs Resource}
\index[general]{Resource!JobDefs}

The JobDefs resource permits all the same directives that can appear in a Job
resource. However, a JobDefs resource does not create a Job, rather it can be
referenced within a Job to provide defaults for that Job. This permits you to
concisely define several nearly identical Jobs, each one referencing a JobDefs
resource which contains the defaults. Only the changes from the defaults need to
be mentioned in each Job.

% \input{autogenerated/bareos-dir-resource-jobdefs-table.tex}
% \input{director-resource-jobdefs-definitions.tex}
% \input{autogenerated/bareos-dir-resource-jobdefs-description.tex}

\section{Schedule Resource}
\label{DirectorResourceSchedule}
\index[general]{Resource!Schedule}
\index[general]{Schedule!Resource}

The Schedule resource provides a means of automatically scheduling a Job as
well as the ability to override the default Level, Pool, Storage and Messages
resources. If a Schedule resource is not referenced in a Job, the Job can only
be run manually. In general, you specify an action to be taken and when.

\input{autogenerated/bareos-dir-resource-schedule-table.tex}
\defDirective{Dir}{Schedule}{Description}{}{}{%
}

\defDirective{Dir}{Schedule}{Enabled}{}{}{%
}

\defDirective{Dir}{Schedule}{Name}{}{}{%
The name of the schedule being defined.
}

\defDirective{Dir}{Schedule}{Run}{}{}{%
The Run directive defines when a Job is to be run, and what overrides if
any to apply.  You may specify multiple {\bf run} directives within a
{\bf Schedule} resource.  If you do, they will all be applied (i.e.
multiple schedules).  If you have two {\bf Run} directives that start at
the same time, two Jobs will start at the same time (well, within one
second of each other).

The {\bf Job-overrides} permit overriding the Level, the Storage, the
Messages, and the Pool specifications provided in the Job resource.  In
addition, the FullPool, the IncrementalPool, and the DifferentialPool
specifications permit overriding the Pool specification according to
what backup Job Level is in effect.

By the use of overrides, you may customize a particular Job.  For
example, you may specify a Messages override for your Incremental
backups that outputs messages to a log file, but for your weekly or
monthly Full backups, you may send the output by email by using a
different Messages override.

{\bf Job-overrides} are specified as: {\bf keyword=value} where the
keyword is Level, Storage, Messages, Pool, FullPool, DifferentialPool,
or IncrementalPool, and the {\bf value} is as defined on the respective
directive formats for the Job resource.  You may specify multiple {\bf
Job-overrides} on one {\bf Run} directive by separating them with one or
more spaces or by separating them with a trailing comma.  For example:

\begin{description}

\item [Level=Full]
\index[dir]{Level}
\index[dir]{Directive!Level}
is all files in the FileSet whether or not  they have changed.

\item [Level=Incremental]
\index[dir]{Level}
\index[dir]{Directive!Level}
is all files that have changed since  the last backup.

\item [Pool=Weekly]
\index[dir]{Pool}
\index[dir]{Directive!Pool}
specifies to use the Pool named {\bf Weekly}.

\item [Storage=DLT\_Drive]
\index[dir]{Storage}
\index[dir]{Directive!Storage}
specifies to use {\bf DLT\_Drive} for  the storage device.

\item [Messages=Verbose]
\index[dir]{Messages}
\index[dir]{Directive!Messages}
specifies to use the {\bf Verbose}  message resource for the Job.

\item [FullPool=Full]
\index[dir]{FullPool}
\index[dir]{Directive!FullPool}
specifies to use the Pool named {\bf Full}  if the job is a full backup, or
is upgraded from another type  to a full backup.

\item [DifferentialPool=Differential]
\index[dir]{DifferentialPool}
\index[dir]{Directive!DifferentialPool}
specifies to use the Pool named {\bf Differential} if the job is a
differential  backup.

\item [IncrementalPool=Incremental]
\index[dir]{IncrementalPool}
\index[dir]{Directive!IncrementalPool}
specifies to use the Pool  named {\bf Incremental} if the job is an
incremental  backup.

\item [Accurate=yes{\textbar}no]
\index[dir]{Accurate}
\index[dir]{Directive!Accurate}
tells Bareos to use or not the Accurate code for the specific job. It can
allow you to save memory and and CPU resources on the catalog server in some
cases.

\end{description}

{\bf Date-time-specification} determines when the  Job is to be run. The
specification is a repetition, and as  a default Bareos is set to run a job at
the beginning of the  hour of every hour of every day of every week of every
month  of every year. This is not normally what you want, so you  must specify
or limit when you want the job to run. Any  specification given is assumed to
be repetitive in nature and  will serve to override or limit the default
repetition. This  is done by specifying masks or times for the hour, day of the
month, day of the week, week of the month, week of the year,  and month when
you want the job to run. By specifying one or  more of the above, you can
define a schedule to repeat at  almost any frequency you want.

Basically, you must supply a {\bf month}, {\bf day}, {\bf hour}, and  {\bf
minute} the Job is to be run. Of these four items to be specified,  {\bf day}
is special in that you may either specify a day of the month  such as 1, 2,
... 31, or you may specify a day of the week such  as Monday, Tuesday, ...
Sunday. Finally, you may also specify a  week qualifier to restrict the
schedule to the first, second, third,  fourth, or fifth week of the month.

For example, if you specify only a day of the week, such as {\bf Tuesday}  the
Job will be run every hour of every Tuesday of every Month. That  is the {\bf
month} and {\bf hour} remain set to the defaults of  every month and all
hours.

Note, by default with no other specification, your job will run  at the
beginning of every hour. If you wish your job to run more than  once in any
given hour, you will need to specify multiple {\bf run}  specifications each
with a different minute.

The date/time to run the Job can be specified in the following way  in
pseudo-BNF:

\begin{longtable}{ l @{ ::= } p{0.5\textwidth} }
\bnfvar{week-keyword}    & 1st \pipe 2nd \pipe 3rd \pipe 4th \pipe 5th \pipe first \pipe
                     second \pipe third \pipe fourth \pipe fifth \\
\bnfvar{wday-keyword}    & sun \pipe mon \pipe tue \pipe wed \pipe thu \pipe fri \pipe sat \pipe
                    sunday \pipe monday \pipe tuesday \pipe wednesday \pipe
                    thursday \pipe friday \pipe saturday \\
\bnfvar{week-of-year-keyword} & w00 \pipe w01 \pipe ... w52 \pipe w53 \\
\bnfvar{month-keyword}   & jan \pipe feb \pipe mar \pipe apr \pipe may \pipe jun \pipe jul \pipe
                    aug \pipe sep \pipe oct \pipe nov \pipe dec \pipe
                    january \pipe february \pipe ... \pipe december \\
\bnfvar{digit}           & 1 \pipe 2 \pipe 3 \pipe 4 \pipe 5 \pipe 6 \pipe 7 \pipe 8 \pipe 9 \pipe 0 \\
\bnfvar{number}          & \bnfvar{digit} \pipe \bnfvar{digit}\bnfvar{number} \\
\bnfvar{12hour}          & 0 \pipe 1 \pipe 2 \pipe ... 12 \\
\bnfvar{hour}            & 0 \pipe 1 \pipe 2 \pipe ... 23 \\
\bnfvar{minute}          & 0 \pipe 1 \pipe 2 \pipe ... 59 \\
\bnfvar{day}             & 1 \pipe 2 \pipe ... 31 \\
\bnfvar{time}            & \bnfvar{hour}:\bnfvar{minute} \pipe
                    \bnfvar{12hour}:\bnfvar{minute}am \pipe
                    \bnfvar{12hour}:\bnfvar{minute}pm \\
\bnfvar{time-spec}       & at \bnfvar{time} \pipe hourly \\
% ??? \bnfvar{date-keyword}    & on \pipe weekly \\
\bnfvar{day-range}       & \bnfvar{day}-\bnfvar{day} \\
\bnfvar{month-range}     & \bnfvar{month-keyword}-\bnfvar{month-keyword} \\
\bnfvar{wday-range}      & \bnfvar{wday-keyword}-\bnfvar{wday-keyword} \\
\bnfvar{range}           & \bnfvar{day-range} \pipe \bnfvar{month-range} \pipe
                          \bnfvar{wday-range} \\
\bnfvar{date}            & \bnfvar{date-keyword} \pipe \bnfvar{day} \pipe \bnfvar{range} \\
\bnfvar{date-spec}       & \bnfvar{date} \pipe \bnfvar{date-spec} \\
\bnfvar{day-spec}        & \bnfvar{day} \pipe \bnfvar{wday-keyword} \pipe
                    \bnfvar{day} \pipe \bnfvar{wday-range} \pipe
                    \bnfvar{week-keyword} \bnfvar{wday-keyword} \pipe
                    \bnfvar{week-keyword} \bnfvar{wday-range} \pipe
                    daily \\
\bnfvar{month-spec}      & \bnfvar{month-keyword} \pipe \bnfvar{month-range} \pipe monthly \\
\bnfvar{date-time-spec}  & \bnfvar{month-spec} \bnfvar{day-spec} \bnfvar{time-spec} \\
\end{longtable}
}

\input{autogenerated/bareos-dir-resource-schedule-description.tex}

Note, the Week of Year specification wnn follows the ISO standard definition
of the week of the year, where Week 1 is the week in which the first Thursday
of the year occurs, or alternatively, the week which contains the 4th of
January. Weeks are numbered w01 to w53. w00 for Bareos is the week that
precedes the first ISO week (i.e. has the first few days of the year if any
occur before Thursday). w00 is not defined by the ISO specification. A week
starts with Monday and ends with Sunday.

According to the NIST (US National Institute of Standards and Technology),
12am and 12pm are ambiguous and can be defined to anything.  However,
12:01am is the same as 00:01 and 12:01pm is the same as 12:01, so Bareos
defines 12am as 00:00 (midnight) and 12pm as 12:00 (noon).  You can avoid
this abiguity (confusion) by using 24 hour time specifications (i.e.  no
am/pm).

An example schedule resource that is named {\bf WeeklyCycle} and runs a job
with level full each Sunday at 2:05am and an incremental job Monday through
Saturday at 2:05am is:

\begin{bconfig}{Schedule Example}
Schedule {
  Name = "WeeklyCycle"
  Run = Level=Full sun at 2:05
  Run = Level=Incremental mon-sat at 2:05
}
\end{bconfig}

An example of a possible monthly cycle is as follows:

\begin{bconfig}{}
Schedule {
  Name = "MonthlyCycle"
  Run = Level=Full Pool=Monthly 1st sun at 2:05
  Run = Level=Differential 2nd-5th sun at 2:05
  Run = Level=Incremental Pool=Daily mon-sat at 2:05
}
\end{bconfig}

The first of every month:

\begin{bconfig}{}
Schedule {
  Name = "First"
  Run = Level=Full on 1 at 2:05
  Run = Level=Incremental on 2-31 at 2:05
}
\end{bconfig}

The last friday of the month (i.e. the last friday in the last week of the month)

\begin{bconfig}{}
Schedule {
  Name = "Last Friday"
  Run = Level=Full last fri at 21:00
}
\end{bconfig}

Every 10 minutes:

\begin{bconfig}{}
Schedule {
  Name = "TenMinutes"
  Run = Level=Full hourly at 0:05
  Run = Level=Full hourly at 0:15
  Run = Level=Full hourly at 0:25
  Run = Level=Full hourly at 0:35
  Run = Level=Full hourly at 0:45
  Run = Level=Full hourly at 0:55
}
\end{bconfig}

The {\bf modulo scheduler} makes it easy to specify schedules like odd or even days/weeks, or more generally every n days or weeks. It is called modulo scheduler because it uses the modulo to determine if the schedule must be run or not. The second variable behind the slash lets you determine in which cycle of days/weeks a job should be run. The first part determines on which day/week the job should be run first. E.g. if you want to run a backup in a 5-week-cycle, starting on week 3, you set it up as w03/w05.

\begin{bconfig}{Schedule Examples: modulo}
Schedule {
  Name = "Odd Days"
  Run = 1/2 at 23:10
}

Schedule {
  Name = "Even Days"
  Run = 2/2 at 23:10
}

Schedule {
  Name = "On the 3rd week in a 5-week-cycle"
  Run = w03/w05 at 23:10
}

Schedule {
  Name = "Odd Weeks"
  Run = w01/w02 at 23:10
}

Schedule {
  Name = "Even Weeks"
  Run = w02/w02 at 23:10
}
\end{bconfig}

\subsection{Technical Notes on Schedules}
\index[general]{Schedule!Technical Notes on Schedules}

Internally Bareos keeps a schedule as a bit mask. There are six masks and a
minute field to each schedule. The masks are hour, day of the month (mday),
month, day of the week (wday), week of the month (wom), and week of the year
(woy). The schedule is initialized to have the bits of each of these masks
set, which means that at the beginning of every hour, the job will run. When
you specify a month for the first time, the mask will be cleared and the bit
corresponding to your selected month will be selected. If you specify a second
month, the bit corresponding to it will also be added to the mask. Thus when
Bareos checks the masks to see if the bits are set corresponding to the
current time, your job will run only in the two months you have set. Likewise,
if you set a time (hour), the hour mask will be cleared, and the hour you
specify will be set in the bit mask and the minutes will be stored in the
minute field.

For any schedule you have defined, you can see how these bits are set by doing
a {\bf show schedules} command in the Console program. Please note that the
bit mask is zero based, and Sunday is the first day of the week (bit zero).



\section{FileSet Resource}
\label{DirectorResourceFileSet}
\label{FileSetResource}
\index[general]{Resource!FileSet}
\index[general]{FileSet!Resource}

The FileSet resource defines what files are to be included or excluded in a
backup job.  A {\bf FileSet} resource is required for each backup Job.  It
consists of a list of files or directories to be included, a list of files
or directories to be excluded and the various backup options such as
compression, encryption, and signatures that are to be applied to each
file.

Any change to the list of the included files will cause Bareos to
automatically create a new FileSet (defined by the name and an MD5 checksum
of the Include/Exclude contents).  Each time a new FileSet is created,
Bareos will ensure that the next backup is always a Full save.

\input{autogenerated/bareos-dir-resource-fileset-table.tex}
\defDirective{Dir}{Fileset}{Description}{}{}{%
Information only.
}

\defDirective{Dir}{Fileset}{Enable VSS}{}{}{%
\index{dir}{Windows!Enable VSS}%
If this directive is set to {\bf yes} the File daemon will be notified
that the user wants to use a Volume Shadow Copy Service (VSS) backup
for this job. This directive is effective only on the Windows File Daemon.
It permits a consistent copy
of open files to be made for cooperating writer applications, and for
applications that are not VSS away, Bareos can at least copy open files.
The Volume Shadow Copy will only be done on Windows drives where the
drive (e.g. C:, D:, ...) is explicitly mentioned in a {\bf File}
directive.
For more information, please see the
\ilink{Windows}{VSS} chapter of this manual.
}

\defDirective{Dir}{Fileset}{Exclude}{}{}{%
Describe the files, that should get excluded from a backup, see section about the \nameref{fileset-exclude}.
}

\defDirective{Dir}{Fileset}{Ignore Fileset Changes}{}{}{%
Normally, if you modify the FileSet Include or Exclude lists,
the next backup will be forced to a Full so that Bareos can
guarantee that any additions or deletions are properly saved.

We strongly recommend against setting this directive to yes,
since doing so may cause you to have an incomplete set of backups.

If this directive is set to {\bf yes}, any changes you make to the
FileSet Include or Exclude lists, will not force a Full during
subsequent backups.

The default is {\bf no}, in which case, if you change the Include or
Exclude, Bareos will force a Full backup to ensure that everything is
properly backed up.
}

\defDirective{Dir}{Fileset}{Include}{}{}{%
Describe the files, that should get included to a backup, see section about the \nameref{fileset-include}.
}

\defDirective{Dir}{Fileset}{Name}{}{}{%
The name of the FileSet resource.
}

\input{autogenerated/bareos-dir-resource-fileset-description.tex}


\subsection{FileSet Include Ressource}
\label{fileset-include}

The Include resource must contain a list of directories and/or files to be
processed in the backup job.

Normally, all files found in all
subdirectories of any directory in the Include File list will be backed up.
Note, see below for the definition of {\textless}file-list{\textgreater}.
The Include resource may also contain one or more Options resources that
specify options such as compression to be applied to all or any subset of
the files found when processing the file-list for backup. Please see
below for more details concerning Options resources.

There can be any number of {\bf Include} resources within the FileSet, each
having its own list of directories or files to be backed up and the backup
options defined by one or more Options resources.  

Please take note of the following items in the FileSet syntax:

\begin{enumerate}
\item There is no equal sign (=) after the Include and before the opening
   brace (\{). The same is true for the Exclude.
\item Each directory (or filename) to be included or excluded is preceded by a {\bf File
   =}.  Previously they were simply listed on separate lines.
\item The Exclude resource does not accept Options.
\item When using wild-cards or regular expressions, directory names are
   always terminated with a slash (/) and filenames have no trailing slash.
\end{enumerate}

\begin{description}
\directive{dir}{File}{ filename \textbar\ dirname \textbar\ \textbar command \textbar\ \textbackslash\textless includefile-client \textbar\ \textless includefile-server }{}{}{}
    The file list
    consists of one file or directory name per line.  Directory names should be
    specified without a trailing slash with Unix path notation.

    Windows users, please take note to specify directories (even c:/...) in
    Unix path notation. If you use Windows conventions, you will most likely
    not be able to restore your files due to the fact that the Windows
    path separator was defined as an escape character long before Windows
    existed, and Bareos adheres to that convention (i.e. means the next character
    appears as itself).

    You should always specify a full path for every directory and file that you
    list in the FileSet.  In addition, on Windows machines, you should {\bf
    always} prefix the directory or filename with the drive specification
    (e.g.  {\bf c:/xxx}) using Unix directory name separators
    (forward slash).  The drive letter itself can be upper or lower case (e.g.
    c:/xxx or C:/xxx).

    Bareos's default for processing directories is to recursively descend in
    the directory saving all files and subdirectories.  Bareos will not by
    default cross filesystems (or mount points in Unix parlance).  This means
    that if you specify the root partition (e.g.  {\bf /}), Bareos will save
    only the root partition and not any of the other mounted filesystems.
    Similarly on Windows systems, you must explicitly specify each of the
    drives you want saved (e.g.
    {\bf c:/} and {\bf d:/} ...). In addition, at least for Windows systems, you
    will most likely want to enclose each specification within double quotes
    particularly if the directory (or file) name contains spaces. The {\bf df}
    command on Unix systems will show you which mount points you must specify to
    save everything. See below for an example.

Take special care not to include a directory twice or Bareos will backup
the same files two times wasting a lot of space on your archive device.
Including a directory twice is very easy to do.  For example:

\begin{bconfig}{File Set}
  Include {
    Options {
      compression=GZIP
    }
    File = /
    File = /usr
  }
\end{bconfig}
on a Unix system where /usr is a subdirectory (rather than a mounted
filesystem) will cause /usr to be backed up twice.

{\bf {\textless}file-list{\textgreater}} is a list of directory and/or filename names
specified with a {\bf File =} directive. To include names containing spaces,
enclose the name between double-quotes. Wild-cards are not interpreted
in file-lists. They can only be specified in Options resources.

There are a number of special cases when specifying directories and files in a
{\bf file-list}. They are:

\begin{itemize}
\item Any name preceded by an at-sign (@) is assumed to be the  name of a
   file, which contains a list of files each preceded by a "File =".  The
   named file is read once when the configuration file is parsed during the
   Director startup.  Note, that the file is read on the Director's machine
   and not on the Client's.  In fact, the @filename can appear anywhere
   within a configuration file where a token would be read, and the contents of
   the named file will be logically inserted in the place of the @filename.
   What must be in the file depends on the location the @filename is
   specified in the conf file.  For example:

\begin{bconfig}{File Set with Include File}
Include {
  Options {
    compression=GZIP
  }
  @/home/files/my-files
}
\end{bconfig}


\item Any name beginning with a vertical bar ({\textbar}) is  assumed to
   be the name of a program.  This program will be executed on the Director's
   machine at the time the Job starts (not when the Director reads the
   configuration file), and any output from that program will be assumed to
   be a list of files or directories, one per line, to be included. Before
   submitting the specified command Bareos will performe
   \ilink{character substitution}{character substitution}.

   This allows you to have a job that, for example, includes all the local
   partitions even if you change the partitioning by adding a disk.  The
   examples below show you how to do this.  However, please note two
   things: \\
   1.  if you want the local filesystems, you probably should be
   using the {\bf fstype} directive and set {\bf onefs=no}.
   \\

   2.  the exact syntax of the command needed in the examples below is very
   system dependent.  For example, on recent Linux systems, you may need to
   add the -P option, on FreeBSD systems, the options will be different as
   well.

   In general, you will need to prefix your command or commands with a {\bf
   sh -c} so that they are invoked by a shell.  This will not be the case
   if you are invoking a script as in the second example below.  Also, you
   must take care to escape (precede with a \textbackslash{}) wild-cards,
   shell character, and to ensure that any spaces in your command are
   escaped as well.  If you use a single quotes (') within a double quote
   ("), Bareos will treat everything between the single quotes as one field
   so it will not be necessary to escape the spaces.  In general, getting
   all the quotes and escapes correct is a real pain as you can see by the
   next example.  As a consequence, it is often easier to put everything in
   a file and simply use the file name within Bareos.  In that case the
   {\bf sh -c} will not be necessary providing the first line of the file
   is {\bf \#!/bin/sh}.

   As an  example:

\begin{bconfig}{File Set with inline script}
Include {
   Options {
     signature = SHA1
   }
   File = "|sh -c 'df -l | grep \"^/dev/hd[ab]\" | grep -v \".*/tmp\" | awk \"{print \\$6}\"'"
}
\end{bconfig}
% workaround for kile editor
\hide{$}
   will produce a list of all the local partitions on a Linux system.
   Quoting is a real problem because you must quote for Bareos  which consists of
   preceding every \textbackslash{} and every " with a \textbackslash{}, and
   you must also quote for the shell command. In the end, it is probably  easier
   just to execute a script file with:

\begin{bconfig}{File Set with external script}
Include {
  Options {
    signature=MD5
  }
  File = "|my_partitions"
}
\end{bconfig}

   where \command{my_partitions} has:

\footnotesize
\begin{verbatim}
#!/bin/sh
df -l | grep "^/dev/hd[ab]" | grep -v ".*/tmp" \
      | awk "{print \$6}"
\end{verbatim}
\normalsize

   If the vertical bar (\verb+|+) in front of \command{my_partitions} is preceded by a
   backslash as in \textbackslash{}\verb+|+, the program will be executed on the
   Client's machine instead of on the Director's machine.
   Please note that if the filename is given within quotes, you
   will need to use two slashes.  An example, provided by John Donagher,
   that backs up all the local UFS partitions on a remote system is:

\begin{bconfig}{File Set with inline script in quotes}
FileSet {
  Name = "All local partitions"
  Include {
    Options {
      signature=SHA1
      onefs=yes
    }
    File = "\\|bash -c \"df -klF ufs | tail +2 | awk '{print \$6}'\""
  }
}
\end{bconfig}

   The above requires two backslash characters after the double quote (one
   preserves  the next one). If you are a Linux user, just change the {\bf ufs}
   to  {\bf ext3} (or your preferred filesystem type), and you will be in
   business.

   If you know what filesystems you have mounted on your system, e.g.
   for Linux only using ext2, ext3 or ext4, you can backup
   all local filesystems using something like:

\begin{bconfig}{File Set to backup all extfs partions}
Include {
   Options {
     signature = SHA1
     onfs=no
     fstype=ext2
   }
   File = /
}
\end{bconfig}

\item Any file-list item preceded by a less-than sign ({\textless})  will be taken
   to be a file. This file will be read on the Director's machine (see
   below for doing it on the Client machine) at the time
   the Job starts, and the  data will be assumed to be a list of directories or
   files,  one per line, to be included. The names should start in  column 1 and
   should not be quoted even if they contain  spaces. This feature allows you to
   modify the external  file and change what will be saved without stopping and
   restarting Bareos as would be necessary if using the @  modifier noted above.
   For example:

\footnotesize
\begin{verbatim}
Include {
  Options {
    signature = SHA1
  }
  File = "</home/files/local-filelist"
}
\end{verbatim}
\normalsize

   If you precede the less-than sign ({\textless}) with a backslash as in
   \textbackslash{}{\textless}, the file-list will be read on the Client machine
   instead of on the Director's machine.  Please note that if the filename
   is given within quotes, you will need to use two slashes.

\footnotesize
\begin{verbatim}
Include {
  Options {
    signature = SHA1
  }
  File = "\\</home/xxx/filelist-on-client"
}
\end{verbatim}
\normalsize

\item     
    \index[general]{Backup!Partitions}
    \index[general]{Backup!Raw Partitions}
    If you explicitly specify a block device such as {\bf /dev/hda1},  then
   Bareos will assume that this  is a raw partition
   to be backed up. In this case, you are strongly  urged to specify a {\bf
   sparse=yes} include option, otherwise, you  will save the whole partition
   rather than just the actual data that  the partition contains. For example:

\begin{bconfig}{Backup Raw Partitions}
Include {
  Options {
    signature=MD5
    sparse=yes
  }
  File = /dev/hd6
}
\end{bconfig}

   will backup the data in device /dev/hd6. Note, the {bf /dev/hd6} must be
   the raw partition itself. Bareos will not back it up as a raw device if
   you specify a symbolic link to a raw device such as my be created by the
   LVM Snapshot utilities.


\item A file-list may not contain wild-cards. Use directives in the
   Options resource if you wish to specify wild-cards or regular expression
   matching.

\end{itemize}



\directive{dir}{Exclude Dir Containing}{filename}{}{}{}
    This directive can be added to the Include section of the FileSet resource.  If the specified
    filename ({\bf filename-string}) is found on the Client in any directory to be
    backed up, the whole directory will be ignored (not backed up).
    We recommend to use the filename \file{.nobackup}, as it is a hidden file on unix
    systems, and explains what is the purpose of the file.

    For example:

    \begin{bconfig}{Exlude Directories containing the file .nobackup}
    # List of files to be backed up
    FileSet {
        Name = "MyFileSet"
        Include {
            Options {
                signature = MD5
            }
            File = /home
            Exclude Dir Containing = .nobackup
        }
    }
    \end{bconfig}

    But in /home, there may be hundreds of directories of users and some
    people want to indicate that they don't want to have certain
    directories backed up. For example, with the above FileSet, if
    the user or sysadmin creates a file named {\bf .nobackup} in
    specific directories, such as

    \begin{verbatim}
    /home/user/www/cache/.nobackup
    /home/user/temp/.nobackup
    \end{verbatim}

    then Bareos will not backup the two directories named:

    \begin{verbatim}
    /home/user/www/cache
    /home/user/temp
    \end{verbatim}

    NOTE: subdirectories will not be backed up.  That is, the directive
    applies to the two directories in question and any children (be they
    files, directories, etc).


\directive{dir}{Plugin}{plugin-name:plugin-parameter1:plugin-parameter2:$\ldots$}{}{}{}
\label{directive-fileset-plugin}
        Instead of only specifying files, a file set can also use plugins.
        Plugins are additional libraries that handle specific requirements.
        The purpose of plugins is to provide an interface to any system program
        for backup and restore. That allows you, for example, to do database backups without a local dump.

        The syntax and semantics of the Plugin directive require
        the first part of the string up to the colon to be the name of the plugin.
        Everything after the first colon is ignored by the File daemon but is passed to the plugin.
        Thus the plugin writer may define the
        meaning of the rest of the string as he wishes.

        For more information, see \nameref{fdPlugins}.

        The program \nameref{bpluginfo} can be used, to retrieve information about a specific plugin.

        Note: It is also possible to define more than one plugin directive in a FileSet to do several database dumps at once.

\directive{dir}{Options}{$\ldots$}{}{}{}
    See the \nameref{fileset-options} section.

\end{description}


\subsubsection{FileSet Options Ressource}
\label{fileset-options}

The Options resource is optional, but when specified, it will contain a
list of {\bf keyword=value} options to be applied to the file-list.
See below for the definition of file-list.
Multiple Options resources may be specified one after another.  As the
files are found in the specified directories, the Options will applied to
the filenames to determine if and how the file should be backed up.  The
wildcard and regular expression pattern matching parts of the
Options resources are checked in the order they are specified in the
FileSet until the first one that matches. Once one matches, the
compression and other flags within the Options specification will
apply to the pattern matched.

A key point is that in the absence of an Option or no other Option is
matched, every file is accepted for backing up. This means that if
you want to exclude something, you must explicitly specify an Option
with an {\bf exclude = yes} and some pattern matching.

Once Bareos determines that the Options resource matches the file under
consideration, that file will be saved without looking at any other Options
resources that may be present.  This means that any wild cards must appear
before an Options resource without wild cards.

If for some reason, Bareos checks all the Options resources to a file under
consideration for backup, but there are no matches (generally because of wild
cards that don't match), Bareos as a default will then backup the file.  This
is quite logical if you consider the case of no Options clause is specified,
where you want everything to be backed up, and it is important to keep in mind
when excluding as mentioned above.

However, one additional point is that in the case that no match was found,
Bareos will use the options found in the last Options resource.  As a
consequence, if you want a particular set of "default" options, you should put
them in an Options resource after any other Options.

It is a good idea to put all your wild-card and regex expressions inside
double quotes to prevent conf file scanning problems.

This is perhaps a bit overwhelming, so there are a number of examples included
below to illustrate how this works.

You find yourself using a lot of Regex statements, which will cost quite a lot
of CPU time, we recommend you simplify them if you can, or better yet
convert them to Wild statements which are much more efficient.

%\input{autogenerated/datatype-options-table.tex}

The directives within an Options resource may be one of the following:

\begin{description}
    \xdirective{Dir}{}{AutoExclude}{\dtYesNo}{}{yes}{14.2.2}{%
        Automatically exclude files not intended for backup.
        Currently only used for Windows, to exclude files defined in the registry key \registrykey{HKEY_LOCAL_MACHINE\SYSTEM\CurrentControlSet\Control\BackupRestore\FilesNotToBackup}, see section \nameref{FilesNotToBackup}.
    }

    \item [compression={\textless}GZIP{\textbar}GZIP1{\textbar}...{\textbar}GZIP9{\textbar}LZO{\textbar}LZFAST{\textbar}LZ4{\textbar}LZ4HC{\textgreater}] \hfill \\
        \index[dir]{compression}
        \index[dir]{Directive!compression}

        Configures the software compression to be used by the File Daemon.
        The compression is done on a file by file basis.
        %If there is a problem reading the tape in a single
        %record of a file, it will at most affect that file and none of the other
        %files on the tape.

        Software compression gets important if you are writing to a device that does not support compression by itself
        (e.g. hard disks). Otherwise, all modern tape drive do support hardware compression.
        Software compression can also be helpful to reduce the required network bandwidth,
        as compression is done on the File Daemon.
        However, using Bareos software compression and device hardware compression together
        is not advised, as trying to compress precompressed data is a very CPU-intense task
        and probably end up in even larger data.

        You can overwrite this option per Storage resource using the \linkResourceDirective{Dir}{Storage}{Allow Compression} = no option.

    \begin{description}
        \item [compression=GZIP] \hfill \\
        All files saved will be software compressed using the GNU ZIP
        compression format.

        Specifying {\bf GZIP} uses the default compression level 6 (i.e.  {\bf
        GZIP} is identical to {\bf GZIP6}).  If you want a different compression
        level (1 through 9), you can specify it by appending the level number
        with no intervening spaces to {\bf GZIP}.  Thus {\bf compression=GZIP1}
        would give minimum compression but the fastest algorithm, and {\bf
        compression=GZIP9} would give the highest level of compression, but
        requires more computation.  According to the GZIP documentation,
        compression levels greater than six generally give very little extra
        compression and are rather CPU intensive.

        \item [compression=LZO] \hfill \\
        All files saved will be software compressed using the LZO
        compression format. The compression is done on a file by file basis by
        the File daemon. Everything else about GZIP is true for LZO.

        LZO provides much faster compression and decompression speed but lower
        compression ratio than GZIP. If your CPU is fast enough you should be able
        to compress your data without making the backup duration longer.

        Note that Bareos only use one compression level LZO1X-1 specified by LZO.

        \item [compression=LZFAST] \hfill \\
        All files saved will be software compressed using the LZFAST
        compression format. The compression is done on a file by file basis by
        the File daemon. Everything else about GZIP is true for LZFAST.

        LZFAST provides much faster compression and decompression speed but lower
        compression ratio than GZIP. If your CPU is fast enough you should be able
        to compress your data without making the backup duration longer.

        \item [compression=LZ4] \hfill \\
        All files saved will be software compressed using the LZ4
        compression format. The compression is done on a file by file basis by
        the File daemon. Everything else about GZIP is true for LZ4.

        LZ4 provides much faster compression and decompression speed but lower
        compression ratio than GZIP. If your CPU is fast enough you should be able
        to compress your data without making the backup duration longer.

        Both LZ4 and LZ4HC have the same decompression speed which is about twice
        the speed of the LZO compression. So for a restore both LZ4 and LZ4HC are
        good candidates.

        \warning{As LZ4 compression is not supported by Bacula, make sure \linkResourceDirective{Fd}{Client}{Compatible} = no.}

        \item [compression=LZ4HC] \hfill \\
        All files saved will be software compressed using the LZ4HC
        compression format. The compression is done on a file by file basis by
        the File daemon. Everything else about GZIP is true for LZ4.

        LZ4HC is the High Compression version of the LZ4 compression. It has
        a higher compression ratio than LZ4 and is more comparable to GZIP-6
        in both compression rate and cpu usage.

        Both LZ4 and LZ4HC have the same decompression speed which is about twice
        the speed of the LZO compression. So for a restore both LZ4 and LZ4HC are
        good candidates.

        \warning{As LZ4 compression is not supported by Bacula, make sure \linkResourceDirective{Fd}{Client}{Compatible} = no.}

    \end{description}



 \item [signature={\textless}SHA1{\textbar}MD5{\textgreater}] \hfill \\
    \begin{description}
        \item [signature=SHA1] \hfill \\
        \index[dir]{signature}
        \index[dir]{SHA1}
        \index[dir]{Directive!signature}
        An SHA1 signature will be computed for all The SHA1 algorithm is
        purported to be some what slower than the MD5 algorithm, but at the same
        time is significantly better from a cryptographic point of view (i.e.
        much fewer collisions, much lower probability of being hacked.) It adds
        four more bytes than the MD5 signature.  We strongly recommend that
        either this option or MD5 be specified as a default for all files.
        Note, only one of the two options MD5 or SHA1 can be computed for any
        file.

        \item [signature=MD5] \hfill \\
        \index[dir]{signature}
        \index[dir]{MD5}
        \index[dir]{Directive!signature}
        An MD5 signature will be computed for all files saved.  Adding this
        option generates about 5\% extra overhead for each file saved.  In
        addition to the additional CPU time, the MD5 signature adds 16 more
        bytes per file to your catalog.  We strongly recommend that this option
        or the SHA1 option be specified as a default for all files.
    \end{description}


\item[basejob={\textless}options{\textgreater}]
\index[dir]{basejob}
\index[dir]{Directive!basejob}

The options letters specified are used when running a {\bf Backup Level=Full}
with BaseJobs. The options letters are the same than in the \textbf{verify=}
option below.

\item[accurate={\textless}options{\textgreater}] \index[dir]{Accurate}
  \index[dir]{Directive!accurate} The options letters specified are used when
  running a {\bf Backup Level=Incremental/Differential} in Accurate mode. The
  options letters are the same than in the \textbf{verify=} option below.

\item [verify={\textless}options{\textgreater}] \hfill \\
\index[dir]{verify}
\index[dir]{Directive!verify}
   The options letters specified are used  when running a {\bf Verify
   Level=Catalog} as well as the  {\bf DiskToCatalog} level job. The options
   letters may be any  combination of the following:

      \begin{description}

      \item {\bf i}
      compare the inodes

      \item {\bf p}
      compare the permission bits

      \item {\bf n}
      compare the number of links

      \item {\bf u}
      compare the user id

      \item {\bf g}
      compare the group id

      \item {\bf s}
      compare the size

      \item {\bf a}
      compare the access time

      \item {\bf m}
      compare the modification time (st\_mtime)

      \item {\bf c}
      compare the change time (st\_ctime)

      \item {\bf d}
      report file size decreases

      \item {\bf 5}
      compare the MD5 signature

      \item {\bf 1}
      compare the SHA1 signature

      \item {\bf A}
      Only for Accurate option, it allows to always backup the file

      \end{description}

   A useful set of general options on the {\bf Level=Catalog}  or {\bf
   Level=DiskToCatalog}  verify is {\bf pins5} i.e. compare permission bits,
   inodes, number  of links, size, and MD5 changes.

\item [onefs=yes{\textbar}no] \hfill \\
\index[dir]{onefs}
\index[dir]{Directive!onefs}
   If set to {\bf yes} (the default), {\bf Bareos} will remain on a single
   file system.  That is it will not backup file systems that are mounted
   on a subdirectory.  If you are using a *nix system, you may not even be
   aware that there are several different filesystems as they are often
   automatically mounted by the OS (e.g.  /dev, /net, /sys, /proc, ...).
   Bareos will inform you when it decides not to
   traverse into another filesystem.  This can be very useful if you forgot
   to backup a particular partition.  An example of the informational
   message in the job report is:

\footnotesize
\begin{verbatim}
rufus-fd: /misc is a different filesystem. Will not descend from / into /misc
rufus-fd: /net is a different filesystem. Will not descend from / into /net
rufus-fd: /var/lib/nfs/rpc_pipefs is a different filesystem. Will not descend from /var/lib/nfs into /var/lib/nfs/rpc_pipefs
rufus-fd: /selinux is a different filesystem. Will not descend from / into /selinux
rufus-fd: /sys is a different filesystem. Will not descend from / into /sys
rufus-fd: /dev is a different filesystem. Will not descend from / into /dev
rufus-fd: /home is a different filesystem. Will not descend from / into /home
\end{verbatim}
\normalsize

   If you wish to backup multiple filesystems, you can  explicitly
   list each filesystem you want saved.  Otherwise, if you set the onefs option
   to {\bf no}, Bareos will backup  all mounted file systems (i.e. traverse mount
   points) that  are found within the {\bf FileSet}. Thus if  you have NFS or
   Samba file systems mounted on a directory listed  in your FileSet, they will
   also be backed up. Normally, it is  preferable to set {\bf onefs=yes} and to
   explicitly name  each filesystem you want backed up. Explicitly naming  the
   filesystems you want backed up avoids the possibility  of getting into a
   infinite loop recursing filesystems.  Another possibility is to
   use {\bf onefs=no} and to set {\bf fstype=ext2, ...}.
   See the example below for more details.

   If you think that Bareos should be backing up a particular directory
   and it is not, and you have {\bf onefs=no} set, before you complain,
   please do:

\footnotesize
\begin{verbatim}
  stat /
  stat <filesystem>
\end{verbatim}
\normalsize

where you replace {\bf filesystem} with the one in question.  If the
{\bf Device:} number is different for / and for your filesystem, then they
are on different filesystems.  E.g.
\footnotesize
\begin{verbatim}
stat /
  File: `/'
  Size: 4096            Blocks: 16         IO Block: 4096   directory
Device: 302h/770d       Inode: 2           Links: 26
Access: (0755/drwxr-xr-x)  Uid: (    0/    root)   Gid: (    0/    root)
Access: 2005-11-10 12:28:01.000000000 +0100
Modify: 2005-09-27 17:52:32.000000000 +0200
Change: 2005-09-27 17:52:32.000000000 +0200

stat /net
  File: `/home'
  Size: 4096            Blocks: 16         IO Block: 4096   directory
Device: 308h/776d       Inode: 2           Links: 7
Access: (0755/drwxr-xr-x)  Uid: (    0/    root)   Gid: (    0/    root)
Access: 2005-11-10 12:28:02.000000000 +0100
Modify: 2005-11-06 12:36:48.000000000 +0100
Change: 2005-11-06 12:36:48.000000000 +0100
\end{verbatim}
\normalsize

   Also be aware that even if you include {\bf /home} in your list
   of files to backup, as you most likely should, you will get the
   informational message that  "/home is a different filesystem" when
   Bareos is processing the {\bf /} directory.  This message does not
   indicate an error. This message means that while examining the
   {\bf File =} referred to in the second part of the message, Bareos will
   not descend into the directory mentioned in the first part of the message.
   However, it is possible that the separate filesystem will be backed up
   despite the message. For example, consider the following FileSet:

\footnotesize
\begin{verbatim}
  File = /
  File = /var
\end{verbatim}
\normalsize

   where {\bf /var} is a separate filesystem.  In this example, you will get a
   message saying that Bareos will not decend from {\bf /} into {\bf /var}.  But
   it is important to realise that Bareos will descend into {\bf /var} from the
   second File directive shown above.  In effect, the warning is bogus,
   but it is supplied to alert you to possible omissions from your FileSet. In
   this example, {\bf /var} will be backed up.  If you changed the FileSet such
   that it did not specify {\bf /var}, then {\bf /var} will not be backed up.

\item [honor nodump flag={\textless}yes{\textbar}no{\textgreater}] \hfill \\
\index[dir]{honornodumpflag}
\index[dir]{Directive!honornodumpflag}
   If your file system supports the {\bf nodump} flag (e. g. most
   BSD-derived systems) Bareos will honor the setting of the flag
   when this option is set to {\bf yes}. Files having this flag set
   will not be included in the backup and will not show up in the
   catalog. For directories with the {\bf nodump} flag set recursion
   is turned off and the directory will be listed in the catalog.
   If the {\bf honor nodump flag} option is not defined
   or set to {\bf no} every file and directory will be eligible for
   backup.

\item [portable=yes{\textbar}no] \hfill \\
\index[dir]{portable}
\index[dir]{Directive!portable}
\label{portable}
   If set to {\bf yes} (default is {\bf no}), the Bareos File daemon will
   backup Win32 files in a portable format, but not all Win32 file
   attributes will be saved and restored.  By default, this option is set
   to {\bf no}, which means that on Win32 systems, the data will be backed
   up using Windows API calls and on WinNT/2K/XP, all the security and
   ownership attributes will be properly backed up (and restored).  However
   this format is not portable to other systems -- e.g.  Unix, Win95/98/Me.
   When backing up Unix systems, this option is ignored, and unless you
   have a specific need to have portable backups, we recommend accept the
   default ({\bf no}) so that the maximum information concerning your files
   is saved.

\item [recurse=yes{\textbar}no] \hfill \\
\index[dir]{recurse}
\index[dir]{Directive!recurse}
   If set to {\bf yes} (the default), Bareos will recurse (or descend) into
   all subdirectories found unless the directory is explicitly excluded
   using an {\bf exclude} definition.  If you set {\bf recurse=no}, Bareos
   will save the subdirectory entries, but not descend into the
   subdirectories, and thus will not save the files or directories
   contained in the subdirectories.  Normally, you will want the default
   ({\bf yes}).

\item [sparse=yes{\textbar}no] \hfill \\
\index[dir]{sparse}
\index[dir]{Directive!sparse}
   Enable special code that checks for sparse files such as created by
   ndbm.  The default is {\bf no}, so no checks are made for sparse files.
   You may specify {\bf sparse=yes} even on files that are not sparse file.
   No harm will be done, but there will be a small additional overhead to
   check for buffers of all zero, and if there is a 32K block of all zeros
   (see below), that block will become a hole in the file, which
   may not be desirable if the original file was not a sparse file.

   {\bf Restrictions:} Bareos reads files in 32K buffers.  If the whole
   buffer is zero, it will be treated as a sparse block and not written to
   tape.  However, if any part of the buffer is non-zero, the whole buffer
   will be written to tape, possibly including some disk sectors (generally
   4098 bytes) that are all zero.  As a consequence, Bareos's detection of
   sparse blocks is in 32K increments rather than the system block size.
   If anyone considers this to be a real problem, please send in a request
   for change with the reason.

   If you are not familiar with sparse files, an example is say a file
   where you wrote 512 bytes at address zero, then 512 bytes at address 1
   million.  The operating system will allocate only two blocks, and the
   empty space or hole will have nothing allocated.  However, when you read
   the sparse file and read the addresses where nothing was written, the OS
   will return all zeros as if the space were allocated, and if you backup
   such a file, a lot of space will be used to write zeros to the volume.
   Worse yet, when you restore the file, all the previously empty space
   will now be allocated using much more disk space.  By turning on the
   {\bf sparse} option, Bareos will specifically look for empty space in
   the file, and any empty space will not be written to the Volume, nor
   will it be restored.  The price to pay for this is that Bareos must
   search each block it reads before writing it.  On a slow system, this
   may be important.  If you suspect you have sparse files, you should
   benchmark the difference or set sparse for only those files that are
   really sparse.

   You probably should not use this option on files or raw disk devices
   that are not really sparse files (i.e. have holes in them).

\item [readfifo=yes{\textbar}no] \hfill \\
\index[dir]{readfifo}
\index[dir]{Directive!readfifo}
\label{readfifo}
   If enabled, tells the Client to read the data on a backup and write the
   data on a restore to any FIFO (pipe) that is explicitly mentioned in the
   FileSet.  In this case, you must have a program already running that
   writes into the FIFO for a backup or reads from the FIFO on a restore.
   This can be accomplished with the {\bf RunBeforeJob} directive.  If this
   is not the case, Bareos will hang indefinitely on reading/writing the
   FIFO. When this is not enabled (default), the Client simply saves the
   directory entry for the FIFO.

   Normally, when Bareos runs a RunBeforeJob, it waits until that
   script terminates, and if the script accesses the FIFO to write
   into it, the Bareos job will block and everything will stall.
   However, Vladimir Stavrinov as supplied tip that allows this feature
   to work correctly.  He simply adds the following to the beginning
   of the RunBeforeJob script:

\begin{verbatim}
   exec > /dev/null
\end{verbatim}


\begin{bconfig}{FileSet with Fifo}
Include {
  Options {
    signature=SHA1
    readfifo=yes
  }
  File = /home/abc/fifo
}
\end{bconfig}

   This feature can be used to do a "hot" database backup.  
   You can use the {\bf RunBeforeJob} to create the fifo
   and to start a program that dynamically reads your database and writes
   it to the fifo.  Bareos will then write it to the Volume. 

   During the restore operation, the inverse is true, after Bareos creates
   the fifo if there was any data stored with it (no need to explicitly
   list it or add any options), that data will be written back to the fifo.
   As a consequence, if any such FIFOs exist in the fileset to be restored,
   you must ensure that there is a reader program or Bareos will block, and
   after one minute, Bareos will time out the write to the fifo and move on
   to the next file.

    If you are planing to use a Fifo for backup, better take a look to the \nameref{bpipe} section.


\item [noatime=yes{\textbar}no] \hfill \\
\index[dir]{noatime}
\index[dir]{Directive!noatime}
   If enabled, and if your Operating System supports the O\_NOATIME file
   open flag, Bareos will open all files to be backed up with this option.
   It makes it possible to read a file without updating the inode atime
   (and also without the inode ctime update which happens if you try to set
   the atime back to its previous value).  It also prevents a race
   condition when two programs are reading the same file, but only one does
   not want to change the atime.  It's most useful for backup programs and
   file integrity checkers (and Bareos can fit on both categories).

   This option is particularly useful for sites where users are sensitive
   to their MailBox file access time.  It replaces both the {\bf keepatime}
   option without the inconveniences of that option (see below).

   If your Operating System does not support this option, it will be
   silently ignored by Bareos.


\item [mtimeonly=yes{\textbar}no] \hfill \\
\index[dir]{mtimeonly}
\index[dir]{Directive!mtimeonly}
   If enabled, tells the Client that the selection of files during
   Incremental and Differential backups should based only on the st\_mtime
   value in the stat() packet.  The default is {\bf no} which means that
   the selection of files to be backed up will be based on both the
   st\_mtime and the st\_ctime values.  In general, it is not recommended
   to use this option.

\item [keepatime=yes{\textbar}no] \hfill \\
\index[dir]{keepatime}
\index[dir]{Directive!keepatime}
   The default is {\bf no}.  When enabled, Bareos will reset the st\_atime
   (access time) field of files that it backs up to their value prior to
   the backup.  This option is not generally recommended as there are very
   few programs that use st\_atime, and the backup overhead is increased
   because of the additional system call necessary to reset the times.
   However, for some files, such as mailboxes, when Bareos backs up the
   file, the user will notice that someone (Bareos) has accessed the
   file. In this, case keepatime can be useful.
   (I'm not sure this works on Win32).

   Note, if you use this feature, when Bareos resets the access time, the
   change time (st\_ctime) will automatically be modified by the system,
   so on the next incremental job, the file will be backed up even if
   it has not changed. As a consequence, you will probably also want
   to use {\bf mtimeonly = yes} as well as keepatime (thanks to
   Rudolf Cejka for this tip).

\item [checkfilechanges=yes{\textbar}no] \hfill \\
\index[dir]{checkfilechanges}
\index[dir]{Directive!checkfilechanges}
   If enabled, the Client will check size, age of each file after
   their backup to see if they have changed during backup. If time
   or size mismatch, an error will raise.

\begin{verbatim}
 zog-fd: Client1.2007-03-31_09.46.21 Error: /tmp/test mtime changed during backup.
\end{verbatim}

   In general, it is recommended to use this option.

\item [hardlinks=yes{\textbar}no] \hfill \\
\index[dir]{hardlinks}
\index[dir]{Directive!hardlinks}
   When enabled (default), this directive will cause hard links to be
   backed up. However, the File daemon keeps track of hard linked files and
   will backup the data only once. The process of keeping track of the
   hard links can be quite expensive if you have lots of them (tens of
   thousands or more). This doesn't occur on normal Unix systems, but if
   you use a program like BackupPC, it can create hundreds of thousands, or
   even millions of hard links. Backups become very long and the File daemon
   will consume a lot of CPU power checking hard links.  In such a case,
   set {\bf hardlinks=no} and hard links will not be backed up.  Note, using
   this option will most likely backup more data and on a restore the file
   system will not be restored identically to the original.

\item [wild={\textless}string{\textgreater}] \hfill \\
\index[dir]{wild}
\index[dir]{Directive!wild}
   Specifies a wild-card string to be applied to the filenames and
   directory names.  Note, if {\bf Exclude} is not enabled, the wild-card
   will select which files are to be included.  If {\bf Exclude=yes} is
   specified, the wild-card will select which files are to be excluded.
   Multiple wild-card directives may be specified, and they will be applied
   in turn until the first one that matches.  Note, if you exclude a
   directory, no files or directories below it will be matched.

   You may want to test your expressions prior to running your
   backup by using the \nameref{bwild} program.
   You can also test your full FileSet definition by using
   the \ilink{estimate}{estimate} command.
   It is recommended to enclose the string in double quotes.

\item [wilddir={\textless}string{\textgreater}] \hfill \\
\index[dir]{wilddir}
\index[dir]{Directive!wilddir}
   Specifies a wild-card string to be applied to directory names only.  No
   filenames will be matched by this directive.  Note, if {\bf Exclude} is
   not enabled, the wild-card will select directories to be
   included.  If {\bf Exclude=yes} is specified, the wild-card will select
   which directories are to be excluded.  Multiple wild-card directives may be
   specified, and they will be applied in turn until the first one that
   matches.  Note, if you exclude a directory, no files or directories
   below it will be matched.

   It is recommended to enclose the string in double quotes.

   You may want to test your expressions prior to running your
   backup by using the \nameref{bwild} program.
   You can also test your full FileSet definition by using
   the \ilink{estimate}{estimate} command.

\item [wildfile={\textless}string{\textgreater}] \hfill \\
\index[dir]{wildfile}
\index[dir]{Directive!wildfile}
   Specifies a wild-card string to be applied to non-directories. That
   is no directory entries will be matched by this directive.
   However, note that the match is done against the full path and filename,
   so your wild-card string must take into account that filenames
   are preceded by the full path.
   If {\bf Exclude}
   is not enabled, the wild-card will select which files are to be
   included.  If {\bf Exclude=yes} is specified, the wild-card will select
   which files are to be excluded.  Multiple wild-card directives may be
   specified, and they will be applied in turn until the first one that
   matches.

   It is recommended to enclose the string in double quotes.

   You may want to test your expressions prior to running your
   backup by using the \nameref{bwild} program.
   You can also test your full FileSet definition by using
   the \ilink{estimate}{estimate} command.
   An example of excluding with the WildFile option on Win32 machines is
   presented below.

\item [regex={\textless}string{\textgreater}] \hfill \\
\index[dir]{regex}
\index[dir]{Directive!regex}

\label{FileRegex}

   Specifies a POSIX extended regular expression to be applied to the
   filenames and directory names, which include the full path.  If {\bf
   Exclude} is not enabled, the regex will select which files are to be
   included.  If {\bf Exclude=yes} is specified, the regex will select
   which files are to be excluded.  Multiple regex directives may be
   specified within an Options resource, and they will be applied in turn
   until the first one that matches.  Note, if you exclude a directory, no
   files or directories below it will be matched.

   It is recommended to enclose the string in double quotes.

   The regex libraries differ from one operating system to
   another, and in addition, regular expressions are complicated,
   so you may want to test your expressions prior to running your
   backup by using the \nameref{bregex} program.
   You can also test your full FileSet definition by using
   the \ilink{estimate}{estimate} command.

   You find yourself using a lot of Regex statements, which will cost quite a lot
   of CPU time, we recommend you simplify them if you can, or better yet
   convert them to Wild statements which are much more efficient.


\item [regexfile={\textless}string{\textgreater}] \hfill \\
\index[dir]{regexfile}
\index[dir]{Directive!regexfile}
   Specifies a POSIX extended regular expression to be applied to
   non-directories. No directories will be matched by this directive.
   However, note that the match is done against the full path and
   filename, so your regex string must take into account that filenames
   are preceded by the full path.
   If {\bf Exclude} is not enabled, the regex will select which files are
   to be included.  If {\bf Exclude=yes} is specified, the regex will
   select which files are to be excluded.  Multiple regex directives may be
   specified, and they will be applied in turn until the first one that
   matches.

   It is recommended to enclose the string in double quotes.

   The regex libraries differ from one operating system to
   another, and in addition, regular expressions are complicated,
   so you may want to test your expressions prior to running your
   backup by using the \nameref{bregex} program.

\item [regexdir={\textless}string{\textgreater}] \hfill \\
\index[dir]{regexdir}
\index[dir]{Directive!regexdir}
   Specifies a POSIX extended regular expression to be applied to directory
   names only.  No filenames will be matched by this directive.  Note, if
   {\bf Exclude} is not enabled, the regex will select directories
   files are to be included.  If {\bf Exclude=yes} is specified, the
   regex will select which files are to be excluded.  Multiple
   regex directives may be specified, and they will be applied in turn
   until the first one that matches.  Note, if you exclude a directory, no
   files or directories below it will be matched.

   It is recommended to enclose the string in double quotes.

   The regex libraries differ from one operating system to
   another, and in addition, regular expressions are complicated,
   so you may want to test your expressions prior to running your
   backup by using the \nameref{bregex} program.

\xdirective{dir}{}{Exclude}{\dtYesNo}{}{no}{}{%
   When enabled, any files matched within the
   Options will be excluded from the backup.
}

\item [aclsupport=yes{\textbar}no] \hfill \\
\index[dir]{aclsupport}
\index[dir]{Directive!aclsupport}
\label{ACLSupport}
   The default is {\bf no}.  If this option is set to yes, and you have the
   POSIX {\bf libacl} installed on your Linux system, Bareos will backup the
   file and directory Unix Access Control Lists (ACL) as defined in IEEE Std
   1003.1e draft 17 and "POSIX.1e" (abandoned).  This feature is
   available on Unix systems only and requires the Linux ACL library. Bareos is
   automatically compiled with ACL support if the {\bf libacl} library is
   installed on your Linux system (shown in config.out).  While restoring the
   files Bareos will try to restore the ACLs, if there is no ACL support
   available on the system, Bareos restores the files and directories but
   not the ACL information.  Please note, if you backup an EXT3 or XFS
   filesystem with ACLs, then you restore them to a different filesystem
   (perhaps reiserfs) that does not have ACLs, the ACLs will be ignored.

   For other operating systems there is support for either POSIX ACLs or
   the more extensible NFSv4 ACLs.

   The ACL stream format between Operation Systems is \textbf{not}
   compatible so for example an ACL saved on Linux cannot be restored on
   Solaris.

   The following Operating Systems are currently supported:

   \begin{enumerate}
   \item AIX (pre-5.3 (POSIX) and post 5.3 (POSIX and NFSv4) ACLs)
   \item Darwin
   \item FreeBSD (POSIX and NFSv4/ZFS ACLs)
   \item HPUX
   \item IRIX
   \item Linux
   \item Solaris (POSIX and NFSv4/ZFS ACLs)
   \item Tru64
   \end{enumerate}

\label{XattrSupport}
\item [xattrsupport=yes{\textbar}no] \hfill \\
\index[dir]{xattrsupport}
\index[dir]{Directive!xattrsupport}
   The default is {\bf no}.  If this option is set to yes, and your
   operating system support either so called Extended Attributes or
   Extensible Attributes Bareos will backup the file and directory
   XATTR data. This feature is available on UNIX only and depends on
   support of some specific library calls in libc.

   The XATTR stream format between Operating Systems is {\bf not}
   compatible so an XATTR saved on Linux cannot for example be restored
   on Solaris.

   On some operating systems ACLs are also stored as Extended Attributes
   (Linux, Darwin, FreeBSD) Bareos checks if you have the aclsupport
   option enabled and if so will not save the same info when saving
   extended attribute information. Thus ACLs are only saved once.

   The following Operating Systems are currently supported:

   \begin{enumerate}
   \item AIX (Extended Attributes)
   \item Darwin (Extended Attributes)
   \item FreeBSD (Extended Attributes)
   \item IRIX (Extended Attributes)
   \item Linux (Extended Attributes)
   \item NetBSD (Extended Attributes)
   \item Solaris (Extended Attributes and Extensible Attributes)
   \item Tru64 (Extended Attributes)
   \end{enumerate}

\item [ignore case=yes{\textbar}no] \hfill \\
\index[dir]{ignore case}
\index[dir]{Directive!ignore case}
   The default is {\bf no}.  On Windows systems, you will almost surely
   want to set this to {\bf yes}.  When this directive is set to {\bf yes}
   all the case of character will be ignored in wild-card and regex
   comparisons.  That is an uppercase A will match a lowercase a.

\item [fstype=filesystem-type] \hfill \\
\index[dir]{fstype}
\index[dir]{Directive!fstype}
   This option allows you to select files and directories by the
   filesystem type.  The permitted filesystem-type names are:

   ext2, jfs, ntfs, proc, reiserfs, xfs, usbdevfs, sysfs, smbfs,
   iso9660.

   You may have multiple Fstype directives, and thus permit matching
   of multiple filesystem types within a single Options resource.  If
   the type specified on the fstype directive does not match the
   filesystem for a particular directive, that directory will not be
   backed up.  This directive can be used to prevent backing up
   non-local filesystems. Normally, when you use this directive, you
   would also set {\bf onefs=no} so that Bareos will traverse filesystems.

   This option is not implemented in Win32 systems.

\item [DriveType=Windows-drive-type] \hfill \\
\index[dir]{DriveType}
\index[dir]{Directive!DriveType}
   This option is effective only on Windows machines and is
   somewhat similar to the Unix/Linux {\bf fstype} described
   above, except that it allows you to select what Windows
   drive types you want to allow.  By default all drive
   types are accepted.

   The permitted drivetype names are:

   removable, fixed, remote, cdrom, ramdisk

   You may have multiple Driveype directives, and thus permit matching
   of multiple drive types within a single Options resource.  If
   the type specified on the drivetype directive does not match the
   filesystem for a particular directive, that directory will not be
   backed up.  This directive can be used to prevent backing up
   non-local filesystems. Normally, when you use this directive, you
   would also set {\bf onefs=no} so that Bareos will traverse filesystems.

   This option is not implemented in Unix/Linux systems.

\item [hfsplussupport=yes{\textbar}no] \hfill \\
\index[dir]{hfsplussupport}
\index[dir]{Directive!hfsplussupport}
   This option allows you to turn on support for Mac OSX HFS plus
   finder information.

\item [strippath={\textless}integer{\textgreater}] \hfill \\
\index[dir]{strippath}
\index[dir]{Directive!strippath}
   This option will cause {\bf integer} paths to be stripped from
   the front of the full path/filename being backed up. This can
   be useful if you are migrating data from another vendor or if
   you have taken a snapshot into some subdirectory.  This directive
   can cause your filenames to be overlayed with regular backup data,
   so should be used only by experts and with great care.

\item [size=sizeoption] \hfill \\
\index[dir]{size}
\index[dir]{Directive!size}
   This option will allow you to select files by their actual size.
   You can select either files smaller than a certain size or bigger
   then a certain size, files of a size in a certain range or files
   of a size which is within 1 \% of its actual size.

   The following settings can be used:

   \begin{enumerate}
   \item {\bf {\textless}size{\textgreater}-{\textless}size{\textgreater}} - Select file in range size - size.
   \item {\bf {\textless}size} - Select files smaller than size.
   \item {\bf {\textgreater}size} - Select files bigger than size.
   \item {\bf size} - Select files which are within 1 \% of size.
   \end{enumerate}

\item [shadowing=none{\textbar}localwarn{\textbar}localremove{\textbar}globalwarn{\textbar}globalremove] \hfill \\
\index[dir]{shadowing}
\index[dir]{Directive!shadowing}
   The default is {\bf none}. This option performs a check within the
   fileset for any file-list entries which are shadowing each other.
   Lets say you specify / and /usr but /usr is not a separate filesystem.
   Then in the normal situation both / and /usr would lead to data being
   backed up twice.

   The following settings can be used:

   \begin{enumerate}
   \item none - Do NO shadowing check
   \item localwarn - Do shadowing check within one include block and warn
   \item localremove - Do shadowing check within one include block and remove duplicates
   \item globalwarn - Do shadowing check between all include blocks and warn
   \item globalremove - Do shadowing check between all include blocks and remove duplicates
   \end{enumerate}

   The local and global part of the setting relate to the fact if the check
   should be performed only within one include block (local) or between multiple
   include blocks of the same fileset (global). The warn and remove part of the
   keyword sets the action e.g. warn the user about shadowing or remove
   the entry shadowing the other.

   Example for a fileset resource with fileset shadow warning enabled:

\begin{bconfig}{FileSet resource with fileset shadow warning enabled}
FileSet {
  Name = "Test Set"
  Include {
    Options {
      signature = MD5
      shadowing = localwarn
    }
  File = /
  File = /usr
  }
}
\end{bconfig}


\item [meta=tag] \hfill \\
\index[dir]{meta}
\index[dir]{Directive!meta}
   This option will add a meta tag to a fileset. These meta tags are used
   by the Native NDMP protocol to pass NDMP backup or restore environment
   variables via the Data Management Agent (DMA) in Bareos to the remote
   NDMP Data Agent. You can have zero or more metatags which are all passed
   to the remote NDMP Data Agent.

\end{description}



\subsection{FileSet Exclude Ressource}
\label{fileset-exclude}
\index[general]{Excluding Files and Directories}

FileSet Exclude-Ressources very similar to Include-Ressources, except that they only allow following directives:

\begin{description}
% file | directoy | |command | \<includefile-client | <includefile-server
\xdirective{dir}{}{File}{ 
  filename \textbar\ 
  directory \textbar\ 
  \textbar command \textbar\ 
  \textbackslash\textless includefile-client \textbar\ 
  \textless includefile-server 
  }{}{}{}{%
    Files to exclude are descripted in the same way as at the \nameref{fileset-include}.
}
\end{description}

For example:

\begin{bconfig}{FileSet using Exclude}
FileSet {
  Name = Exclusion_example
  Include {
    Options {
      Signature = SHA1
    }
    File = /
    File = /boot
    File = /home
    File = /rescue
    File = /usr
  }
  Exclude {
    File = /proc
    File = /tmp                          # Don't add trailing /
    File = .journal
    File = .autofsck
  }
}
\end{bconfig}

Another way to exclude files and directories is to use the \configdirective{Exclude} option from the Include section.


\subsection{FileSet Examples}
\index[general]{Example!FileSet}
\index[general]{FileSet!Example}

The following is an example of a valid FileSet resource definition.  Note,
the first Include pulls in the contents of the file \file{/etc/backup.list}
when Bareos is started (i.e.  the @), and that file must have each filename
to be backed up preceded by a {\bf File =} and on a separate line.

\begin{bconfig}{FileSet using import}
FileSet {
  Name = "Full Set"
  Include {
    Options {
      Compression=GZIP
      signature=SHA1
      Sparse = yes
    }
    @/etc/backup.list
  }
  Include {
     Options {
        wildfile = "*.o"
        wildfile = "*.exe"
        Exclude = yes
     }
     File = /root/myfile
     File = /usr/lib/another_file
  }
}
\end{bconfig}

In the above example, all the files contained in \file{/etc/backup.list} will
be compressed with GZIP compression, an SHA1 signature will be computed on the
file's contents (its data), and sparse file handling will apply.

The two directories \file{/root/myfile} and \file{/usr/lib/another_file} will also be saved
without any options, but all files in those directories with the extensions
\file{.o} and \file{.exe} will be excluded.

Let's say that you now want to exclude the directory \file{/tmp}. The simplest way
to do so is to add an exclude directive that lists \file{/tmp}.  The example
above would then become:

\begin{bconfig}{extended FileSet excluding /tmp}
FileSet {
  Name = "Full Set"
  Include {
    Options {
      Compression=GZIP
      signature=SHA1
      Sparse = yes
    }
    @/etc/backup.list
  }
  Include {
     Options {
        wildfile = "*.o"
        wildfile = "*.exe"
        Exclude = yes
     }
     File = /root/myfile
     File = /usr/lib/another_file
  }
  Exclude {
     File = /tmp                          # don't add trailing /
  }
}
\end{bconfig}

You can add wild-cards to the File directives listed in the Exclude
directory, but you need to take care because if you exclude a directory,
it and all files and directories below it will also be excluded.

Now lets take a slight variation on the above and suppose
you want to save all your whole filesystem except \file{/tmp}.
The problem that comes up is that Bareos will not normally
cross from one filesystem to another.
Doing a \command{df} command, you get the following output:

\begin{commands}{df}
<command>df</command>
Filesystem      1k-blocks      Used Available Use% Mounted on
/dev/hda5         5044156    439232   4348692  10% /
/dev/hda1           62193      4935     54047   9% /boot
/dev/hda9        20161172   5524660  13612372  29% /home
/dev/hda2           62217      6843     52161  12% /rescue
/dev/hda8         5044156     42548   4745376   1% /tmp
/dev/hda6         5044156   2613132   2174792  55% /usr
none               127708         0    127708   0% /dev/shm
//minimatou/c$   14099200   9895424   4203776  71% /mnt/mmatou
lmatou:/          1554264    215884   1258056  15% /mnt/matou
lmatou:/home      2478140   1589952    760072  68% /mnt/matou/home
lmatou:/usr       1981000   1199960    678628  64% /mnt/matou/usr
lpmatou:/          995116    484112    459596  52% /mnt/pmatou
lpmatou:/home    19222656   2787880  15458228  16% /mnt/pmatou/home
lpmatou:/usr      2478140   2038764    311260  87% /mnt/pmatou/usr
deuter:/          4806936     97684   4465064   3% /mnt/deuter
deuter:/home      4806904    280100   4282620   7% /mnt/deuter/home
deuter:/files    44133352  27652876  14238608  67% /mnt/deuter/files
\end{commands}
\hide{$}

And we see that there are a number of separate filesystems (/ /boot
/home /rescue /tmp and /usr not to mention mounted systems).
If you specify only {\bf /} in your Include list, Bareos will only save the
Filesystem {\bf /dev/hda5}. To save all filesystems except {\bf /tmp} with
out including any of the Samba or NFS mounted systems, and explicitly
excluding a /tmp, /proc, .journal, and .autofsck, which you will not want to
be saved and restored, you can use the following:

\begin{bconfig}{FileSet mount points}
FileSet {
  Name = Include_example
  Include {
    Options {
       wilddir = /proc
       wilddir = /tmp
       wildfile = "/.journal"
       wildfile = "/.autofsck"
       exclude = yes
    }
    File = /
    File = /boot
    File = /home
    File = /rescue
    File = /usr
  }
}
\end{bconfig}

Since \file{/tmp} is on its own filesystem and it was not explicitly named in the
Include list, it is not really needed in the exclude list. It is better to
list it in the Exclude list for clarity, and in case the disks are changed so
that it is no longer in its own partition.

Now, lets assume you only want to backup .Z and .gz files and nothing
else. This is a bit trickier because Bareos by default will select
everything to backup, so we must exclude everything but .Z and .gz files.
If we take the first example above and make the obvious modifications
to it, we might come up with a FileSet that looks like this:

\begin{bconfig}{Non-working example}
FileSet {
  Name = "Full Set"
  Include {                    !!!!!!!!!!!!
     Options {                    This
        wildfile = "*.Z"          example
        wildfile = "*.gz"         doesn't
                                  work
     }                          !!!!!!!!!!!!
     File = /myfile
  }
}
\end{bconfig}

The *.Z and *.gz files will indeed be backed up, but all other files
that are not matched by the Options directives will automatically
be backed up too (i.e. that is the default rule).

To accomplish what we want, we must explicitly exclude all other files.
We do this with the following:

\begin{bconfig}{Exclude all except specific wildcards}
FileSet {
  Name = "Full Set"
  Include {
     Options {
        wildfile = "*.Z"
        wildfile = "*.gz"
     }
     Options {
        Exclude = yes
        RegexFile = ".*"
     }
     File = /myfile
  }
}
\end{bconfig}

The "trick" here was to add a RegexFile expression that matches
all files. It does not match directory names, so all directories in
/myfile will be backed up (the directory entry) and any *.Z and *.gz
files contained in them. If you know that certain directories do
not contain any *.Z or *.gz files and you do not want the directory
entries backed up, you will need to explicitly exclude those directories.
Backing up a directory entries is not very expensive.

Bareos uses the system regex library and some of them are
different on different OSes. The above has been reported not to work
on FreeBSD. This can be tested by using the \bcommand{estimate}{job=job-name
listing} command in the console and adapting the RegexFile expression
appropriately.

Please be aware that allowing Bareos to traverse or change file systems can be
{\bf very} dangerous. For example, with the following:

\begin{bconfig}{backup all filesystem below /mnt/matou (use with care)}
FileSet {
  Name = "Bad example"
  Include {
    Options {
      onefs=no
    }
    File = /mnt/matou
  }
}
\end{bconfig}

you will be backing up an NFS mounted partition ({\bf /mnt/matou}), and since
{\bf onefs} is set to {\bf no}, Bareos will traverse file systems. Now if {\bf
/mnt/matou} has the current machine's file systems mounted, as is often the
case, you will get yourself into a recursive loop and the backup will never
end.

As a final example, let's say that you have only one or two
subdirectories of /home that you want to backup.  For example,
you want to backup only subdirectories beginning with the letter
a and the letter b -- i.e. \file{/home/a*} and \file{/home/b*}. 
Now, you might first try:
\begin{bconfig}{Non-working example}
FileSet {
  Name = "Full Set"
  Include {
     Options {
        wilddir = "/home/a*"
        wilddir = "/home/b*"
     }
     File = /home
  }
}
\end{bconfig}

The problem is that the above will include everything in /home.  To get
things to work correctly, you need to start with the idea of exclusion
instead of inclusion.  So, you could simply exclude all directories
except the two you want to use:
\begin{bconfig}{Exclude by regex}
FileSet {
  Name = "Full Set"
  Include {
     Options {
        RegexDir = "^/home/[c-z]"
        exclude = yes
     }
     File = /home
  }
}
\end{bconfig}

And assuming that all subdirectories start with a lowercase letter, this
would work.

An alternative would be to include the two subdirectories desired and
exclude everything else:
\begin{bconfig}{Include and Exclude}
FileSet {
  Name = "Full Set"
  Include {
     Options {
        wilddir = "/home/a*"
        wilddir = "/home/b*"
     }
     Options {
        RegexDir = ".*"
        exclude = yes
     }
     File = /home
  }
}
\end{bconfig}


The following example shows how to back up only the My Pictures directory inside
the My Documents directory for all users in C:/Documents and Settings, i.e.
everything matching the pattern:

\file{C:/Documents and Settings/*/My Documents/My Pictures/*}

To understand how this can be achieved, there are two important points to
remember:

Firstly, Bareos walks over the filesystem depth-first starting from the File =
lines.  It stops descending when a directory is excluded, so you must include
all ancestor directories of each directory containing files to be included.

Secondly, each directory and file is compared to the Options clauses in the
order they appear in the FileSet.  When a match is found, no further clauses
are compared and the directory or file is either included or excluded.

The FileSet resource definition below implements this by including specifc
directories and files and excluding everything else.

\begin{bconfig}{Include/Exclude example}
FileSet {
  Name = "AllPictures"

  Include {

    File  = "C:/Documents and Settings"

    Options {
      signature = SHA1
      verify = s1
      IgnoreCase = yes

      # Include all users' directories so we reach the inner ones.  Unlike a
      # WildDir pattern ending in *, this RegExDir only matches the top-level
      # directories and not any inner ones.
      RegExDir = "^C:/Documents and Settings/[^/]+$"

      # Ditto all users' My Documents directories.
      WildDir = "C:/Documents and Settings/*/My Documents"

      # Ditto all users' My Documents/My Pictures directories.
      WildDir = "C:/Documents and Settings/*/My Documents/My Pictures"

      # Include the contents of the My Documents/My Pictures directories and
      # any subdirectories.
      Wild = "C:/Documents and Settings/*/My Documents/My Pictures/*"
    }

    Options {
      Exclude = yes
      IgnoreCase = yes

      # Exclude everything else, in particular any files at the top level and
      # any other directories or files in the users' directories.
      Wild = "C:/Documents and Settings/*"
    }
  }
}
\end{bconfig}
\hide{$}



\subsection{Windows FileSets}
\index[general]{Windows!FileSet}
\index[general]{FileSet!Windows}
\label{win32}
If you are entering Windows file names, the directory path may be preceded by
the drive and a colon (as in c:). However, the path separators must be
specified in Unix convention (i.e. forward slash (/)). If you wish to include
a quote in a file name, precede the quote with a backslash
(\textbackslash{}). For example you might use the following
for a Windows machine to backup the "My Documents" directory:

\begin{bconfig}{Windows FileSet}
FileSet {
  Name = "Windows Set"
  Include {
    Options {
       WildFile = "*.obj"
       WildFile = "*.exe"
       exclude = yes
     }
     File = "c:/My Documents"
  }
}
\end{bconfig}

For exclude lists to work correctly on Windows, you must observe the following
rules:

\begin{itemize}
\item Filenames are case sensitive, so you must use the correct case.
\item To exclude a directory, you must not have a trailing slash on the
   directory name.
\item If you have spaces in your filename, you must enclose the entire name
   in double-quote characters ("). Trying to use a backslash before  the space
   will not work.
\item If you are using the old Exclude syntax (noted below), you may not
   specify a drive letter in the exclude.  The new syntax noted above
   should work fine including driver letters.
\end{itemize}

Thanks to Thiago Lima for summarizing the above items for us. If you are
having difficulties getting includes or excludes to work, you might want to
try using the {\bf estimate job=xxx listing} command documented in the
\ilink{Console chapter}{estimate} of this manual.

On Win32 systems, if you move a directory or file or rename a file into the
set of files being backed up, and a Full backup has already been made, Bareos
will not know there are new files to be saved during an Incremental or
Differential backup (blame Microsoft, not us). To avoid this problem, please
{\bf copy} any new directory or files into the backup area. If you do not have
enough disk to copy the directory or files, move them, but then initiate a
Full backup.


\paragraph*{Example Fileset for Windows}
\index[general]{FileSet!Windows Example}
\index[general]{Windows!FileSet!Example}

The following example demostrates a Windows FileSet.
It backups all data from all fixed drives and only excludes some Windows temporary data.

\begin{bconfig}{Windows All Drives FileSet}
FileSet {
  Name = "Windows All Drives"
  Enable VSS = yes
  Include {
    Options {
      Signature = MD5
      Drive Type = fixed
      IgnoreCase = yes
      WildFile = "[A-Z]:/pagefile.sys"
      WildDir = "[A-Z]:/RECYCLER"
      WildDir = "[A-Z]:/$RECYCLE.BIN"
      WildDir = "[A-Z]:/System Volume Information"
      Exclude = yes
    }
    File = /
  }
}
\end{bconfig}

% workaround for kile editor
\hide{$}

\variable{File = /} includes all Windows drives.
Using \variable{Drive Type = fixed} excludes drives like USB-Stick or CD-ROM Drive.
Using \variable{WildDir = "[A-Z]:/RECYCLER"} excludes the backup of the directory \path|RECYCLER| from all drives.


\subsection{Testing Your FileSet}
\index[general]{FileSet!Testing Your}
\index[general]{Testing Your FileSet}

If you wish to get an idea of what your FileSet will really backup or if your
exclusion rules will work correctly, you can test it by using the
\ilink{estimate}{estimate} command.

As an example, suppose you add the following test FileSet:

\begin{bconfig}{FileSet for all *.c files}
FileSet {
  Name = Test
  Include {
    File = /home/xxx/test
    Options {
       regex = ".*\\.c$"
    }
  }
}
\end{bconfig}
\hide{$}

You could then add some test files to the directory {\bf /home/xxx/test}
and use the following command in the console:

\begin{bconsole}{estimate}
estimate job=<any-job-name> listing client=<desired-client> fileset=Test
\end{bconsole}

to give you a listing of all files that match.  In the above
example, it should be only files with names ending in  {\bf .c}.



\section{Client Resource}
\label{DirectorResourceClient}
\index[general]{Resource!Client}
\index[general]{Client Resource}

The Client (or FileDaemon) resource defines the attributes of the Clients that are served by
this Director; that is the machines that are to be backed up. You will need
one Client resource definition for each machine to be backed up.

\input{autogenerated/bareos-dir-resource-client-table.tex}
\defDirective{Dir}{Client}{Address}{}{}{%
Where the address is a host name, a fully qualified domain name, or a
network address in dotted quad notation for a Bareos File server daemon.
This directive is required.
}

\defDirective{Dir}{Client}{Authtype}{}{}{%
Specifies the authentication type that must be supplied when connecting to
a backup protocol that uses a specific authentication type.

The following values are allowed:
\begin{enumerate}
\item None - Use no password
\item Clear - Use clear text password
\item MD5 - Use MD5 hashing
\end{enumerate}
}

\defDirective{Dir}{Client}{Autoprune}{}{}{%
\label{AutoPrune}
If AutoPrune is set to  {\bf yes} (default is no), Bareos
will  automatically apply the File retention period and the Job  retention
period for the Client at the end of the Job.  If you leave the default {\bf AutoPrune = no},
pruning will not be done, and your Catalog will grow in size each time you
run a Job.  Pruning affects only information in the catalog and not data
stored in the backup archives (on Volumes), but if pruning deletes all data
referring to a certain volume, the volume is regarded as empty and will possibly
be overwritten before the volume retention has expired.
}

\defDirective{Dir}{Client}{Catalog}{}{}{%
This specifies the  name of the catalog resource to be used for this Client.
If none is specified the first defined catalog is used.
}

\defDirective{Dir}{Client}{Description}{}{}{%
}

\defDirective{Dir}{Client}{Enabled}{}{}{%
}

\defDirective{Dir}{Client}{FD Address}{}{}{%
}

\defDirective{Dir}{Client}{FD Password}{}{}{%
}

\defDirective{Dir}{Client}{FD Port}{}{}{%
Where the port is a port  number at which the Bareos File server daemon can
be contacted.  The default is 9102. For NDMP backups set this to 10000.
}

\defDirective{Dir}{Client}{File Retention}{}{}{%
The File Retention directive defines the length of time that  Bareos will
keep File records in the Catalog database after the End time of the
Job corresponding to the File records.
When this time period expires, and if
{\bf AutoPrune} is set to  {\bf yes} Bareos will prune (remove) File records
that  are older than the specified File Retention period. Note, this  affects
only records in the catalog database. It does not  affect your archive
backups.

File records  may actually be retained for a shorter period than you specify
on  this directive if you specify either a shorter \linkResourceDirective{Dir}{Client}{Job Retention}  or a
shorter \linkResourceDirective{Dir}{Pool}{Volume Retention} period. The shortest  retention period of the
three takes precedence.  The time may be expressed in seconds, minutes,
hours, days, weeks, months, quarters, or years. See the
\ilink{ Configuration chapter}{Time} of this  manual for
additional details of time specification.

The  default is 60 days.
}

\defDirective{Dir}{Client}{Hard Quota}{}{}{%
This is the maximal amount this client can backup before any backup Job
will be aborted.
}

\defDirective{Dir}{Client}{Heartbeat Interval}{}{}{%
}

\defDirective{Dir}{Client}{Job Retention}{}{}{%
The Job Retention directive defines the length of time that  Bareos will keep
Job records in the Catalog database after the Job End time.  When
this time period expires, and if {\bf AutoPrune} is set to {\bf yes}
Bareos will prune (remove) Job records that are older than the specified
File Retention period.  As with the other retention periods, this
affects only records in the catalog and not data in your archive backup.

If a Job record is selected for pruning, all associated File and JobMedia
records will also be pruned regardless of the File Retention period set.
As a consequence, you normally will set the File retention period to be
less than the Job retention period.  The Job retention period can actually
be less than the value you specify here if you set the {\bf Volume
Retention} directive in the Pool resource to a smaller duration.  This is
because the Job retention period and the Volume retention period are
independently applied, so the smaller of the two takes precedence.

The Job retention period is specified as seconds,  minutes, hours, days,
weeks, months,  quarters, or years.  See the
\ilink{ Configuration chapter}{Time} of this manual for
additional details of  time specification.

The default is 180 days.
}

\defDirective{Dir}{Client}{Maximum Bandwidth Per Job}{}{}{%
The speed parameter specifies the maximum allowed bandwidth that a job may use
when started for this Client. The speed parameter should be specified in
k/s, Kb/s, m/s or Mb/s.
}

\defDirective{Dir}{Client}{Maximum Concurrent Jobs}{}{}{%
where {\textless}number{\textgreater}  is the maximum number of Jobs with the current Client
that  can run concurrently. Note, this directive limits only Jobs  for Clients
with the same name as the resource in which it appears. Any  other
restrictions on the maximum concurrent jobs such as in  the Director, Job, or
Storage resources will also apply in addition to  any limit specified here.
The  default is set to 1, but you may set it to a larger number.
}

\defDirective{Dir}{Client}{Name}{}{}{%
The client name which will be used in the  Job resource directive or in the
console run command.
}

\defDirective{Dir}{Client}{NDMP Block Size}{}{}{%
This directive sets the default NDMP blocksize for this client.
}

\defDirective{Dir}{Client}{NDMP Log Level}{}{}{%
This directive sets the loglevel for the NDMP protocol library.
}

\defDirective{Dir}{Client}{Passive}{}{13.2}{%
The normal way of initializing the data channel (the channel where the backup data itself is transported)
is done by the file daemon (client) that connects to the storage daemon.

By using the client passive mode, the initialization of the datachannel is reversed, so that the storage daemon connects to the filedaemon.

See chapter \ilink{Passive Client}{PassiveClient}.
}

\defDirective{Dir}{Client}{Password}{}{}{%
This is the password to be  used when establishing a connection with the File
services, so  the Client configuration file on the machine to be backed up
must  have the same password defined for this Director.
If you have either \file{/dev/random} or {\bf bc} on your machine,
Bareos will generate a random  password during the configuration process,
otherwise it will  be left blank.

The password is plain text.  It is not generated through any special
process, but it is preferable for security reasons to make the text
random.
}

\defDirective{Dir}{Client}{Port}{}{}{%
}

\defDirective{Dir}{Client}{Protocol}{Native{\textbar}NDMP}{13.2.0}{%
The backup protocol to use to run the Job.

Currently the director understand the following protocols:
\begin{enumerate}
\item Native - The native Bareos protocol
\item NDMP - The NDMP protocol
\end{enumerate}
}

\defDirective{Dir}{Client}{Quota Include Failed Jobs}{}{}{%
When calculating the amount a client used take into consideration any failed Jobs.
Default {\bf Yes}.
}

\defDirective{Dir}{Client}{Soft Quota}{}{}{%
This is the amount after which there will be a warning issued that a client
is over his softquota. A client can keep doing backups until it hits the
hardquota or when the Soft Quota Graceperiod is expired.
}

\defDirective{Dir}{Client}{Soft Quota Grace Period}{}{}{%
Time allowed for a client to be over its softquota before it will be enforced.
}

\defDirective{Dir}{Client}{Strict Quotas}{}{}{%
The directive Strict Quotas determines, if after the Grace Time Period is over,
the Burst Limit is enforced (Strict Quotas = {\bf No}) or
the Soft Limit is enforced (Strict Quotas = {\bf Yes}).
}

\defDirective{Dir}{Client}{TLS Allowed CN}{}{}{%
}

\defDirective{Dir}{Client}{TLS Authenticate}{}{}{%
}

\defDirective{Dir}{Client}{TLS CA Certificate Dir}{}{}{%
}

\defDirective{Dir}{Client}{TLS CA Certificate File}{}{}{%
}

\defDirective{Dir}{Client}{TLS Certificate}{}{}{%
}

\defDirective{Dir}{Client}{TLS Certificate Revocation List}{}{}{%
}

\defDirective{Dir}{Client}{TLS Enable}{}{}{%
Bareos can be configured to encrypt all its network traffic.
See chapter \nameref{TlsDirectives} to see,
how the Bareos Director (and the other components) must be configured to use TLS.
}

\defDirective{Dir}{Client}{TLS Key}{}{}{%
}

\defDirective{Dir}{Client}{TLS Require}{}{}{%
}

\defDirective{Dir}{Client}{Username}{}{}{%
Specifies the username that must be supplied when authenticating.
Only used for the non Native protocols at the moment.

}

\input{autogenerated/bareos-dir-resource-client-description.tex}

The following is an example of a valid Client resource definition:

\begin{bconfig}{Minimal client resource definition in bareos-dir.conf}
Client {
  Name = client1-fd
  Address = client1.example.com
  Password = "secret"
}
\end{bconfig}

The following is an example of a Quota Configuration in Client resource:

\begin{bconfig}{Quota Configuration in Client resource}
Client {
  Name = client1-fd
  Address = client1.example.com
  Password = "secret"

  # Quota
  Soft Quota = 50 mb
  Soft Quota Grace Period = 2 days
  Strict Quotas = Yes
  Hard Quota = 150 mb
  Quota Include Failed Jobs = yes
}
\end{bconfig}


\section{Storage Resource}
\label{DirectorResourceStorage}
\index[general]{Resource!Storage}
\index[general]{Storage Resource}

The Storage resource defines which Storage daemons are available for use by
the Director.

\input{autogenerated/bareos-dir-resource-storage-table.tex}
\defDirective{Dir}{Storage}{Address}{}{}{%
Where the address is a host name,  a {\bf fully qualified domain name}, or an
{\bf IP address}. Please note  that the {\textless}address{\textgreater} as specified here
will be transmitted to  the File daemon who will then use it to contact the
Storage daemon. Hence,  it is {\bf not}, a good idea to use {\bf localhost} as
the  name but rather a fully qualified machine name or an IP address.  This
directive is required.
}

\defDirective{Dir}{Storage}{Allow Compression}{}{}{%
This directive is optional, and if you specify {\bf No},
it will cause backups jobs running on this storage resource to run
without client File Daemon compression.  This effectively overrides
compression options in FileSets used by jobs which use this storage
resource.
\label{AllowCompression}
}

\defDirective{Dir}{Storage}{Auth Type}{}{}{%
Specifies the authentication type that must be supplied when connecting to
a backup protocol that uses a specific authentication type.
}

\defDirective{Dir}{Storage}{Auto Changer}{}{}{%
When \linkResourceDirective{Dir}{Storage}{Device} refers to an Auto Changer (\linkResourceDirective{Sd}{Device}{Autochanger}),
this directive must be set to \parameter{yes}.

If you specify \parameter{yes},
\begin{itemize}
  \item Volume management command like \bcommand{label}{} or \bcommand{add}{} will request a Autochanger Slot number.
  \item Bareos will prefer Volumes, that are in a Auto Changer slot.
    If none of theses volumes can be used, even after recycling, pruning, ...,
    Bareos will search for any volume of the same \linkResourceDirective{Dir}{Storage}{Media Type} whether or not in the magazine.
\end{itemize}

Please consult the \nameref{AutochangersChapter} chapter for details.
}

\defDirective{Dir}{Storage}{Collect Statistics}{}{}{%
Collect statistic information. These information will be collected by the Director (see \linkResourceDirective{Dir}{Director}{Statistics Collect Interval}) and stored in the Catalog.
}

\defDirective{Dir}{Storage}{Description}{}{}{%
Information.
}

\defDirective{Dir}{Storage}{Device}{}{}{%

If \linkResourceDirective{Dir}{Job}{Protocol} is not \parameter{NDMP_NATIVE} (default is \linkResourceDirectiveValue{Dir}{Job}{Protocol}{Native}), this directive refers to one or multiple \linkResourceDirective{Sd}{Device}{Name}
or a single \linkResourceDirective{Sd}{Autochanger}{Name}.

If an Autochanger should be used, it had to refer to a configured \linkResourceDirective{Sd}{Autochanger}{Name}.
In this case, also set \linkResourceDirectiveValue{Dir}{Storage}{Auto Changer}{yes}.

Otherwise it refers to one or more configured \linkResourceDirective{Sd}{Device}{Name}, see  \nameref{sec:MultipleStorageDevices}.

This name is not the physical device name, but the logical device name as
defined in the \bareosSd resource.

If \resourceDirectiveValue{Dir}{Job}{Protocol}{NDMP_NATIVE}, it refers to tape devices on the NDMP \TapeAgent, see \nameref{sec:NdmpNative}.
}



\defDirective{Dir}{Storage}{Enabled}{}{}{%
}

\defDirective{Dir}{Storage}{Heartbeat Interval}{}{}{%
This directive is optional and if specified will cause the Director to
set a keepalive interval (heartbeat) in seconds on each of the sockets
it opens for the Storage resource.  This value will override any
specified at the Director level.  It is implemented only on systems
(Linux, ...) that provide the {\bf setsockopt} TCP\_KEEPIDLE function.
The default value is zero, which means no change is made to the socket.
}


\defDirective{Dir}{Storage}{Lan Address}{}{}{%
This directive might be useful in network setups where the \bareosDir and \bareosFd need different addresses to communicate with the \bareosSd.

For details, see \nameref{LanAddress}.

This directive corresponds to \linkResourceDirective{Dir}{Client}{Lan Address}.
}

\defDirective{Dir}{Storage}{Maximum Bandwidth Per Job}{}{}{%
}

\defDirective{Dir}{Storage}{Maximum Concurrent Jobs}{}{}{%
This directive specifies the maximum number of Jobs with the current
Storage resource that can run concurrently.  Note, this directive limits
only Jobs for Jobs using this Storage daemon.  Any other restrictions on
the maximum concurrent jobs such as in the Director, Job or Client
resources will also apply in addition to any limit specified here.

If you set the Storage daemon's number of concurrent jobs greater than one,
we recommend that you read \nameref{ConcurrentJobs} and/or
turn data spooling on as documented in \nameref{SpoolingChapter}.
}

\defDirective{Dir}{Storage}{Maximum Concurrent Read Jobs}{}{}{%
This directive specifies the maximum number of Jobs with the current
Storage resource that can read concurrently.
}

\defDirective{Dir}{Storage}{Media Type}{}{}{%
This directive specifies the Media Type to be used to store the data.
This is an arbitrary string of characters up to 127 maximum that you
define.  It can be anything you want.  However, it is best to make it
descriptive of the storage media (e.g.  File, DAT, "HP DLT8000", 8mm,
...).  In addition, it is essential that you make the {\bf Media Type}
specification unique for each storage media type.  If you have two DDS-4
drives that have incompatible formats, or if you have a DDS-4 drive and
a DDS-4 autochanger, you almost certainly should specify different {\bf
Media Types}.  During a restore, assuming a {\bf DDS-4} Media Type is
associated with the Job, Bareos can decide to use any Storage daemon
that supports Media Type {\bf DDS-4} and on any drive that supports it.

If you are writing to disk Volumes, you must make doubly sure that each
Device resource defined in the Storage daemon (and hence in the
Director's conf file) has a unique media type.  Otherwise Bareos
may assume, these Volumes can be mounted and read by any Storage daemon File device.

Currently Bareos permits only a single Media Type per Storage
Device definition. Consequently, if
you have a drive that supports more than one Media Type, you can
give a unique string to Volumes with different intrinsic Media
Type (Media Type = DDS-3-4 for DDS-3 and DDS-4 types), but then
those volumes will only be mounted on drives indicated with the
dual type (DDS-3-4).

If you want to tie Bareos to using a single Storage daemon or drive, you
must specify a unique Media Type for that drive.  This is an important
point that should be carefully understood.  Note, this applies equally
to Disk Volumes.  If you define more than one disk Device resource in
your Storage daemon's conf file, the Volumes on those two devices are in
fact incompatible because one can not be mounted on the other device
since they are found in different directories.  For this reason, you
probably should use two different Media Types for your two disk Devices
(even though you might think of them as both being File types).  You can
find more on this subject in the \ilink{Basic Volume
Management}{DiskChapter} chapter of this manual.

The {\bf MediaType} specified in the Director's Storage resource, {\bf
must} correspond to the {\bf Media Type} specified in the {\bf Device}
resource of the {\bf Storage daemon} configuration file.  This directive
is required, and it is used by the Director and the Storage daemon to
ensure that a Volume automatically selected from the Pool corresponds to
the physical device.  If a Storage daemon handles multiple devices (e.g.
will write to various file Volumes on different partitions), this
directive allows you to specify exactly which device.

As mentioned above, the value specified in the Director's Storage
resource must agree with the value specified in the Device resource in
the {\bf Storage daemon's} configuration file.  It is also an additional
check so that you don't try to write data for a DLT onto an 8mm device.
\label{MediaType}
}

\defDirective{Dir}{Storage}{Name}{}{}{%
The name of the storage resource. This  name appears on the Storage directive
specified in the Job resource and is required.
}

\defDirective{Dir}{Storage}{Paired Storage}{}{}{%
For NDMP backups this points to the definition of the Native Storage
that is accesses via the NDMP protocol. For now we only support NDMP
backups and restores to access Native Storage Daemons via the NDMP
protocol. In the future we might allow to use Native NDMP storage which
is not bound to a Bareos Storage Daemon.
}

\defDirective{Dir}{Storage}{Password}{}{}{%
This is the password to be used  when establishing a connection with the
Storage services. This  same password also must appear in the Director
resource of the Storage  daemon's configuration file. This directive is
required.

The password is plain text.
}

\defDirective{Dir}{Storage}{Port}{}{}{%
Where port is the port to use to  contact the storage daemon for information
and to start jobs.  This same port number must appear in the Storage resource
of the  Storage daemon's configuration file.
}

\defDirective{Dir}{Storage}{Protocol}{}{}{%
}

\defDirective{Dir}{Storage}{SD Address}{}{}{%
}

\defDirective{Dir}{Storage}{SD Password}{}{}{%
}

\defDirective{Dir}{Storage}{SD Port}{}{}{%
}

\defDirective{Dir}{Storage}{Sdd Port}{}{}{%
}

\defDirective{Dir}{Storage}{TLS Authenticate}{}{}{%
}

\defDirective{Dir}{Storage}{TLS CA Certificate File}{}{}{%
}

\defDirective{Dir}{Storage}{TLS CA Certificate Dir}{}{}{%
}

\defDirective{Dir}{Storage}{TLS Certificate}{}{}{%
}

\defDirective{Dir}{Storage}{TLS Certificate Revocation List}{}{}{%
}

\defDirective{Dir}{Storage}{TLS Enable}{}{}{%
Bareos can be configured to encrypt all its network traffic.
For details, refer to chapter \nameref{TlsDirectives}.
}

\defDirective{Dir}{Storage}{TLS Key}{}{}{%
}

\defDirective{Dir}{Storage}{TLS Require}{}{}{%
}

\defDirective{Dir}{Storage}{Username}{}{}{%
}

\input{autogenerated/bareos-dir-resource-storage-description.tex}

The following is an example of a valid Storage resource definition:

\begin{bconfig}{Storage resource (tape) example}
Storage {
  Name = DLTDrive
  Address = lpmatou
  Password = storage_password # password for Storage daemon
  Device = "HP DLT 80"    # same as Device in Storage daemon
  Media Type = DLT8000    # same as MediaType in Storage daemon
}
\end{bconfig}

\section{Pool Resource}
\label{DirectorResourcePool}
\index[general]{Resource!Pool}
\index[general]{Pool Resource}

The Pool resource defines the set of storage Volumes (tapes or files) to be
used by Bareos to write the data. By configuring different Pools, you can
determine which set of Volumes (media) receives the backup data. This permits,
for example, to store all full backup data on one set of Volumes and all
incremental backups on another set of Volumes. Alternatively, you could assign
a different set of Volumes to each machine that you backup. This is most
easily done by defining multiple Pools.

Another important aspect of a Pool is that it contains the default attributes
(Maximum Jobs, Retention Period, Recycle flag, ...) that will be given to a
Volume when it is created. This avoids the need for you to answer a large
number of questions when labeling a new Volume. Each of these attributes can
later be changed on a Volume by Volume basis using the \bcommand{update}{} command in
the console program. Note that you must explicitly specify which Pool Bareos
is to use with each Job. Bareos will not automatically search for the correct
Pool.

To use a Pool, there are three distinct steps. First the Pool must be defined
in the Director's configuration. Then the Pool must be written to the
Catalog database. This is done automatically by the Director each time that it
starts. Finally, if you change the Pool definition in the Director's
configuration file and restart Bareos, the pool will be updated alternatively
you can use the \bcommand{update}{pool} console command to refresh the database
image. It is this database image rather than the Director's resource image
that is used for the default Volume attributes. Note, for the pool to be
automatically created or updated, it must be explicitly referenced by a Job
resource.

If automatic labeling is not enabled (see \nameref{AutomaticLabeling})
the physical media must be manually labeled.
The labeling can either be done with
the \bcommand{label}{} command in the console program or using the \command{btape}
program. The preferred method is to use the \bcommand{label}{} command in the console program.
Generally, automatic labeling is enabled for \linkResourceDirectiveValue{Sd}{Device}{Device Type}{File}
and disabled for \linkResourceDirectiveValue{Sd}{Device}{Device Type}{Tape}.

Finally, you must add Volume names (and their attributes) to the Pool. For
Volumes to be used by Bareos they must be of the same \linkResourceDirective{Sd}{Device}{Media Type} as the
archive device specified for the job (i.e. if you are going to back up to a
DLT device, the Pool must have DLT volumes defined since 8mm volumes cannot be
mounted on a DLT drive). The \linkResourceDirective*{Sd}{Device}{Media Type} has particular importance if you
are backing up to files. When running a Job, you must explicitly specify which
Pool to use. Bareos will then automatically select the next Volume to use from
the Pool, but it will ensure that the \linkResourceDirective*{Sd}{Device}{Media Type} of any Volume selected
from the Pool is identical to that required by the Storage resource you have
specified for the Job.

If you use the \bcommand{label}{} command in the console program to label the
Volumes, they will automatically be added to the Pool, so this last step is
not normally required.

It is also possible to add Volumes to the database without explicitly labeling
the physical volume. This is done with the \bcommand{add}{} console command.

As previously mentioned, each time Bareos starts, it scans all the Pools
associated with each Catalog, and if the database record does not already
exist, it will be created from the Pool Resource definition.
If you change the Pool definition, you manually have to call \bcommand{update}{pool} command in
the console program to propagate the changes to existing volumes.

The Pool Resource defined in the Director's configuration
may contain the following directives:

\input{autogenerated/bareos-dir-resource-pool-table.tex}
\defDirective{Dir}{Pool}{Action On Purge}{}{}{%
This directive \configline{Action On Purge=Truncate} instructs Bareos to truncate the
volume when it is purged with the \configline{Purge Volume Action=Truncate}
command. It is useful to prevent disk based volumes from consuming too much
space.

\begin{bconfig}{}^^J
Pool \{^^J
\  Name = Default^^J
\  Action On Purge = Truncate^^J
\  ...^^J
\}^^J
\end{bconfig}

You can schedule the truncate operation at the end of your CatalogBackup job
like in this example:

\begin{bconfig}{}^^J
Job \{^^J
\ Name = CatalogBackup^^J
\ ...^^J
\ RunScript \{^^J
\ \   RunsWhen=After^^J
\ \   RunsOnClient=No^^J
\ \   Console = "purge volume action=all allpools storage=File"^^J
\ \}^^J
\}^^J
\end{bconfig}
}

\defDirective{Dir}{Pool}{Autoprune}{}{}{%
\label{PoolAutoPrune}%
If AutoPrune is set to {\bf yes}, Bareos  will automatically
apply the Volume Retention period when new
Volume is needed and no appendable Volumes exist in the Pool.  Volume
pruning causes expired Jobs (older than the {\bf Volume Retention}
period) to be deleted from the Catalog and permits possible recycling of
the Volume.
}

\defDirective{Dir}{Pool}{Catalog}{}{}{%
This specifies the name of the catalog resource to be used for this Pool.
When a catalog is defined in a Pool it will override the definition in
the client (and the Catalog definition in a Job since
\sinceVersion{dir}{Job catalog overwriten by Pool catalog}{13.4.0}). e.g.
this catalog setting takes precedence over any other definition.
}

\defDirective{Dir}{Pool}{Catalog Files}{}{}{%
This directive defines whether or not you want the names of the files
that were saved to be put into the catalog.  The default is {\bf yes}.
The advantage of specifying {\bf Catalog Files = No} is that you will
have a significantly smaller Catalog database.  The disadvantage is that
you will not be able to produce a Catalog listing of the files backed up
for each Job (this is often called Browsing).  Also, without the File
entries in the catalog, you will not be able to use the Console {\bf
restore} command nor any other command that references File entries.
}

\defDirective{Dir}{Pool}{Cleaning Prefix}{}{}{%
This directive defines a prefix string, which if it matches the
beginning of a Volume name during labeling of a Volume, the Volume will
be defined with the VolStatus set to {\bf Cleaning} and thus Bareos will
never attempt to use this tape.  This is primarily for use with
autochangers that accept barcodes where the convention is that barcodes
beginning with {\bf CLN} are treated as cleaning tapes.

The default value for this directive is consequently set to {\bf CLN}, so
that in most cases the cleaning tapes are automatically recognized without
configuration.
If you use another prefix for your cleaning tapes, you can set this directive
accordingly.
}

\defDirective{Dir}{Pool}{Copy Pool}{}{}{%
}

\defDirective{Dir}{Pool}{Description}{}{}{%
}

\defDirective{Dir}{Pool}{File Retention}{}{}{%
The File Retention directive defines the length of time that  Bareos will
keep File records in the Catalog database after the End time of the
Job corresponding to the File records.

This directive takes precedence over Client directives of the same name. For
example, you can decide to increase Retention times for Archive or OffSite
Pool.

Note, this affects only records in the catalog database. It does not affect
your archive backups.

For more information see Client documentation about
\linkResourceDirective{Dir}{Client}{File Retention}
}

\defDirective{Dir}{Pool}{Job Retention}{}{}{%
The Job Retention directive defines the length of time that Bareos will keep
Job records in the Catalog database after the Job End time.  As with the
other retention periods, this affects only records in the catalog and not
data in your archive backup.

This directive takes precedence over Client directives of the same name.
For example, you can decide to increase Retention times for Archive or
OffSite Pool.

For more information see Client side documentation
\linkResourceDirective{Dir}{Client}{Job Retention}
}

\defDirective{Dir}{Pool}{Label Format}{}{}{%
\label{Label}%
This directive specifies the format of the labels contained in this
pool.  The format directive is used as a sort of template to create new
Volume names during automatic Volume labeling.

The {\bf format} should be specified in double quotes, and consists of
letters, numbers and the special characters hyphen ({\bf -}), underscore
({\bf \_}), colon ({\bf :}), and period ({\bf .}), which are the legal
characters for a Volume name.  The {\bf format} should be enclosed in
double quotes (").

In addition, the format may contain a number of variable expansion
characters which will be expanded by a complex algorithm allowing you to
create Volume names of many different formats.  In all cases, the
expansion process must resolve to the set of characters noted above that
are legal Volume names.  Generally, these variable expansion characters
begin with a dollar sign ({\bf \$}) or a left bracket ({\bf [}).  If you
specify variable expansion characters, you should always enclose the
format with double quote characters ({\bf "}).  For more details on
variable expansion, please see the \ilink{Variable
Expansion}{VarsChapter} Chapter of this manual.

If no variable expansion characters are found in the string, the Volume
name will be formed from the {\bf format} string appended with the
a unique number that increases.  If you do not remove volumes from the
pool, this number should be the number of volumes plus one, but this
is not guaranteed. The unique number will be edited as four
digits with leading zeros.  For example, with a {\bf Label Format =
"File-"}, the first volumes will be named {\bf File-0001}, {\bf
File-0002}, ...

With the exception of Job specific variables, you can test your {\bf
LabelFormat} by using the \ilink{var command}{var} the Console Chapter
of this manual.

In almost all cases, you should enclose the format specification (part
after the equal sign) in double quotes.
}

\defDirective{Dir}{Pool}{Label Type}{}{}{%
}

\defDirective{Dir}{Pool}{Maximum Blocksize}{}{14.2.0}{%
The \configdirective{Maximum Block Size} can be defined here or at \ilink{Storage Device resource}{storage-device-maximumblocksize}.
If not defined, its default is 63 KB.
Increasing this value could improve the throughput of writing to tapes a lot, see \ilink{Setting Block Sizes}{setblocksizes} chapter.
}

\defDirective{Dir}{Pool}{Maximum Volume Bytes}{}{}{%
This directive specifies the maximum number of bytes that can be written
to the Volume.  If you specify zero (the default), there is no limit
except the physical size of the Volume.  Otherwise, when the number of
bytes written to the Volume equals {\bf size} the Volume will be marked
{\bf Used}.  When the Volume is marked {\bf Used} it can no longer be
used for appending Jobs, much like the {\bf Full} status but it can be
recycled if recycling is enabled, and thus the Volume can be re-used
after recycling.  This value is checked and the {\bf Used} status set
while the job is writing to the particular volume.

This directive is particularly useful for restricting the size
of disk volumes, and will work correctly even in the case of
multiple simultaneous jobs writing to the volume.

The value defined by this directive in the bareos-dir.conf file is the
default value used when a Volume is created.  Once the volume is
created, changing the value in the bareos-dir.conf file will not change
what is stored for the Volume.  To change the value for an existing
Volume you must use the {\bf update} command in the Console.
}

\defDirective{Dir}{Pool}{Maximum Volume Files}{}{}{%
This directive specifies the maximum number of files that can be written
to the Volume.  If you specify zero (the default), there is no limit.
Otherwise, when the number of files written to the Volume equals {\bf
positive-integer} the Volume will be marked {\bf Used}.  When the Volume
is marked {\bf Used} it can no longer be used for appending Jobs, much
like the {\bf Full} status but it can be recycled if recycling is
enabled and thus used again.  This value is checked and the {\bf Used}
status is set only at the end of a job that writes to the particular
volume.

The value defined by this directive in the bareos-dir.conf file is the
default value used when a Volume is created.  Once the volume is
created, changing the value in the bareos-dir.conf file will not change
what is stored for the Volume.  To change the value for an existing
Volume you must use the {\bf update} command in the Console.
}

\defDirective{Dir}{Pool}{Maximum Volume Jobs}{}{}{%
This directive specifies the maximum number of Jobs that can be written
to the Volume.  If you specify zero (the default), there is no limit.
Otherwise, when the number of Jobs backed up to the Volume equals {\bf
positive-integer} the Volume will be marked {\bf Used}.  When the Volume
is marked {\bf Used} it can no longer be used for appending Jobs, much
like the {\bf Full} status but it can be recycled if recycling is
enabled, and thus used again.  By setting {\bf MaximumVolumeJobs} to
one, you get the same effect as setting {\bf UseVolumeOnce = yes}.

The value defined by this directive in the  bareos-dir.conf
file is the default value used when a Volume  is created. Once the volume is
created, changing the value  in the bareos-dir.conf file will not change what
is stored  for the Volume. To change the value for an existing Volume  you
must use the {\bf update} command in the Console.

If you are running multiple simultaneous jobs, this directive may not
work correctly because when a drive is reserved for a job, this
directive is not taken into account, so multiple jobs may try to
start writing to the Volume. At some point, when the Media record is
updated, multiple simultaneous jobs may fail since the Volume can no
longer be written.
}

\defDirective{Dir}{Pool}{Maximum Volumes}{}{}{%
\label{MaxVolumes}%
This directive specifies the maximum number of volumes (tapes or files)
contained in the pool.  This directive is optional, if omitted or set to
zero, any number of volumes will be permitted.  In general, this
directive is useful for Autochangers where there is a fixed number of
Volumes, or for File storage where you wish to ensure that the backups
made to disk files do not become too numerous or consume too much space.
}

\defDirective{Dir}{Pool}{Migration High Bytes}{}{}{%
}

\defDirective{Dir}{Pool}{Migration Low Bytes}{}{}{%
}

\defDirective{Dir}{Pool}{Migration Time}{}{}{%
}

\defDirective{Dir}{Pool}{Minimum Blocksize}{}{}{%
The \configdirective{Minimum Block Size} can be defined here or at \ilink{Storage Device resource}{storage-device-minimumblocksize}.
For details, see chapter \ilink{Setting Block Sizes}{setblocksizes}.
}

\defDirective{Dir}{Pool}{Name}{}{}{%
The name of the pool.
}

\defDirective{Dir}{Pool}{Next Pool}{}{}{%
This directive specifies the Next pool a Migration or Copy Job
and a Virtual Backup Job will write their data too.
}

\defDirective{Dir}{Pool}{Pool Type}{}{}{%
This directive defines the pool type, which corresponds to the type of
Job being run.  It is required and may be one of the following:

\begin{description}
  \item [Backup]
  \item [*Archive]
  \item [*Cloned]
  \item [*Migration]
  \item [*Copy]
  \item [*Save]
\end{description}
Note, only Backup is currently implemented.
}

\defDirective{Dir}{Pool}{Purge Oldest Volume}{}{}{%
\label{PurgeOldest}%
This directive instructs the Director to search for the oldest used
Volume in the Pool when another Volume is requested by the Storage
daemon and none are available.  The catalog is then {\bf purged}
irrespective of retention periods of all Files and Jobs written to this
Volume.  The Volume is then recycled and will be used as the next Volume
to be written.  This directive overrides any Job, File, or Volume
retention periods that you may have specified.

This directive can be useful if you have a fixed number of Volumes in
the Pool and you want to cycle through them and reusing the oldest one
when all Volumes are full, but you don't want to worry about setting
proper retention periods.  However, by using this option you risk losing
valuable data.

Please be aware that {\bf Purge Oldest Volume} disregards all retention
periods. If you have only a single Volume defined and you turn this
variable on, that Volume will always be immediately overwritten when it
fills!  So at a minimum, ensure that you have a decent number of Volumes
in your Pool before running any jobs.  If you want retention periods to
apply do not use this directive.  To specify a retention period, use the
{\bf Volume Retention} directive (see above).

We {\bf highly} recommend against using this directive, because it is
sure that some day, Bareos will recycle a Volume that contains current
data.  The default is {\bf no}.
}

\defDirective{Dir}{Pool}{Recycle}{}{}{%
\label{PoolRecycle}%
This directive specifies whether or not Purged Volumes may be recycled.
If it is set to {\bf yes} (default) and Bareos needs a volume but finds
none that are appendable, it will search for and recycle (reuse) Purged
Volumes (i.e.  volumes with all the Jobs and Files expired and thus
deleted from the Catalog).  If the Volume is recycled, all previous data
written to that Volume will be overwritten. If Recycle is set to {\bf
no}, the Volume will not be recycled, and hence, the data will remain
valid.  If you want to reuse (re-write) the Volume, and the recycle flag
is no (0 in the catalog), you may manually set the recycle flag (update
command) for a Volume to be reused.

Please note that the value defined by this directive in the
bareos-dir.conf file is the default value used when a Volume is created.
Once the volume is created, changing the value in the bareos-dir.conf
file will not change what is stored for the Volume.  To change the value
for an existing Volume you must use the {\bf update} command in the
Console.

When all Job and File records have been pruned or purged from the
catalog for a particular Volume, if that Volume is marked as
Append, Full, Used, or Error, it will then be marked as Purged. Only
Volumes marked as Purged will be considered to be converted to the
Recycled state if the {\bf Recycle} directive is set to {\bf yes}.
}

\defDirective{Dir}{Pool}{Recycle Current Volume}{}{}{%
\label{RecycleCurrent}%
If Bareos needs a new Volume, this directive instructs Bareos to Prune
the volume respecting the Job and File retention periods.  If all Jobs
are pruned (i.e.  the volume is Purged), then the Volume is recycled and
will be used as the next Volume to be written.  This directive respects
any Job, File, or Volume retention periods that you may have specified,
and thus it is {\bf much} better to use it rather than the Purge Oldest
Volume directive.

This directive can be useful if you have: a fixed number of Volumes in
the Pool, you want to cycle through them, and you have specified
retention periods that prune Volumes before you have cycled through the
Volume in the Pool.

However, if you use this directive and have only one Volume in the Pool,
you will immediately recycle your Volume if you fill it and Bareos needs
another one.  Thus your backup will be totally invalid.  Please use this
directive with care.  The default is {\bf no}.
}

\defDirective{Dir}{Pool}{Recycle Oldest Volume}{}{}{%
\label{RecycleOldest}%
This directive instructs the Director to search for the oldest used
Volume in the Pool when another Volume is requested by the Storage
daemon and none are available.  The catalog is then {\bf pruned}
respecting the retention periods of all Files and Jobs written to this
Volume.  If all Jobs are pruned (i.e. the volume is Purged), then the
Volume is recycled and will be used as the next Volume to be written.
This directive respects any Job, File, or Volume retention periods that
you may have specified, and as such it is {\bf much} better to use this
directive than the Purge Oldest Volume.

This directive can be useful if you have a fixed number of Volumes in the
Pool and you want to cycle through them and you have specified the correct
retention periods.

However, if you use this directive and have only one
Volume in the Pool, you will immediately recycle your Volume if you fill
it and Bareos needs another one. Thus your backup will be totally invalid.
Please use this directive with care. The default is {\bf no}.
}

\defDirective{Dir}{Pool}{Recycle Pool}{}{}{%
\label{PoolRecyclePool}%
This directive defines to which pool
the Volume will be placed (moved) when it is recycled. Without
this directive, a Volume will remain in the same pool when it is
recycled. With this directive, it can be moved automatically to any
existing pool during a recycle. This directive is probably most
useful when defined in the Scratch pool, so that volumes will
be recycled back into the Scratch pool. For more on the see the
\ilink{Scratch Pool}{TheScratchPool} section of this manual.

Although this directive is called RecyclePool, the Volume in
question is actually moved from its current pool to the one
you specify on this directive when Bareos prunes the Volume and
discovers that there are no records left in the catalog and hence
marks it as {\bf Purged}.
}

\defDirective{Dir}{Pool}{Scratch Pool}{}{}{%
\label{PoolScratchPool}%
This directive permits to specify a dedicate \textsl{Scratch} for the
current pool. This pool will replace the special pool named \textsl{Scrach}
for volume selection. For more information about \textsl{Scratch} see
\ilink{Scratch Pool}{TheScratchPool} section of this manual. This is useful
when using multiple storage sharing the same mediatype or when you want to
dedicate volumes to a particular set of pool.
}

\defDirective{Dir}{Pool}{Storage}{}{}{%
The Storage directive defines the name of the storage services where you
want to backup the FileSet data.  For additional details, see the
\nameref{DirectorResourceStorage} of this manual.
The Storage resource may also be specified in the Job resource,
but the value, if any, in the Pool resource overrides any value
in the Job. This Storage resource definition is not required by either
the Job resource or in the Pool, but it must be specified in
one or the other.  If not configuration error will result.
}

\defDirective{Dir}{Pool}{Use Catalog}{}{}{%
}

\defDirective{Dir}{Pool}{Use Volume Once}{}{}{%
The default is {\bf no}.
Use \configdirective{Maximum Volume Jobs = 1} instead.
}

\defDirective{Dir}{Pool}{Volume Retention}{}{}{%
\label{VolRetention}%
The Volume Retention directive defines the length of time that
Bareos will keep records associated with the Volume in
the Catalog database after the End time of each Job written to the
Volume.  When this time period expires, and if {\bf AutoPrune} is set to
{\bf yes} Bareos may prune (remove) Job records that are older than the
specified Volume Retention period if it is necessary to free up a
Volume.  Recycling will not occur until it is absolutely necessary to
free up a volume (i.e. no other writable volume exists).
All File records associated with pruned Jobs are also
pruned.  The time may be specified as seconds, minutes, hours, days,
weeks, months, quarters, or years.  The {\bf Volume Retention} is
applied independently of the {\bf Job Retention} and the {\bf File
Retention} periods defined in the Client resource.  This means that all
the retentions periods are applied in turn and that the shorter period
is the one that effectively takes precedence.  Note, that when the {\bf
Volume Retention} period has been reached, and it is necessary to obtain
a new volume, Bareos will prune both the Job and the File records.  This
pruning could also occur during a {\bf status dir} command because it
uses similar algorithms for finding the next available Volume.

It is important to know that when the Volume Retention period expires,
Bareos does not automatically recycle a Volume. It attempts to keep the
Volume data intact as long as possible before over writing the Volume.

By defining multiple Pools with different Volume Retention periods, you
may effectively have a set of tapes that is recycled weekly, another
Pool of tapes that is recycled monthly and so on.  However, one must
keep in mind that if your {\bf Volume Retention} period is too short, it
may prune the last valid Full backup, and hence until the next Full
backup is done, you will not have a complete backup of your system, and
in addition, the next Incremental or Differential backup will be
promoted to a Full backup.  As a consequence, the minimum {\bf Volume
Retention} period should be at twice the interval of your Full backups.
This means that if you do a Full backup once a month, the minimum Volume
retention period should be two months.

The default Volume retention period is 365 days, and either the default
or the value defined by this directive in the bareos-dir.conf file is
the default value used when a Volume is created.  Once the volume is
created, changing the value in the \file{bareos-dir.conf} file will not change
what is stored for the Volume.  To change the value for an existing
Volume you must use the {\bf update} command in the Console.
}

\defDirective{Dir}{Pool}{Volume Use Duration}{}{}{%
The Volume Use Duration directive defines the time period that the
Volume can be written beginning from the time of first data write to the
Volume.  If the time-period specified is zero (the default), the Volume
can be written indefinitely.  Otherwise, the next time a job
runs that wants to access this Volume, and the time period from the
first write to the volume (the first Job written) exceeds the
time-period-specification, the Volume will be marked {\bf Used}, which
means that no more Jobs can be appended to the Volume, but it may be
recycled if recycling is enabled.
% Using the command {\bf
% status dir} applies algorithms similar to running jobs, so
% during such a command, the Volume status may also be changed.
Once the Volume is
recycled, it will be available for use again.

You might use this directive, for example, if you have a Volume used for
Incremental backups, and Volumes used for Weekly Full backups.  Once the
Full backup is done, you will want to use a different Incremental
Volume.  This can be accomplished by setting the Volume Use Duration for
the Incremental Volume to six days.  I.e.  it will be used for the 6
days following a Full save, then a different Incremental volume will be
used.  Be careful about setting the duration to short periods such as 23
hours, or you might experience problems of Bareos waiting for a tape
over the weekend only to complete the backups Monday morning when an
operator mounts a new tape.

% The use duration is checked and the {\bf Used} status is set only at the
% end of a job that writes to the particular volume, which means that even
% though the use duration may have expired, the catalog entry will not be
% updated until the next job that uses this volume is run. This
% directive is not intended to be used to limit volume sizes
% and will not work correctly (i.e. will fail jobs) if the use
% duration expires while multiple simultaneous jobs are writing
% to the volume.

Please note that the value defined by this directive in the  bareos-dir.conf
file is the default value used when a Volume  is created. Once the volume is
created, changing the value  in the bareos-dir.conf file will not change what
is stored  for the Volume. To change the value for an existing Volume  you
must use the
\ilink{\bf update volume}{UpdateCommand} command in the Console.
}

\input{autogenerated/bareos-dir-resource-pool-description.tex}

The following is an example of a valid Pool resource definition:

\begin{bconfig}{Pool resource example}
Pool {
  Name = Default
  Pool Type = Backup
}
\end{bconfig}


\subsection{Scratch Pool}
\label{TheScratchPool}
\index[general]{Scratch Pool}
\index[general]{Pool!Scratch}

In general, you can give your Pools any name you wish, but there is one
important restriction: the Pool named {\bf Scratch}, if it exists behaves
like a scratch pool of Volumes in that when Bareos needs a new Volume for
writing and it cannot find one, it will look in the Scratch pool, and if
it finds an available Volume, it will move it out of the Scratch pool into
the Pool currently being used by the job.

\section{Catalog Resource}
\label{DirectorResourceCatalog}
\index[general]{Resource!Catalog}
\index[general]{Catalog Resource}

The Catalog Resource defines what catalog to use for the current job.
Currently, Bareos can only handle a single database server (SQLite, MySQL,
PostgreSQL) that is defined when configuring {\bf Bareos}.  However, there
may be as many Catalogs (databases) defined as you wish.  For example, you
may want each Client to have its own Catalog database, or you may want
backup jobs to use one database and verify or restore jobs to use another
database.

Since SQLite is compiled in, it always runs on the same machine
as the Director and the database must be directly accessible (mounted) from
the Director.  However, since both MySQL and PostgreSQL are networked
databases, they may reside either on the same machine as the Director
or on a different machine on the network.  See below for more details.

\input{autogenerated/bareos-dir-resource-catalog-table.tex}
\defDirective{Dir}{Catalog}{Address}{}{}{%
}

\defDirective{Dir}{Catalog}{DB Address}{}{}{%
}

\defDirective{Dir}{Catalog}{DB Driver}{}{}{%
}

\defDirective{Dir}{Catalog}{DB Name}{}{}{%
}

\defDirective{Dir}{Catalog}{DB Password}{}{}{%
}

\defDirective{Dir}{Catalog}{DB Port}{}{}{%
}

\defDirective{Dir}{Catalog}{DB Socket}{}{}{%
}

\defDirective{Dir}{Catalog}{DB User}{}{}{%
}

\defDirective{Dir}{Catalog}{Description}{}{}{%
}

\defDirective{Dir}{Catalog}{Disable Batch Insert}{}{}{%
}

\defDirective{Dir}{Catalog}{Idle Timeout}{}{}{%
}

\defDirective{Dir}{Catalog}{Inc Connections}{}{}{%
}

\defDirective{Dir}{Catalog}{Max Connections}{}{}{%
}

\defDirective{Dir}{Catalog}{Min Connections}{}{}{%
}

\defDirective{Dir}{Catalog}{Multiple Connections}{}{}{%
}

\defDirective{Dir}{Catalog}{Name}{}{}{%
}

\defDirective{Dir}{Catalog}{Password}{}{}{%
}

\defDirective{Dir}{Catalog}{User}{}{}{%
}

\defDirective{Dir}{Catalog}{Validate Timeout}{}{}{%
}


\input{autogenerated/bareos-dir-resource-catalog-description.tex}

The following is an example of a valid Catalog resource definition:

\begin{bconfig}{Catalog Resource for Sqlite}
Catalog
{
  Name = SQLite
  DB Driver = sqlite
  DB Name = bareos;
  DB User = bareos;
  DB Password = ""
}
\end{bconfig}

or for a Catalog on another machine:

\begin{bconfig}{Catalog Resource for remote MySQL}
Catalog
{
  Name = MySQL
  DB Driver = mysql
  DB Name = bareos
  DB User = bareos
  DB Password = "secret"
  DB Address = remote.example.com
  DB Port = 1234
}
\end{bconfig}

\section{Messages Resource}
\label{DirectorResourceMessages}
\index[general]{Resource!Messages}
\index[general]{Messages Resource}

For the details of the Messages Resource, please see the
\nameref{MessagesChapter} of this manual.

\section{Console Resource}
\label{DirectorResourceConsole}
\index[general]{Console Resource}
\index[general]{Resource!Console}

There are three different kinds of consoles, which the administrator or
user can use to interact with the Director. These three kinds of consoles
comprise three different security levels.

\begin{description}
\item[Default Console] \index[dir]{Console!Default Console}
the first console type is an \bquote{anonymous} or \bquote{default}  console,
which has full privileges.  There is no console resource necessary for
this type since the password is specified in the Director's resource and
consequently such consoles do not have a name as defined on a \configdirective{Name} directive.
Typically you would use it only for  administrators.

\item[Named Console] \index[dir]{Named Console} \index[dir]{Console!Named Console} \index[dir]{Console!Restricted Console}
the second type of console, is a
\bquote{named} console (also called \bquote{Restricted Console}) defined within a Console resource in both the Director's
configuration file and in the Console's configuration file.  Both the
names and the passwords in these two entries must match much as is the
case for Client programs.

This second type of console begins with absolutely no privileges except
those explicitly specified in the Director's Console resource.  Thus you
can have multiple Consoles with different names and passwords, sort of
like multiple users, each with different privileges.  As a default,
these consoles can do absolutely nothing -- no commands whatsoever.  You
give them privileges or rather access to commands and resources by
specifying access control lists in the Director's Console resource.  The
ACLs are specified by a directive followed by a list of access names.
Examples of this are shown below.

\begin{itemize}
\item The third type of console is similar to the above mentioned  one in that
it requires a Console resource definition in both the Director and the
Console.  In addition, if the console name, provided on the 
\linkResourceDirective{Dir}{Console}{Name} directive, 
is the same as a Client name, that console is permitted to
use the \bcommand{SetIP}{} command to change the Address directive in the
Director's client resource to the IP address of the Console.  This
permits portables or other machines using DHCP (non-fixed IP addresses)
to "notify" the Director of their current IP address.
\end{itemize}
\end{description}

The Console resource is optional and need not be specified. The following
directives are permitted within these resources:

\input{autogenerated/bareos-dir-resource-console-table.tex}
\defDirective{Dir}{Console}{Catalog ACL}{}{}{%
This directive is used to specify a list of Catalog resource names that
can be accessed by the console.
}

\defDirective{Dir}{Console}{Client ACL}{}{}{%
This directive is used to  specify a list of Client resource names that can be accessed by  the console.
}

\defDirective{Dir}{Console}{Command ACL}{}{}{%
This directive is used to specify a list of of console commands that can
be executed by the console.
}

\defDirective{Dir}{Console}{Description}{}{}{%
}

\defDirective{Dir}{Console}{Fileset ACL}{}{}{%
This directive is used to specify a list of FileSet resource names that
can be accessed by the console.
}

\defDirective{Dir}{Console}{Job ACL}{}{}{%
This directive is used to specify a list of Job resource names that can
be accessed by the console.  Without this directive, the console cannot
access any of the Director's Job resources.  Multiple Job resource names
may be specified by separating them with commas, and/or by specifying
multiple JobACL directives.  For example, the directive may be specified
as:
\bconfigInput{config/DirConsoleJobACL1.conf}
With the above specification, the console can access the Director's  resources
for the four jobs named on the JobACL directives,  but for no others.
}

\defDirective{Dir}{Console}{Name}{}{}{%
The name of the console. This  name must match the name specified in the
Console's configuration  resource (much as is the case with Client
definitions).
}

\defDirective{Dir}{Console}{Password}{}{}{%
Specifies the password that must be supplied for a named Bareos Console
to be authorized.  The same password must appear in the {\bf Console}
resource of the Console configuration file.  For added security, the
password is never actually passed across the network but rather a
challenge response hash code created with the password.  This directive
is required.

The password is plain text.  It is not generated through any special
process.  However, it is preferable for security reasons to choose
random text.
}

\defDirective{Dir}{Console}{Pluginoptions ACL}{}{}{%
}

\defDirective{Dir}{Console}{Pool ACL}{}{}{%
This directive is used to  specify a list of Pool resource names that can be
accessed by the console.
}

\defDirective{Dir}{Console}{Run ACL}{}{}{%
}

\defDirective{Dir}{Console}{Schedule ACL}{}{}{%
This directive is used to  specify a list of Schedule resource names that can
be accessed by the console.
}

\defDirective{Dir}{Console}{Storage ACL}{}{}{%
This directive is used to  specify a list of Storage resource names that can
be accessed by  the console.
}

\defDirective{Dir}{Console}{TLS Allowed CN}{}{}{%
}

\defDirective{Dir}{Console}{TLS Authenticate}{}{}{%
}

\defDirective{Dir}{Console}{TLS CA Certificate Dir}{}{}{%
}

\defDirective{Dir}{Console}{TLS CA Certificate File}{}{}{%
}

\defDirective{Dir}{Console}{TLS Certificate}{}{}{%
}

\defDirective{Dir}{Console}{TLS Certificate Revocation List}{}{}{%
}

\defDirective{Dir}{Console}{TLS DH File}{}{}{%
}

\defDirective{Dir}{Console}{TLS Enable}{}{}{%
}

\defDirective{Dir}{Console}{TLS Key}{}{}{%
}

\defDirective{Dir}{Console}{TLS Require}{}{}{%
}

\defDirective{Dir}{Console}{TLS Verify Peer}{}{}{%
}

\defDirective{Dir}{Console}{Where ACL}{}{}{%
This directive permits you to specify where a restricted console
can restore files. If this directive is not specified, only the
default restore location is permitted (normally \file{/tmp/bareos-restores}.
If {\bf all} is specified any path the
user enters will be accepted (not very secure), any other
value specified (there may be multiple WhereACL directives) will
restrict the user to use that path. For example, on a Unix system,
if you specify "/", the file will be restored to the original
location.  This directive is untested.
}


\input{autogenerated/bareos-dir-resource-console-description.tex}

The example at \nameref{sec:ConsoleAccessExample} shows how to use a console resource for a connection from a client like \command{bconsole}.

\section{Profile Resource}
\label{DirectorResourceProfile}
\index[general]{Profile Resource}
\index[general]{Resource!Profile}

The Profile Resource defines a set of ACLs. \nameref{DirectorResourceConsole}s can be tight to one or more profiles (\linkResourceDirective{Dir}{Console}{Profile}),
making it easier to use a common set of ACLs.

\input{autogenerated/bareos-dir-resource-profile-table.tex}
\defDirective{Dir}{Profile}{Catalog ACL}{}{}{%
}

\defDirective{Dir}{Profile}{Client ACL}{}{}{%
}

\defDirective{Dir}{Profile}{Command ACL}{}{}{%
}

\defDirective{Dir}{Profile}{Description}{}{}{%
}

\defDirective{Dir}{Profile}{File Set ACL}{}{}{%
}

\defDirective{Dir}{Profile}{Job ACL}{}{}{%
}

\defDirective{Dir}{Profile}{Name}{}{}{%
}

\defDirective{Dir}{Profile}{Plugin Options ACL}{}{}{%
}

\defDirective{Dir}{Profile}{Pool ACL}{}{}{%
}

\defDirective{Dir}{Profile}{Run ACL}{}{}{%
}

\defDirective{Dir}{Profile}{Schedule ACL}{}{}{%
}

\defDirective{Dir}{Profile}{Storage ACL}{}{}{%
}

\defDirective{Dir}{Profile}{Where ACL}{}{}{%
}


\input{autogenerated/bareos-dir-resource-profile-description.tex}

\section{Counter Resource}
\label{DirectorResourceCounter}
\index[general]{Resource!Counter}
\index[general]{Counter Resource}

The Counter Resource defines a counter variable that can be accessed by
variable expansion used for creating Volume labels with the \linkResourceDirective{Dir}{Pool}{Label Format}
directive.

\input{autogenerated/bareos-dir-resource-counter-table.tex}
\defDirective{Dir}{Counter}{Catalog}{}{}{%
}

\defDirective{Dir}{Counter}{Description}{}{}{%
}

\defDirective{Dir}{Counter}{Maximum}{}{}{%
}

\defDirective{Dir}{Counter}{Minimum}{}{}{%
}

\defDirective{Dir}{Counter}{Name}{}{}{%
}

\defDirective{Dir}{Counter}{Wrap Counter}{}{}{%
}


\input{autogenerated/bareos-dir-resource-counter-description.tex}

\section{Example Director Configuration File}
\label{SampleDirectorConfiguration}
\index[general]{Configuration!Director!Example}
\index[dir]{Configuration File Example}

See below an example of a full Director configuration file:

\bconfigInput{bareos-dir.conf.in}
