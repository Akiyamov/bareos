%%
%%

\chapter{Daemon Protocol}
\label{_ChapterStart2}
\index{Protocol!Daemon }
\index{Daemon Protocol }

\section{General}
\index{General }
\addcontentsline{toc}{subsection}{General}

This document describes the protocols used between the various daemons. As
Bacula has developed, it has become quite out of date. The general idea still
holds true, but the details of the fields for each command, and indeed the
commands themselves have changed considerably.

It is intended to be a technical discussion of the general daemon protocols
and as such is not targeted at end users but rather at developers and system
administrators that want or need to know more of the working details of {\bf
Bacula}.

\section{Low Level Network Protocol}
\index{Protocol!Low Level Network }
\index{Low Level Network Protocol }
\addcontentsline{toc}{subsection}{Low Level Network Protocol}

At the lowest level, the network protocol is handled by {\bf BSOCK} packets
which contain a lot of information about the status of the network connection:
who is at the other end, etc. Each basic {\bf Bacula} network read or write
actually consists of two low level network read/writes. The first write always
sends four bytes of data in machine independent byte order. If data is to
follow, the first four bytes are a positive non-zero integer indicating the
length of the data that follow in the subsequent write. If the four byte
integer is zero or negative, it indicates a special request, a sort of network
signaling capability. In this case, no data packet will follow. The low level
BSOCK routines expect that only a single thread is accessing the socket at a
time. It is advised that multiple threads do not read/write the same socket.
If you must do this, you must provide some sort of locking mechanism. It would
not be appropriate for efficiency reasons to make every call to the BSOCK
routines lock and unlock the packet.

\section{General Daemon Protocol}
\index{General Daemon Protocol }
\index{Protocol!General Daemon }
\addcontentsline{toc}{subsection}{General Daemon Protocol}

In general, all the daemons follow the following global rules. There may be
exceptions depending on the specific case. Normally, one daemon will be
sending commands to another daemon (specifically, the Director to the Storage
daemon and the Director to the File daemon).

\begin{itemize}
\item Commands are always ASCII commands that are  upper/lower case dependent
   as well as space sensitive.
\item All binary data is converted into ASCII (either with printf statements
   or  using base64 encoding).
\item All responses to commands sent are always  prefixed with a return
   numeric code where codes in the 1000's are  reserved for the Director, the
   2000's are reserved for the  File daemon, and the 3000's are reserved for the
Storage daemon.
\item Any response that is not prefixed with a numeric  code is a command (or
   subcommand if you like) coming  from the other end. For example, while the
   Director is  corresponding with the Storage daemon, the Storage daemon  can
request Catalog services from the Director. This  convention permits each side
to send commands to the  other daemon while simultaneously responding to
commands.
\item Any response that is of zero length, depending on the context,  either
   terminates the data stream being sent or terminates command mode  prior to
   closing the connection.
\item Any response that is of negative length is a special sign that  normally
   requires a response. For example, during data transfer from the  File daemon
   to the Storage daemon, normally the File daemon  sends continuously without
intervening reads. However, periodically,  the File daemon will send a packet
of length -1 indicating that  the current data stream is complete and that the
Storage  daemon should respond to the packet with an OK, ABORT JOB,  PAUSE,
etc. This permits the File daemon to efficiently send  data while at the same
time occasionally ``polling''  the Storage daemon for his status or any
special requests.

Currently, these negative lengths are specific to the daemon, but  shortly,
the range 0 to -999 will be standard daemon wide signals,  while -1000 to
-1999 will be for Director user, -2000 to -2999  for the File daemon, and
-3000 to -3999 for the Storage  daemon.
\end{itemize}

\section{The Protocol Used Between the Director and the Storage Daemon}
\index{Daemon!Protocol Used Between the Director and the Storage }
\index{Protocol Used Between the Director and the Storage Daemon }
\addcontentsline{toc}{subsection}{Protocol Used Between the Director and the
Storage Daemon}

Before sending commands to the File daemon, the Director opens a Message
channel with the Storage daemon, identifies itself and presents its password.
If the password check is OK, the Storage daemon accepts the Director. The
Director then passes the Storage daemon, the JobId to be run as well as the
File daemon authorization (append, read all, or read for a specific session).
The Storage daemon will then pass back to the Director a enabling key for this
JobId that must be presented by the File daemon when opening the job. Until
this process is complete, the Storage daemon is not available for use by File
daemons.

\footnotesize
\begin{verbatim}
SD: listens
DR: makes connection
DR: Hello <Director-name> calling <password>
SD: 3000 OK Hello
DR: JobId=nnn Allow=(append, read) Session=(*, SessionId)
                    (Session not implemented yet)
SD: 3000 OK Job Authorization=<password>
DR: use device=<device-name> media_type=<media-type>
        pool_name=<pool-name> pool_type=<pool_type>
SD: 3000 OK use device
\end{verbatim}
\normalsize

For the Director to be authorized, the {\textless}Director-name{\textgreater} and the
{\textless}password{\textgreater} must match the values in one of the Storage daemon's
Director resources (there may be several Directors that can access a single
Storage daemon).

\section{The Protocol Used Between the Director and the File Daemon}
\index{Daemon!Protocol Used Between the Director and the File }
\index{Protocol Used Between the Director and the File Daemon }
\addcontentsline{toc}{subsection}{Protocol Used Between the Director and the
File Daemon}

A typical conversation might look like the following:

\footnotesize
\begin{verbatim}
FD: listens
DR: makes connection
DR: Hello <Director-name> calling <password>
FD: 2000 OK Hello
DR: JobId=nnn Authorization=<password>
FD: 2000 OK Job
DR: storage address = <Storage daemon address> port = <port-number>
          name = <DeviceName> mediatype = <MediaType>
FD: 2000 OK storage
DR: include
DR: <directory1>
DR: <directory2>
    ...
DR: Null packet
FD: 2000 OK include
DR: exclude
DR: <directory1>
DR: <directory2>
    ...
DR: Null packet
FD: 2000 OK exclude
DR: full
FD: 2000 OK full
DR: save
FD: 2000 OK save
FD: Attribute record for each file as sent to the
    Storage daemon (described above).
FD: Null packet
FD: <append close responses from Storage daemon>
    e.g.
    3000 OK Volumes = <number of volumes>
    3001 Volume = <volume-id> <start file> <start block>
         <end file> <end block> <volume session-id>
    3002 Volume data = <date/time of last write> <Number bytes written>
         <number errors>
    ... additional Volume / Volume data pairs for volumes 2 .. n
FD: Null packet
FD: close socket
\end{verbatim}
\normalsize

\section{The Save Protocol Between the File Daemon and the Storage Daemon}
\index{Save Protocol Between the File Daemon and the Storage Daemon }
\index{Daemon!Save Protocol Between the File Daemon and the Storage }
\addcontentsline{toc}{subsection}{Save Protocol Between the File Daemon and
the Storage Daemon}

Once the Director has send a {\bf save} command to the File daemon, the File
daemon will contact the Storage daemon to begin the save.

In what follows: FD: refers to information set via the network from the File
daemon to the Storage daemon, and SD: refers to information set from the
Storage daemon to the File daemon.

\subsection{Command and Control Information}
\index{Information!Command and Control }
\index{Command and Control Information }
\addcontentsline{toc}{subsubsection}{Command and Control Information}

Command and control information is exchanged in human readable ASCII commands.


\footnotesize
\begin{verbatim}
FD: listens
SD: makes connection
FD: append open session = <JobId> [<password>]
SD: 3000 OK ticket = <number>
FD: append data <ticket-number>
SD: 3000 OK data address = <IPaddress> port = <port>
\end{verbatim}
\normalsize

\subsection{Data Information}
\index{Information!Data }
\index{Data Information }
\addcontentsline{toc}{subsubsection}{Data Information}

The Data information consists of the file attributes and data to the Storage
daemon. For the most part, the data information is sent one way: from the File
daemon to the Storage daemon. This allows the File daemon to transfer
information as fast as possible without a lot of handshaking and network
overhead.

However, from time to time, the File daemon needs to do a sort of checkpoint
of the situation to ensure that everything is going well with the Storage
daemon. To do so, the File daemon sends a packet with a negative length
indicating that he wishes the Storage daemon to respond by sending a packet of
information to the File daemon. The File daemon then waits to receive a packet
from the Storage daemon before continuing.

All data sent are in binary format except for the header packet, which is in
ASCII. There are two packet types used data transfer mode: a header packet,
the contents of which are known to the Storage daemon, and a data packet, the
contents of which are never examined by the Storage daemon.

The first data packet to the Storage daemon will be an ASCII header packet
consisting of the following data.

{\textless}File-Index{\textgreater} {\textless}Stream-Id{\textgreater} {\textless}Info{\textgreater} where {\bf
{\textless}File-Index{\textgreater}} is a sequential number beginning from one that
increments with each file (or directory) sent.

where {\bf {\textless}Stream-Id{\textgreater}} will be 1 for the Attributes record and 2 for
uncompressed File data. 3 is reserved for the MD5 signature for the file.

where {\bf {\textless}Info{\textgreater}} transmit information about the Stream to the
Storage Daemon. It is a character string field where each character has a
meaning. The only character currently defined is 0 (zero), which is simply a
place holder (a no op). In the future, there may be codes indicating
compressed data, encrypted data, etc.

Immediately following the header packet, the Storage daemon will expect any
number of data packets. The series of data packets is terminated by a zero
length packet, which indicates to the Storage daemon that the next packet will
be another header packet. As previously mentioned, a negative length packet is
a request for the Storage daemon to temporarily enter command mode and send a
reply to the File daemon. Thus an actual conversation might contain the
following exchanges:

\footnotesize
\begin{verbatim}
FD: <1 1 0> (header packet)
FD: <data packet containing file-attributes>
FD: Null packet
FD: <1 2 0>
FD: <multiple data packets containing the file data>
FD: Packet length = -1
SD: 3000 OK
FD: <2 1 0>
FD: <data packet containing file-attributes>
FD: Null packet
FD: <2 2 0>
FD: <multiple data packets containing the file data>
FD: Null packet
FD: Null packet
FD: append end session <ticket-number>
SD: 3000 OK end
FD: append close session <ticket-number>
SD: 3000 OK Volumes = <number of volumes>
SD: 3001 Volume = <volumeid> <start file> <start block>
     <end file> <end block> <volume session-id>
SD: 3002 Volume data = <date/time of last write> <Number bytes written>
     <number errors>
SD: ... additional Volume / Volume data pairs for
     volumes 2 .. n
FD: close socket
\end{verbatim}
\normalsize

The information returned to the File daemon by the Storage daemon in response
to the {\bf append close session} is transmit in turn to the Director.
