%%
%%

\chapter{File Services Daemon}
\label{_ChapterStart11}
\index{File Services Daemon }
\index{Daemon!File Services }
\addcontentsline{toc}{section}{File Services Daemon}

Please note, this section is somewhat out of date as the code has evolved
significantly. The basic idea has not changed though.

This chapter is intended to be a technical discussion of the File daemon
services and as such is not targeted at end users but rather at developers and
system administrators that want or need to know more of the working details of
{\bf Bacula}.

The {\bf Bacula File Services} consist of the programs that run on the system
to be backed up and provide the interface between the Host File system and
Bacula -- in particular, the Director and the Storage services.

When time comes for a backup, the Director gets in touch with the File daemon
on the client machine and hands it a set of ``marching orders'' which, if
written in English, might be something like the following:

OK, {\bf File daemon}, it's time for your daily incremental backup. I want you
to get in touch with the Storage daemon on host archive.mysite.com and perform
the following save operations with the designated options. You'll note that
I've attached include and exclude lists and patterns you should apply when
backing up the file system. As this is an incremental backup, you should save
only files modified since the time you started your last backup which, as you
may recall, was 2000-11-19-06:43:38. Please let me know when you're done and
how it went. Thank you.

So, having been handed everything it needs to decide what to dump and where to
store it, the File daemon doesn't need to have any further contact with the
Director until the backup is complete providing there are no errors. If there
are errors, the error messages will be delivered immediately to the Director.
While the backup is proceeding, the File daemon will send the file coordinates
and data for each file being backed up to the Storage daemon, which will in
turn pass the file coordinates to the Director to put in the catalog.

During a {\bf Verify} of the catalog, the situation is different, since the
File daemon will have an exchange with the Director for each file, and will
not contact the Storage daemon.

A {\bf Restore} operation will be very similar to the {\bf Backup} except that
during the {\bf Restore} the Storage daemon will not send storage coordinates
to the Director since the Director presumably already has them. On the other
hand, any error messages from either the Storage daemon or File daemon will
normally be sent directly to the Directory (this, of course, depends on how
the Message resource is defined).

\section{Commands Received from the Director for a Backup}
\index{Backup!Commands Received from the Director for a }
\index{Commands Received from the Director for a Backup }
\addcontentsline{toc}{subsection}{Commands Received from the Director for a
Backup}

To be written ...

\section{Commands Received from the Director for a Restore}
\index{Commands Received from the Director for a Restore }
\index{Restore!Commands Received from the Director for a }
\addcontentsline{toc}{subsection}{Commands Received from the Director for a
Restore}

To be written ...
