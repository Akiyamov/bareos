
\chapter{ANSI and IBM Tape Labels}
\label{AnsiLabelsChapter}
\index[general]{ANSI and IBM Tape Labels} 
\index[general]{Labels!Tape}

Bacula supports ANSI or IBM tape labels as long as you
enable it.  In fact, with the proper configuration, you can
force Bacula to require ANSI or IBM labels.

Bacula can create an ANSI or IBM label, but if Check Labels is
enabled (see below), Bacula will look for an existing label, and
if it is found, it will keep the label. Consequently, you
can label the tapes with programs other than Bacula, and Bacula
will recognize and support them.

Even though Bacula will recognize and write ANSI and IBM labels, 
it always writes its own tape labels as well.

When using ANSI or IBM tape labeling, you must restrict your Volume
names to a maximum of six characters.  

If you have labeled your Volumes outside of Bacula, then the
ANSI/IBM label will be recognized by Bacula only if you have created
the HDR1 label with {\bf BACULA.DATA} in the Filename field (starting
with character 5).  If Bacula writes the labels, it will use
this information to recognize the tape as a Bacula tape.  This allows
ANSI/IBM labeled tapes to be used at sites with multiple machines
and multiple backup programs.


\section{Director Pool Directive}

\begin{description}
\item [ Label Type = ANSI | IBM | Bacula]  
  This directive is implemented in the Director Pool resource and in the SD Device
  resource.  If it is specified in the SD Device resource, it will take
  precedence over the value passed from the Director to the SD. The default
  is Label Type = Bacula.
\end{description}

\section{Storage Daemon Device Directives}

\begin{description}
\item [ Label Type = ANSI | IBM | Bacula]  
  This directive is implemented in the Director Pool resource and in the SD Device
  resource.  If it is specified in the the SD Device resource, it will take
  precedence over the value passed from the Director to the SD.
 
\item [Check Labels = yes | no]
  This directive is implemented in the the SD Device resource.  If you intend
  to read ANSI or IBM labels, this *must* be set.  Even if the volume is
  not ANSI labeled, you can set this to yes, and Bacula will check the
  label type. Without this directive set to yes, Bacula will assume that
  labels are of Bacula type and will not check for ANSI or IBM labels.
  In other words, if there is a possibility of Bacula encountering an
  ANSI/IBM label, you must set this to yes.
\end{description}
