%%
%%

\chapter{Bacula Memory Management}
\label{_ChapterStart7}
\index{Management!Bacula Memory}
\index{Bacula Memory Management}
\addcontentsline{toc}{section}{Bacula Memory Management}

\section{General}
\index{General}
\addcontentsline{toc}{subsection}{General}

This document describes the memory management routines that are used in Bacula
and is meant to be a technical discussion for developers rather than part of
the user manual. 

Since Bacula may be called upon to handle filenames of varying and more or
less arbitrary length, special attention needs to be used in the code to
ensure that memory buffers are sufficiently large. There are four
possibilities for memory usage within {\bf Bacula}. Each will be described in
turn. They are: 

\begin{itemize}
\item Statically allocated memory. 
\item Dynamically allocated memory using malloc() and free(). 
\item Non-pooled memory. 
\item Pooled memory. 
   \end{itemize}

\subsection{Statically Allocated Memory}
\index{Statically Allocated Memory}
\index{Memory!Statically Allocated}
\addcontentsline{toc}{subsubsection}{Statically Allocated Memory}

Statically allocated memory is of the form: 

\footnotesize
\begin{verbatim}
char buffer[MAXSTRING];
\end{verbatim}
\normalsize

The use of this kind of memory is discouraged except when you are 100\% sure
that the strings to be used will be of a fixed length. One example of where
this is appropriate is for {\bf Bacula} resource names, which are currently
limited to 127 characters (MAX\_NAME\_LENGTH). Although this maximum size may
change, particularly to accommodate Unicode, it will remain a relatively small
value. 

\subsection{Dynamically Allocated Memory}
\index{Dynamically Allocated Memory}
\index{Memory!Dynamically Allocated}
\addcontentsline{toc}{subsubsection}{Dynamically Allocated Memory}

Dynamically allocated memory is obtained using the standard malloc() routines.
As in: 

\footnotesize
\begin{verbatim}
char *buf;
buf = malloc(256);
\end{verbatim}
\normalsize

This kind of memory can be released with: 

\footnotesize
\begin{verbatim}
free(buf);
\end{verbatim}
\normalsize

It is recommended to use this kind of memory only when you are sure that you
know the memory size needed and the memory will be used for short periods of
time -- that is it would not be appropriate to use statically allocated
memory. An example might be to obtain a large memory buffer for reading and
writing files. When {\bf SmartAlloc} is enabled, the memory obtained by
malloc() will automatically be checked for buffer overwrite (overflow) during
the free() call, and all malloc'ed memory that is not released prior to
termination of the program will be reported as Orphaned memory. 

\subsection{Pooled and Non-pooled Memory}
\index{Memory!Pooled and Non-pooled}
\index{Pooled and Non-pooled Memory}
\addcontentsline{toc}{subsubsection}{Pooled and Non-pooled Memory}

In order to facility the handling of arbitrary length filenames and to
efficiently handle a high volume of dynamic memory usage, we have implemented
routines between the C code and the malloc routines. The first is called
``Pooled'' memory, and is memory, which once allocated and then released, is
not returned to the system memory pool, but rather retained in a Bacula memory
pool. The next request to acquire pooled memory will return any free memory
block. In addition, each memory block has its current size associated with the
block allowing for easy checking if the buffer is of sufficient size. This
kind of memory would normally be used in high volume situations (lots of
malloc()s and free()s) where the buffer length may have to frequently change
to adapt to varying filename lengths. 

The non-pooled memory is handled by routines similar to those used for pooled
memory, allowing for easy size checking. However, non-pooled memory is
returned to the system rather than being saved in the Bacula pool. This kind
of memory would normally be used in low volume situations (few malloc()s and
free()s), but where the size of the buffer might have to be adjusted
frequently. 

\paragraph*{Types of Memory Pool:}

Currently there are three memory pool types: 

\begin{itemize}
\item PM\_NOPOOL -- non-pooled memory. 
\item PM\_FNAME -- a filename pool. 
\item PM\_MESSAGE -- a message buffer pool. 
\item PM\_EMSG -- error message buffer pool. 
   \end{itemize}

\paragraph*{Getting Memory:}

To get memory, one uses: 

\footnotesize
\begin{verbatim}
void *get_pool_memory(pool);
\end{verbatim}
\normalsize

where {\bf pool} is one of the above mentioned pool names. The size of the
memory returned will be determined by the system to be most appropriate for
the application. 

If you wish non-pooled memory, you may alternatively call: 

\footnotesize
\begin{verbatim}
void *get_memory(size_t size);
\end{verbatim}
\normalsize

The buffer length will be set to the size specified, and it will be assigned
to the PM\_NOPOOL pool (no pooling). 

\paragraph*{Releasing Memory:}

To free memory acquired by either of the above two calls, use: 

\footnotesize
\begin{verbatim}
void free_pool_memory(void *buffer);
\end{verbatim}
\normalsize

where buffer is the memory buffer returned when the memory was acquired. If
the memory was originally allocated as type PM\_NOPOOL, it will be released to
the system, otherwise, it will be placed on the appropriate Bacula memory pool
free chain to be used in a subsequent call for memory from that pool. 

\paragraph*{Determining the Memory Size:}

To determine the memory buffer size, use: 

\footnotesize
\begin{verbatim}
size_t sizeof_pool_memory(void *buffer);
\end{verbatim}
\normalsize

\paragraph*{Resizing Pool Memory:}

To resize pool memory, use: 

\footnotesize
\begin{verbatim}
void *realloc_pool_memory(void *buffer);
\end{verbatim}
\normalsize

The buffer will be reallocated, and the contents of the original buffer will
be preserved, but the address of the buffer may change. 

\paragraph*{Automatic Size Adjustment:}

To have the system check and if necessary adjust the size of your pooled
memory buffer, use: 

\footnotesize
\begin{verbatim}
void *check_pool_memory_size(void *buffer, size_t new-size);
\end{verbatim}
\normalsize

where {\bf new-size} is the buffer length needed. Note, if the buffer is
already equal to or larger than {\bf new-size} no buffer size change will
occur. However, if a buffer size change is needed, the original contents of
the buffer will be preserved, but the buffer address may change. Many of the
low level Bacula subroutines expect to be passed a pool memory buffer and use
this call to ensure the buffer they use is sufficiently large. 

\paragraph*{Releasing All Pooled Memory:}

In order to avoid orphaned buffer error messages when terminating the program,
use: 

\footnotesize
\begin{verbatim}
void close_memory_pool();
\end{verbatim}
\normalsize

to free all unused memory retained in the Bacula memory pool. Note, any memory
not returned to the pool via free\_pool\_memory() will not be released by this
call. 

\paragraph*{Pooled Memory Statistics:}

For debugging purposes and performance tuning, the following call will print
the current memory pool statistics: 

\footnotesize
\begin{verbatim}
void print_memory_pool_stats();
\end{verbatim}
\normalsize

an example output is: 

\footnotesize
\begin{verbatim}
Pool  Maxsize  Maxused  Inuse
   0      256        0      0
   1      256        1      0
   2      256        1      0
\end{verbatim}
\normalsize
