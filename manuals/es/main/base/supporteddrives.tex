%%
%%

\chapter{Unidades de Tape Soportadas}
\label{DrivesSoportados}
\index{Unidades! de Tape Soportadas}
\index{Unidades de Tape Soportadas}

Bacula utiliza llamadas estándar al sistema operativo (read,write, ioctl) para interactuar
con las unidades de tape. Como consecuencia de esto, es necesario contar con el driver
de tape adecuado y desarrollado para el sistema operativo seleccionado. Bacula trabaja
perfectamente bien con unidades de tape SCSI en equipos FreeBSD, Linux, Solaris y
Windows, y puede funcionar en otras máquinas UNIX, sin embargo, esto no se ha probado
todavía. Recientemente, han surgido nuevos drives que utilizan interfaces IDE, ATAPI
o SATA, en vez de SCSI. En Linux, el drive de OnStream, que utiliza el driver OSST
es uno de estos ejemplos, es renocido que trabaja con Bacula. En adición, un gran
número de unidades de cinta (es decir, OS drivers) se ha visto que trabajan en sistemas
Windows. Sin embargo, las unidades de tape no-SCSI (diferentes a OnStream) que utilizan
drivers ide-scis, ide-tape o no-scsi, no funcionan de manera correcta con Bacula
(o cualquier otra aplicación que demande una unidad de tape) hasta hoy (abril 2007).
Si ha seleccionado un tape drive no-SCSI para utilizar con Bacula, hay una gran probabilidad
que la misma no funcione. El equipo de desarrollo de Bacula, está trabajando en conjunto
con los programadores del kernel para rectificar esta situación, sin embargo, esto
no se resolverá en un futuro cercano.

Incluso, si su drive aparece en el listado siguiente, por favor, chequee el capítulo
\ilink{Probando el Tape}{btape1} de este manual, para revisar los procedimientos
que se pueden utilizar para verificar si la unidad de cinta trabajará con Bacula.
Si su drive está en un modo de bloque fijo, este puede aparentar que trabaja con
Bacula, hasta que se intenta realizar una restauración y Bacula necesita posicionarse
en el tape. Solo se puede estar seguro, siguiendo los procedimientos sugeridos anteriormente
y probar.

Es muy difícil suministrar una lista de unidades de tape soportados, o drives conocidos
para trabajar con Bacula, debido al limitado feedback de los usuarios (por lo tanto, si 
usted utiliza Bacula en un drive diferente, por favor, permitanos conocer su experiencia). 
Basado en dicho feedback, los siguiente drives son conocidos que trabajan con
Bacula. Un signo menos \{''-''\} en la columna, significa desconocido.

\addcontentsline{lot}{table}{Unidades de Tape Soportadas}
\begin{longtable}{|p{2.0in}|l|l|p{2.5in}|l|}
 \hline 
\multicolumn{1}{|c| }{\bf SO } & \multicolumn{1}{c| }{\bf Man. } &
\multicolumn{1}{c| }{\bf Media } & \multicolumn{1}{c| }{\bf Modelo } &
\multicolumn{1}{c| }{\bf Capacidad } \\
 \hline {- } & {ADIC } & {DLT } & {Adic Scalar 100 DLT } & {100GB  } \\
 \hline {- } & {ADIC } & {DLT } & {Adic Fastor 22 DLT } & {-  } \\
 \hline {FreeBSD 5.4-RELEASE-p1 amd64 } & {Certance} & {LTO } & {AdicCertance CL400 LTO Ultrium 2 } & {200GB  } \\
 \hline {- } & {- } & {DDS } & {Compaq DDS 2,3,4 } & {-  } \\
 \hline {SuSE 8.1 Pro} & {Compaq} & {AIT } & {Compaq AIT 35 LVD } & {35/70GB } \\
 \hline {- } & {Exabyte } & {-  } & {Exabyte drives less than 10 years old } & {-  } \\
 \hline {- } & {Exabyte } & {-  } & {Exabyte VXA drives } & {-  } \\
 \hline {- } & {HP } & {Travan 4 } & {Colorado T4000S } & {-  } \\
 \hline {- } & {HP } & {DLT } & {HP DLT drives } & {-  } \\
 \hline {- } & {HP } & {LTO } & {HP LTO Ultrium drives } & {-  } \\
 \hline {- } & {IBM} & {??} & {3480, 3480XL, 3490, 3490E, 3580 and 3590 drives} & {-  } \\
 \hline {FreeBSD 4.10 RELEASE } & {HP } & {DAT } & {HP StorageWorks DAT72i } & {-  } \\
 \hline {- } & {Overland } & {LTO } & {LoaderXpress LTO } & {-  } \\
 \hline {- } & {Overland } & {- } & {Neo2000 } & {-  } \\
 \hline {- } & {OnStream } & {- } & {OnStream drives (see below) } & {-  } \\
 \hline {FreeBSD 4.11-Release} & {Quantum } & {SDLT } & {SDLT320 } & {160/320GB  } \\
 \hline {- } & {Quantum } & {DLT } & {DLT-8000 } & {40/80GB  } \\
 \hline {Linux } & {Seagate } & {DDS-4 } & {Scorpio 40 } & {20/40GB  } \\
 \hline {FreeBSD 4.9 STABLE } & {Seagate } & {DDS-4 } & {STA2401LW } & {20/40GB  } \\
 \hline {FreeBSD 5.2.1 pthreads patched RELEASE } & {Seagate } & {AIT-1 } & {STA1701W} & {35/70GB  } \\
 \hline {Linux } & {Sony } & {DDS-2,3,4 } & {- } & {4-40GB  } \\
 \hline {Linux } & {Tandberg } & {- } & {Tandbert MLR3 } & {-  } \\
 \hline {FreeBSD } & {Tandberg } & {- } & {Tandberg SLR6 } & {-  } \\
 \hline {Solaris } & {Tandberg } & {- } & {Tandberg SLR75 } & {- } \\
 \hline 

\end{longtable}

Hay una lista de \ilink{librerías de cintas con recambio automático soportadas}{Modelos},
en el capítulo de Autochangers Soportados de este documento, se pueden ubicar otras
unidades de tape que trabajan con Bacula.

\section{Unidades de Tape No Soportadas}
\label{DrivesNoSoportados}
\index[general]{Unidades de Tape No Soportadas}
\index[general]{Unidades! de Tape No Soportados }

Anteriormente, las unidades de tape OnStream IDE-SCSI no trabajaban con Bacula. Desde
la versión 1.33 y la versión de driver de kernel de osst 0.9.14 o superior, ahora
se puede trabajar. Por favor, revise el capítulo de pruebas para definir un tamaño
de bloque fijo.

Los tapes QIC son conocidos por tener un conjunto de particularidades (tamaño fijo
de bloque y una marca de EOF en vez de dos para terminar la cinta). Como consecuencia
de esto, se necesitan tomar algunas consideraciones para configurarlas y hacer que
las mismas trabajen correctamente con Bacula.

\section{Los usuarios FreeBSD deben tener precaución!!!}
\index[general]{Los usuarios FreeBSD deben tener precaución!!!}
\index[general]{Precaución!Los usuarios FreeBSD deben tener }

A menos que se haya aplicado el patch de las librerías pthreads en los sistemas FreeBSD
en versiones 4.11, se perderá data cuando Bacula se extienda sobre varios tapes.
Esto se debe a que las librerías que no cuentan con el patch, presentan errores para
retornar un status de advertencia a Bacula, cuando el fin de la cinta está próximo.
Este problema está corregido en los sistemas FreeBSD liberados después de la versión
4.11. Por favor, revise el capítulo \ilink{Probando el Tape}{FreeBSDTapes} de
este manual, para obtener información \textbf{importante} acerca de cómo configurar
su unidad de cinta, de manera que sea compatible con Bacula.

\section{Autochangers Soportados}
\index[general]{Autochangers!Soportados }
\index[general]{Autochangers Soportados}

Para mayor información acerca de los autochangers soportados, por favor vea la sección
\ilink{Autochangers conocidos que trabajan con Bacula}{Modelos}, de este manual.

\section{Especificaciones de Tape}
\index[general]{Especificaciones!deTape}
\index[general]{Especificaciones de Tape}

Si usted quiere conocer qué unidad de tape debe comprar para trabajar con Bacula,
realmente no se lo podemos decir. Sin embargo, podemos indicarle si usted va a comprar
un drive, debe evitar los drives DDS. La tecnología es más vieja y las unidades de
tape DDS necesitan limpieza frecuente. Los drives DLT, generalmente son mejores (tecnología
más nueva) y no requieren una limpieza tan frecuente.

A continuación, encontrará una tabla con las especificaciones de tapes DLT y LTO
que le darán una idea de la capacidad y velocidad de las cintas modernas. Las capacidades
listadas están en la capacidad nativa del tape, sin compresión. Todos los drives
modernos cuentan con compresión de hardware, y los fabricantes, generalmente, muestran
las capacidades utilizando una razón de compresión de 2:1. La razón de compresión
actual, en mayor parte, depende de la data a la cual se le hace respaldo, pero una
razón de 1.5:1 es un número mas razonable (es decir, multiplique el valor mostrado
por 1.5, para obtener un promedio bruto de lo que usted probablemente observará).
Las ratas de transferencia son redondeadas a los GB/hr mas cercanos. Todos los
valores son suministrados por varios fabricantes.

El tipo de media (Media Type), representa como está diseñado por el fabricante, y
usted no requiere utilizar (pero podría) el mismo nombre en sus recursos de configuración
de Bacula.

\begin{tabular}{|c|c|c|c}
Tipo de Media & Tipo de Drive & Capacidad de la Media & Rata de Transferencia \\ \hline
DDS-1              & DAT        & 2 GB &        ?? GB/hr   \\ \hline
DDS-2              & DAT        & 4 GB &        ?? GB/hr   \\ \hline
DDS-3              & DAT        & 12 GB &       5.4 GB/hr   \\ \hline
Travan 40          & Travan     & 20 GB &       ?? GB/hr    \\ \hline
DDS-4              & DAT        & 20 GB &       11 GB/hr    \\ \hline
VXA-1              & Exabyte    & 33 GB &       11 GB/hr    \\ \hline
DAT-72             & DAT        & 36 GB &       13 GB/hr    \\ \hline
DLT IV             & DLT8000    & 40 GB  &      22 GB/hr    \\ \hline
VXA-2              & Exabyte    & 80 GB &       22 GB/hr    \\ \hline
Half-high Ultrium 1 & LTO 1      & 100 GB &      27 GB/hr    \\ \hline
Ultrium 1          & LTO 1      & 100 GB &      54 GB/hr    \\ \hline
Super DLT 1        & SDLT 220   & 110 GB &      40 GB/hr    \\ \hline
VXA-3              & Exabyte    & 160 GB &      43 GB/hr    \\ \hline
Super DLT I        & SDLT 320   & 160 GB &      58 GB/hr    \\ \hline
Ultrium 2          & LTO 2      & 200 GB &      108 GB/hr   \\ \hline
Super DLT II       & SDLT 600   & 300 GB &      127 GB/hr   \\ \hline
VXA-4              & Exabyte    & 320 GB &      86 GB/hr    \\ \hline
Ultrium 3          & LTO 3      & 400 GB &      216 GB/hr   \\ \hline
\end{tabular}
