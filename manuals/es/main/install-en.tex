%%
%%

\chapter{Installing Bacula}
\label{InstallChapter}
\index[general]{Bacula!Installing}
\index[general]{Installing Bacula}

In general, you will need the Bacula source release, and if you want to run
a Windows client, you will need the Bacula Windows binary release.
However, Bacula needs certain third party packages (such as {\bf MySQL},
{\bf PostgreSQL}, or {\bf SQLite} to build and run
properly depending on the
options you specify.  Normally, {\bf MySQL} and {\bf PostgreSQL} are
packages that can be installed on your distribution.  However, if you do
not have them, to simplify your task, we have combined a number of these
packages into three {\bf depkgs} releases (Dependency Packages).  This can
vastly simplify your life by providing you with all the necessary packages
rather than requiring you to find them on the Web, load them, and install
them.

\section{Source Release Files}
\index[general]{Source Files}
\index[general]{Release Files}
 Beginning with Bacula 1.38.0, the source code has been broken into
 four separate tar files each corresponding to a different module in
 the Bacula SVN. The released files are:

\begin{description}
\item [bacula-3.0.3.tar.gz]
  This is the primary source code release for Bacula. On each
  release the version number (3.0.3) will be updated.

\item [bacula-docs-3.0.3.tar.gz]
  This file contains a copy of the docs directory with the
  documents prebuild. English HTML directory, single HTML
  file, and pdf file. The French and German translations
  are in progress, but are not built.

\item [bacula-gui-3.0.3.tar.gz]
  This file contains the non-core GUI programs. Currently,
  it contains bacula-web, a PHP program for producing management
  viewing of your Bacula job status in a browser; and bimagemgr
  a browser program for burning CDROM images with Bacula Volumes.

\item [bacula-rescue-3.0.3.tar.gz]
  This is the Bacula Rescue CDROM code. Note, the version number
  of this package is not tied to the Bacula release version, so
  it will be different.  Using this code, you can burn a CDROM
  with your system configuration and containing a statically
  linked version of the File daemon. This can permit you to easily
  repartition and reformat your hard disks and reload your
  system with Bacula in the case of a hard disk failure.
  Unfortunately this rescue disk does not properly boot for
  all Linux distributions. The problem is that the boot procedure
  can vary significantly between distributions, and even within
  a distribution, they are a moving target.

  This package evolves slower than the Bacula source code,
  so there may not always be a new release of the rescue package when
  making minor updates to the Bacula code. For example, when releasing
  Bacula version 3.0.3, the rescue package may still be at a prior
  version if there were no updates.

\item [winbacula-3.0.3.exe]
  This file is the 32 bit Windows installer for installing
  the Windows client (File daemon) on a Windows machine.
  This client will also run on 64 bit Windows machines.
  Beginning with Bacula version 1.39.20, this executable will
  also optionally load the Win32 Director and the Win32 
  Storage daemon.

\item [win64bacula-3.0.3.exe]
  This file is the 64 bit Windows installer for installing
  the Windows client (File daemon) on a Windows machine.
  This client will only run on 64 bit Windows OS machines.
  It will not run on 32 bit machines or 32 bit Windows OSes.
  The win64bacula release is necessary for Volume Shadow
  Copy (VSS) to work on Win64 OSes.  This installer
  installs only the FD, the Director and Storage daemon
  are not included.

\end{description}

\label{upgrading1}
\section{Upgrading Bacula}
\index[general]{Bacula!Upgrading}
\index[general]{Upgrading Bacula}
\index[general]{Upgrading}

If you are upgrading from one Bacula version to another, you should first
carefully read the ReleaseNotes of all major versions between your current
version and the version to which you are upgrading.  In many upgrades,
especially for minor patch upgrades (e.g. between 3.0.0 and 3.0.1) there
will be no database upgrade, and hence the process is rather simple.

With version 3.0.0 and later, you {\bf must} ensure that on any one
machine that all components of Bacula are running on exactly the
same version.  Prior to version 3.0.0, it was possible to run a 
lower level FD with a newer Director and SD.  This is no longer the
case.  

As always, we attempt to support older File daemons. This avoids the
need to do a simultaneous upgrade of many machines. For exactly what
older versions of the FD are supported, please see the ReleaseNotes 
for the new version.  In any case, you must always upgrade both the
Director and the Storage daemon at the same time, and you must also
upgrade any File daemon that is running on the same machine as a Director
or a Storage daemon (see the prior paragraph).

If the Bacula catalog
database has been upgraded (as it is almost every major release), you will
either need to reinitialize your database starting from scratch (not
normally a good idea), or save an ASCII copy of your database, then proceed
to upgrade it. If you are upgrading two major versions (e.g. 1.36 to 2.0)
then life will be more complicated because you must do two database
upgrades. See below for more on this.

Upgrading the catalog is normally done after Bacula is build and installed
by:

\begin{verbatim}
cd <installed-scripts-dir> (default /etc/bacula)
./update_bacula_tables
\end{verbatim}

This update script can also be find in the Bacula source src/cats
directory.

If there are several database upgrades between your version and the
version to which you are upgrading, you will need to apply each database
upgrade script. For your convenience, you can find all the old upgrade scripts
in the {\bf upgradedb} directory of the source code. You will need to edit the
scripts to correspond to your system configuration. The final upgrade script,
if any, can be applied as noted above.

If you are upgrading from one major version to another, you will need to
replace all your components at the same time as generally the inter-daemon
protocol will change. However, within any particular release (e.g. version
1.32.x) unless there is an oversight or bug, the daemon protocol will not
change. If this is confusing, simply read the ReleaseNotes very carefully as
they will note if all daemons must be upgraded at the same time. 

Finally, please note that in general it is not necessary or desirable
to do a {\bf make uninstall} before doing an upgrade providing you are careful
not to change the installation directories. In fact, if you do so, you will 
most likely delete all your conf files, which could be disastrous.
The normal procedure during an upgrade is simply:

\begin{verbatim}
./configure (your options)
make
make install
\end{verbatim}

In general none of your existing .conf or .sql files will be overwritten,
and you must do both the {\bf make} and {\bf make install}  commands, a
{\bf make install} without the preceding {\bf make} will not work.
  
For additional information on upgrading, please see the \ilink{Upgrading Bacula
Versions}{upgrading} in the Tips chapter of this manual.

\section{Releases Numbering}
\index[general]{Release Numbering}
\index[general]{Version Numbering}
Every Bacula release whether beta or production has a different number  
as well as the date of the release build. The numbering system follows
traditional Open Source conventions in that it is of the form.

\begin{verbatim}
major.minor.release
\end{verbatim}

For example:
\begin{verbatim}
1.38.11
\end{verbatim}

where each component (major, minor, patch) is a number.
The major number is currently 1 and normally does not change
very frequently.  The minor number starts at 0 and increases
each for each production release by 2 (i.e. it is always an
even number for a production release), and the patch number is
starts at zero each time the minor number changes.  The patch
number is increased each time a bug fix (or fixes) is released
to production.

So, as of this date (10 September 2006), the current production Bacula
release is version 1.38.11.  If there are bug fixes, the next release
will be 1.38.12 (i.e. the patch number has increased by one).

For all patch releases where the minor version number does not change,
the database and all the daemons will be compatible.  That means that
you can safely run a 1.38.0 Director with a 1.38.11 Client.  Of course,
in this case, the Director may have bugs that are not fixed. Generally,
within a minor release (some minor releases are not so minor), all
patch numbers are officially released to production. This means that while
the current Bacula version is 1.38.11, versions 1.38.0, 1.38.1, ... 1.38.10
have all been previously released.

When the minor number is odd, it indicates that the package is under 
development and thus may not be stable. For example, while the current 
production release of Bacula is currently 1.38.11, the current development
version is 1.39.22. All patch versions of the development code are 
available in the SVN (source repository).  However, not all patch versions
of the development code (odd minor version) are officially released. When
they are released, they are released as beta versions (see below for a
definition of what beta means for Bacula releases).                     

In general when the minor number increases from one production release
to the next (i.e. 1.38.x to 1.40.0), the catalog database must be upgraded,
the Director and Storage daemon must always be on the same minor release
number, and often (not always), the Clients must also be on the same minor
release. As often as possible, we attempt to make new releases that are
downwards compatible with prior clients, but this is not always possible.
You must check the release notes.  In general, you will have fewer problems
if you always run all the components on the same minor version number (i.e.
all either 1.38.x or 1.40.x but not mixed).


\label{BetaReleases}
\section*{Beta Releases}
\index[general]{Beta Releases}
Towards the end of the development cycle, which typically runs
one year from a major release to another, there will be several beta
releases of the development code prior to a production release.  
As noted above, beta versions always have odd minor version numbers
(e.g 1.37.x or 1.39.x). 
The purpose of the beta releases is to allow early adopter users to test
the new code.  Beta releases are made with the following considerations:

\begin{itemize}
\item The code passes the regression testing on FreeBSD, Linux, and Solaris
  machines.

\item There are no known major bugs, or on the rare occasion that 
  there are, they will be documented or already in the bugs database.

\item Some of the new code/features may not yet be tested.

\item Bugs are expected to be found, especially in the new
  code before the final production release.

\item The code will have been run in production in at least one small
  site (mine).

\item The Win32 client will have been run in production at least
  one night at that small site.

\item The documentation in the manual is unlikely to be complete especially
  for the new features, and the Release Notes may not be fully
  organized.

\item Beta code is not generally recommended for everyone, but
  rather for early adopters.
\end{itemize}


\label{Dependency}
\section{Dependency Packages}
\index[general]{Dependency Packages}
\index[general]{Packages!Dependency}

As discussed above, we have combined a number of third party packages that
Bacula might need into the {\bf depkgs} release. You can,
of course, get the latest packages from the original authors or 
from your operating system supplier. The locations of
where we obtained the packages are in the README file in each package.
However, be aware that the packages in the depkgs files have been tested by us
for compatibility with Bacula. 

Typically, a dependency package will be named {\bf depkgs-ddMMMyy.tar.gz}
where {\bf dd} is the day we release it, {\bf MMM}
is the abbreviated month (e.g. Jan), and {\bf yy} is the year. An actual
example is: {\bf depkgs-24Jul09.tar.gz}. To install and build this package (if
needed), you do the following: 

\begin{enumerate}
\item Create a {\bf bacula} directory, into which you will place  both the
   Bacula source as well as the dependency package.  
\item Detar the {\bf depkgs} into the {\bf bacula} directory.  
\item cd bacula/depkgs  
\item make 
\end{enumerate}

Although the exact composition of the dependency packages may change from time
to time, the current makeup is the following: 

\addcontentsline{lot}{table}{Dependency Packages}
\begin{longtable}{|l|l|l|}
 \hline 
\multicolumn{1}{|c| }{\bf 3rd Party Package} & \multicolumn{1}{c| }{\bf depkgs}
     & \multicolumn{1}{c| }{\bf depkgs-qt} \\
 \hline {SQLite3 } & \multicolumn{1}{c| }{X } & \multicolumn{1}{c| }{ }\\
 \hline {mtx } & \multicolumn{1}{c| }{X } & \multicolumn{1}{c| }{ } \\
 \hline {qt4 } & \multicolumn{1}{c| }{ } & \multicolumn{1}{c| }{X } \\
 \hline 
\end{longtable}

Note, some of these packages are quite large, so that building them can be a
bit time consuming. The above instructions will build all the packages
contained in the directory. However, when building Bacula, it will take only
those pieces that it actually needs. 

Alternatively, you can make just the packages that are needed. For example, 

\footnotesize
\begin{verbatim}
cd bacula/depkgs
make sqlite
\end{verbatim}
\normalsize

will configure and build only the SQLite package. 

You should build the packages that you will require in {\bf depkgs} a     
prior to configuring and building Bacula, since Bacula will need
them during the build process. 

For more information on the {\bf depkgs-qt} package, please read the
INSTALL file in the main directory of that package. If you are going to 
build Qt4 using {\bf depkgs-qt}, you must source the {\bf qt4-paths} file
included in the package prior to building Bacula. Please read the INSTALL
file for more details.

Even if you do not use SQLite, you might find it worthwhile to build {\bf mtx}
because the {\bf tapeinfo} program that comes with it can often provide you
with valuable information about your SCSI tape drive (e.g. compression,
min/max block sizes, ...). Note, most distros provide {\bf mtx} as part of 
their release.

The {\bf depkgs1} package is depreciated and previously contained
readline, which should be available on all operating systems.

The {\bf depkgs-win32} package is deprecated and no longer used in 
Bacula version 1.39.x and later. It was previously used to build
the native Win32 client program, but this program is now built on Linux
systems using cross-compiling.  All the tools and third party libraries
are automatically downloaded by executing the appropriate scripts.  See
src/win32/README.mingw32 for more details.

\section{Supported Operating Systems}
\label{Systems}
\index[general]{Systems!Supported Operating}
\index[general]{Supported Operating Systems}

Please see the 
\ilink{ Supported Operating Systems}{SupportedOSes} section
of the QuickStart chapter of this manual. 

\section{Building Bacula from Source}
\label{Building}
\index[general]{Source!Building Bacula from}
\index[general]{Building Bacula from Source}

The basic installation is rather simple. 

\begin{enumerate}
\item Install and build any {\bf depkgs} as noted above. This
   should be unnecessary on most modern Operating Systems.

\item Configure and install MySQL or PostgreSQL (if desired). 
   \ilink{Installing and Configuring MySQL Phase I}{MySqlChapter} or  
   \ilink{Installing and Configuring PostgreSQL Phase
   I}{PostgreSqlChapter}.  If you are installing from rpms, and are
   using MySQL, please be sure to install  {\bf mysql-devel}, so that the MySQL
   header files are available  while compiling Bacula. In addition, the MySQL
   client  library {\bf mysqlclient} requires the gzip compression library  {\bf
   libz.a} or {\bf libz.so}. If you are using rpm packages,  these libraries are
   in the {\bf libz-devel} package. On Debian  systems, you will need to load the
   {\bf zlib1g-dev} package. If  you are not using rpms or debs, you will need to
   find the  appropriate package for your system.  

   Note, if you already have a running MySQL or PostgreSQL on your system, you 
   can skip this phase provided that you have built the thread  safe libraries.
   And you have already installed the additional  rpms noted above.  

   SQLite is not supported on Solaris. This is because it
   frequently fails with bus errors.  However SQLite3 may work.

\item Detar the Bacula source code preferably into the {\bf bacula}  directory
   discussed above.  

\item {\bf cd} to the directory containing the source code.  

\item ./configure (with appropriate options as described below). Any
   path names you specify as options on the ./configure command line
   must be absolute paths and not relative.

\item Check the output of ./configure very carefully, especially  the Install
   binaries and Install config directories.  If they are not correct,
   please rerun ./configure until they  are. The output from ./configure is
   stored in {\bf config.out}  and can be re-displayed at any time without
   rerunning the  ./configure by doing {\bf cat config.out}.  

\item If after running ./configure once, you decide to change options  and
   re-run it, that is perfectly fine, but before re-running it,  you should run: 

\footnotesize
\begin{verbatim}
      make distclean
\end{verbatim}
\normalsize

so that you are sure to start from scratch and not have a  mixture of the two
options. This is because ./configure  caches much of the information. The {\bf
make distclean}  is also critical if you move the source directory from one 
machine to another. If the {\bf make distclean} fails,  just ignore it and
continue on.  

\item make  
   If you get errors while linking in the Storage daemon directory
   (src/stored), it is probably because you have not loaded the static
   libraries on your system.  I noticed this problem on a Solaris system.
   To correct it, make sure that you have not added {\bf
   {-} {-}enable-static-tools} to the {\bf ./configure} command.

   If you skip this step ({\bf make}) and proceed immediately to the {\bf
   make install} you are making two serious errors: 1.  your install will
   fail because Bacula requires a {\bf make} before a {\bf make install}.
   2.  you are depriving yourself of the chance to make sure there are no
   errors before beginning to write files to your system directories.
                                 

\item make install  
   Please be sure you have done a {\bf make} before entering this command,
   and that everything has properly compiled and linked without errors.


\item If you are new to Bacula, we {\bf strongly} recommend that you skip
   the next step and use the default configuration files, then run the
   example program in the next chapter, then come back and modify your
   configuration files to suit your particular needs.

\item Customize the configuration files for each of the three daemons 
   (Directory, File, Storage) and for the Console program.  For the details
   of how to do this, please see \ilink{Setting Up Bacula Configuration
   Files}{ConfigureChapter} in the Configuration chapter of this manual.  We
   recommend that you start by modifying the default configuration files
   supplied, making the minimum changes necessary.  Complete customization
   can be done after you have Bacula up and running.  Please take care when
   modifying passwords, which were randomly generated, and the {\bf Name}s
   as the passwords and names must agree between the configuration files
   for security reasons.  

\label{CreateDatabase}
\item Create the Bacula MySQL database and tables
   (if using MySQL)
      \ilink{Installing and Configuring MySQL Phase II}{mysql_phase2} or 
      create the Bacula PostgreSQL database and tables  
   \ilink{Configuring PostgreSQL
   II}{PostgreSQL_configure} or alternatively  if you are using
   SQLite \ilink{Installing and Configuring SQLite Phase II}{phase2}.  

\item Start Bacula ({\bf ./bacula start}) Note. the next chapter  shows you
   how to do this in detail.  

\item Interface with Bacula using the Console program  

\item For the previous two items, please follow the instructions  in the 
   \ilink{Running Bacula}{TutorialChapter} chapter of  this manual,
   where you will run a simple backup and do a  restore. Do this before you make
   heavy modifications to the  configuration files so that you are sure that
   Bacula works  and are familiar with it. After that changing the conf files 
   will be easier.  

\item If after installing Bacula, you decide to "move it", that is  to
   install it in a different set of directories, proceed  as follows:  

\footnotesize
\begin{verbatim}
      make uninstall
      make distclean
      ./configure (your-new-options)
      make
      make install
      
\end{verbatim}
\normalsize

\end{enumerate}

If all goes well, the {\bf ./configure} will correctly determine which
operating system you are running and configure the source code appropriately.
Currently, FreeBSD, Linux (Red Hat), and Solaris are supported. The Bacula
client (File daemon) is reported to work with MacOS X 10.3 is if 
readline support is not enabled (default) when building the client.       

If you install Bacula on more than one system, and they are identical, you can
simply transfer the source tree to that other system and do a "make
install". However, if there are differences in the libraries or OS versions,
or you wish to install on a different OS, you should start from the original
compress tar file. If you do transfer the source tree, and you have previously
done a ./configure command, you MUST do: 

\footnotesize
\begin{verbatim}
make distclean
\end{verbatim}
\normalsize

prior to doing your new ./configure. This is because the GNU autoconf tools
cache the configuration, and if you re-use a configuration for a Linux machine
on a Solaris, you can be sure your build will fail. To avoid this, as
mentioned above, either start from the tar file, or do a "make distclean". 

In general, you will probably want to supply a more complicated {\bf
configure} statement to ensure that the modules you want are built and that
everything is placed into the correct directories. 

For example, on Fedora, Red Hat, or SuSE one could use the following: 

\footnotesize
\begin{verbatim}
CFLAGS="-g -Wall" \
  ./configure \
    --sbindir=$HOME/bacula/bin \
    --sysconfdir=$HOME/bacula/bin \
    --with-pid-dir=$HOME/bacula/bin/working \
    --with-subsys-dir=$HOME/bacula/bin/working \
    --with-mysql \
    --with-working-dir=$HOME/bacula/bin/working \
    --with-dump-email=$USER
\end{verbatim}
\normalsize

The advantage of using the above configuration to start is that
everything will be put into a single directory, which you can later delete
once you have run the examples in the next chapter and learned how Bacula
works. In addition, the above can be installed and run as non-root. 

For the developer's convenience, I have added a {\bf defaultconfig} script to
the {\bf examples} directory. This script contains the statements that you
would normally use, and each developer/user may modify them to suit his needs.
You should find additional useful examples in this directory as well. 

The {\bf \verb:--:enable-conio} or {\bf \verb:--:enable-readline} options are
useful because they provide a command line history, editing capability for the
Console program and tab completion on various option. If you have included
either option in the build, either the {\bf termcap} or the {\bf ncurses}
package will be needed to link. On most systems, including Red Hat and SuSE,
you should include the ncurses package.  If Bacula's configure process finds
the ncurses libraries, it will use those rather than the termcap library.  On
some systems, such as SuSE, the termcap library is not in the standard library
directory.  As a consequence, the option may be disabled or you may get an
error message such as:

\footnotesize
\begin{verbatim}
/usr/lib/gcc-lib/i586-suse-linux/3.3.1/.../ld:
cannot find -ltermcap
collect2: ld returned 1 exit status
\end{verbatim}
\normalsize

while building the Bacula Console. In that case, you will need to set the {\bf
LDFLAGS} environment variable prior to building. 

\footnotesize
\begin{verbatim}
export LDFLAGS="-L/usr/lib/termcap"
\end{verbatim}
\normalsize

The same library requirements apply if you wish to use the readline subroutines
for command line editing, history and tab completion or if you are using a
MySQL library that requires encryption. If you need encryption, you can either
export the appropriate additional library options as shown above or,
alternatively, you can include them directly on the ./configure line as in:

\footnotesize
\begin{verbatim}
LDFLAGS="-lssl -lcyrpto" \
   ./configure <your-options>
\end{verbatim}
\normalsize

On some systems such as Mandriva, readline tends to
gobble up prompts, which makes it totally useless. If this happens to you, use
the disable option, or if you are using version 1.33 and above try using {\bf
\verb:--:enable-conio} to use a built-in readline replacement. You will still need
either the termcap or the ncurses library, but it is unlikely that the {\bf conio}
package will gobble up prompts. 

readline is no longer supported after version 1.34.  The code within Bacula
remains, so it should be usable, and if users submit patches for it, we will
be happy to apply them.  However, due to the fact that each version of
readline seems to be incompatible with previous versions, and that there
are significant differences between systems, we can no longer afford to
support it.

\section{What Database to Use?}
\label{DB}
\index[general]{What Database to Use?}
\index[general]{Use!What Database to}

Before building Bacula you need to decide if you want to use SQLite, MySQL, or
PostgreSQL. If you are not already running MySQL or PostgreSQL, you might
want to start by testing with SQLite (not supported on Solaris).
This will greatly simplify the setup for you
because SQLite is compiled into Bacula an requires no administration. It
performs well and is suitable for small to medium sized installations (maximum
10-20 machines). However, we should note that a number of users have
had unexplained database corruption with SQLite. For that reason, we
recommend that you install either MySQL or PostgreSQL for production
work.

If you wish to use MySQL as the Bacula catalog, please see the 
\ilink{Installing and Configuring MySQL}{MySqlChapter} chapter of
this manual. You will need to install MySQL prior to continuing with the
configuration of Bacula. MySQL is a high quality database that is very
efficient and is suitable for any sized installation. It is slightly more
complicated than SQLite to setup and administer because it has a number of
sophisticated features such as userids and passwords. It runs as a separate
process, is truly professional and can manage a database of any size. 

If you wish to use PostgreSQL as the Bacula catalog, please see the 
\ilink{Installing and Configuring PostgreSQL}{PostgreSqlChapter}
chapter of this manual. You will need to install PostgreSQL prior to
continuing with the configuration of Bacula. PostgreSQL is very similar to
MySQL, though it tends to be slightly more SQL92 compliant and has many more
advanced features such as transactions, stored procedures, and the such. It
requires a certain knowledge to install and maintain. 

If you wish to use SQLite as the Bacula catalog, please see 
\ilink{Installing and Configuring SQLite}{SqlLiteChapter} chapter of
this manual. SQLite is not supported on Solaris.

\section{Quick Start}
\index[general]{Quick Start}
\index[general]{Start!Quick}

There are a number of options and important considerations given below
that you can skip for the moment if you have not had any problems building
Bacula with a simplified configuration as shown above. 
      
If the ./configure process is unable to find specific libraries (e.g.    
libintl, you should ensure that the appropriate package is installed on
your system. Alternatively, if the package is installed in a non-standard
location (as far as Bacula is concerned), then there is generally an
option listed below (or listed with "./configure {-} {-}help" that will
permit you to specify the directory that should be searched. In other
cases, there are options that will permit you to disable to feature 
(e.g. {-} {-}disable-nls).

If you want to dive right into it, we recommend you skip to the next chapter,
and run the example program. It will teach you a lot about Bacula and as an
example can be installed into a single directory (for easy removal) and run as
non-root. If you have any problems or when you want to do a real installation,
come back to this chapter and read the details presented below. 

\section{Configure Options}
\label{Options}
\index[general]{Options!Configure}
\index[general]{Configure Options}

The following command line options are available for {\bf configure} to
customize your installation. 

\begin{description}
\item [ \--prefix=\lt{}patch\gt{}]
   \index[general]{{-}prefix}
   This option is meant to allow you to direct where the architecture
   independent files should be placed.  However, we find this a somewhat
   vague concept, and so we have not implemented this option other than
   what ./configure does by default.  As a consequence, we suggest that
   you avoid it. We have provided options that allow you to explicitly
   specify the directories for each of the major categories of installation
   files.
\item [ {-}{\-}sbindir=\lt{}binary-path\gt{}]
   \index[general]{{-}{\-}sbindir}
   Defines where the Bacula  binary (executable) files will be placed during a
   {\bf make  install} command.  

\item [ {-}{\-}sysconfdir=\lt{}config-path\gt{}]
   \index[general]{{-}{\-}sysconfdir}
   Defines where the Bacula configuration files should be placed during a
   {\bf make install} command.

\item [ {-}{\-}mandir=\lt{}path\gt{}]
   \index[general]{{-}{\-}mandir}
   Note, as of Bacula version 1.39.14, the meaning of any path
   specified on this option is change from prior versions. It
   now specifies the top level man directory.
   Previously the mandir specified the full path to where you
   wanted the man files installed.
   The man files will be installed in gzip'ed format under
   mandir/man1 and mandir/man8 as appropriate.
   For the install to succeed you must have {\bf gzip} installed
   on your system.

   By default, Bacula will install the Unix man pages in 
   /usr/share/man/man1 and /usr/share/man/man8.  
   If you wish the man page to be installed in
   a different location, use this option to specify the path.
   Note, the main HTML and PDF Bacula documents are in a separate
   tar file that is not part of the source distribution.

\item [ {-}{\-}datadir=\lt{}path\gt{} ]
   \index[general]{{-}{\-}datadir}
   If you translate Bacula or parts of Bacula into a different language
   you may specify the location of the po files using the {\bf
   {-}{\-}datadir} option. You must manually install any po files as
   Bacula does not (yet) automatically do so.

\item [ {-}{\-}disable-ipv6 ]
   \index[general]{{-}{\-}disable-ipv6}

\item [ {-}{\-}enable-smartalloc ]
   \index[general]{{-}{\-}enable-smartalloc}
   This enables the inclusion of the Smartalloc orphaned buffer detection
   code.  This option is highly recommended.  Because we never build
   without this option, you may experience problems if it is not enabled.
   In this case, simply re-enable the option.  We strongly recommend
   keeping this option enabled as it helps detect memory leaks.  This
   configuration parameter is used while building Bacula

\item [ {-}{\-}enable-bat ]
   \label{enablebat}
   \index[general]{{-}{\-}enable-bat}
   If you have Qt4 >= 4.3.4 installed on your computer including the
   libqt4 and libqt4-devel (libqt4-dev on Debian) libraries, and you want
   to use the Bacula Administration Tool (bat) GUI Console interface to
   Bacula, you must specify this option.  Doing so will build everything in
   the {\bf src/qt-console} directory.  The build with enable-bat will work
   only with a full Bacula build (i.e. it will not work with a client-only
   build). 

   Qt4 is available on OpenSUSE 10.2, CentOS 5, Fedora, and Debian. If it
   is not available on your system, you can download the {\bf depkgs-qt}
   package from the Bacula Source Forge download area and build it.
   See the
   INSTALL file in that package for more details. In particular to use
   the Qt4 built by {\bf depkgs-qt} you {\bf must} source the file
   {\bf qt4-paths}.

\item [ {-}{\-}enable-batch-insert ]
   \index[general]{{-}{\-}enable-batch-insert}
   This option enables batch inserts of the attribute records (default) in
    the catalog database, which is much faster (10 times or more) than
   without this option for large numbers of files. However, this option
   will automatically be disabled if your SQL libraries are not
   thread safe. If you find that batch mode is not enabled on your Bacula
   installation, then your database most likely does not support threads.

   SQLite2 is not thread safe.  Batch insert cannot be enabled when using
   SQLite2

   On most systems, MySQL, PostgreSQL and SQLite3 are thread safe.

   To verify that your PostgreSQL is thread safe, you can try this
   (change the path to point to your particular installed libpq.a;
   these commands were issued on FreeBSD 6.2):

\begin{verbatim}
$ nm /usr/local/lib/libpq.a | grep PQputCopyData
00001b08 T PQputCopyData
$ nm /usr/local/lib/libpq.a | grep mutex
         U pthread_mutex_lock
         U pthread_mutex_unlock
         U pthread_mutex_init
         U pthread_mutex_lock
         U pthread_mutex_unlock
\end{verbatim}

   The above example shows a libpq that contains the required function
   PQputCopyData and is thread enabled (i.e. the pthread\_mutex* entries).
   If you do not see PQputCopyData, your version of PostgreSQL is too old
   to allow batch insert.  If you do not see the mutex entries, then thread
   support has not been enabled. Our tests indicate you usually need to
   change the configuration options and recompile/reinstall the PostgreSQL
   client software to get thread support.

   Bacula always links to the thread safe MySQL libraries.

   Running with Batch Insert turned on is recommended because it can
   significantly improve attribute insertion times. However, it does 
   put a significantly larger part of the work on your SQL engine, so
   you may need to pay more attention to tuning it. In particular,   
   Batch Insert can require large temporary table space, and consequently,
   the default location (often /tmp) may run out of space causing errors.
   For MySQL, the location is set in my.conf with "tmpdir".  You may also
   want to increase the memory available to your SQL engine to further
   improve performance during Batch Inserts.

\item [ {-}{\-}enable-bwx-console ]
   \index[general]{{-}{\-}enable-bwx-console}
   If you have wxWidgets installed on your computer and you want to use the
   wxWidgets GUI Console interface to Bacula, you must specify this option.
   Doing so will build everything in the {\bf src/wx-console} directory.
   This could also be useful to users who want a GUI Console and don't want
   to install QT, as wxWidgets can work with GTK+, Motif or even X11
   libraries.

\item [ {-}{\-}enable-tray-monitor ]
   \index[general]{{-}{\-}enable-tray-monitor}
   If you have GTK installed on your computer, you run a graphical
   environment or a window manager compatible with the FreeDesktop system
   tray standard (like KDE and GNOME) and you want to use a GUI to monitor
   Bacula daemons, you must specify this option.  Doing so will build
   everything in the {\bf src/tray-monitor} directory. Note, due to 
   restrictions on what can be linked with GPLed code, we were forced to
   remove the egg code that dealt with the tray icons and replace it by
   calls to the GTK+ API, and unfortunately, the tray icon API necessary
   was not implemented until GTK version 2.10 or later.

\item [ {-}{\-}enable-static-tools]
   \index[general]{{-}{\-}enable-static-tools}
   This option causes the linker to link the Storage daemon utility tools
   ({\bf bls}, {\bf bextract}, and {\bf bscan}) statically.  This permits
   using them without having the shared libraries loaded.  If you have
   problems linking in the {\bf src/stored} directory, make sure you have
   not enabled this option, or explicitly disable static linking by adding
   {\bf \verb:--:disable-static-tools}.

\item [ {-}{\-}enable-static-fd]
   \index[general]{{-}{\-}enable-static-fd}
   This option causes the make process to build a {\bf static-bacula-fd} in
   addition to the standard File daemon.  This static version will include
   statically linked libraries and is required for the Bare Metal recovery.
   This option is largely superseded by using {\bf make static-bacula-fd}
   from with in the {\bf src/filed} directory.  Also, the {\bf
   \verb:--:enable-client-only} option described below is useful for just
   building a client so that all the other parts of the program are not
   compiled.   
     
   When linking a static binary, the linker needs the static versions
   of all the libraries that are used, so frequently users will 
   experience linking errors when this option is used. The first 
   thing to do is to make sure you have the static glibc library 
   installed on your system. The second thing to do is the make sure
   you do not specify {\bf {-}{\-}openssl} or {\bf {-}{\-}with-python}
   on your ./configure statement as these options require additional
   libraries. You may be able to enable those options, but you will
   need to load additional static libraries.


\item [ {-}{\-}enable-static-sd]
   \index[general]{{-}{\-}enable-static-sd}
   This option causes the make process to build a {\bf static-bacula-sd} in
   addition to the standard Storage daemon.  This static version will
   include statically linked libraries and could be useful during a Bare
   Metal recovery.

   When linking a static binary, the linker needs the static versions
   of all the libraries that are used, so frequently users will 
   experience linking errors when this option is used. The first 
   thing to do is to make sure you have the static glibc library 
   installed on your system. The second thing to do is the make sure
   you do not specify {\bf {-}{\-}openssl} or {\bf {-}{\-}with-python}
   on your ./configure statement as these options require additional
   libraries. You may be able to enable those options, but you will
   need to load additional static libraries.


\item [ {-}{\-}enable-static-dir]
   \index[general]{{-}{\-}enable-static-dir}
   This option causes the make process to build a {\bf static-bacula-dir}
   in addition to the standard Director.  This static version will include
   statically linked libraries and could be useful during a Bare Metal
   recovery.

   When linking a static binary, the linker needs the static versions
   of all the libraries that are used, so frequently users will 
   experience linking errors when this option is used. The first 
   thing to do is to make sure you have the static glibc library 
   installed on your system. The second thing to do is the make sure
   you do not specify {\bf {-}{\-}openssl} or {\bf {-}{\-}with-python}
   on your ./configure statement as these options require additional
   libraries. You may be able to enable those options, but you will
   need to load additional static libraries.


\item [ {-}{\-}enable-static-cons]
   \index[general]{{-}{\-}enable-static-cons}
   This option causes the make process to build a {\bf static-console} in
   addition to the standard console.  This static version will include
   statically linked libraries and could be useful during a Bare Metal
   recovery.

   When linking a static binary, the linker needs the static versions
   of all the libraries that are used, so frequently users will 
   experience linking errors when this option is used. The first 
   thing to do is to make sure you have the static glibc library 
   installed on your system. The second thing to do is the make sure
   you do not specify {\bf {-}{\-}openssl} or {\bf {-}{\-}with-python}
   on your ./configure statement as these options require additional
   libraries. You may be able to enable those options, but you will
   need to load additional static libraries.


\item [ {-}{\-}enable-client-only]
   \index[general]{{-}{\-}enable-client-only}
   This option causes the make process to build only the File daemon and
   the libraries that it needs.  None of the other daemons, storage tools,
   nor the console will be built.  Likewise a {\bf make install} will then
   only install the File daemon.  To cause all daemons to be built, you
   will need to do a configuration without this option.  This option
   greatly facilitates building a Client on a client only machine.

   When linking a static binary, the linker needs the static versions
   of all the libraries that are used, so frequently users will 
   experience linking errors when this option is used. The first 
   thing to do is to make sure you have the static glibc library 
   installed on your system. The second thing to do is the make sure
   you do not specify {\bf {-}{\-}openssl} or {\bf {-}{\-}with-python}
   on your ./configure statement as these options require additional
   libraries. You may be able to enable those options, but you will
   need to load additional static libraries.

\item [ {-}{\-}enable-build-dird]
   \index[general]{{-}{\-}enable-build-dird}
   This option causes the make process to build the Director and the
   Director's tools. By default, this option is on, but you may turn
   it off by using {\bf {-}{\-}disable-build-dird} to prevent the
   Director from being built.

\item [ {-}{\-}enable-build-stored]
   \index[general]{{-}{\-}enable-build-stored}
   This option causes the make process to build the Storage daemon.
   By default, this option is on, but you may turn
   it off by using {\bf {-}{\-}disable-build-stored} to prevent the
   Storage daemon from being built.


\item [ {-}{\-}enable-largefile]
   \index[general]{{-}{\-}enable-largefile}
   This option (default) causes  Bacula to be built with 64 bit file address
   support if it  is available on your system. This permits Bacula to read and 
   write files greater than 2 GBytes in size. You may disable this  feature and
   revert to 32 bit file addresses by using  {\bf \verb:--:disable-largefile}.  

\item [ {-}{\-}disable-nls]
   \index[general]{{-}{\-}disable-nls}
   By default, Bacula uses the GNU Native Language Support (NLS) libraries. On
   some machines, these libraries may not be present or may not function 
   correctly (especially on non-Linux implementations). In such cases, you
   may specify {\bf {-}{\-}disable-nls} to disable use of those libraries.
   In such a case, Bacula will revert to using English.

\item [ {-}{\-}disable-ipv6 ]
   \index[general]{{-}{\-}disable-ipv6}
   By default, Bacula enables IPv6 protocol. On some systems, the files
   for IPv6 may exist, but the functionality could be turned off in the
   kernel. In that case, in order to correctly build Bacula, you will
   explicitly need to use this option so that Bacula does not attempt
   to reference OS function calls that do not exist.

\item [ {-}{\-}with-sqlite3=\lt{}sqlite3-path\gt{}]
   \index[general]{{-}{\-}with-sqlite3}
   This enables use of the SQLite version 3.x database.  The {\bf
   sqlite3-path} is not normally specified as Bacula looks for the
   necessary components in a standard location ({\bf depkgs/sqlite3}).  See
   \ilink{Installing and Configuring SQLite}{SqlLiteChapter} chapter of
   this manual for more details. SQLite3 is not supported on Solaris.

\item [ {-}{\-}with-mysql=\lt{}mysql-path\gt{}]
   \index[general]{{-}{\-}with-mysql}
   This enables building of the Catalog services for Bacula.  It assumes
   that MySQL is running on your system, and expects it to be installed in
   the {\bf mysql-path} that you specify.  Normally, if MySQL is installed
   in a standard system location, you can simply use {\bf {-}{\-}with-mysql}
   with no path specification.  If you do use this option, please proceed
   to installing MySQL in the \ilink{Installing and Configuring
   MySQL}{MySqlChapter} chapter before proceeding with the configuration.

   See the note below under the {-}{\-}with-postgresql item.

\item [ {-}{\-}with-postgresql=\lt{}path\gt{}]
   \index[general]{{-}{\-}with-postgresql}
   This provides an explicit path to the PostgreSQL libraries if Bacula
   cannot find it by default.  Normally to build with PostgreSQL, you would
   simply use {\bf {-}{\-}with-postgresql}.

   Note, for Bacula to be configured properly, you must specify one
   of the four database options supported.  That is:
   {-}{\-}with-sqlite, {-}{\-}with-sqlite3, {-}{\-}with-mysql, or
   {-}{\-}with-postgresql, otherwise the ./configure will fail.

\item [ {-}{\-}with-openssl=\lt{}path\gt{}]
   This configuration option is necessary if you want to enable TLS (ssl),
   which encrypts the communications within       
   Bacula or if you want to use File Daemon PKI data encryption.
   Normally, the {\bf path} specification is not necessary since
   the configuration searches for the OpenSSL libraries in standard system
   locations. Enabling OpenSSL in Bacula permits secure communications
   between the daemons and/or data encryption in the File daemon.
   For more information on using TLS, please see the
   \ilink{Bacula TLS -- Communications Encryption}{CommEncryption} chapter
   of this manual.
   For more information on using PKI data encryption, please see the
   \ilink{Bacula PKI -- Data Encryption}{DataEncryption}
   chapter of this manual.

\item [ {-}{\-}with-python=\lt{}path\gt{}]
   \index[general]{{-}{\-}with-python}
   This option enables Bacula support for Python.  If no path is supplied,
   configure will search the standard library locations for Python 2.2,
   2.3, 2.4, or 2.5.  If it cannot find the library, you will need to
   supply a path to your Python library directory.  Please see the
   \ilink{Python chapter}{PythonChapter} for the details of using Python
   scripting.

\item [ {-}{\-}with-libintl-prefix=\lt{}DIR\gt{}]
   \index[general]{{-}{\-}with-libintl-prefix}
   This option may be used to tell Bacula to search DIR/include and
   DIR/lib for the libintl headers and libraries needed for Native
   Language Support (NLS).

\item [ {-}{\-}enable-conio]
   \index[general]{{-}{\-}enable-conio}
   Tells Bacula to enable building the small, light weight readline
   replacement routine.  It is generally much easier to configure than
   readline, although, like readline, it needs either the termcap or
   ncurses library.

\item [ {-}{\-}with-readline=\lt{}readline-path\gt{}]
   \index[general]{{-}{\-}with-readline}
   Tells Bacula where {\bf readline} is installed.  Normally, Bacula will
   find readline if it is in a standard library.  If it is not found and no
   {-}{\-}with-readline is specified, readline will be disabled.  This
   option affects the Bacula build.  Readline provides the Console program
   with a command line history and editing capability and is no longer
   supported, so you are on your own if you have problems.

\item [ {-}{\-}enable-readline]
   \index[general]{{-}{\-}enable-readline}
   Tells Bacula to enable readline support.  It is normally disabled due to the
   large number of configuration  problems and the fact that the package seems to
   change in incompatible  ways from version to version.  

\item [ {-}{\-}with-tcp-wrappers=\lt{}path\gt{}]
   \index[general]{{-}{\-}with-tcp-wrappers}
   \index[general]{TCP Wrappers}
   \index[general]{Wrappers!TCP}
   \index[general]{libwrappers}
   This specifies that you  want TCP wrappers (man hosts\_access(5)) compiled in.
   The path is optional since  Bacula will normally find the libraries in the
   standard locations.  This option affects the Bacula build.  In specifying your
   restrictions in the {\bf /etc/hosts.allow}  or {\bf /etc/hosts.deny} files, do
   not use the {\bf twist}  option (hosts\_options(5)) or the Bacula process will
   be terminated. Note, when setting up your {\bf /etc/hosts.allow}
   or {\bf /etc/hosts.deny}, you must identify the Bacula daemon in
   question with the name you give it in your conf file rather than the
   name of the executable.
   
   For more information on configuring and testing TCP wrappers, please  see the 
   \ilink{Configuring and Testing TCP Wrappers}{wrappers}  section
   in the Security Chapter.  

   On SuSE, the libwrappers libraries needed to link Bacula are
   contained in the tcpd-devel package. On Red Hat, the package is named
   tcp\_wrappers.

\item [ {-}{\-}with-archivedir=\lt{}path\gt{} ]
   \index[general]{{-}{\-}with-archivedir}
   The directory used for disk-based backups.  Default value is /tmp.
   This parameter sets the default values in the bacula-dir.conf and bacula-sd.conf
   configuration files.  For example, it sets the Where directive for the
   default restore job and the Archive Device directive for the FileStorage
   device.

   This option is designed primarily for use in regression testing.
   Most users can safely ignore this option.

\item [ {-}{\-}with-working-dir=\lt{}working-directory-path\gt{} ]
   \index[general]{{-}{\-}with-working-dir}
   This option is mandatory and specifies a directory  into which Bacula may
   safely place files that  will remain between Bacula executions. For example, 
   if the internal database is used, Bacula will keep  those files in this
   directory.  This option is only used to modify the daemon  configuration
   files. You may also accomplish the same  thing by directly editing them later.
   The working directory  is not automatically created by the install process, so
   you  must ensure that it exists before using Bacula for the  first time. 

\item [ {-}{\-}with-base-port=\lt{}port=number\gt{}]
   \index[general]{{-}{\-}with-base-port}
   In order to run,  Bacula needs three TCP/IP ports (one for the Bacula 
   Console, one for the Storage daemon, and one for the File daemon).  The {\bf
   \verb:--:with-baseport} option will automatically assign three  ports beginning at
   the base port address specified. You may  also change the port number in the
   resulting configuration  files. However, you need to take care that the
   numbers  correspond correctly in each of the three daemon configuration 
   files. The default base port is 9101, which assigns ports 9101  through 9103.
   These ports (9101, 9102, and 9103) have been  officially assigned to Bacula by
   IANA.  This option is only used  to modify the daemon configuration files. You
   may also accomplish  the same thing by directly editing them later. 

\item [ {-}{\-}with-dump-email=\lt{}email-address\gt{}]
   \index[general]{{-}{\-}with-dump-email}
   This option specifies  the email address where any core dumps should be set.
   This option  is normally only used by developers.  

\item [ {-}{\-}with-pid-dir=\lt{}PATH\gt{}  ]
   \index[general]{{-}{\-}with-pid-dir}
   This specifies where Bacula should place the process id  file during
   execution. The default is: {\bf /var/run}.  This directory is not created by
   the install process, so  you must ensure that it exists before using Bacula
   the  first time.  

\item [ {-}{\-}with-subsys-dir=\lt{}PATH\gt{}]
   \index[general]{{-}{\-}with-subsys-dir}
   This specifies where Bacula should place the subsystem lock  file during
   execution. The default is {\bf /var/run/subsys}.  Please make sure that you do
   not specify the same directory  for this directory and for the {\bf sbindir}
   directory.  This directory is used only within the autostart scripts.  The
   subsys directory is not created by the Bacula install,  so you must be sure to
   create it before using Bacula. 

\item [ {-}{\-}with-dir-password=\lt{}Password\gt{}]
   \index[general]{{-}{\-}with-dir-password}
   This option allows you to specify the password used to  access the Director
   (normally from the Console program).  If it is not specified, configure will
   automatically create a random  password.  

\item [ {-}{\-}with-fd-password=\lt{}Password\gt{} ]
   \index[general]{{-}{\-}with-fd-password}
   This option allows you to specify the password used to  access the File daemon
   (normally called from the Director).  If it is not specified, configure will
   automatically create a random  password.  

\item [ {-}{\-}with-sd-password=\lt{}Password\gt{} ]
   \index[general]{{-}{\-}with-sd-password}
   This option allows you to specify the password used to access the Storage daemon
   (normally called from the Director).  If it is not specified, configure will
   automatically create a random  password.  

\item [ {-}{\-}with-dir-user=\lt{}User\gt{} ]
   \index[general]{{-}{\-}with-dir-user}
   This option allows you to specify the Userid used to run the Director.  The
   Director must be started as root, but doesn't need to run as root, and
   after doing preliminary initializations, it can "drop" to the UserId
   specified on this option.  
   If you specify this option, you must
   create the User prior to running {\bf make install}, because the
   working directory owner will be set to {\bf User}.
                       
\item [ {-}{\-}with-dir-group=\lt{}Group\gt{} ]
   \index[general]{{-}{\-}with-dir-group}
   This option allows you to specify the GroupId used to  run the Director. The
   Director must be started as root, but  doesn't need to run as root, and  after
   doing preliminary initializations, it can "drop"  to the GroupId specified
   on this option. 
   If you specify this option, you must
   create the Group prior to running {\bf make install}, because the
   working directory group will be set to {\bf Group}.

\item [ {-}{\-}with-sd-user=\lt{}User\gt{} ]
   \index[general]{{-}{\-}with-sd-user}
   This option allows you to specify the Userid used to  run the Storage daemon.
   The Storage daemon must be started as root, but  doesn't need to run as root,
   and  after doing preliminary initializations, it can "drop"  to the UserId
   specified on this option. If you use this option,  you will need to take care
   that the Storage daemon has access  to all the devices (tape drives, ...) that
   it needs. 

\item [ {-}{\-}with-sd-group=\lt{}Group\gt{} ]
   \index[general]{{-}{\-}with-sd-group}
   This option allows you to specify the GroupId used to  run the Storage daemon.
   The Storage daemon must be started as root, but  doesn't need to run as root,
   and  after doing preliminary initializations, it can "drop"  to the GroupId
   specified on this option. 

\item [ {-}{\-}with-fd-user=\lt{}User\gt{} ]
   \index[general]{{-}{\-}with-fd-user}
   This option allows you to specify the Userid used to  run the File daemon. The
   File daemon must be started as root,  and in most cases, it needs to run as
   root, so this option is  used only in very special cases,  after doing
   preliminary initializations, it can "drop"  to the UserId specified on this
   option. 

\item [ {-}{\-}with-fd-group=\lt{}Group\gt{} ]
   \index[general]{{-}{\-}with-fd-group}
   This option allows you to specify the GroupId used to  run the File daemon.
   The File daemon must be started as root, and  in most cases, it must be run as
   root, however,  after doing preliminary initializations, it can "drop"  to
   the GroupId specified on this option. 

\item [ {-}{\-}with-mon-dir-password=\lt{}Password\gt{}]
   \index[general]{{-}{\-}with-mon-dir-password}
   This option allows you to specify the password used to  access the Directory
   from the monitor.  If it is not specified, configure will
   automatically create a random  password.  

\item [ {-}{\-}with-mon-fd-password=\lt{}Password\gt{} ]
   \index[general]{{-}{\-}with-mon-fd-password}
   This option allows you to specify the password used to  access the File daemon
   from the Monitor.  If it is not specified, configure will
   automatically create a random  password.  

\item [ {-}{\-}with-mon-sd-password=\lt{}Password\gt{} ]
   \index[general]{{-}{\-}with-mon-sd-password}
   This option allows you to specify the password used to  access the
   Storage daemon from the Monitor. If it is not specified, configure will
   automatically create a random  password.  

\item [ {-}{\-}with-db-name=\lt{}database-name\gt{} ]
   \index[general]{{-}{\-}with-db-name}
   This option allows you to specify the database name to be used in
   the conf files.  The default is bacula.

\item [ {-}{\-}with-db-user=\lt{}database-user\gt{} ]
   \index[general]{{-}{\-}with-db-user}
   This option allows you to specify the database user name to be used in
   the conf files.  The default is bacula.

\end{description}

Note, many other options are presented when you do a {\bf ./configure
\verb:--:help}, but they are not implemented.

\section{Recommended Options for Most Systems}
\index[general]{Systems!Recommended Options for Most}
\index[general]{Recommended Options for Most Systems}

For most systems, we recommend starting with the following options: 

\footnotesize
\begin{verbatim}
./configure \
  --enable-smartalloc \
  --sbindir=$HOME/bacula/bin \
  --sysconfdir=$HOME/bacula/bin \
  --with-pid-dir=$HOME/bacula/bin/working \
  --with-subsys-dir=$HOME/bacula/bin/working \
  --with-mysql=$HOME/mysql \
  --with-working-dir=$HOME/bacula/working
\end{verbatim}
\normalsize

If you want to install Bacula in an installation directory rather than run it
out of the build directory (as developers will do most of the time), you
should also include the \verb:--:sbindir and \verb:--:sysconfdir options with appropriate
paths. Neither are necessary if you do not use "make install" as is the case
for most development work. The install process will create the sbindir and
sysconfdir if they do not exist, but it will not automatically create the
pid-dir, subsys-dir, or working-dir, so you must ensure that they exist before
running Bacula for the first time.

\section{Red Hat}
\index[general]{Red Hat}

Using SQLite: 

\footnotesize
\begin{verbatim}
 
CFLAGS="-g -Wall" ./configure \
  --sbindir=$HOME/bacula/bin \
  --sysconfdir=$HOME/bacula/bin \
  --enable-smartalloc \
  --with-sqlite=$HOME/bacula/depkgs/sqlite \
  --with-working-dir=$HOME/bacula/working \
  --with-pid-dir=$HOME/bacula/bin/working \
  --with-subsys-dir=$HOME/bacula/bin/working \
  --enable-bat \
  --enable-conio
\end{verbatim}
\normalsize

or 

\footnotesize
\begin{verbatim}
 
CFLAGS="-g -Wall" ./configure \
  --sbindir=$HOME/bacula/bin \
  --sysconfdir=$HOME/bacula/bin \
  --enable-smartalloc \
  --with-mysql=$HOME/mysql \
  --with-working-dir=$HOME/bacula/working
  --with-pid-dir=$HOME/bacula/bin/working \
  --with-subsys-dir=$HOME/bacula/bin/working
  --enable-conio
\end{verbatim}
\normalsize

or finally, a completely traditional Red Hat Linux install: 

\footnotesize
\begin{verbatim}
CFLAGS="-g -Wall" ./configure \
  --sbindir=/usr/sbin \
  --sysconfdir=/etc/bacula \
  --with-scriptdir=/etc/bacula \
  --enable-smartalloc \
  --enable-bat \
  --with-mysql \
  --with-working-dir=/var/bacula \
  --with-pid-dir=/var/run \
  --enable-conio
\end{verbatim}
\normalsize

Note, Bacula assumes that /var/bacula, /var/run, and /var/lock/subsys exist so
it will not automatically create them during the install process. 

\section{Solaris}
\index[general]{Solaris}

To build Bacula from source, you will need the following installed on your
system (they are not by default): libiconv, gcc 3.3.2, stdc++, libgcc (for
stdc++ and gcc\_s libraries), make 3.8 or later. 

You will probably also need to: Add /usr/local/bin to PATH and Add
/usr/ccs/bin to PATH for ar. 

It is possible to build Bacula on Solaris with the Solaris compiler, but
we recommend using GNU C++ if possible.  

A typical configuration command might look like:

\footnotesize
\begin{verbatim}
#!/bin/sh
CFLAGS="-g" ./configure \
  --sbindir=$HOME/bacula/bin \
  --sysconfdir=$HOME/bacula/bin \
  --with-mysql=$HOME/mysql \
  --enable-smartalloc \
  --with-pid-dir=$HOME/bacula/bin/working \
  --with-subsys-dir=$HOME/bacula/bin/working \
  --with-working-dir=$HOME/bacula/working
\end{verbatim}
\normalsize

As mentioned above, the install process will create the sbindir and sysconfdir
if they do not exist, but it will not automatically create the pid-dir,
subsys-dir, or working-dir, so you must ensure that they exist before running
Bacula for the first time.

Note, you may need to install the following packages to build Bacula
from source:
\footnotesize
\begin{verbatim}
SUNWbinutils,
SUNWarc,
SUNWhea,
SUNWGcc,
SUNWGnutls
SUNWGnutls-devel
SUNWGmake
SUNWgccruntime
SUNWlibgcrypt
SUNWzlib
SUNWzlibs
SUNWbinutilsS
SUNWGmakeS
SUNWlibm

export 
PATH=/usr/bin::/usr/ccs/bin:/etc:/usr/openwin/bin:/usr/local/bin:/usr/sfw/bin:/opt/sfw/bin:/usr/ucb:/usr/sbin
\end{verbatim}
\normalsize

If you have installed special software not normally in the Solaris
libraries, such as OpenSSL, or the packages shown above, then you may need
to add {\bf /usr/sfw/lib} to the library search path.  Probably the
simplest way to do so is to run:

\footnotesize
\begin{verbatim}
setenv LDFLAGS "-L/usr/sfw/lib -R/usr/sfw/lib"
\end{verbatim}
\normalsize

Prior to running the ./configure command.

Alternatively, you can set the LD\_LIBARY\_PATH and/or the LD\_RUN\_PATH
environment variables appropriately.

It is also possible to use the {\bf crle} program to set the library
search path.  However, this should be used with caution.

\section{FreeBSD}
\index[general]{FreeBSD}

Please see: 
\elink{The FreeBSD Diary}{http://www.freebsddiary.org/bacula.php} for a
detailed description on how to make Bacula work on your system. In addition,
users of FreeBSD prior to 4.9-STABLE dated Mon Dec 29 15:18:01 2003 UTC who
plan to use tape devices, please see the 
\ilink{Tape Testing Chapter}{FreeBSDTapes} of this manual for
{\bf important} information on how to configure your tape drive for
compatibility with Bacula. 

If you are using Bacula with MySQL, you should take care to compile MySQL with
FreeBSD native threads rather than LinuxThreads, since Bacula is normally built
with FreeBSD native threads rather than LinuxTreads. Mixing the two will
probably not work. 

\section{Win32}
\index[general]{Win32}

To install the binary Win32 version of the File daemon please see the 
\ilink{Win32 Installation Chapter}{Win32Chapter} in this document. 

\section{One File Configure Script}
\index[general]{Script!One File Configure}
\index[general]{One Files Configure Script}

The following script could be used if you want to put everything
in a single file:

\footnotesize
\begin{verbatim}
#!/bin/sh
CFLAGS="-g -Wall" \
  ./configure \
    --sbindir=$HOME/bacula/bin \
    --sysconfdir=$HOME/bacula/bin \
    --mandir=$HOME/bacula/bin \
    --enable-smartalloc \
    --enable-bat \
    --enable-bwx-console \
    --enable-tray-monitor \
    --with-pid-dir=$HOME/bacula/bin/working \
    --with-subsys-dir=$HOME/bacula/bin/working \
    --with-mysql \
    --with-working-dir=$HOME/bacula/bin/working \
    --with-dump-email=$USER@your-site.com \
    --with-job-email=$USER@your-site.com \
    --with-smtp-host=mail.your-site.com
exit 0
\end{verbatim}
\normalsize

You may also want to put the following entries in your {\bf /etc/services}
file as it will make viewing the connections made by Bacula easier to
recognize (i.e. netstat -a): 

\footnotesize
\begin{verbatim}
bacula-dir      9101/tcp
bacula-fd       9102/tcp
bacula-sd       9103/tcp
\end{verbatim}
\normalsize

\section{Installing Bacula}
\index[general]{Bacula!Installing}
\index[general]{Installing Bacula}

Before setting up your configuration files, you will want to install Bacula in
its final location. Simply enter: 

\footnotesize
\begin{verbatim}
make install
\end{verbatim}
\normalsize

If you have previously installed Bacula, the old binaries will be overwritten,
but the old configuration files will remain unchanged, and the "new"
configuration files will be appended with a {\bf .new}. Generally if you have
previously installed and run Bacula you will want to discard or ignore the
configuration files with the appended {\bf .new}. 

\section{Building a File Daemon or Client}
\index[general]{Client!Building a File Daemon or}
\index[general]{Building a File Daemon or Client}

If you run the Director and the Storage daemon on one machine and you wish to
back up another machine, you must have a copy of the File daemon for that
machine. If the machine and the Operating System are identical, you can simply
copy the Bacula File daemon binary file {\bf bacula-fd} as well as its
configuration file {\bf bacula-fd.conf} then modify the name and password in
the conf file to be unique. Be sure to make corresponding additions to the
Director's configuration file ({\bf bacula-dir.conf}). 

If the architecture or the OS level are different, you will need to build a
File daemon on the Client machine. To do so, you can use the same {\bf
./configure} command as you did for your main program, starting either from a
fresh copy of the source tree, or using {\bf make\ distclean} before the {\bf
./configure}. 

Since the File daemon does not access the Catalog database, you can remove
the {\bf \verb:--:with-mysql} or {\bf \verb:--:with-sqlite} options, then
add {\bf \verb:--:enable-client-only}.  This will compile only the
necessary libraries and the client programs and thus avoids the necessity
of installing one or another of those database programs to build the File
daemon.  With the above option, you simply enter {\bf make} and just the
client will be built.

\label{autostart}
\section{Auto Starting the Daemons}
\index[general]{Daemons!Auto Starting the}
\index[general]{Auto Starting the Daemons}

If you wish the daemons to be automatically started and stopped when your
system is booted (a good idea), one more step is necessary. First, the
./configure process must recognize your system -- that is it must be a
supported platform and not {\bf unknown}, then you must install the platform
dependent files by doing: 

\footnotesize
\begin{verbatim}
(become root)
make install-autostart
\end{verbatim}
\normalsize

Please note, that the auto-start feature is implemented only on systems
that we officially support (currently, FreeBSD, Red Hat/Fedora Linux, and
Solaris), and has only been fully tested on Fedora Linux.

The {\bf make install-autostart} will cause the appropriate startup scripts
to be installed with the necessary symbolic links.  On Red Hat/Fedora Linux
systems, these scripts reside in {\bf /etc/rc.d/init.d/bacula-dir} {\bf
/etc/rc.d/init.d/bacula-fd}, and {\bf /etc/rc.d/init.d/bacula-sd}.  However
the exact location depends on what operating system you are using.

If you only wish to install the File daemon, you may do so with: 

\footnotesize
\begin{verbatim}
make install-autostart-fd
\end{verbatim}
\normalsize

\section{Other Make Notes}
\index[general]{Notes!Other Make}
\index[general]{Other Make Notes}

To simply build a new executable in any directory, enter: 

\footnotesize
\begin{verbatim}
make
\end{verbatim}
\normalsize

To clean out all the objects and binaries (including the files named 1, 2, or
3, which are development temporary files), enter: 

\footnotesize
\begin{verbatim}
make clean
\end{verbatim}
\normalsize

To really clean out everything for distribution, enter: 

\footnotesize
\begin{verbatim}
make distclean
\end{verbatim}
\normalsize

note, this cleans out the Makefiles and is normally done from the top level
directory to prepare for distribution of the source. To recover from this
state, you must redo the {\bf ./configure} in the top level directory, since
all the Makefiles will be deleted. 

To add a new file in a subdirectory, edit the Makefile.in in that directory,
then simply do a {\bf make}. In most cases, the make will rebuild the Makefile
from the new Makefile.in. In some case, you may need to issue the {\bf make} a
second time. In extreme cases, cd to the top level directory and enter: {\bf
make Makefiles}. 

To add dependencies: 

\footnotesize
\begin{verbatim}
make depend
\end{verbatim}
\normalsize

The {\bf make depend} appends the header file dependencies for each of the
object files to Makefile and Makefile.in. This command should be done in each
directory where you change the dependencies. Normally, it only needs to be run
when you add or delete source or header files. {\bf make depend} is normally
automatically invoked during the configuration process. 

To install: 

\footnotesize
\begin{verbatim}
make install
\end{verbatim}
\normalsize

This not normally done if you are developing Bacula, but is used if you are
going to run it to backup your system. 

After doing a {\bf make install} the following files will be installed on your
system (more or less). The exact files and location (directory) for each file
depends on your {\bf ./configure} command (e.g. if you are using SQLite instead
of MySQL, some of the files will be different).

NOTE: it is quite probable that this list is out of date.  But it is a
starting point.

\footnotesize
\begin{verbatim}
bacula
bacula-dir
bacula-dir.conf
bacula-fd
bacula-fd.conf
bacula-sd
bacula-sd.conf
bacula-tray-monitor
tray-monitor.conf
bextract
bls
bscan
btape
btraceback
btraceback.gdb
bconsole
bconsole.conf
create_mysql_database
dbcheck
delete_catalog_backup
drop_bacula_tables
drop_mysql_tables
make_bacula_tables
make_catalog_backup
make_mysql_tables
mtx-changer
query.sql
bsmtp
startmysql
stopmysql
bwx-console
bwx-console.conf
9 man pages
\end{verbatim}
\normalsize

\label{monitor}

\section{Installing Tray Monitor}
\index[general]{Monitor!Installing Tray}
\index[general]{Installing Tray Monitor}

The Tray Monitor is already installed if you used the {\bf
\verb:--:enable-tray-monitor} configure option and ran {\bf make install}.

As you don't run your graphical environment as root (if you do, you should
change that bad habit), don't forget to allow your user to read {\bf
tray-monitor.conf}, and to execute {\bf bacula-tray-monitor} (this is not a
security issue).

Then log into your graphical environment (KDE, GNOME or something else), run
{\bf bacula-tray-monitor} as your user, and see if a cassette icon appears
somewhere on the screen, usually on the task bar.
If it doesn't, follow the instructions below related to your environment or
window manager. 

\subsection{GNOME}
\index[general]{GNOME}

System tray, or notification area if you use the GNOME terminology, has been
supported in GNOME since version 2.2. To activate it, right-click on one of
your panels, open the menu {\bf Add to this Panel}, then {\bf Utility} and
finally click on {\bf Notification Area}. 

\subsection{KDE}
\index[general]{KDE}

System tray has been supported in KDE since version 3.1. To activate it,
right-click on one of your panels, open the menu {\bf Add}, then {\bf Applet}
and finally click on {\bf System Tray}. 

\subsection{Other window managers}
\index[general]{Managers!Other window}
\index[general]{Other window managers}

Read the documentation to know if the Freedesktop system tray standard is
supported by your window manager, and if applicable, how to activate it. 

\section{Modifying the Bacula Configuration Files}
\index[general]{Modifying the Bacula Configuration Files}
\index[general]{Files!Modifying the Bacula Configuration}

See the chapter 
\ilink{Configuring Bacula}{ConfigureChapter} in this manual for
instructions on how to set Bacula configuration files. 
